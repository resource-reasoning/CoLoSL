\section{\colosl worlds and assertions}
\label{sec:logic}

We formally introduce the assertions of \colosl and their models,
starting with the latter.

\subsection{Worlds}

\paragraph{Overview}
%% Incomplete, subjective view of the shared state forces us to revisit many
%% semantic concepts related to shared resources and interference.
%%  In doing so, we find that, although we pay a price of subtlety for
%% some definitions, our more general setting makes for cleaner
%% definitions (TODO: not good; not sure what I want to say here,
%% something like ``because we have to be more general, we cannot
%% weasel out of some difficulties, but that actually makes the
%% setting simpler in some respects, eg no token games, ability to
%% manipulate the interference relations, etc.'').
A \emph{world} is a 4-tuple $(l,g,\lmod,\gmod)$ where $l$ and $g$ are
\emph{logical states} and $\lmod$ and $\gmod$ are \emph{action
  models}; let us explain the role of each component informally. The
\emph{local (logical) state}, or simply local state, $l$ represents
the locally owned resources of a thread (\textit{e.g.} in
\fig{fig:concurrentInc}, the capability $\{\token a_{\li y}\}$ for
$\mathbb{P}_{\li y}$, and the variables \li{x}, \li{y}, \li{z} at the
beginning of $\mathbb{P}$). The \emph{shared logical state}, or shared
state, $g$ represents the \emph{whole} current shared state
(\textit{e.g.} variables \li{x}, \li{y}, \li{z} in
$\mathbb{P}_{\li{y}}$), and is subject to interferences as described
by the action models (which represent, \textit{e.g.}, the various
interference assertions $I$, $I_{\li{y}}$, \ldots).

An action model is a partial function from \emph{capabilities} to sets
of \emph{actions}. An action is a pair $(p,q)$ of logical states where
$p$ is the \emph{pre-state} of the action and $q$ its
\emph{post-state}.  The \emph{local action model} $\lmod$ corresponds
directly to the (semantic interpretation of) an interference relation
$I$. The \emph{global action model} $\gmod$ extrapolates the effect of
some of the actions in $\lmod$ to the global shared state
$g$. Although worlds do not put further constraints on the
relationship between $\lmod$ and $\gmod$, they will be linked more
tightly in the semantics of assertions (\S\ref{sec:assertions}): as we
shall see, $\gmod$ records at least those actions that matter for the
current subjective view.

As $g$ represents the global shared state, we will also require that
actions in $\lmod$ and $\gmod$ are \emph{confined} to $g$.  We write
$g \containI \lmod$ to denote that the part of state mutated by
actions in $\lmod$ is always fully contained in $g$. As threads may
uniterally decide to bring in part of their local states into the
shared state at any point (by \eqref{eq:extend}), this ensures that
existing actions cannot affect future extra pieces of shared state. We
will similarly require that the new actions associated with newly
shared state are confined to that extension in the same sense, hence
extending the shared state cannot retroactively invalidate the views
of other threads. As we shall see, confinement does not however
prohibit \emph{referring} to existing parts of the shared state in the
new actions; rather, it only safeguards against mutation of the
already shared resources through new actions.

Finally, the composition of two worlds will be defined whenever their
local states are disjoint and they agree on all other three
components, hence have identical knowledge of the shared state and
possible interferences.

\paragraph{Worlds and actions}
We start by defining logical states, which are \colosl's notion of
\emph{resource}, in the standard separation logic sense. Logical
states have two components: one describes machine states; the other
represents \emph{capabilities}. The latter are inspired by the
capabilities in deny-guarantee reasoning~\cite{dg}: a thread in
possession of a given capability is allowed to perform the associated
actions (as prescribed by the \emph{action model} components of each
world, defined below), while any capability \emph{not} owned by a
thread means that the environment can perform the action.\footnote{In
  general, capabilities could be fractionally owned, in which case
  ownership of a \emph{fraction} of a capability grants the right to
  perform the action to both the thread and the environment, while a
  fully-owned capability \emph{denies} the right to the environment to
  perform the associated action. For ease of exposition, we only
  consider the case of fully-owned capabilities in our formal
  development. See the technical report for the general case.}

\begin{definition}[Logical states]
  A \emph{logical state} is a tuple $((@s, h), \ca{})$, also written
  $(@s,h,\ca{})$, of a finite partial \emph{stack} $@s\in\set{Stack}$
  associating program variables with values, a heap $h \in \set{Heap}$
  associating heap locations with values, and a capability $\ca{}
  \in\Caps$.
  \begin{mathpar}
    \set{Stack} == \set{PVar} --`_{\m{fin}} \set{Val}

    \set{Heap} == \set{Loc} --`_{\m{fin}} \set{Val}

    \Heaps == \set{Stack}\times \set{Heap}

    \Caps == \powerset(\set{Token})

    \LStates == \Heaps\times \Caps
  \end{mathpar}
  We write $\unitL$ for the logical state $(\emptyset, \unitH,
  \unitCap)$. The \emph{composition of logical states} $ \composeL :
  \LStates \times \LStates \rightharpoonup \LStates $ is defined as
  the disjoint union $\uplus$ of their individual components:
  \[
  (@s, h,\ca{}) \composeL (@s', h', \ca{}') \eqdef
  (@s\uplus @s', h\composeH h', \ca{}\composeCap \ca{}')
  \]
\end{definition}
Although, for this presentation, we build logical states from a fixed
$\Heaps$ and $\Caps$, the general setting allows these two objects to
be instantiated with \emph{any} separation algebras (\textit{i.e.}
cancellative partial commutative monoids~\cite{asl}) satisfying a
technical \emph{cross-split} condition~\cite{colosl-tr14}.

In the following, we write $l_1\leq l_2$ when there exists $l$ such
that $l\composeL l_1 = l_2$. We also write $l_2 - l_1$ to denote the
unique (by cancellativity) such $l$. When $l_1$ and $l_2$ are
compatible according to $\composeL$, we write $l_1\compatL l_2$. In
the rest of this paper, we will use $l$ to denote either an arbitrary
logical state or one that is to be understood as a local state, and
$g$ for a (global) shared state.

An action is simply a pair of logical states describing the pre- and
post-states of the action, and an action model describes the set of
actions enabled by each capability.

\begin{definition}[Action]
  An action $a\in\set{Action}$ is a pair of logical states:
  \[
  \set{Action} == \LStates \times \LStates
  \]
\end{definition}

\begin{definition}[Action models]
An \emph{action model} is a partial function from capabilities to
actions:
\[
\lmod \in \AMods \eqdef \Caps \rightharpoonup \pset{\set{Action}}
\]
We write $\unitAM$ for an action model with empty domain.
\end{definition}


\paragraph{The effect of actions} As mentioned in the beginning of the
section, we require that the effect of actions in the local and global
action models are \emph{confined} to the shared state. This means
that, given an action $a = (p,q)$ and a shared state $g$, whenever the
precondition $p$ is \emph{compatible} with $g$ then the part of the
state mutated by the state is contained in $g$. Compatibility of $p$
and $g$ is interpreted \emph{loosely}: it merely means that $p$ and
$g$ agree on the resource they have in common. In particular, $g$ need
not contain $p$ entirely for $a$ to take effect. This relaxation is
due to the fact that other threads may extend the shared state; in
particular, the extension may provide the missing resources for $p$ to
be contained in the shared state, thus allowing the extending thread
to perform action $a$. Crucially, however, it is not the case that any
part of $p$ is allowed to be missing from $g$. Rather, only parts of
$p$ that are not mutated by $a$, \textit{i.e.}, which can also be
found in $q$, can escape $g$'s grasp. Mutated parts must always be
contained in $g$, so that extensions need not worry about existing
actions interfering with new resources that were never shared
beforehand. We start by characterising the active and passive parts of
actions, then move on to define the effect of actions. This will
enable us to define our confinement condition, and then the set of
well-formed worlds.

An action $a = (p, q)$ is typically of the form $(p'\composeL c,
q'\composeL c)$, \textit{i.e.} part of the state $c$ required for
the action is \emph{passive} and acts as a mere \emph{catalyst} for
the action: it has to be present for the action to take effect, but is
left unchanged by the action. The \emph{active} part of the action is then
the pair $(p',q')$, which should be maximal in the sense that no
further, non-empty catalyst can be found in $(p',q')$.

\begin{definition}[Active part of an action]
  Given a pair of logical states $(p, q)$, its \emph{active part}
  $\updateFP{(p,q)}$ is defined as the pair $(p', q')$ such that:
  \[
  \E c p = p' \composeL c /| q = q' \composeL c /| \V{c'} c'\leq p' /|
  c'\leq q' => c' = \unitL
  \]
\end{definition}

For instance, the interpretations of the actions associated with token
$\token a_{\li{x}}$ in the previous section all have the same active
parts
$([x:v],\emptyset,\emptyset),([x:v+1],\emptyset,\emptyset)$. Their
catalysts are of the form $([z:v],\emptyset,\emptyset)$ in
the interpretations of $I$, $I_{\li{x}}$, and $I_{\li{z}}$, and
$([y:v,z:v],\emptyset,\emptyset)$ in $I_{\li{y}}$.


The compatibility of an action prestate $p$ and the shared state $g$
will be defined using the following notion of \emph{intersection} of
logical states.

\begin{definition}[Intersection]
The \emph{intersection} function over logical states 
$
\meetL : \left(\LStates \times \LStates \right) \rightarrow \pset{\LStates}
$
is defined as:
\[
l_1 \meetL l_2 \eqdef 
\left\{ 
l  \mid
\exsts{l', l_1', l_2'}\! l_1 = l \composeL l_1' \land l_2 = l \composeL l_2' \land l \composeL l_1' \composeL l_2' = l'
\right\}
\]
%% The \emph{maximal intersection} $l_1 \maxMeetL l_2$ of $l_1$ and $l_2$
%% is defined as the largest element of the intersection of $l_1$ and
%% $l_2$, when it exists.
\end{definition}

%% One can check that $l_1\maxMeetL l_2$ indeed yields at most one
%% element for any given $l_1$ and $l_2$~\cite{colosl-tr14}.
One can check that $l_1\meetL l_2$ yields at most one element for any
given $l_1$ and $l_2$~\cite{colosl-tr14}. Thus, we will usually omit
the set notation.\footnote{In the general case, separation algebras
  may lack the \emph{disjointness} property, in which case
  intersection between two states may yield more than one state. Thus,
  our general definition introduces \emph{maximal intersections} to
  restore unicity. Since in this paper our logical states satisfy disjointness,
  intersection is enough for our exposition.}

\begin{definition}[Action application]
  The \emph{application} $a[g]$ of an action $a$ on a logical state
  $g$ is defined provided that there is $l$ such that
  \[
    \m{fst}(a) \meetL g /=\emptyset /|
    g = \m{fst}(\updateFP{a}) \composeL l /|
    \m{snd}(\updateFP{a})\compatL l
  \]
  When that is the case, we write $a[g]$ for the (uniquely defined)
  logical state $\m{snd}(\updateFP{a})\composeL l$. We write
  $\m{potential}(a,g)$ to denote that $a[g]$ is defined.
\end{definition}

\begin{definition}[Action confinement]
  \label{def:actconf}
  An action $a$ is \emph{confined} to a logical state $g$, written
  $g\containI a$, if
  \[
  \m{fst}(a)\meetL g \neq\emptyset => \m{fst}(\updateFP{a})\leq g
  \]
\end{definition}

Observe that $\m{potential}(a,g) => g\containI a$.  As discussed, only
the \emph{active} precondition of $a$, \textit{i.e.}  the part
actually mutated by the action, has to be contained in $g$.  The
condition we will require on actions with respect to the global shared
world $g$ is that the actions of the local action model $\lmod$ are
confined in all possible \emph{futures} of $g$, \textit{i.e.} all the
shared states that result from $g$ after any number of applications of
actions in $\lmod$. For some of these actions, the active pre-state
might be empty. In that case, we find that we do not need to take the
action into account when considering the effect of all actions on the
shared state, even though it is potentially enabled. We thus introduce
the following notion of \emph{visible actions}.

\begin{definition}[Visible actions]
  An action $a$ is called \emph{visible in $g$}, written
  $\m{visible}(a,g)$ when
  \[
  \E{l} \m{fst}(\updateFP{a})\meetL g = l /| l/=\unitL
  \]
\end{definition}


We are now ready to define our confinement condition on actions.
Inspired by the LRG logic~\cite{lrg}, we introduce the concept of
locally fenced action models to capture all possible states reachable
from the current state via some number of action applications. A set
of states $\fence{}$ \emph{fences} an action model if it is invariant
under interferences perpetrated by the corresponding actions. We write
$\m{rg}(f)$ to denote the \emph{range} (or codomain) of a function
$f$.

\begin{definition}[Locally-fenced action model]
  An action model $\lmod \in \AMods$ is \emph{locally fenced} by
  $\fence{} \in \pset{\LStates}$, written $\fence{} \strictfences \lmod$,
  iff, for all $g \in \fence{}$ and all $a\in\m{rg}(\lmod)$,
\[
\begin{array}{L}
  g\containI a /|
  (\m{potential}(a,g) /| \m{visible}(a,g) => a[g]\in\fence{})
\end{array}
\]
\end{definition}

\begin{definition}[Action model confinement]
  An action model $\lmod$ is \emph{confined} to a logical state
  $l$, written $l\containI\lmod$, if there exists a fence $\fence{}$
  such that $l\in\fence{}$ and $\fence{}\strictfences{}\lmod$.
\end{definition}



A world $(l,g,\lmod,\gmod)$ is well-formed if $l$ and $g$ are
compatible, the capabilities found in $l\composeL g$ exist in the
local action model $\lmod$, actions in the global action model $\gmod$
correspond to actions in $\lmod$, and $\gmod$ is confined to $g$.

\begin{definition}[Well-formedness]
  A 4-tuple $(l, g, \lmod, \gmod)$ is \emph{well-formed},
  written $\wf{(l, g, \lmod, \gmod)}$, if
  \[
  \begin{array}{L}
    (\exsts{\h{},\ca{}}
    l \composeL g = (\h{},\ca{}) \land \ca{}\subseteq \dom{\lmod})
    /|\null\\
    (\V{\ca{}}\V{a\in\gmod(\ca{})}\E{a'\in\lmod(\ca{})}
    \updateFP{a} = \updateFP{a'}) /| g \containI \lmod
  \end{array}
  \]
\end{definition}

Note that we also have that $g\containI \gmod$.

\begin{definition}[Worlds]
  \label{def:worlds}
  The set of \emph{worlds} is defined as
  \begin{mathpar}
    \Worlds \eqdef \{w\in
    \LStates\times\LStates\times\AMods\times\AMods ||| \p{wf}(w)\}
  \end{mathpar}
  The \emph{composition} $\composeW$ between worlds is defined as
  \begin{align*}
    &\qquad (l,g,\lmod,\gmod) \composeW
    (l',g',\lmod',\gmod') \\
    &\eqdef
    \begin{cases}
      (l\composeL l', g, \lmod, \gmod) &
      \begin{array}[t]{L}
        \text{if }
        g = g' \text{, }
        \lmod = \lmod' \text{, and } \gmod = \gmod'\\[-.7ex]
        \text{and }\p{wf}((l\composeL l', g, \lmod, \gmod))
      \end{array}\\
      \textit{undefined}&\text{otherwise}
    \end{cases}
  \end{align*}
\end{definition}

The set of worlds with composition $\composeW$ forms a separation
algebra with multiple units, which are all the well-formed states of
the form $(\unitL,g,\lmod,\gmod)$.


\subsection{Assertions}
\label{sec:assertions}

Our assertions extend standard separation logic with \emph{subjective
  views}, as introduced in the previous section. \colosl is parametric
with respect to the machine states and capability assertions and can be
instantiated with any assertion language over states $\Heaps$ and
capabilities $\Caps$. Here, we instantiate these languages for our
particular instantiation with respect to our model. We assume infinite
disjoint sets $\set{PVar}$, $\set{LVar}$, and $\set{Token}$ of program
variables, logical variables, and tokens, respectively.

\begin{definition}[Assertion syntax]
  \label{def:assertions}
  The assertions of \colosl are elements of $\Assertions$ described by
  the grammar below, where $\li x$ ranges over program variables, $x$
  over logical variables, and $\token a$ over tokens.
  \begin{align*}
    E &::= x ||| 0, 1, \ldots ||| E_1 + E_2 ||| \cdots
    \qquad\qquad
    K ::= a ||| K * K\\
    A &::=\m{false} \mid E_1 = E_2 ||| \emp ||| \li{x}|-> E |||
    E_1 |-> E_2 \mid [K] \\
    P, Q  &::= 
    A \mid P \Rightarrow Q \mid \exsts{x} P \mid
     P * Q \mid P --* Q \mid P \sepish Q% \mid P \intersect Q
     \mid \shared{P}{I} \\
    I &::= \emptyset ||| \interAss{K}{\vec{y}}{P}{Q}, I
  \end{align*}
\end{definition}

This syntax follows from standard separation logic with variables as
resource~\cite{entcs06} (notice that expressions $E$ do not allow
program variables), with the exception of shared-state assertions
$\shared P I$. $\emp$ is true of the units of $\composeW$. Predicates
are interpreted over the logic state of a world: $\li{x}|-> E$
(resp.\ $E_1|->E_2$) is true of the singleton stack (resp.\ heap)
where only \li{x} (resp.\ address $E_1$) is allocated and has value
$E$ (resp.\ points to $E_2$); capability assertions $[K]$ describe the
capability component of the local state: $\token a$ is true of the
singleton $\{a\}$, while the $*$ of capabilities is their disjoint
union; the spatial connectives $*$ and $--*$ are standard in
separation logic~\cite{rey02}, and $P**Q$ is the \emph{overlapping
  conjunction}, which states that the world can be split into three
disjoint sub-worlds, where the first two satisfy $P$ and the last two
satisfy $Q$~\cite{rey-slnotes}; classical predicates and connectives
have their standard classical meaning. Interference assertions $I$
describe actions enabled by a given capability, in the form of a pre
and post-condition. A shared state assertion $\shared P I$ is true of
$(l,g,\lmod,\gmod)$ when $l = \unitL$, a subjective view $s$ satisfying $P$ and
undergoing interferences $I$ can be found in the global shared state
$g$, \textit{i.e.} $g = s\composeL r$ for some $r$, and $I$, $\lmod$,
and $\gmod$ \emph{agree} given the decomposition $s$, $r$ in the
following sense:
\begin{enumerate}
\item
  Every action in $I$ is reflected in $\lmod{}$.
\item
  Every action in $\lmod$ with visible effect on $s$ is reflected in
  $I$.
\item
  Every action in $I$ that is enabled in $g$ is reflected is $\gmod$.
\item
  And the above is true after any number of action applications in
  $I$ that affect $g$ and any number of action applications in
  $\gmod$ that affect $r$ but not $s$.
\end{enumerate}

%% , and spatial predicates and connectives
%% have been informally introduced in the previous section

The semantics of \colosl assertions is given by a forcing relation
$w,\lenv|= P$ between a world $w$, an interpretation of logical
variables $\lenv\in\LEnv == \set{LVar} --> \set{Val}$ and a formula
$P$. We use three auxiliary forcing relations. The first one
$\ca{},\lenv|=_K K$ interprets capability assertions. The second one
$l,\lenv\slsat P$ interprets formulas $P$ in the usual separation
logic sense over a logical state $l$ (and ignores shared state
assertions). The third one $s,\lenv|=_{g,\lmod,\gmod} P$ interprets
assertions over a \emph{subjective view} $s$ that is part of the
global shared state $g$, subject to action models $\lmod$ and
$\gmod$. This third form of satisfaction is needed to deal with
nesting of boxed formulas.\footnote{This presentation with several
  forcing relations differs from the usual CAP
  presentation~\cite{cap-ecoop10}, where formulas are interpreted over
  worlds that are not necessarily well-formed, and then cut down to
  well-formed ones. We found that the CAP presentation strays away
  from separation logic models in some ways. For instance, in CAP, $*$
  and $--*$ and not adjoints of each other; they are in our
  presentation.}

Moreover, since logical connectives are interpreted uniformly in all
cases, we write $|=_\dagger$ for either of the three satisfaction
relations, and then write $u$ for elements of either $\Worlds$ or
$\LStates$, and $\gray$ for either $\composeW$ or $\composeL$,
depending on whether the satisfaction relation is
$|=$, or either $\slsat$ or $|=_{l,\lmod,\gmod}$,
respectively.

\begin{definition}[Assertion semantics]
  Given a logical environment $\lenv$, the semantics of \colosl
  assertions is as follows, where $\semI[(.)]{.}: \LEnv --> \AMods$
  denotes the semantics of interference assertions:
\[
\begin{array}{R>{\null}l@{\ \,}c@{\ \,}l}
  (l,g,\lmod,\gmod), \lenv &|= A &\text{iff}& l,\lenv \slsat A\\
  (l,g,\lmod,\gmod), \lenv &|= \shared P I &\text{iff}&
  l = \unitH\text{ and }
  \exsts{s,r}
  g = s\composeL r
  \text{ and}\\
  &&&s, \lenv |=_{g,\lmod,\gmod} P\text{ and }
  \extendsAM{\lmod, \gmod}{s}{r}{\semI[\lenv]{I}}
\end{array}
\]
\[
\begin{array}{R>{\null}l@{\ \,}c@{\ \,}l}
  s, \lenv &|=_{g,\lmod,\gmod} A &\text{iff}& s, \lenv \slsat A\\
  s, \lenv &|=_{g,\lmod,\gmod} \shared P I &\text{iff}&
  (s,g,\lmod,\gmod), \lenv |= \shared P I\\

  u,\lenv &|=_\dagger P => Q
  &\text{iff}& u,\lenv |=_\dagger P\text{ implies }u,\lenv |=_\dagger Q\\
  u,\lenv &|=_\dagger \exsts x P
  &\text{iff}& \exsts v u, [\lenv|||x:v] |=_\dagger P\\
  u, \lenv &|=_\dagger P_1 * P_2 &\text{iff}&
  \exsts{u_1,u_2} u = u_1\gray u_2\text{ and}\\
  &&& u_1, \lenv |=_\dagger P_1 \text{ and }u_2, \lenv |=_\dagger P_2\\
  u, \lenv &|=_\dagger P --* Q &\text{iff}&
  \for{u'} u', \lenv |=_\dagger P \text{ and }
  u \sharp u'\\
  &&&\text{ implies }u\gray u', \lenv |=_\dagger Q\\
  u, \lenv &|=_\dagger P_1 ** P_2 &\text{iff}&
  \exsts{u',u_1,u_2} u = u'\gray u_1\gray u_2\text{ and}\\
  &&&
  u'\gray u_1, \lenv |=_\dagger P_1 \text{ and }
  u'\gray u_2, \lenv |=_\dagger P_2\\

  l,\lenv &\slsat E_1 = E_2
  &\text{iff}& [|E_1|]_{\lenv} = [|E_2|]_{\lenv}\\
  l, \lenv &\slsat \li{x}|->E
  &\text{iff}&
  l =
  ([\li x: [|E|]_{\lenv}],\unitH,\unitCap)\\
  l, \lenv &\slsat E_1|->E_2 
  &\text{iff}&
  l =
  (\emptyset,[[|E_1|]_{\lenv}: [|E_2|]_{\lenv}],\unitCap)\\
  l, \lenv &\slsat [K]
  &\text{iff}&
  l = (\emptyset,\unitH, \ca{}) /| \ca{},\lenv|=_K K\\
%  &&\cdots\\
  l,\lenv &\slsat \m{false}
  && \text{never}\\
  l, \lenv &\slsat \emp &\text{iff}&   l, \lenv \slsat \shared P I
  \text{ iff } l = \unitL\\

  \ca{},\lenv &|=_K \token a
  &\text{iff}& \ca{} = \{\token a\}\\
  \ca{},\lenv &|=_K K_1 * K_2
  &\text{iff}& \E{\ca{1},\ca{2}} \ca{} = \ca{1}\uplus\ca{2} /|\null\\
  &&& \ca{1},\lenv|=_K K_1 /| \ca{2}|=_K K_2
\end{array}
\]
\vspace{-1em}
\begin{align*}
  \semI[\lenv]{I}(\ca{}) &==
  \left\{
  \begin{array}{@{}l@{\ }|@{\ }r@{}}
    (p,q)&
    \begin{array}{@{}l@{}}
      \interAss{\ca{}}{\vec{y}}{P}{Q} \in I /|\null\\
      \exsts{\vec{v}}
      p,[\lenv|||\vec y:\vec v] \slsat P \land
      q,[\lenv|||\vec y:\vec v] \slsat Q
    \end{array}
  \end{array}
  \right\}
  \end{align*}
\end{definition}


The semantics of separation logic predicates and connectives is
standard and depends only on the local state.  $\shared{P}{I}$ states
that $P$ holds for only a sub-state $s$ of the global shared state
$s\composeL r$. The interference associated with $s$ is given by
interference assertion $\semI[\lenv]{I}$ such that the global and
local action models $\lmod$ and $\gmod$ are \emph{closed}
under $\semI[\lenv]{I}$ with respect to the subjective view $s$ and
context $r$. This will be formalised just after this lemma.
%Since shared state assertions are partial subjective description of the shared state, separating conjunction between them behaves as overlapping conjunction 

\begin{lemma}
  \label{lem:assertionFacts}
  The following formulas are valid according to the semantics above
  (where $x$ does not appear free in $I$):
  \begin{align*}
    \shared{P * \shared{Q}{I'}}{I} &<=> \shared{P}{I} *
    \shared{Q}{I'}&
    \E x \shared P I &<=>\shared{\E x P} I\\
    (P => Q) &=> \shared{P}{I} => \shared{Q}{I}&
    \shared{P}{I} &=> \emp
  \end{align*}
\end{lemma}
\begin{proof}
  Immediate.
\end{proof}





%%%%%%%%%%%%%%%%%%%%%%%%%%%%%%%%%%%
\paragraph{Action model closure}
Let us now turn to the definition of action model closure, as
informally introduced at the beginning of this section. First, we need
to revisit the effect of actions to take into account the splitting of
the global shared state into a subjective state and a context.

\begin{definition}[Action application (cont.)]
  The application of action $a$ on the subjective state $s$ together
  with the context $r$, written $a[s,r]$, is defined
  provided that there are $s'$ and $r'$ such that
  \[
  \begin{array}{L}
  \m{fst}(a)\meetL (s\composeL r) /= \emptyset /|
  \E{p_s,p_r}
  s = p_s \composeL s' /|
  r = p_r \composeL r' /|\null\\
  \quad
  \m{fst}(\updateFP{a}) = p_s\composeL p_r /|
  \m{snd}(\updateFP{a})\composeL s'\composeL r'\text{ is defined}\\
  \end{array}
  \]
  When that is the case, we write $a[s,r]$ for
  $(\m{snd}(\updateFP{a})\composeL s',r')$.
\end{definition}

We observe that $\m{fst}(a[s,r]) \composeL \m{snd}(a[s,r]) =
a[s\composeL r]$.  In our informal description, we have mentioned that
some actions must be reflected in some action models. Here is the
formal definition.

\begin{definition}
  An action $a$ is \emph{reflected} in a set of actions $A$ from a state
  $l$, written $\m{reflected}(a,l,A)$, if
  \[
  \V r \m{fst}(a)\leq l\composeL r =>
  \E{a'\in A} \updateFP{a'} = \updateFP{a} /| \m{fst}(a')\leq
  l\composeL r
  \]
\end{definition}

Let us now give the formal definition of action model closure. We
annotate parts of the definition with the items of our informal
discussion below Defn.~\ref{def:assertions} that they implement. We
write
\[
\m{enabled}(a,g) == \m{potential}(a,g) /| \m{fst}(a)\leq g
\]
\textit{i.e.}\ $a$ can actually happen in $g$ since $g$ holds all the
resources in the prestate of $a$.

\begin{definition}[Action model closure]
  \label{def:actclos}
  A pair $(\lmod, \gmod)$ of action models is \emph{closed} under a
  subjective state $s$, context $r$, and action model $\lmod'$,
  written $\extendsAM{\lmod, \gmod}{s}{r}{\lmod'}$, if
  $\extendsAMUpto{\lmod, \gmod}{n}{s}{r}{\lmod '}$ for all
  $n\in\Nats$, where $\downarrow_n$ is defined recursively as follows:
\[
\begin{array}{L}
  \extendsAMUpto{\lmod, \gmod}{\mathrlap{0}\phantom{n}}{s}{r}{\lmod'}
  \iffdef
  \lmod' \subseteq \lmod
  \hfill\textit{(1.)}\\
  \extendsAMUpto{\lmod, \gmod}{n+1}{s}{r}{\lmod'} \iffdef (\V{\ca{}}
  \V{a\in \lmod'(\ca{})}\\
  (\m{potential}(a,s\composeL r) =>
  \extendsAMUptob{\lmod,
    \gmod}{n}{a[s, r]}{\lmod'}) \land\hfill \textit{(4.)}\\
  \quad\m{enabled}(a,s\composeL r)
  => (s\composeL r,
  a[s\composeL r])\in \gmod(\ca{}))
  /|\null\hfill \textit{(3.)}\\
  \V{\ca{}}\V{a\in \lmod(\ca{})}
  \m{potential}(a,s\composeL r) =>\null\\
  \ \m{reflected}(a,s\composeL r,\lmod'(\ca{})) |/\null\hfill \textit{(2.)}\\
  \ (\neg\m{visible}(a,s) /| \m{fst}(a[s,r]) = s /|
  \extendsAMUptob{\lmod, \gmod}{n}{a[s, r]}{\lmod'})\quad \textit{(4.)}
  %% \hspace*{0.1cm}\for{\ca{}, l', r', t, p, q, p', q', p_l, p_r}\\
  %% \hspace*{0.2cm}(p, q) \in \lmod'(\ca{}) 
  %% \land \updateFP{p, q} = (p', q')
  %% \land p \leq  l \composeL r \composeL t \land\null\\
  %% \hspace*{0.2cm}p' = p_l \composeL p_r 
  %% \land p' \maxMeetL l = p_l 
  %% \land l = p_l \composeL l' 
  %% \land r = p_r \composeL r' =>\\
  %% \hspace*{0.4cm} \extendsAMUpto{\lmod, \gmod}{n}{q' \composeL l'}{r'}{\lmod'} \land\null\\
  %% \hspace*{0.4cm}t = \unitL => 
  %% \left(
  %% \begin{array}{l}
  %%   (p' \composeL l' \composeL r', q' \composeL l' \composeL r') \in \lmod(\ca{})\\
  %%   \lor\; q' \composeL l' \composeL r' \text{ is undefined} 
  %% \end{array}
  %% \right)\\
  %% \hspace*{2cm} \land\\
  %% \hspace*{0.1cm}\for{\ca{}, p, q, p', q', l', r'} \\
  %% \hspace*{0.2cm} (p, q) \in \gmod(\ca{})
  %% \land \updateFP{p, q} = (p', q')
  %% \land p \composeL l' = l \composeL r \composeL r' =>\\
  %% \hspace*{0.4cm}\exsts{f, c'} (p' \composeL f, q' \composeL f) \in \lmod'(\ca{}) \land p' \composeL f \composeL c' =  l \composeL r \composeL r' \;\;\lor\\
  %% \hspace*{0.4cm}q \composeL l' \text{ is undefined} \;\;\lor\\
  %% \hspace*{0.4cm}p' \maxMeetL l = \unitL \land \exsts{r''} q \composeL l' = l \composeL r' \composeL r'' \land \extendsAMUpto{\lmod, \gmod}{n}{l}{r''}{\lmod'}
\end{array}
\]
\end{definition}

We make further observations about this definition. First, the closure
relation must hold after any number of either of the following two
steps on $g = s\composeL r$:
\begin{itemize}
\item
  An action $a$ of $\lmod'$ is potentially enabled in $g$, and $g$ is
  updated to $a[g']$.
\item
  An action $a$ of $\lmod$ is potentially enabled in $g$ but invisible
  as far as $s$ is concerned, and $g$ is updated to $a[g']$.
\end{itemize}
This makes our assertions robust with respect to future extensions of
the shared state, where potentially enabled actions may be become
enabled using additional catalyst that is not immediately
present. Second, $\gmod$ only has to record actions from $\lmod'$
that are enabled in any $g$ resulting from these steps, hence
immediately possible. Finally, $\lmod'$ (and thus interference
assertions) need not reflect actions that have no visible effect on
the subjective state.

%% in particular, the rely and guarantee relations will be
%% computed from actions in the global action model. The local action
%% model ensures that different threads agree on the set of all possible
%% actions (but whether an action is actually relevant to a given world
%% is given by the global action model).

%% The above states that the $(\lmod, \gmod)$ pair is closed under $(s, r, \lmod')$ if the closure relation holds for any number of steps $n \in \Nats$ where 1) $s$ denotes the subjective view of the shared state, 2) $r$ denotes the context, 3) $s \composeL r$ captures the entire shared state  and 4) a step corresponds to the occurrence of an action as prescribed in either the current action model $\lmod'$ or $\gmod$. The relation is satisfied trivially for no steps ($n = 0$). On the other hand, for an arbitrary $n\in \Nats$ the relation holds iff:
%% \begin{enumerate}
%% 	\item Given a capability $\ca{}$, its action $(p, q) \in \lmod'(\ca{})$ and its update footprint $(p', q')$, if the action precondition $p$ is satisfiable through an arbitrary extension of the shared region ($t$), then after one step the closure relation holds for the resultant subjective state $q' \composeL s'$ and context $r'$ where $s'$ and $r'$ capture the residue subjective and context states, respectively; that is parts of the states unchanged by the action. Moreover, if the action requires no extension ($t = \unitL$) and thus the action precondition is satisfied by the current shared state $s \composeL r$, then the action should be included in the global action model $\lmod$ unless the resultant state $q' \composeL s' \composeL r'$ is undefined.
	
%% 	\item Given a capability $\ca{}$, its action $(p, q) \in \gmod$ and its update footprint $(p', q')$, if the action precondition is satisfiable by the current shared state then \emph{either} 1) a similar action is visible to $\lmod'$, \emph{or} 2) the resultant state is undefined and thus the action can never take place, \emph{or} 3) the subjective state remains unchanged by the action $(p' \maxMeetL s = \unitL )$ and after one step the closure relation holds for the subjective state $s$ and the resultant context $r'$.
%% \end{enumerate}
%% Intuitively, the closure relation ensures that given the current subjective state $s$ and context $r$, all possible actions of $\lmod'$ are reflected in the global action model $\lmod$ \emph{and} for all actions that are possible and are yet unbeknownst to $\lmod'$, the subjective view $s$ remains unaffected.

%% As illustrated above, the global behaviour $\lmod$ relies on the division between the subjective state $s$ and the context $r$. As such, it is essential to calculate it upon defining the semantics of $\shared{P}{I}$ when the boundary between the subjective state described by $P$ and the context is discernible.  In other words, we cannot correctly identify the global action model associated with  a shared state $s$ and a localised action model $\gmod$ without knowledge of the division between the subjective state and the context. 

This completes the definition of the semantics of assertions. We can
now show that the logical principles of \colosl are sound. The proof
of the two semantic principles \eqref{eq:shift} and \eqref{eq:extend}
is delayed to the next section, where $===>$ will be defined.

\begin{lemma}\label{lem:semprinciples}
  The logical implications \eqref{eq:forget}, \eqref{eq:split}, and \eqref{eq:merge} are \emph{valid}.
\end{lemma}
\begin{proof}
The case of \eqref{eq:split} is straightforward from the semantics of
assertions.  For \eqref{eq:forget}, it suffices to remark that
whenever the action model closure relation holds for a subjective
state $s\leq g$ then it holds of any $s'\leq s$. For \eqref{eq:merge},
we remark that the two worlds corresponding to each subjective view
need to have the same global action model to be compatible, which
ensures closure under the combined subjective views. The full proof is
provided in~\cite{colosl-tr14}.
\end{proof}

The following examples illustrate the need for both local and global
action models in worlds. The first one showcases subtle reasoning
about subjective views enabled by our logic thanks to the global
action model.

\begin{example}
  The following entailment is valid, where $I \eqdef (\token{a}:
  \left\{\cell{x}{0} * \cell{y}{0} \swap \cell{x}{0} *
  \cell{y}{1}\right\})$:
  \[
  \shared{(x|->0 * y|-> 0) |/ x|->1}{I} * \shared{y|->0}{\emptyset}
  |-
  \shared{x|->1}{I} * \shared{y|->0}{\emptyset}
  \]
  Assertion $\shared{(x|->0 * y|-> 0) |/ x|->1}{I}$ is true of states
  $(\unitL,g,\lmod,\gmod)$ where parts of $g$ satisfies $x|->0 * y|-> 0$, in
  which case $\gmod$ has to include the action associated with $\token
  a$ in $I$ according to our closure condition, and of states
  $(\unitL,g',\lmod',\gmod')$ where parts of $g'$ satisfies $x|->1$, in which
  case $\gmod'$ need not contain the action. The second conjunct
  $\shared{y|->0}{\emptyset}$ does not allow the action to appear in
  the global action models, because it potentially affects $y|->0$ yet
  is not included in $\emptyset$. Thus, combining the two conjuncts
  allows us to deduce that $x|->0 * y|-> 0$ is not possible.
\end{example}

%% Extrapolating the effect of a local action $(p,q)$ over global shared
%% states cannot be realised by the naïve approach that simply ``frames
%% on'' arbitrary pieces of state $r$ to yield $(p\composeL r, q\composeL
%% r)$.

%% \begin{example}[]\label{ex:closure}
%%   Let $P \eqdef \shared{\cell{y}{2} \lor \cell{z}{3}}{I} *
%%   \shared{\cell{y}{2}}{\emptyset}$ denote the subjective view of the
%%   current thread of the shared state with $I \eqdef (\token{a}:
%%   \left\{\cell{x}{1} * \cell{y}{2} \swap \cell{x}{2}\right\})$. The
%%   action $\token{a}$ states that if the shared state contains the
%%   resource $\cell{x}{1} * \cell{y}{2}$, then a thread in possession of
%%   the $[\token{a}]$ capability can update the value of $x$ such that
%%   $\cell{x}{2}$ and claim the $\cell{y}{2}$ resource by moving it into
%%   its local state. On the other hand, $\cell{y}{2}$ is an immutable
%%   shared resource since its associated interference corresponds to
%%   $\emptyset$.

%%   In calculating the semantics of $P$, we need to find a global action
%%   model $\lmod$ that would encompass the behaviour of actions in $I
%%   \cup \emptyset$. A na\"\i ve attempt at calculating $\lmod$ would
%%   be to define it such that $\lmod(\{\token a\}) =
%%   \left\{(\cell{x}{1} \composeL \cell{y}{2} \composeL r, \cell{x}{2}
%%   \composeL r ) \mid r \in \LStates \right\}$. That is, given the
%%   localised behaviour of $\{\token a\}$ as prescribed in
%%   $\semI[\lenv]{I}$, we extend it with arbitrary logical states so
%%   that if the shared state contains the $\cell{x}{1} \composeL
%%   \cell{y}{2}$ resource, regardless of the rest of the shared state
%%   ($r$), the action can be carried out accordingly while leaving $r$
%%   untouched. Since the shared state currently contains the
%%   $\cell{y}{2}$ resource, with this interpretation when the shared
%%   state also contains $\cell{x}{1}$, it is possible to perform the
%%   action \token{a} and remove $\cell{y}{2}$ from the shared
%%   state. However, this does not agree with the current view, since as
%%   explained above $\cell{y}{2}$ is an immutable shared resource that
%%   cannot be removed from the shared state.
%% %in exchange for $\cell{w}{4}$. Since the current thread holds the $\cell{y}{2}$ resource locally, it extends the shared state with this resource such that its subjective view is captured by $P' \eqdef \shared{\cell{x}{1} * \cell{z}{3}}{I} * \shared{\cell{y}{2}}{\emptyset}$. With this simple extension, $\cell{y}{2}$ is now an immutable shared resource since its associated interference corresponds to $\emptyset$.
%% %Given a logical environment $\lenv$, let $\ca{} \in \semK[\lenv]{\token{X}}$. In calculating the semantics of $P$, we need to find a global action model $\lmod$ that would encompass the behaviour of actions in $I \cup \emptyset$. A na\"\i ve attempt at calculating $\lmod$ would be to define it such that $\lmod(\ca{}) = \left\{(\cell{x}{1} \composeL \cell{y}{2} \composeL r, \cell{x}{2} \composeL \cell{w}{4} \composeL r ) \mid r \in \LStates \right\}$. That is, given the localised behaviour of $\ca{}$ as prescribed in $\semI[\lenv]{I}$, we extend it with arbitrary logical states so that if the shared state contains the $\cell{x}{1} \composeL \cell{y}{2}$ resource, regardless of the rest of the shared state ($r$), the action can be carried out accordingly while leaving $r$ untouched. With this interpretation,  
%% \end{example}

%% The solution is to take the subjective view of the shared state into
%% account when extrapolating the global effects of local actions, as we
%% have done in our definition of action model closure.


%% \paragraph{On the need for local and global action models}

%% Recall that a shared state assertion $\shared{R}{I}$ subjectively
%% describes parts of the shared state that satisfy $R$. On the other
%% hand, an action of the form $P \swap Q$ specified in $I$ captures
%% the behaviour of the associated update in a \emph{localised}
%% manner. That is, if \emph{parts} of the shared state satisfy the
%% action precondition $P$, after the execution of the action, the
%% shared state is mutated such that the said parts satisfy the action
%% postcondition $Q$ and the rest of the shared state remains
%% unchanged. As such, we interpret interference assertions over the
%% entire shared state; however as we illustrate with the following
%% example, recording the global interpretation of actions alone in
%% the underlying model is not enough as the ways in which the shared
%% state can be updated may vary with its extension.

\begin{example}
  Let $P \eqdef \cell{x}{0} * \shared{\cell{y}{0}}{I}$, where $I$ is
  as in the previous example. Since $\cell{x}{0}$ is present in the
  local state, and the local and shared states are always disjoint,
  the action associated with $\token a$ cannot be carried out by
  either the current thread or the environment; thus the global action
  model $\gmod$ of worlds satisfying $P$ need not account for this
  action, \textit{e.g.}  $\gmod(\{\token a\}) = \emptyset$ is
  possible.

  Anticipating on the next section, consider now the situation where
  the current thread extends the shared state with the $\cell{x}{0}$
  resource previously held locally and specifies no actions for it,
  yielding the view such that $P' == \shared{\cell{x}{0}}{\emptyset} *
  \shared{\cell{y}{0}}{I}$. Since the new shared state contains both
  $x$ and $y$, it might be possible for a thread in possession of
  $\token{a}$ to perform the associated action. In particular,
  similarly to our previous example, the following entailment is
  valid:
  \[
  P' *  \shared{y|->0}{\emptyset} |- \m{false}
  \]
  If our worlds did not record local action models, from the global
  action models of the previous state alone we could not have made
  that deduction, as some of them ``forgot'' about the action which
  was impossible at the time.
\end{example}


%% On the other hand, it may seem sufficient for the model to track the
%% localised action behaviour $(\gmod)$ alone as the global behaviour can
%% be calculated from the localised one. However, as we illustrate in
%% \S\ref{subsec:amodClosure}, calculation of the global behaviour
%% depends on the subjective view of the shared state and the context
%% (parts of the shared state not visible) as well as the localised
%% action behaviour. As a result, given $\lenv \in \LEnv$, when defining
%% the semantics of a subjective shared assertion $\shared{P}{I}$, we
%% calculate the global behaviour based on the subjective view as
%% prescribed by $P$, the localised behaviour $\semI[\lenv]{I}$ and the
%% choice of compatible contexts. 

%%  More precisely, given $\shared{P}{I}$, the contents of the shared
%% state are of the form $l \composeL r$ where $l$ denotes a
%% subjective view as captured by $P$ while $r$ represents the context
%% or parts of the shared state not visible by the subjective view. In
%% order to remedy the problem illustrated in \ex~\ref{ex:closure},
%% when calculating the global behaviour of actions, rather than
%% extending their localised behaviour with arbitrary frames, we
%% relate them to the current subjective view $l$ and the context
%% $r$. This is captured by the \emph{closure} relation as defined
%% below.


%\begin{example}[]\label{ex:closure}
%Let $\tau_1$ and $\tau_2$ denote two distinct threads with $P \eqdef \token{X} * \cell{w}{4} * \shared{\cell{x}{1} * \cell{y}{2} \;\lor\; \cell{x}{1} * \cell{z}{3}}{I}$ and $Q \eqdef \cell{y}{2} * \shared{\cell{x}{1} * \cell{z}{3}}{I}$ as their subjective views of the shared state respectively with $I \eqdef (\token{X}: \left\{\cell{x}{1} * \cell{y}{2} \swap \cell{x}{1} * \cell{w}{4}\right\})$. The action associated with $\token{X}$ states that if the shared state contains the resource $\cell{x}{1} * \cell{y}{2}$, then an action in possession of the $\cell{w}{4}$ resource can claim the $\cell{y}{2}$ resource in exchange for $\cell{w}{4}$. Given a logical environment $\lenv$, let $\ca{} \in \semK[\lenv]{\token{X}}$ and $\lmod$ denote the global interpretation of $I$. A naive attempt at calculating $\lmod$ would be to define it such that $\lmod(\ca{}) = \left\{(\cell{x}{1} \composeL \cell{y}{2} \composeL r, \cell{x}{1} \composeL \cell{w}{4} \composeL r ) \mid r \in \LStates \right\}$
%Since $\tau_2$ holds the $\cell{y}{2}$ resource locally, Suppose that at this point the current thread extends the shared state with 
%\end{example}
