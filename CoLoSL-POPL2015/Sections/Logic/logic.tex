\section{The \colosl Logic}\label{sec:logic}
\todo introductory paragraph?\\
We present our assertion language and program logic. 

\subsection{\colosl Assertions}
Recall from \S\ref{TODO} that our proof judgements are of the form
$\hoare{P} C {Q}$ where $P$ and $Q$ are \emph{assertions} from the set
\Assertions. Assertions are constructed from standard connectives and
quantifiers of propositional and separation logic together with
\emph{heap assertions} in \HAssertions and capability assertions in
\KAssertions. Assertions describe both thread-local and \emph{shared}
resources; we write $\shared{P}{I}$ to denote a shared resource
characterised by $P$ where $I$ indicates an \emph{interference
  assertion} in \IAssertions. \colosl is parametric with respect to
the heap and capability assertions and can be instantiated with any
assertion language so long as they are interpreted over elements of a
separation algebra~\cite{asl}. Typically, we take
\HAssertions to be the standard heap assertions~\cite{rey02} and
use capability assertions in the style of tokens in~\cite{TODO}
such as \token{X}, \token{Y} and \token{Z} in \S\ref{sec:intuition}.
% 
\begin{definition}[\colosl Assertions]\label{def:assertions}
The assertions of \colosl are defined as follows where $\heapAss{} \in \HAssertions$, $\capAss{} \in \KAssertions$ and $x, y$ range over logical variables interpreted over a \emph{logical environment} $\lenv \in \LEnv$.
%
\[
\begin{array}{r@{}l}
	P, Q  ::=\null& \emp \mid \heapAss{} \mid \capAss{} \mid P * Q \mid P \ewand Q \:|\; P \sepish Q \mid \shared{P}{I} \mid \\
	& \exsts{x} P \mid \m{false} \mid P \Rightarrow Q \\
	I ::=\null& \interAss{\capAss{}}{\vec{y}}{P}{Q} \mid I_1,I_2
\end{array}
\]
%
In $\Assertions$, $P \ewand Q$ denotes the existential magic wand of separation logic~\cite{TODO} while $P \sepish Q$ denotes the overlapping conjunction or ``sepish''. The $\shared{P}{I}$ assertion is \emph{subjective} and describes only parts of the shared state that satisfy $P$. The overall (global) shared state can be expressed as $P * \m{true}$ where $\m{true}$ represents all states \emph{compatible} with $P$. For instance, the $S_X$ predicate in \fig\ref{fig:concurrentIncSubjectiveSpec} describes parts of the shared state that contain the $x$ and $z$ variables, while the $G$ predicate of \fig~\ref{fig:concurrentIncCoLoSLSpec} defines parts of the shared state that contain the $z$ variable as well as $x$ and $y$. Both assertions are subjective views of the same underlying shared state and are compatible with one another; the former however yields a weaker and more localised description.

In the interference assertion $\interAss{\capAss{}}{\vec{y}}{P}{Q}$, the capability assertion $\capAss{}$ is associated with \emph{action} $\exsts{\vec{y}} P \swap Q$, describing the transformation of the shared resource satisfying $P$ into one captured by $Q$. The existential quantification $\exists \vec{y}$ allows variables to be shared between $P$ and $Q$. For instance, the action associated with \token{X} in \fig\ref{fig:concurrentIncCoLoSLSpec} increments the value of $x$ from $v$ to $v+1$. The initial value ($v$) of $x$ does not matter ; however the resulting value must be the original value plus one ($v+1$). Interference assertions are not unique and capability assertions can be associated with multiple actions. This is particularly evident in the case of interference shifting of example in \S\ref{sec:intuition} where through action rewriting, \token{Z} is associated with multiple compatible actions. For brevity, we often parametrise capability assertions and interference environments. However, these parametrisations are merely a short-hand and can be substituted for their full definitions. 
\end{definition}
%
%
%
%
\subsection{Assertion Model}
Recall that \colosl assertions are parametric with respect to the heap and capability assertions so long as they are interpreted over elements of a separation algebra. 
%
\begin{parameter}[Heap Separation Algebra]\label{par:heapSA}
Let $(\Heaps, \composeH, \unitH)$ be any separation algebra with the cross-split property~\cite{dockins2009fresh}\footnote{%
$	
\begin{array}[t]{l}
	\for{h_1, h_2, h_3, h_4 \in \Heaps}
	h_1 \composeH h_2 = h_3 \composeH h_4 \implies \\
	\hspace*{0.3cm}	
	\begin{array}{l l}
		\exsts{h_{13}, h_{14}, h_{23}, h_{24}} &
		h_1 = h_{13} \composeH h_{14} \;\land\; h_2 = h_{23} \composeH h_{24} \;\land\\
		& h_3 = h_{13} \composeH h_{23} \;\land\; h_4 = h_{14} \composeH h_{24}
	\end{array}
\end{array}
$%
} representing heaps ranged over by $\h{}$. Assume a set of heap assertions $\HAssertions$ ranged over by $\heapAss{}$ and an associated semantics function:
%
\[
	\semH[\blankArg]{.} : \HAssertions \rightarrow \LEnv \rightarrow \pset{\Heaps}
\]
%
\end{parameter}
%
In the examples of this paper, we take $\Heaps$ to be the standard heap model: that is, the set of finite partial functions from addresses to values where $\composeH$ denotes union of functions with disjoint domains.
%
%
\begin{parameter}[Capability Separation Algebra]\label{par:capSA}
Let $\left(\Caps, \composeCap, \unitCap\right)$ be any separation algebra with the cross-split property, representing capability resources ranged over by $\ca{}$. Assume a set of capability assertions $\KAssertions$ ranged over by $\capAss{}$ and an associated semantics function:
%
\[
	\semK[\blankArg]{.} : \KAssertions \rightarrow \LEnv \rightarrow \pset{\Caps}
\]
%
\end{parameter}
%
%We often take $\Caps$ to be the set of functions from \emph{token names} to permission values from the range $[0,1]$; that is, $\Caps: \textsf{Token} \rightarrow [0,1]$ where $\composeCap$ is defined as addition with the proviso that the sum is within $[0,1]$ and is undefined otherwise. Intuitively, $1$ represents exclusive permission, $0$ represents no permission and any other value denotes a non-exclusive permission.
%We often take $\Caps$ to be the set of functions from \emph{token names} to permission values from $\{0,1\}$; that is, $\Caps: \textsf{Token} \rightarrow \{0,1\}$ where $\composeCap$ is defined as addition. Intuitively, $1$ represents exclusive capability while $0$ represents no capability.
%
We often take $\Caps$ to be the set of \emph{token names}; that is, $\Caps: \textsf{Token}$ where $\composeCap$ is defined for any two distinct elements. 
%For instance, in the example of \S\ref{sec:intuition}, we had $\token{X}, \token{Y}, \token{Z} \in \textsf{Token}$.

For simplicity, in this paper we assume that the separation algebras of both heaps and capability resources (\param s~\ref{par:heapSA} and~\ref{par:capSA}) have the disjointness property\footnote{$\for{h, h' \in \Heaps} h \composeH h = h' \implies h = h' = \unitH$}. However, the theory of \colosl as presented in~\cite{techreport} does not require the disjointness property and can be instantiated with any separation algebra.

Our assertions distinguish between thread-local and shared resources, where the former is private to the thread and the latter is shared amongst all threads and is thus subject to interference. Both thread-local and shared resources are described through heap and capability assertions. Heap assertions are used to represent the current state of the memory while capability assertions are used to specify how shared resources can be manipulated. We thus define \emph{logical states} which pair heaps with capabilities to represent both thread-local and shared states.
%
\begin{definition}[Logical States]A \emph{logical state} is a pair consisting of a heap and a capability:
%
\[
	l, s \in \LStates \eqdef \Heaps \times \Caps
\]
%
We write $\unitL$ for the logical state $(\unitH, \unitCap)$. Given a logical state $l$, we write $\heapPart{l}$ and $\capPart{l}$ for the first and second projections, respectively. The \emph{composition of logical states}
%
$
	\composeL : \LStates \times \LStates \rightharpoonup \LStates
$
%
is defined as:
%
\[
	l \composeL l' \eqdef (\heapPart{l} \composeH \heapPart{l'},\; \capPart{l} \composeCap \capPart{l'})
\]
%
$(\LStates, \composeL, \unitL)$ form the \emph{separation algebra of logical states}.
\end{definition}
%
%
Recall that interference assertions specify how shared resources can be manipulate by various threads through capability assertions. That is, the thread in possession of the necessary capabilities in its local (private) state, can carry out the associated action and update the shared state accordingly. We interpret interference assertions as \emph{action models} that associate capabilities with actions. 
%
\begin{definition}[Action Models]
An \emph{action model} is a partial function associating capabilities with actions. An action is expressed as a relation between the original logical state and the new logical state reflecting the effects of the action:
%
\[
	\amod{} \in \AMods \eqdef \Caps \rightharpoonup \pset{\LStates \times \LStates}
\]
%
We write $\unitAM$ for an action model with empty domain.
%
\end{definition}
%
%
\begin{definition}[Worlds]
A \emph{world} is a 4-tuple of a logical state representing the thread-local state, a second logical state representing the shared state, a global action model and a local action model, subject to a well-formedness condition.
%
\[
	w \in \Worlds \eqdef 
	\left\{
	\hspace*{-0.15cm}
	\begin{array}{l | l}
		(l, s, \amod{}, \amod{L}) & 
 		\begin{array}{l}
	 		l, s \in \LStates \;\land\; \amod{}, \amod{L} \in \AMods\\
	 		\land\; \wf{l, s, \amod{}, \amod{L}} 
	 	\end{array}
	 	\hspace*{-0.1cm}
	\end{array} 
	\hspace*{-0.15cm}
	\right\}
\]
% 
Given a world $w$, we write $\localPart{w}$, $\sharedPart{w}$, $\amodelPart{w}$ and $\amodelPartL{w}$ for the first, second, third and fourth projections, respectively.
The \emph{composition of worlds} is defined as
%
\[
	w \composeW w' \eqdef (\localPart{w} \composeL \localPart{w'},\; \sharedPart{w} \composeS \sharedPart{w'},\; \amodelPart{w} \composeAM \amodelPart{w'},\; \amodelPartL{w} \composeAM \amodelPartL{w'})
\]
%
We defer the formal definition of \wf{.} to the proceeding section. Informally, well-formedness ensures that the local and shared state are disjointed from one another and that the updates permitted by the action model are well-behaved with respect to future extensions of the shared state. Later in \S\ref{TODO}, we justify the need for both local and global action models.

Finally, given a world $w \in \Worlds$, its \emph{collapse into a logical state},
%
$
	\collapseW{.}: \Worlds \rightarrow \LStates
$
%
, is defined as:
%
\[
	\collapseW{l, s, \amod{}, \amod{L}} \eqdef l \composeL s
\]
%
\end{definition}
%
%
%
%
\subsection{Assertion Semantics}\label{subsec:semantics}
We proceed with the semantics of \colosl assertions as defined in \defin\ref{def:assertions}. We first define an intermediate semantics $\intermediateSem[(.)]{.}$ that does not enforce well-formedness and then describe the semantics $\sem[(.)]{.}$ by restricting them to the set of well-formed worlds. We then define the semantics of interference assertions $\semI[(.)]{.}$. 
%
%
\begin{definition}[Assertion Semantics]
Given a logical environment $\lenv \in \LEnv$, the \emph{intermediate semantics} of \colosl assertions $\intermediateSem[(.)]{.}: \Assertions \times \LEnv \rightarrow \pset{\LStates \times \LStates \times \AMods \times \AMods}$ are given as follows where $\semI[(.)]{.}$ denotes the semantics of interference assertions described in \defin ~\ref{def:interferenceSemantics}.
%
\[
\hspace*{-0.2cm}
\begin{array}{r c l}
	\intermediateSem{\heapAss{}} 
	& \hspace*{-0.3cm} \eqdef \hspace*{-0.3cm} & 
	\left\{
	\hspace*{-0.2cm}
	\begin{array}{l | l}
	 	\left(\left(h, \unitCap \right), s, \amod{}, \amod{L} \right) &
	 	\hspace*{-0.15cm}
		\begin{array}{l}
		 	h \in \semH{\heapAss{}} \land s \in \LStates \\
		  \land\; \amod{}, \amod{L} \in \AMods
		\end{array}
		\hspace*{-0.1cm}
	\end{array}
	\hspace*{-0.2cm}
	\right\}\\
	
	
	\intermediateSem{\capAss{}} 
	& \hspace*{-0.3cm} \eqdef \hspace*{-0.3cm} & 
	\left\{
	\hspace*{-0.2cm}
	\begin{array}{l | l}
	 	\left(\left(\unitH, \ca{} \right), s, \amod{}, \amod{L} \right) &
	 	\hspace*{-0.15cm}
		\begin{array}{l}
		 	\ca{} \in \semK{\capAss{}} \land s \in \LStates \\
		  \land\; \amod{}, \amod{L} \in \AMods
		\end{array}
		\hspace*{-0.1cm}
	\end{array}
	\hspace*{-0.2cm}
	\right\}\\
	
	
	
	\intermediateSem{\emp} 
	& \hspace*{-0.3cm} \eqdef \hspace*{-0.3cm} & 
	\left\{
	\hspace*{-0.2cm}
	\begin{array}{l | l}
	 	\left(\unitL, s, \amod{}, \amod{L} \right) &
	 	\hspace*{-0.15cm}
		\begin{array}{l}
		 	s \in \LStates 
		  \;\land\; \amod{}, \amod{L} \in \AMods
		\end{array}
		\hspace*{-0.1cm}
	\end{array}
	\hspace*{-0.2cm}
	\right\}\\



	\intermediateSem{P * Q} 
	& \hspace*{-0.3cm} \eqdef \hspace*{-0.3cm} &
	\left\{ 
	\hspace*{-0.2cm}
	\begin{array}{l | l}
		w_1 \composeW w_2& 
		\hspace*{-0.15cm}
		\begin{array}{l}
				 w_1 \in \intermediateSem{P} \;\land\;
				 w_2 \in \intermediateSem{Q} 
		\end{array}
		\hspace*{-0.1cm}
	\end{array} 
	\hspace*{-0.2cm}
	\right\}\\
	 
%	\left\{ 
%	\hspace*{-0.2cm}
%	\begin{array}{l | l}
%		(p \composeL q, s, \amod{}, \amod{L})& 
%		\hspace*{-0.15cm}
%		\begin{array}{l}
%				 (p, s, \amod{}, \amod{L}) \in \intermediateSem{P} \;\land\\
%				 (q, s, \amod{}, \amod{L}) \in \intermediateSem{Q} 
%		\end{array}
%		\hspace*{-0.1cm}
%	\end{array} 
%	\hspace*{-0.2cm}
%	\right\}\\
	
	
	
	
	\intermediateSem{P \ewand Q} 
	& \hspace*{-0.3cm} \eqdef \hspace*{-0.3cm} & 
	\left\{ 
	\hspace*{-0.2cm}
	\begin{array}{l | l}
		w & 
		\hspace*{-0.15cm}
		\begin{array}{l l}
			\exsts{w_1, w_2} 
			& \hspace*{-0.3cm} 
				w_2 = w \composeW w_1 \;\land\\
				
			& \hspace*{-0.3cm} 
				w_1 \in \intermediateSem[\lenv]{P} \;\land\;
				w_2 \in \intermediateSem[\lenv]{Q}
				
		\end{array}
		\hspace*{-0.1cm}
	\end{array} 
	\hspace*{-0.2cm}
	\right\}\\
	
	\intermediateSem{P \!\sepish\! Q} 
	& \hspace*{-0.3cm} \eqdef \hspace*{-0.3cm} & 
	\left\{ 
	\hspace*{-0.2cm}
	\begin{array}{l | l}
		(p \composeL c \composeL q, s, \amod{}, \amod{L}) & 
		\hspace*{-0.15cm}
		\begin{array}{l}
				 (p \composeL c, s, \amod{}, \amod{L}) \in \intermediateSem{P}\land \\
				 (c \composeL q , s, \amod{}, \amod{L}) \in \intermediateSem{Q} 
		\end{array}
		\hspace*{-0.1cm}
	\end{array} 
	\hspace*{-0.2cm}
	\right\}\\
	
	
	
	
	
	\intermediateSem{\shared{P}{I}} 
	& \hspace*{-0.3cm} \eqdef \hspace*{-0.3cm} & 
	\left\{ 
	\hspace*{-0.2cm}
	\begin{array}{l | l}
		(\unitL, l \composeL r, \amod{}, \amod{L}) & 
		\hspace*{-0.15cm}
		\begin{array}{l}
			(l, l \composeL r, \amod{}, \amod{L}) \in \intermediateSem{P}\\
			\land\; r \in \LStates \\
			\land\; \extendsAM{\amod{}, \amod{L}}{l}{r}{\semI{I} }
		\end{array}
		\hspace*{-0.1cm}
	\end{array} 
	\hspace*{-0.2cm}
	\right\}\\
	
	

\end{array}
\]
%
The semantics of propositional formulae are standard and are omitted here. The semantics of heap and capability assertions are given in terms of their respective semantics functions (\param s ~\ref{par:heapSA} and ~\ref{par:capSA}). Note that the semantics of $\heapAss{}$, $\capAss{}$ and \emp use arbitrary shared states $s$ and action models $\amod{}, \amod{L}$. The semantics of separation logic connectives $*, \ewand$ and $\sepish$ are standard. 

$\shared{P}{I}$ states that $P$ holds for \emph{parts} of the shared state and thus the overall shared state is given by extension of the subjective view $l$ with an arbitrary logical state $r$ representing the \emph{context} not visible by the current subjective view. The interference associated with this part of the shared state is given by interference assertion $\semI[\lenv]{I}$ such that the global and local action models $\amod{}, \amod{L}$ are \emph{closed} under $\semI[\lenv]{I}$ with respect to the subjective view $l$ and context $r$. In \S\ref{subsec:localGlobalActionModels} we justify the need for global and local action models and formalise action model closuer $\extendsAM{\amod{}, \amod{L}}{l}{r}{\amod{}'}$. 
%Since shared state assertions are partial subjective description of the shared state, separating conjunction between them behaves as overlapping conjunction 

We permit nesting of shared state assertions since it is particularly useful in describing inductively defined predicates, as we illustrate in \S\ref{sec:examples}. However our semantics of shared states does not include nesting; in our semantics, nested assertions are equivalent to flattened assertion with all nested assertions moved to the top level. That is, 
%
\[
	\shared{P * \shared{Q}{I'}}{I} \iff \shared{P}{I} * \shared{Q}{I'}
\]
%


The \emph{semantics} of \colosl assertions $\sem[(.)]{.}: \Assertions \times \LEnv \rightarrow \pset{\Worlds}$ is defined as:
%
\[
	\sem[\lenv]{P} \eqdef 
	\left\{ 
		(l, s, \amod{}, \amod{L}) 
		\mid
		(l, s, \amod{}, \amod{L}) \in \intermediateSem[\lenv]{P} \cap \Worlds	
	\right\}
\]
%
\end{definition}
%
%
\begin{definition}[Localised Interference Semantics]\label{def:interferenceSemantics}
The \emph{semantic of interference assertions}
%
$
	\semI[(.)]{.} : \IAssertions \times \LEnv \rightharpoonup (\Caps \rightharpoonup  \LStates \times \LStates )
$
%
is defined as:
%
\[
\begin{array}{r l}
	\for{\kappa} \semI[\lenv]{I}(\ca{}) =
	 	&
%	 	\hspace*{0.2cm}
	 	\left\{
		\begin{array}{l | r}
			
			\hspace*{-0.2cm}(p, q)\hspace*{-0.1cm} & 
			\hspace*{-0.2cm}
			\begin{array}{l}
				\exsts{\interAss{\capAss{}}{\vec{y}}{P}{Q} \in I} \exsts{\vec{v}}\\
	
				\hspace*{0.2cm}
				\ca{} \in \semK[\lenv]{\capAss{}} \;\land\\
				
				\hspace*{0.2cm}
				(p, -, -, -) \in \intermediateSem[\lenv\text{[} \vec{y} \mapsto \vec{v} \text{]}]{P} \;\land\\
				

				
				\hspace*{0.2cm}
				(q, -, -, -) \in \intermediateSem[\lenv\text{[} \vec{y} \mapsto \vec{v} \text{]}]{Q} 
				
			\end{array}
		\end{array}
	\hspace*{-0.2cm}
	\right\}

\end{array}
\]
%
\end{definition}
%
%
%
%
\subsection{Shared State Extension}\label{subsec:extension}
Recall from \S\ref{subsec:extend} that a thread can always \emph{extend} the shared state by handing over some of its private resources. In doing so, it can also extend the interference relation by actions that specify how the newly shared resources can be manipulated. However, the newly added behaviour must \emph{agree} with the existing actions and not invalidate other threads' view of how the existing resources can be mutated. We impose a locality condition on the newly added behaviour to ensure sound extension of the shared state. This is informally illustrated in the following example. 
%
\begin{example}\label{ex:badExtension}
Let $P \eqdef \cell{x}{1} * \shared{\cell{y}{1} \lor \cell{y}{2}}{I}$ denote the view of the current thread with $I \eqdef \left(\token{Y}: \left\{\cell{y}{1} \swap \cell{y}{2}\right\} \right)$. Since the current thread owns the location addressed by $x$, it can extend the shared state as $Q \eqdef \token{X} * \shared{\left(\cell{y}{1} \!\lor\! \cell{y}{2} \right) * \cell{x}{1}}{I \cup I'}$ where 
$
	I' \eqdef 
		\left(
			\token{X}: 
			\left\{
			\hspace*{-0.15cm}
			\begin{array}{l} 
				\cell{x}{1} \swap \cell{x}{2}\\
				\cell{y}{1} \swap \cell{y}{3}
			\end{array}
			\hspace*{-0.15cm}
			\right\}
		 \right)
$.
In extending the shared state, the current thread also extended the interference allowed on the shared state by adding two new actions associated with the newly generated capability resource $\token{X}$ as given in $I'$. The first action specifies how the value of location $x$ can be updated. Since location $x$ was previously owned privately by the current thread and was hence not visible to other threads, this new action will not invalidate their view of the shared state. On the other hand, the second action introduces a new way in which the value of location $y$ can be mutated. To other threads the only updates allowed on location $y$ are done through the $\token{Y}$ capability as specified in $I$ and thus this new behaviour is unbeknownst to them. As such, this action violates the view of other threads and does not agree with the existing interference.
\end{example}
%
%
In order to ensure sound extension of the shared state, we require that the \emph{update footprint} of extended actions are confined to the locally owned resources. That is, the resources mutated by the action do not \emph{overlap} with those already in the shared state.  
%We proceed with the definitions necessary for formalising what it means for actions to have a local update footprint.
%
\begin{definition}[Overlap]
The \emph{overlap} function over logical states 
%
$
\meetL : \left(\LStates \times \LStates \right) \rightarrow \pset{\LStates}
$
%
is defined as:
%
\[
	l_1 \meetL l_2 \eqdef 
	\left\{ 
	\hspace*{-0.15cm}
	\begin{array}{l | l}
	 	l 
	 	\!\!&\!
	  \exsts{l', l'', l_3}\! l_1 = l \composeL l' \land l_2 = l \composeL l'' \land l \composeL l' \composeL l'' = l_3 
	\end{array}
	\hspace*{-0.15cm}
	\right\}
\]
%
The \emph{maximal overlap} is defined as:
%
\[
	l_1 \maxMeetL l_2 = l \iffdef l \in (l_1 \meetL l_2) \land \for{l' \in (l_1 \meetL l_2)} l' \leq l
\]
%
The $\leq$ relation denotes the standard ordering by extension where 
%
\[
	l_1 \leq l \iffdef \exsts{l_2} l_1 \composeL l_2 = l
\]
%
\end{definition}
%
%
\begin{definition}[Update footprint]
Given a pair of logical states $(p, q)$, its \emph{update footprint} is defined as:
%
\[
\begin{array}{l l l}
	\updateFP{p,q} = (l_1, l_2) \iffdef  
	&\hspace*{-0.2cm}	\exsts{f}\hspace*{-0.2cm}
	&\hspace*{-0.2cm} p = l_1 \composeL f \land q = l_2 \composeL f \land\\
	\hspace*{-0.2cm}
	&\hspace*{-0.2cm}&
	\hspace*{-0.2cm}\for{l'} l' \leq l_1 \land l' \leq l_2 \implies l' = \unitL
\end{array}
\]
%
\end{definition}
\todo\ Need a better segue here.\\
%

In order to ensure that the new actions agree with the existing actions, we need to guarantee that the update footprint of the new actions is local with respect to the \emph{current} state and \emph{all states reachable} from the current state. Inspired by the LRG logic~\cite{lrg} we introduce the concept of invariant-fenced action models to capture all possible states reachable from the current state.
%
\begin{definition}[Invariant Fenced Action Model]
An action model $\amod{} \in \AMods$ is \emph{fenced by an invariant} $\fence{} \in \pset{\LStates}$ iff
%
\[
\hspace*{-0.15cm}
\begin{array}{l l}
	\fence{} \fences \amod{} \iffdef \hspace*{-0.2cm}
	 &\hspace*{-0.2cm} \for{\ca{}, p, q, l, l_1, l_2, l_3, c}\\
	 
	 & 
	 	(p, q) \in \amod{}(\ca{}) 
	 	\land \updateFP{p, q} = (l_1, l_2) \land
	 	l \in \fence{} \land
	 	p \meetL l \not= \emptyset \\
	 	
	 & 
	 	\land\; l_3 > \unitL \land
	 	 l_1 \maxMeetL l = l_3 \land
	 	 l = l_3 \composeL c
	 	 \implies\\
	 	 
	 & \hspace*{1cm} l_2 \composeL c \in \fence{}	 
\end{array}
\]
%
That is, given any capability $\ca{}$, and an action associated with it $(p, q) \in \amod{}(\ca{})$, where its update footprint is given by $(l_1, l_2)$; and given any state $l \in \fence{}$ that is compatible with the action precondition ($l \meetL p \not= \emptyset$), if there is an overlap between $l$ and the update precondition ($l_1 \maxMeetL l = l_3 > \unitL$), then the state resulting from the update ($l_2 \composeL c$) is also in $\fence{}$.
For instance in example of \S\ref{sec:intuition}, $I_X$ as specified in \fig\ref{fig:concurrentIncSubjectiveSpec} is fenced by $\fence{X} \eqdef \left\{\cell{x}{v} * \cell{z}{v} \mid v \in \{0, \cdots, 10\} \right\} \cup \left\{\cell{x}{v+1} * \cell{z}{v} \mid v \in \{0 ,\cdots, 9 \}\right\}$.
% 
\end{definition}
%
%
We now formalise what it means for the update footprint of actions to be local with respect to an invariant.
%
\begin{definition}[Locally-fenced Action Model]
An action model $\amod{} \in \AMods$ is \emph{locally-fenced} by $\fence{} \in \pset{\LStates}$ iff
%
\[
\hspace*{-0.2cm}
\begin{array}{l l}
	\fence{} \strictfences \amod{} \iffdef 
	&
		\hspace*{-0.3cm}
		\fence{} \fences \amod{} \land
		\for{\ca{}, p, q, l_1, l_2, l}\\
	&
		\hspace*{-0.15cm}
		l \in \fence{} \land 
		(p, q) \in \amod{}(\ca{}) \land 
		\updateFP{p, q} = (l_1, l_2) \land 
		p \meetL l \not= \emptyset\\
	&
		\hspace*{0.5cm}
		\implies
		l_1 \leq l
\end{array}
\]
%
The above states that $\amod{}$ is fenced by invariant $\fence{}$ and given any capability $\ca{}$ and an action associated with it $(p, q) \in \amod{}(\ca{})$ where its update footprint is given by $(l_1, l_2)$; and given any state $l \in \fence{}$ that is compatible with the action precondition $p \meetL l \not= \emptyset$, the update precondition $l_1$ is \emph{contained} within $l$: $l_1 \leq l$. 
\end{definition}
%
%
We're now in a position to formalise well-formedness.
%
\begin{definition}[Well-formedness]
A world $(l, s, \amod{}, \amod{L}) \in \Worlds$ is \emph{well-formed} iff
%
\[
\begin{array}{l l l}
	\wf{l, s, \amod{}, \amod{L}} \iffdef 
	\hspace*{-0.2cm} & \hspace*{-0.2cm}
	\exsts{r, \fence{}} 
	\hspace*{-0.1cm} & \hspace*{-0.1cm} 
		s \in \fence{} \;\land\; 
		\fence{} \strictfences \amod{L} \;\land\;
		\collapseW{l, s, \amod{}, \amod{L}} = r\\
	
	\hspace*{-0.2cm} & \hspace*{-0.2cm} 
	& \hspace*{-0.1cm} 	
	\land\; \contains{\dom{\amod{L}}}{\capPart{r}}
\end{array}
\]
%
That is, there exists an invariant $\fence{}$ that contains the current shared state $s$ and the update footprint of the local action model $\amod{L}$ is local with respect to $\fence{}$. The local and shared states are disjoint from one an other and finally that the capability resources of both local and shared states are contained within the domain of $\amod{L}$ where
%
\[
\hspace*{-0.2cm}
\begin{array}{l l}
	\contains{S}{\ca{}}\!\! \iffdef & \hspace*{-0.6cm}
	\begin{array}{l}
		\exsts{K}\! \ca{} = \prod\limits_{\ca{i} \in K}^{\composeCap} \ca{i} \land
		\for{\ca{i} \!\in K}\!\! \exsts{\ca{}' \!\in S}\! \ca{i} \leq \ca{}'
	\end{array}
\end{array}
\]
%
\end{definition}
%
%
%
%
\subsection{Local and Global Action Models}\label{subsec:localGlobalActionModels}
Recall that a shared state assertion $\shared{R}{I}$ subjectively describes parts of the shared state that satisfy $R$. On the other hand, an action of the form $P \swap Q$ specified in $I$ captures the behaviour of the associated update in a \emph{localised} manner. That is, if \emph{parts} of the shared state satisfy the action precondition $P$, after the execution of the action, the shared state is mutated such that the said parts satisfy the action postcondition $Q$ and the rest of the shared state remains unchanged. As such, we interpret interference assertions over the entire shared state; however as we illustrate with the following example, recording the global interpretation of actions alone in the underlying model is not enough as the ways in which the shared state can be updated may vary with its extension. 
%
\begin{example}[]Let $P \eqdef \cell{w}{1} * \shared{\cell{x}{2}}{I}$ denote the view of the current thread with $I \eqdef \left(\token{X}: \left\{\cell{x}{2} * \cell{w}{1} \swap \cell{x}{3} * \cell{w}{1}\right\}\right)$. The action associated with \token{X} states that if the shared state contains the resource $\cell{x}{2} * \cell{w}{1}$, then the value of $x$ may be mutated such that $\cell{x}{3}$ while $w$ as well as other parts of the shared state remain unchanged. Given a logical environment $\lenv$, let $\amod{0} = \semI[\lenv]{I}$ denote the localised interpretation of $I$ (\defin\ref{def:interferenceSemantics}); and $\ca{\textsf{X}} \in  \semK[\lenv]{\token{X}}$ represent a capability captured by the assertion \token{X} (\param~\ref{par:capSA}). In calculating the semantics of $P$, we need to interpret the behaviour of actions specified in $I$ over the entire shared state.

Since the current thread locally holds the $\cell{w}{1}$ resource and the contents of thread-local and shared states are always disjoint, we can safely deduce that $\cell{w}{2}$ is not a part of the shared state. Consequently, the action associated with $\ca{\textsf{X}}$ cannot be carried out by either the current thread or the environment; thus any $\amod{}'$ where $\amod{}'(\ca{\textsf{X}}) = \emptyset$, yields a valid extension of $\amod{0}$ to the entire shared state.

Suppose at this point the current thread extends the shared state with the $\cell{w}{1}$ resource such that $\shared{\cell{x}{2}}{I} * \shared{\cell{w}{1}}{\emptyset}$ denotes its current view. Since the shared state now contains both $\cell{x}{2}$ and $\cell{w}{1}$ resources, it should be possible for a thread in possession of \token{X} to perform the associated action. However, this is not reflected in the global interpretation $\amod{}'$ and hence $\amod{}'$ needs to be recalculated upon extension of the shared state. On the other hand, from $\amod{}'$ alone and without keeping track of the localised interpretation $\amod{0}$, we cannot recalculate the global behaviour since $\amod{}'$ is lossy and does not reflect the original behaviour prescribed to $\ca{\textsf{X}}$.
\end{example}
%
%
As demonstrated by the above example, it is necessary to record the localised interpretation of interference assertions at the model level. As a result, given a world $(l, s, \amod{}, \amod{L})$, the third component $(\amod{})$ denotes the global behaviour of actions while the fourth component $(\amod{L})$ represents the set of all localised action behaviour.
%
%
\subsection{Action Model Closure}
Recall that interference assertions are interpreted into localised action models from which global action models are then calculated over the entire shared state. However, as we demonstrate through \ex~\ref{ex:closure}, in calculating the global action model we cannot simply extend the action pre- and post-conditions with arbitrary frames. 
%
\begin{example}[]\label{ex:closure}
Let $P \eqdef \shared{\cell{y}{2} \lor \cell{z}{3}}{I} * \shared{\cell{y}{2}}{\emptyset}$ denote the subjective view of the current thread of the shared state with $I \eqdef (\token{X}: \left\{\cell{x}{1} * \cell{y}{2} \swap \cell{x}{2}\right\})$. The action associated with $\token{X}$ states that if the shared state contains the resource $\cell{x}{1} * \cell{y}{2}$, then a thread in possession of the $\token{X}$ resource can update the value of $x$ such that $\cell{x}{2}$ and claim the $\cell{y}{2}$ resource by moving it into its local state. On the other hand, $\cell{y}{2}$ is an immutable shared resource since its associated interference corresponds to $\emptyset$.

Given a logical environment $\lenv$, let $\ca{} \in \semK[\lenv]{\token{X}}$. In calculating the semantics of $P$, we need to find a global action model $\amod{}$ that would encompass the behaviour of actions in $I \cup \emptyset$. A na\"\i ve attempt at calculating $\amod{}$ would be to define it such that $\amod{}(\ca{}) = \left\{(\cell{x}{1} \composeL \cell{y}{2} \composeL r, \cell{x}{2} \composeL r ) \mid r \in \LStates \right\}$. That is, given the localised behaviour of $\ca{}$ as prescribed in $\semI[\lenv]{I}$, we extend it with arbitrary logical states so that if the shared state contains the $\cell{x}{1} \composeL \cell{y}{2}$ resource, regardless of the rest of the shared state ($r$), the action can be carried out accordingly while leaving $r$ untouched. Since the shared state currently contains the $\cell{y}{2}$ resource, with this interpretation when the shared state also contains $\cell{x}{1}$, it is possible to perform the action of \token{X} and remove $\cell{y}{2}$ from the shared state. However, this does not agree with the current view, since as explained above $\cell{y}{2}$ is an immutable shared resource that cannot be removed from the shared state. 
%in exchange for $\cell{w}{4}$. Since the current thread holds the $\cell{y}{2}$ resource locally, it extends the shared state with this resource such that its subjective view is captured by $P' \eqdef \shared{\cell{x}{1} * \cell{z}{3}}{I} * \shared{\cell{y}{2}}{\emptyset}$. With this simple extension, $\cell{y}{2}$ is now an immutable shared resource since its associated interference corresponds to $\emptyset$.
%
%Given a logical environment $\lenv$, let $\ca{} \in \semK[\lenv]{\token{X}}$. In calculating the semantics of $P$, we need to find a global action model $\amod{}$ that would encompass the behaviour of actions in $I \cup \emptyset$. A na\"\i ve attempt at calculating $\amod{}$ would be to define it such that $\amod{}(\ca{}) = \left\{(\cell{x}{1} \composeL \cell{y}{2} \composeL r, \cell{x}{2} \composeL \cell{w}{4} \composeL r ) \mid r \in \LStates \right\}$. That is, given the localised behaviour of $\ca{}$ as prescribed in $\semI[\lenv]{I}$, we extend it with arbitrary logical states so that if the shared state contains the $\cell{x}{1} \composeL \cell{y}{2}$ resource, regardless of the rest of the shared state ($r$), the action can be carried out accordingly while leaving $r$ untouched. With this interpretation,  
\end{example}
%
%
Recall from \S\ref{subsec:semantics} that when defining the semantics of $\shared{P}{I}$, the contents of the shared state are of the form $l \composeL r$ where $l$ denotes a subjective view as captured by $P$ while $r$ represents the context or parts of the shared state not visible by the subjective view. In order to remedy the problem illustrated in \ex~\ref{ex:closure}, when calculating the global behaviour of actions, rather than extending their localised behaviour with arbitrary frames, we relate them to the current subjective view $l$ and the context $r$. This is captured by the \emph{closure} relation as defined below.
%
%
%\begin{example}[]\label{ex:closure}
%Let $\tau_1$ and $\tau_2$ denote two distinct threads with $P \eqdef \token{X} * \cell{w}{4} * \shared{\cell{x}{1} * \cell{y}{2} \;\lor\; \cell{x}{1} * \cell{z}{3}}{I}$ and $Q \eqdef \cell{y}{2} * \shared{\cell{x}{1} * \cell{z}{3}}{I}$ as their subjective views of the shared state respectively with $I \eqdef (\token{X}: \left\{\cell{x}{1} * \cell{y}{2} \swap \cell{x}{1} * \cell{w}{4}\right\})$. The action associated with $\token{X}$ states that if the shared state contains the resource $\cell{x}{1} * \cell{y}{2}$, then an action in possession of the $\cell{w}{4}$ resource can claim the $\cell{y}{2}$ resource in exchange for $\cell{w}{4}$. Given a logical environment $\lenv$, let $\ca{} \in \semK[\lenv]{\token{X}}$ and $\amod{}$ denote the global interpretation of $I$. A naive attempt at calculating $\amod{}$ would be to define it such that $\amod{}(\ca{}) = \left\{(\cell{x}{1} \composeL \cell{y}{2} \composeL r, \cell{x}{1} \composeL \cell{w}{4} \composeL r ) \mid r \in \LStates \right\}$
%
%Since $\tau_2$ holds the $\cell{y}{2}$ resource locally, Suppose that at this point the current thread extends the shared state with 
%\end{example}
\begin{definition}[Action Model Closure]
%
\[
\hspace*{-0.2cm}
\begin{array}{l}
	\extendsAM{\amod{}, \amod{L}}{l}{r}{\amod{}'} \iffdef \for{n \in \Nats} \extendsAMUpto{\amod{}, \amod{L}}{n}{l}{r}{\amod{}'}  \;\land\; \amod{}' \subseteq \amod{L}\\
	
	
	\extendsAMUpto{\amod{}, \amod{L}}{0}{l}{r}{\amod{}'} \iffdef true \\ 



	\extendsAMUpto{\amod{}, \amod{L}}{n}{l}{r}{\amod{}'} \iffdef \\ 

	\hspace*{0.1cm}\for{\ca{}, c, d, t, p, q, l_1, l_2, l_3, l_4}\\
	
	\hspace*{0.2cm}(p, q) \in \amod{}'(\ca{}) 
	\;\land\; \updateFP{p, q} = (l_1, l_2)
	\;\land\; p \leq  l \composeL r \composeL t \;\land\\
	
	\hspace*{0.2cm}l_1 = l_3 \composeL l_4 
	\;\land\; l_1 \maxMeetL l = l_3 
	\;\land\; l = l_3 \composeL c 
	\;\land\; r = l_4 \composeL d \implies\\
	
	\hspace*{0.4cm} \extendsAMUpto{\amod{}, \amod{L}}{(n-1)}{l_2 \composeL c}{d}{\amod{}'} \;\land\\
	\hspace*{0.4cm}t = \unitL \implies 
	\left(
	\begin{array}{l}
		(l_1 \composeL c \composeL d, l_2 \composeL c \composeL d) \in \amod{}(\ca{})\\
		\lor\; l_2 \composeL c \composeL d \text{ is undefined} 
	\end{array}
	\right)\\
	
		\hspace*{2cm} \land\\
	

  \hspace*{0.1cm}\for{\ca{}, p, q, l_1, l_2, c, d} \\
  
  \hspace*{0.2cm} (p, q) \in \amod{L}(\ca{})
  \land\; \updateFP{p, q} = (l_1, l_2)
  \land\; p \composeL c = l \composeL r \composeL d \implies\\
  
  \hspace*{0.4cm}\exsts{f, c'} (l_1 \composeL f, l_2 \composeL f) \in \amod{}'(\ca{}) \land\; l_1 \composeL f \composeL c' =  l \composeL r \composeL d \;\;\lor\\
	

		\hspace*{0.4cm}q \composeL c \text{ is undefined} \;\;\lor\\
		\hspace*{0.4cm}l_1 \maxMeetL l = \unitL \land \exsts{r'}\! q \composeL c = l \composeL r' \composeL d \land \extendsAMUpto{\amod{}, \amod{L}}{(n-1)}{l}{r'}{\amod{}'}

\end{array}
\]
%
\todo Explain the formalisation.
\end{definition}
%
%
%
%
\subsection{Action Shifting}
\todo Intuitive explanation
\begin{definition}[Action Shifting]
%
\[
\begin{array}{l}
	\amod{} \weakenI{F}  \amod{1} \iffdef \\
	\hspace*{0.2cm}\for{\ca{}, l, p, q, l_1, l_2}\;\; l \in F \;\land\; \updateFP{p, q} = (l_1, l_2) \implies\\
	\hspace*{0.2cm}
	\begin{array}{l}
		(p, q) \in \amod{1}(\ca{}) 
		\;\land\; p \leq l \composeL r \implies \\
		\hspace*{0.4cm}\exsts{f} (l_1 \composeL f, l_2 \composeL f) \in \amod{}(\ca{}) \;\land\; l_1 \composeL f \leq l \composeL r \\

		\hspace*{1cm} \land\\
		
		(p, q) \in \amod{}(\ca{})
		\;\land\; p \leq l \composeL r \implies \\
		\hspace*{0.2cm}
		\begin{array}{l}
			\exsts{f} (l_1 \composeL f, l_2 \composeL f) \in \amod{1}(\ca{}) \;\land\; l_1 \composeL f \leq l \composeL r \\
			\hspace*{1cm} \lor\\
			\left( l \meetL l_1 \right) = \left\{ \unitL \right\}
		\end{array}

	\end{array}
\end{array}
\]
%
\end{definition}
