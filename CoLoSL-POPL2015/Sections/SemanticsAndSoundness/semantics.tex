\vspace{-.5ex}
\section{\colosl Program Logic}
\vspace{-.5ex}
\label{sec:semantics}

This section introduces the core concepts behind our program logic. We
start by defining what the rely and guarantee conditions of each
thread are in terms of their action models. This allows us to define
\emph{stability} against rely conditions, \emph{repartitioning},
which logically represents a thread's atomic actions (and have to be in
the guarantee condition), and \emph{semantic implication}. Equipped
with these notions, we can justify the \eqref{eq:shift} and
\eqref{eq:extend} principles. We conclude this section with a sketch
of the soundness of \colosl.

\vspace{-.5ex}
\subsection{Proof System}
\vspace{-.5ex}

\paragraph{Rely}
The rely relation represents potential interferences from the
environment. Although the rely will be different for every program
(and indeed, every thread), it is always defined in the same way,
which we can break down into three kinds of possible interferences. In
the remainder of this section, given a logical state $l =
((@s,h),\ca{})$, we write $\heapPart{l}$, $\capPart{l}$ for
$\m{fst}(l)$ and $\m{snd}(l)$, respectively.

The first relation, $\extendR$, extends the shared state $g$ with new
state $g'$, along with a new interference $\lmod'$ on $g'$. We introduce
the following notation to describe extensions that respect all
previous subjective views of a world (and may change the global action
model in the process provided it preserves action closure with respect
to both the old and the new local action models):
\begin{definition}[Action model extension]\label{def:amodExtension}
An action model $\gmod'$ \emph{extends} ($g, \lmod, \gmod$) with
$(g', \lmod')$, written
$\expandsAM{\gmod'}{g'}{\lmod'}{g}{\lmod}{\gmod}$, iff for all
$\lmod_0 \subseteq \lmod$ and $s, r$ such that $s \composeL r = g$:
\[
\extendsAM{\lmod, \gmod}{s}{r}{\lmod_0} \implies \extendsAM{\lmod \cup \lmod', \gmod'}{s}{r\composeL g'}{\lmod_0}
\]
\end{definition}
Then the extension rely, $\extendR$, is defined as
\vspace{-1ex}
\[
  \extendR \eqdef
  \left\{
  \begin{array}{@{} l @{\hspace*{2pt}} | @{\hspace*{2pt}} r @{}}
    \left(
    \begin{array}{@{} l @{}}
      (l , g, \lmod, \gmod),\\
      \left(
      \begin{array}{@{} l @{}}
	l,
	g \composeL g',\\
	\lmod \cup \lmod', \gmod'
      \end{array}
      \right)
    \end{array}
    \right)
    &
    \begin{array}{@{} l @{}}
      \capPart{g'} \subseteq \dom{\lmod'} \cup \dom{\lmod} \land \\
      \extendsAM{\lmod \cup \lmod', \gmod'}{g'}{g}{\lmod'} \land \\
      \expandsAM{\gmod'}{g'}{\lmod'}{g}{\lmod}{\gmod}
    \end{array}
  \end{array}
  \right\}
\]
%% Intuitively, the rely relation describes all possible updates by the environment. At any one point, a thread in the environment may extend the shared state with additional resources ($g'$), introduce new interference to describe how the new resources can be manipulated ($\lmod_0$) and consequently rewrite the global action model ($\gmod'$). This is captured by the \extendR\ relation. For this to be possible, the resultant action model pair $(\gmod', \lmod \cup \lmod_0)$ must be closed under the newly added resources and interference: $\extendsAM{\gmod', \lmod \cup \lmod_0}{g'}{g}{\lmod_0}$. Additionally, for all subjective state $l_1$, context $r$ and action model $\lmod_1$ under which the old action model pair was closed, the new action model pair must also  be closed when the context is extended with the new resources $\extendsAM{\gmod', \lmod \cup \lmod_0}{l_1}{r \composeL g'}{\lmod_1}$. 


The second kind of interference is the \emph{update} of the global
state according to actions in the global action model whose capability
is ``owned by the environment'', \textit{i.e.}, nowhere to be found
in the current local and shared states.
\vspace{-1ex}
\[	
  \updateR \eqdef
  \left\{
  \begin{array}{@{} l @{\hspace*{2pt}} | @{\hspace*{2pt}} r @{}}
    \left(
    \begin{array}{@{} l @{}}
      (l, g, \lmod, \gmod),\\
      (l, g', \lmod, \gmod)
    \end{array}
    \right)
    &
    \begin{array}{@{} l @{} }
      \begin{array}{@{} l @{}}
	\exsts{\ca{} }\left( \capPart{l} \composeCap \capPart{g} \right) \disjoint \ca{} \land 
	(g, g') \in \gmod(\ca{}) \\
      \end{array}	
    \end{array}
  \end{array}
  \right\}
\]	
%% Moreover, at any one point the environment may update the shared state by performing an action for which it has the sufficient capability ($\ca{}$). This is described by the \updateR\ relation. For this to be possible, the thread performing the action must have the $\ca{}$ capability in its own local state and thus the $\ca{}$ capability must be disjoint from both shared and local states of the current world. 

The third and last kind of interference is the \emph{shifting} of the
local interference relation to a new one that allows more action while
preserving all current subjective views (as expressed by the global
action model):
\vspace{-1ex}
\[
  \shiftR \eqdef
  \left\{
  \begin{array}{@{} l @{\hspace*{2pt}} | @{\hspace*{2pt}} r @{}}
    \left(
    \begin{array}{@{} l @{}}
      (l, g, \lmod, \gmod),
      \left( l, g, \lmod \cup \lmod', \gmod \right)
    \end{array}
    \right)
    &
    \expandsAM{\gmod}{\unitL}{\lmod'}{g}{\lmod}{\gmod}
  \end{array}
  \right\}
\]
%% Finally, the environment may extend the local action model by shifting (rewriting) some of the existing behaviour. This is modelled by the \shiftR\ relation. This is only possible if for any subjective state $l$, context $r$ and action model $\gmod'$ under which the old action model pair was closed, the new action model pair is also closed: $\extendsAM{\gmod, \lmod \cup \lmod_0}{l}{r}{\gmod'}$. 


\begin{definition}[Rely]
  The \emph{rely} relation $\rely: \pset{\Worlds \times \Worlds}$ is
  defined as follows, where $(\cdot)^{\text{*}}$ denotes reflexive
  transitive closure:
\vspace{-1ex}
  \[
\vspace{-1ex}
  \rely \eqdef  \left(\updateR \cup \extendR \cup \shiftR \right)^{\text{*}}
  \]
\end{definition}

The rely relation enables us to define the stability of assertions
with respect to the environment actions.
\begin{definition}[Stability]
  An assertion $P$ is \emph{stable} ($\stable{P}$) if, for all $\lenv
  \in \LEnv$ and $w, w' \in \Worlds$, if $w, \lenv |= P$ and $(w, w')
  \in \rely$, then $w', \lenv |= P$.
%\[	
%\begin{array}{r @{} l}
%	\stable{P} \iffdef &
%	\for{\lenv}\for{w, w'} \\
%	& w \in \sem[\lenv]{P} \land (w, w') \in \rely \implies
%	 w' \in \sem[\lenv]{P}
%\end{array}
%\]
\end{definition}

Proving that an assertion is stable is not always obvious, in
particular when there are lots of actions to consider (all those in
$\extendR$, $\updateR$, $\shiftR$); as it turns out, we only need to
check stability against update actions in $\updateR$, as expressed by
the following lemma.

\begin{lemma}[Stability]
  If an assertion $P$ is stable with respect to actions in $\updateR$,
  then it is stable.
%% An assertion $P$ is stable if it cannot be falsified by possible updates on the shared state from other threads:
%% %
%% \[
%% \begin{array}{l}
%% 	\for{\lenv}\for{w, w'} 
%%     w, \lenv |= P \land (w, w') \in \updateR \implies
%% 	 	w', \lenv |= P\\
	 
%% 	 \implies \stable{P}
%% \end{array}	 
%% \]
\end{lemma}

\paragraph{Guarantee}
We now define the guarantee relation that describes all possible
updates the current thread can perform. In some sense, the guarantee
relation is the dual of rely: the actions in the guarantee of one
thread are included in the rely of concurrently running threads. Thus,
it should come as no surprise that transitions in the guarantee can be
categorised using three categories which resonate with those of the
rely. The \emph{extension} guarantee is similar to the extension rely
except that new capabilities corresponding to the new shared region
are materialised in the local and shared state, and a part of the
local state is moved into the shared state:
\vspace{-1ex}
\[
\extendG \eqdef
\left\{
\begin{array}{L @{\hspace*{2pt}} | @{\hspace*{2pt}} R}
  \left(
  \begin{array}{@{} l @{}}
    (l \composeL l', g, \lmod, \gmod),\\
    \left(
    \begin{array}{@{} l @{}}
      l \composeL (\unitH, \ca{1}),\\
      g \composeL g',\\
      \lmod \cup \lmod', \gmod'
    \end{array}
    \right)
  \end{array}
  \right)
  &
  \begin{array}{L}
    g' = l' \composeL (\unitH, \ca{2})  \land\\
    \ca{1} \composeCap \ca{2} \subseteq \dom{\lmod'} \land \\
    \ca{1} \composeCap \ca{2} \compatL \dom{\lmod} \land \\
    \extendsAM{\lmod \cup \lmod', \gmod'}{g'}{g}{\lmod'} \land \\
    \expandsAM{\gmod'}{g'}{\lmod'}{g}{\lmod}{\gmod}
  \end{array}
\end{array}
\right\}
\]
%% The current thread may at any point extend the shared state with some of its locally held resources $l'$ and in doing so introduce new interference to describe how the new resources may be mutated ($\lmod_0$) and generate  new capabilities ($\ca{1} \composeCap \ca{2} \subseteq \dom{\lmod_0}$) that facilitate the new interference. This is captured by the \extendG\ relation and is analogous to \extendR.  The generated capabilities must be fresh ($\ca{1} \composeCap \ca{2}  \compatL \dom{\lmod}$). The last two conjuncts  enforce closure of the new action models and can be justified as in the case of \extendR.



The \emph{update guarantee} $\updateG$ is more involved than its
$\updateR$ counterpart, because updates in the guarantee may move
resources from the local state into the shared state (similarly to
extensions above) at the same time that it mutates them as prescribed
by an enabled action. Intuitively, we want to enforce that resources
are not created ``out of thin air'' in the process. This can be
expressed as preserving the \emph{orthogonal} of the combination of
the local and global states, \textit{i.e.}, the set of states
compatible with that combination.

\begin{definition}[Orthogonal]\label{def:orthogonal}
The \emph{orthogonal} of a state  $\heapSize{.}: \Heaps \rightarrow
\pset{\Heaps}$ is defined as the set of all states that are disjoint
from it:
\vspace{-1ex}
\[
\vspace{-1ex}
\heapSize{\h{}} \eqdef \left\{ \h{}' \mid \h{} \compatL \h{}' \right\}
\]
\end{definition}

Moreover, update actions are not allowed to introduce new
capabilities, as this is the sole privilege of extension actions.
\vspace{-1ex}
\[
	\updateG \eqdef
 	\left\{
	\begin{array}{@{} l @{\hspace*{2pt}} | @{\hspace*{2pt}} r @{}}
	   	\left(
	   	\begin{array}{@{} l @{}}
	     	(l, g, \lmod, \gmod),\\
	     	(l', g', \lmod, \gmod)
	   	\end{array}
		\right)
	  	&
	  	\begin{array}{@{} l @{} }
	  		\capPart{(l' \composeL s')}  = \capPart{(l \composeL g)} \land\\
		  	\left(
		  	\begin{array}{@{} l @{} }
		 		g = g' \lor\\
		 	  	\left(
		 	  	\begin{array}{@{} l @{}}
		 	  		\exsts{\ca{} \subseteq \capPart{l}}
		 	  		(g, g') \in \gmod(\ca{}) \land\\
		 	  	
		 	  	\heapSize{\heapPart{\left(l \composeL g\right)}} \subseteq 
		 	  	\heapSize{\heapPart{\left(l' \composeL g'\right)}}
		 	  	\end{array}	
		 	  	\right)
		 	\end{array}
	   		\right)
   		\end{array}
 	\end{array}
	\right\}
\]
%% The current thread can update its local state arbitrarily and may
%% update the shared state by performing an action for which it holds the
%% necessary capability $\ca{}$ in its local state. This is described by
%% the \updateG\ relation. For this to be possible, the total amount of
%% capabilities held locally and in the shared state must not grow. That
%% is, the current thread may not generate new capabilities arbitrarily
%% since capabilities may only be generated upon extension of the shared
%% state as captured in \extendG. Similarly, if the shared state is being
%% updated ($g \not= g'$),  then the heap resources (the domain of the
%% heap) may not grow ($\heapSize{\heapPart{(l \composeL g)}} \subseteq
%% \heapSize{\heapPart{(l' \composeL g')}}$) where the $\heapSize{.}$
%% function is as given in \defin\ref{def:orthogonal}. The intuition
%% behind this constraint is to ensure that threads do not generate
%% resources out of thin air in order to enable actions. Note that this
%% does not stop a thread from local allocation/deallocation of
%% resources. In the case of allocation, the desired effect can be
%% achieved by first allocating the new resources locally (\textit{i.e.}
%% extending the local state from $l$ to $l \composeL l''$), while
%% leaving the shared state untouched; and then performing the relevant
%% action on the shared state (\textit{i.e.} updating $g$ to $g'$)
%% without extending the heap resources, as illustrated in the following
%% sequence.
%% \[
%% \begin{array}{l}
%% 	(l, g, \gmod, \lmod) \rightarrow (l \composeL l'', g, \gmod, \lmod) \rightarrow (l', g', \gmod, \lmod)\\
%% 	\text{ where } \heapSize{\heapPart{(l \composeL l'' \composeL g)}} \subseteq \heapSize{\heapPart{(l' \composeL g')}}
%% \end{array}
%% \]
%% Deallocation of local resources can be achieved in a similar fashion.
%That is, all previously compatible frames remain compatible. We
%illustrate the need for this constraint in \ex
%~\ref{ex:compatibleFrames}.

Lastly, the current thread may extend the local action model by
shifting some of the existing interference. This is modelled by the
\shiftG relation which is analogous to \shiftR.
\[
\shiftG \eqdef
\left\{
\begin{array}{@{} l @{\hspace*{2pt}} | @{\hspace*{2pt}} r @{}}
  \left(
  \begin{array}{@{} l @{}}
    (l, g, \lmod, \gmod),
    \left( l, g, \lmod \cup \lmod_0, \gmod  \right)
  \end{array}
  \right)
  &
  \expandsAM{\gmod}{\unitL}{\lmod_0}{g}{\lmod}{\gmod}
\end{array}
\right\}
\]

\begin{definition}[Guarantee]
The \emph{guarantee} relation $\guarantee: \powerset (\Worlds \times \Worlds)$ is defined as
\[
\guarantee \eqdef  \left( \updateG \cup \extendG \cup \shiftG \right)^{\text{*}}
\]
\end{definition}

Using the guarantee relation, we introduce the notion of
\emph{repartitioning} $\repartitions{P}{Q}{R_1}{R_2}$. This relation
holds whenever, from any world satisfying $P$, if whenever parts of
the composition of its local and shared states that satisfies $R_1$ is
exchanged for one satisfying $R_2$, it is possible to split the
resulting logical state into a local and shared part again, in such a
way that the resulting transition is in $\guarantee$.

\begin{definition}[Repartitioning] \label{def:repartitioning}
  We write $\repartitions{P}{Q}{R_1}{R_2}$ if, for every $\lenv \in
  \LEnv$, and world $w_1 = (l_1,g_1,\lmod_1,\gmod_1)$ such that $w_1,
  \lenv |= P$, there exists states $m_1,m' \in \Heaps$ such that
  $(m_1, \emptyset), \lenv \slsat R_1$ and
\begin{itemize} 
\item $m_1 \composeH m' = \heapPart{(l_1\composeL g_1)}$; and
\item for every $m_2$ where $(m_2, \emptyset), \lenv \slsat R_2$,
  there exists a world $w_2 = (l_2,g_2,\lmod_2,\gmod_2)$ such that
  $w_2, \lenv |= Q$,
  $m_2 \composeH m' = \heapPart{(l_2 \composeL g_2)}$, and
  $(w_1, w_2) \in \guarantee$.
\end{itemize}
\end{definition}

We write $P \semimplies Q$ for $\repartitions{P}{Q}{\emp}{\emp} $, in
which case the repartitioning has no ``side effect'' and simply
shuffles resources around between the local and shared state or
modifies the action models. This is the case for (\shiftRule) and
(\extendRule), whose proof will be given in the next section.

%\[
%\begin{array}{@{} r l @{}}
%	\repartitions{P}{Q}{p}{q} \iffdef 
%	& \for{\lenv} \repartitions{\sem[\lenv]{P}}{\sem[\lenv]{Q}}{\semH[\lenv]{\heapAss{1}}}{\semH[\lenv]{\heapAss{2}}}\\\\
%	
%	\repartitions{W_1}{W_2}{H_{1}}{H_{2}} \iffdef &
%	\begin{array}{l}
%		\for{w_1 \in W_1} \exsts{h_1 \in H_1\ , h'}\\
%		\hspace*{0.5cm} h_1 \composeH h' = \heapPart{(\collapseW{w_1})} \;\land\\
%		\hspace*{0.5cm} \for{h_2 \in H_2} \exsts{w_2 \in W_2}\\
%		\hspace*{1cm} h_2 \composeH h' = \heapPart{(\collapseW{w_2})} \;\land\; (w_1, w_2) \in \guarantee
%	\end{array}
%\end{array}
%\]
%%
%
%

\paragraph{Proof Rules}
Our proof rules are of the form $\entails \{P\}\ C\ \{Q\}$ and carry
an implicit assumption that the pre- and post-conditions of their
judgements are stable. Most of the proof rules are standard and
omitted for space reasons; they include the rules of concurrent
separation logic~\cite{csl-tcs} (including the \proofRule{Parallel}
rule mentioned in the introduction). We use two additional rules,
similar to their CAP counterparts~\cite{cap-ecoop10}, which are as
follows, where $\entails_{\textsf{SL}}$ denotes the standard
sequential separation logic judgement:
%% \begin{definition}[Proof rules]
%%   \label{def:proofRules}
%%   The \emph{atomic} proof judgement, \proofRule{Atom}, is defined as
%%   follows where
%% %
\begin{mathpar}
	\infer[\proofRule{Atom}]{
		\entails \{P\} \atomic {C} \{Q\}
	}{
		\entails_{\textsf{SL}} \{R_1\}\ C\ \{R_2\} &
		\repartitions{P}{Q}{R_1}{R_2}
	}
%% \]
%% The \emph{rule of consequence}, \proofRule{Conseq}, is as given below.
%% \[	

	\infer[\proofRule{Conseq}]{
		\entails \{P\}\ C\ \{Q\}
	}{
		P \semimplies P' &
		\entails \{P'\}\ C\ \{Q'\} &
		Q' \semimplies Q
	}
\end{mathpar}
%% In both judgements, $\semimplies$ denotes the repartitioning of the state as described in \defin\ref{def:repartitioning}.
%% \end{definition}
%
%
%\subsection{Program logic}
%We build the \colosl\ program logic on top of the views framework~\cite{views} instantiated as follows. Our view semigroup is the separation algebra of worlds as described in \defin\ref{def:worlds}. 
%The machine states and the reification of worlds are given in \defin\ref{def:machineStates} and \ref{def:reification}\footnote{
%Generally, since \colosl\ is parametric in the separation algebra of heaps, the set of machine states and the reification function over heaps are also parameterised. Further details can be found in~\cite{colosl-tr14}.
%}. 
%The grammar of \colosl\ atomic commands is provided in \defin\ref{def:atomicCommands}; we show the soundness of \colosl\ atomic commands in the companion technical report~\cite{colosl-tr14}. 
%The axiomatisation of \colosl\ atomic commands are as per \defin\ref{def:proofRules}.
%%
%%
%\begin{definition}[Machine States]\label{def:machineStates}
%The set of \colosl\ \emph{Machine States} are $\MStates \eqdef \Heaps$.
%\end{definition}
%%
%%
%%
%%
%\paragraph{Programming language}
% We use the programming language of the views framework \cite{views} instantiated with a set of atomic commands. \colosl\ is parametric in the set of atomic commands: we provide an atomic construct $\atomic{.}$ that can be applied to any sequential command and enforce atomic behaviour. We proceed with the grammar of \colosl atomic commands.
%%
%\begin{definition}[Atomic commands]\label{def:atomicCommands}
%The \emph{atomic commands} of \colosl, $\atom{} \in \Atoms$, are defined by the following grammar
%%
%\begin{mathpar}
%	\atom{} ::= \atomic{\seq{}}
%	
%	\seq{} ::= \bc{} \mid \li{skip} \mid \seq{1};\seq{2} \mid \seq{1}+\seq{2} \mid \seq{}^{\text{*}}
%\end{mathpar}
%%
%where $\bc{} \in \Basics$ denotes a set of basic commands that can be instantiated with any set of sequential commands. In the examples of this paper, our basic commands are given by the following grammar.
%%
%\[
%\bc{} ::= \li{x}:= E \mid \li{assume}(E)
%\]
%%
%\end{definition}
%
%\paragraph{Soundness}\label{subsec:soundness}
%In~\cite{colosl-tr14} we provide the semantics of the \colosl\ programming language; we demonstrate that the \colosl\ program logic is sound with respect to its semantics by appealing to the views framework~\cite{views}.


\vspace{-.5ex}
\subsection{Interference Manipulations}\label{subsec:extension}
\vspace{-.5ex}


In this section we formalise the requirements of the \eqref{eq:extend}
and \eqref{eq:shift} semantic implications and show that they are
valid.

\paragraph{Shared State Extension}
When extending the shared state using currently owned local resources,
one specifies a new interference assertion over these newly shared
resources. While in \colosl the new interferences may mention parts of
the shared state beyond the newly added resources (in particular
the existing shared state), they must not allow visible updates to those
parts, so as not to invalidate other threads' views of existing
resources. We thus impose a locality condition on the newly added
behaviour to ensure sound extension of the shared state, similarly to
the confinement constraint of local fences of
\defin\ref{def:actconf}. We first motivate this constraint with an example.

\begin{example}\label{ex:badExtension}
Let $P \eqdef \cell{x}{1} * \shared{\cell{y}{1} \lor \cell{y}{2}}{I}$
denote the view of the current thread with $I \eqdef \left(\token{b}:
\left\{\cell{y}{1} \swap \cell{y}{2}\right\} \right)$. Since the
current thread owns the location addressed by $x$, it can extend the
shared state as $Q \eqdef [\token{a}] * \shared{\left(\cell{y}{1} |/
  \cell{y}{2} \right) * \cell{x}{1}}{I \cup I'}$ where $ I' \eqdef
\left( \token{a}: \cell{x}{1} \swap \cell{x}{2} \right) $.  In
extending the shared state, the current thread also extended the
interference allowed on the shared state by adding a new action
associated with the newly generated capability resource $[\token{a}]$,
as given in $I'$, which updates the value of location $x$. Since
location $x$ was previously owned privately by the current thread and
was hence not visible to other threads, this new action will not
invalidate their view of the shared state, hence this extension is a
valid one.

\sloppy
If, on the
other hand, $I'$ is replaced with
$I''
\eqdef \left( \token{a}:
\left\{
  \cell{y}{1} \swap \cell{y}{3}
\right\}\right)$, where location $y$ can be mutated, the situation above
is not allowed.  Indeed, other threads may rely on the fact that the
only updates allowed on location $y$ are done through the
$[\token{b}]$ capability as specified in $I$, and would be spooked by
this new possible behaviour they were not aware of (as it is not
in $I$).
\end{example}

\fussy
In order to ensure sound extension of the shared state, we require in
\eqref{eq:extend} that the newly introduced interferences are confined
to the locally owned resources.

\begin{definition}
  A set of states $\mathcal P$ \emph{contains} an action model
  $\gmod$, written $\mathcal P \containI \gmod$, if
  \[
  \E{\fence{}} \mathcal P\subseteq \fence{} /| \fence{} \strictfences \gmod
  \]
\end{definition}

The $P\containI I$ notation used in \eqref{eq:extend} is then a
straightforward lift of this definition to assertions $P \in
\Assertions$ and interference assertions $I \in \IAssertions$:
\[
P \containI I \iffdef \for{\lenv, \ca{}} \left\{l \mid l, \lenv \slsat P \right\} \containI \semI[\lenv]{I}(\ca{})
\]

\begin{lemma}
  The semantic implication (\extendRule) is valid.
\end{lemma}
\begin{proof}[Proof (sketch)]
  It suffices to show that the repartitioning ($\semimplies$) is valid
  and allowed by the guarantee relation. We can show that transitions
  corresponding to applications of \eqref{eq:extend} are contained in
  the \extendG relation.
\end{proof}


\paragraph{Action Shifting}
Notice that, according to our rely and guarantee conditions, the local
action model may only grow with time. However, that is not to say that
the same holds of interference relations in shared state assertions:
as seen in \S\ref{sec:intuition}, they may forget about actions that
are either redundant or not relevant to the current subjective
view via \emph{shifting}. As for the action model closure relation
(\defin~\ref{def:actclos}), this needs to be the case not only
from the current subjective view but also for any evolution of it
according to potential actions, both from the thread and the
environment. To capture this set of possible futures, we refine our
notion of local fences (\defin~\ref{def:localFence}), which was
defined in the context of the global shared state, to the case where
we consider only a subjective state within the global shared state.
For this, we need to also refine a second time our notion of action
application to ignore the context of a subjective state, which as far
as a subjective view is concerned could be anything.

%% Recall from \S\ref{sec:intuition} that \colosl\ allows for forgetting
%% of those actions that do not affect the subjective view of the shared
%% state, as well as rewriting the behaviour of actions to gain a more
%% accurate account of their effect. This is achieved through
%% \emph{action shifting}, which allows the swapping of one global action
%% model for another one that is equivalent. Here, being equivalent means
%% that there is some fence that sees them as equivalent, in that actions
%% from the new global action model contains only actions with the same
%% effect from the old one, and moreover any action from the old one that
%% affects the subjective state has to remain in the new one. Since we
%% are working at the level of a subjective view and not the whole shared
%% state, we need to refine our notion of action application and fences.

\begin{definition}[Subjective action application]
  The \emph{subjective application} $a(s)$ of an action $a$ on a
  logical state $s$ is defined as follows:
  \[
  a(s) == \m{fst} \left( a[s, \m{fst}(\updateFP{a}) - (s\meetL \m{fst}(\updateFP{a}))] \right)
  \]
\end{definition}

Note that, in contrast with $a[l]$, only part of the active
precondition $p' = \m{fst}(\updateFP{a})$ has to intersect with the
subjective view $s$ for $a(s)$ to apply. Thus, we fabricate a context
that will satisfy the rest of $p'$ to be able to reuse our $a[s,r]$
definition (\defin\ref{def:actionApplicationPair}). We observe that $\m{snd} \left( a[s, \m{fst}(\updateFP{a}) - (s\meetL \m{fst}(\updateFP{a}))] \right) = \unitL$ always, since it is fabricated as the minimum context needed. 

\begin{definition}[Fenced action model]
  An action model $\lmod \in \AMods$ is \emph{fenced} by $\fence{}
  \in \pset{\LStates}$, written $\fence{} \fences \lmod$, if, for
  all $l \in \fence{}$ and all $\ca{}\in\m{rg}(\lmod)$,
\[
\begin{array}{L}
  a(l)\text{ is defined} /| \m{visible}(a,l) => a(l)\in\fence{}
  %% (p, q) \in \gmod(\ca{}) /|
  %% p \meetL l \not= \emptyset /|
  %% \updateFP{p, q} = (p', q') \land\\
  %% p' \maxMeetL l = p'' \land
  %% p'' /= \unitL \land
  %% l = p'' \composeL l' /|
  %% q' \compatL l'
  %% =>
  %% q' \composeL l' \in \fence{}
\end{array}
\]
\end{definition}

In contrast with local fences, fences do not require that actions be
confined inside the subjective state.  We lift the notion of fences to
assertions as follows, given $\fenceAss{} \in \Assertions$ and $I \in
\IAssertions$:
\begin{align*}
  \fenceAss{} \fences I &\iffdef \for{\lenv,\ca{}}
  \{ l \mid l, \lenv \slsat F \} \fences \semI[\lenv]{I}(\ca{})
  %% \fenceAss{} \strictfences I &\iffdef \for{\lenv,\ca{}}
  %% \{ l \mid l, \lenv \slsat F \} \strictfences \semI[\lenv]{I}(\ca{})
\end{align*}


For instance, a possible fence for the interference assertion
$I_{\li{y}}$ of \fig\ref{fig:concurrentInc} is denoted by the assertion
\[
F_{\li{y}} \eqdef
\bigvee_{v = 0}^{10} (x|-> v * y|->v) |/ (x|->v+1 * y|->v)
\]
%% \[
%% \fence{\li{y}} \eqdef\begin{array}[t]{L}
%% \left\{(([\li{x}:v,\li{y}:v],\emptyset),\emptyset) \mid v \in \{0,
%% \ldots, 10\} \right\} \cup\\
%% \left\{(([\li{x}:v+1,\li{y}:v],\emptyset),\emptyset) \mid v \in \{0,
%% \ldots, 10\} \right\}
%% \end{array}
%% \]

%% TODO:
%% \julescomment{Find better/more examples, perhaps of non-fencing as
%%   well?}


%%%%%%%%%%%%%%%%%%%%%
%% \paragraph{Fences}
%% A subjective state represents the instantaneous view of a thread at a
%% particular instant. However, this view is subject to interferences by
%% other threads, as well as by the thread's own actions, as specified by
%% an action model. To reason on such states, it makes sense to consider
%% all its potential future incarnations with respect to these
%% modifications.  

%% %  F |> s <==> \forall p, q, k, l, l', p',
%% %    (p,q)\in s(k) /\ l o l' o p' is defined /\ l o l'\in F /\ l o p' = p /\ l\neq\emptyset
%% %      => \exists f, l'', q'. l = l'' o f /\ q = f o q' /\
%% %            (l''\neq\emptyset /\ q # l' => l' o q\in F)


%% We are now ready to describe what effect an action $a$ can have on a
%% \emph{subjective} state $l$: because the subjective state is not the
%% whole shared state, the precondition of $a$ only needs to be
%% \emph{compatible} with $l$. When that is the case, the action consumes
%% the part of its active precondition that intersects with $l$ and plugs
%% in its active postcondition instead, provided that the result is
%% defined.





\begin{definition}[Action shifting]
  Given $\lmod, \lmod' \in \AMods$ and $\mathcal{P} \in
  \pset{\LStates}$, $\lmod'$ is a \emph{shifting} of $\lmod$ with
  respect to $\mathcal{P}$, written $\lmod \weakenI{\mathcal{P}}
  \lmod'$, if there exists a fence $\fence{}$ such that
  \[
  \begin{array}{L}
    \mathcal{P} \subseteq \fence{} \land \fence{} \fences \lmod
    \land
    \for{l\in \fence{}}\for{\ca{}}\\
    \ (\for{a\in\lmod'(\ca{})}
    \m{reflected}(a,l,\lmod(\ca{}))) /|\null\\
    \ \for{a\in\lmod(\ca{})}
    a(l)\text{ is defined} {/|}\m{visible}(a,l) {=>}
    \m{reflected}(a,l,\lmod'(\ca{}))
    %% \for{\ca{}, l, p, q, p', q', l'} l \in F \land \updateFP{p, q} = (p', q') =>\\
    %% \left(\begin{array}{L}
    %%   (p, q) \in \gmod'(\ca{}) 
    %%   \land p \leq l \composeL r => \\
    %%   \hspace*{0.4cm}\exsts{f} (p' \composeL f, q' \composeL f) \in
    %%   \gmod(\ca{}) \land p' \composeL f \leq l \composeL r
    %% \end{array}\right) /|\\
    %% \left(\begin{array}{L}
    %% (p, q) \in \gmod(\ca{})
    %%   \land p \composeL l' = l \composeL r /| q'\compatL l' /| l \meetL p' /= \{ \unitL \} => \\
    %%   \hspace*{0.4cm}\exsts{f} (p' \composeL f, q' \composeL f) \in \gmod'(\ca{}) \land p' \composeL f \leq l \composeL r
    %% \end{array}\right)
  \end{array}
  \]
\end{definition}

The first conjunct expresses the fact that the new action model
$\lmod'$ cannot introduce new actions not present in $\lmod$, and the
second one that $\lmod'$ has to include all the visible potential
actions of $\lmod$. We lift $\weakenI{}$ to assertions as follows:
\[
I \weakenI{P} I' == \V{\lenv,\ca{}} \semI[\lenv]{I} \weakenI{\{s|||
  s,\lenv\slsat P\}} \semI[\lenv]{I'}
\]


%% Intuitively, there exists an invariant $\fence{}$ that contains the states in $\mathcal{P}$ and encompasses the behaviour of actions in $\gmod$ and 
%% \begin{enumerate}
%% 	\item if an action in $\gmod'$ is possible given a state $l \in \fence{}$ and an arbitrary context $r$ (its precondition is satisfiable by $l \composeL r$), then there exists a similar action with the same update footprint in $\gmod$ whose precondition is also satisfiable by $l \composeL r$. 
%% 	\item if an action in $\gmod$ is possible given a state $l \in \fence{}$ and an arbitrary context $r$ (its precondition is satisfiable by $l \composeL r$), then \emph{either}  there exists a similar action with the same update footprint in $\gmod'$ whose precondition is also satisfiable by $l \composeL r$; \emph{or} the action does not affect $l$, that is  $l \meetL p' = \{\unitL\}$ where $(p', q')$ denotes the update footprint of the action; \emph{or}  the resultant state ($q' \composeL l'$) is undefined.
%% \end{enumerate} 


\begin{lemma}
  The semantic implications \eqref{eq:shift} is valid.
\end{lemma}
\begin{proof}[Proof (sketch)]
  It suffices to show that the repartitioning ($\semimplies$) is valid
  and allowed by the guarantee relation. We can show that the transitions
  corresponding to applications of \eqref{eq:shift} are contained in
  the \shiftG relation.
\end{proof}

%We provide a set of rules in \fig\ref{fig:shiftRules} that reduce action shifting to logical entailments. While these rules are not complete, we found them sufficient to reason about our examples. The $\approx^R$ notation in the conclusion of \proofRule{Expand/Contract}, denotes that the shifting is valid both ways (\textit{i.e.} $I_1 \approx^R I_2$ iff $I_1 \weakenI{R} I_2$ and $I_2 \weakenI{R} I_1$). We write $\exact{P}$ to denote that the assertion $P$ is \emph{exact}. That is, there exists $l$ such that for all $\lenv$ and $l'$ where $l', \lenv \slsat P$ then $l = l'$. 
%%
%%
%\begin{figure*}
%\hrule\vspace*{5pt}
%\begin{mathpar}
%	\infer[\proofRule{Hide}]{
%		I \cup \left\{ \capAss{}: \left\{ P * \fenceAss{}  \swap Q * \fenceAss{}  \right\}\right\} \weakenI{R} 
%		I 	
%	}{
%		R \entails \fenceAss{} * \fenceAss{}' 
%		& \fenceAss{} * \fenceAss{}' \fences  I \cup \left\{ \capAss{}: \left\{P * \fenceAss{}  \swap Q * \fenceAss{}  \right\}\right\}
%		&\exact{\fenceAss{}}
%%		& P * \fenceAss{} * \fenceAss{}' \not\entails  false
%		& P \sepish \fenceAss{}' \entails P * \fenceAss{}' 
%		& \precise{P}
%		& \precise{\fenceAss{}'}
%	}
%
%
%	\infer[\proofRule{False-L}]{	
%		I \cup \left\{ \capAss{}: \left\{ P \swap Q\right\}\right\} \weakenI{R} 
%		I 	
%	}{
%		R \entails \fenceAss{} 
%		& \fenceAss{} \fences  I \cup \left\{ \capAss{}: \left\{ P \swap Q \right\}\right\}
%		&F \sepish P \entails \m{false}
%	}
%	
%	
%	\infer[\proofRule{False-R}]{	
%		I \cup \left\{ \capAss{}: \left\{ P \swap Q\right\}\right\} \weakenI{R} 
%		I 	
%	}{
%		R \entails \fenceAss{} 
%		& \fenceAss{} \fences  I \cup \left\{ \capAss{}: \left\{ P \swap Q \right\}\right\}
%		&F \sepish Q \entails \m{false}
%	}
%
%
%	\infer[\proofRule{Expand/Contract}]{
%%		\left( I \cup \left\{ \capAss{}: P \swap Q\right\}\right) \approx^R \left( I \cup \left\{ \capAss{}: \bigcup\limits_{i \in J} P * R_i \swap Q * R_i \right\} \right) 	
%		I \cup \left\{ \capAss{}: P \swap Q\right\} \;\approx^R\;  I \cup \left\{ \capAss{}: \left\{P * R_i \swap Q * R_i \mid i \in J\right\} \right\}	
%	}
%	{	
%		R \entails \fenceAss{} 
%		& \fenceAss{} \fences I \cup \left\{ \capAss{}: \left\{ P \swap Q \right\}\right\}
%		&\bigwedge\limits_{i \in J} \exact{R_i}
%		& \fenceAss{} \sepish P \vdash \bigvee\limits_{i \in J} \fenceAss{} \sepish \left(P * R_i \right)
%	}
%\end{mathpar}
%%\[
%%%\infer{
%%%	\left(I \cup \left\{ \capAss{}: P \swap Q\right\}\right) \weakenI{R} 
%%%	I 	
%%%}{
%%%	R \entails \fenceAss{} 
%%%	& \fenceAss{} \fences  I \cup \left\{ \capAss{}: P \swap Q\right\}
%%%	&\exact{\fenceAss{} \intersect Q}
%%%	& \fenceAss{} \intersect P \vdash \fenceAss{} \intersect Q
%%%}\\\\
%%\]
%\hrule
%%\vspace*{5pt}
%\caption{Action shifting rules.}
%\label{fig:shiftRules}
%\end{figure*}


%
\vspace{-.5ex}
\subsection{Soundness}
\vspace{-.5ex}
In this section we show that the program logic of \colosl\ is sound. We proceed with the definition of a \emph{valid} triple that relates the proof rules (Hoare triples) to the operational semantics of \colosl. In what follows, we write $C, m \rightarrow^{\text{*}} C', m'$ to denote the operational semantics relation where $C, C' \in \textsf{Comm}$ and $m, m' \in \Heaps$. We define a reification function that relates a \colosl\ world $w \in \Worlds$, to concrete states in \Heaps.
%We proceed by  providing the definition of a reification function that maps \colosl\ worlds onto states.
%
\begin{definition}[Reification]\label{def:reification}
The \emph{reification of worlds}, $\reifyW{.}: \Worlds \rightarrow \pset{\Heaps}$ is defined as:
%
\[
	\reifyW{(l, g, \gmod, \lmod)} \eqdef \heapPart{(l \composeL g)}
\]
%
\end{definition}
%
%
\begin{definition}[Valid triple] A triple is \emph{valid}, written $|= \{P\}\ C\ \{Q\}$, iff for all $\lenv \in \LEnv$, $w \in  \Worlds$ and  $\h{}, \h{}' \in \Heaps$,
%
\[
\begin{array}{l l}
	\text{if} & w, \lenv |= P  \land \h{} \in \reifyW{w} \land C, \h{} \rightarrow^{\text{*}} \li{skip}, \h{}' \\
	
	\text{then} & \exsts{w'} w', \lenv |= Q \land \h{}' \in \reifyW{w'}
\end{array}
\]
%
\end{definition}
%
%
%
%
\begin{theorem}[Soundness]
%
The \colosl\ program logic is sound. That is, if $|-\!\{P\}\ C\ \{Q\}$ then $|=\!\{P\}\ C\ \{Q\}$.
%
\begin{proof}(Sketch)
We build the \colosl\ program logic on top of the views framework~\cite{views} and provide the full details of  how we instantiate the framework in~\cite{colosl-tr14}. To establish the soundness of \colosl, it then suffices to show that the atomic triple in the conclusion of (\proofRule{Atom}) rule is valid; the full proof is presented in~\cite{colosl-tr14}. 
\renewcommand{\qed}{}
\end{proof}
%
\end{theorem}
%
%
%\subsection{Axiomatisation of \colosl principles}

%\paragraph{Action shifting}
%We do not give the semantic model of interferences here. Instead, we
%give rules that reduce it to logical entailments. We delay the
%semantic definitions until \S\ref{sec:soundness}, in which we will
%show that the rules we propose here are indeed sound.
