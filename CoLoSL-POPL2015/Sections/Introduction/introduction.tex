\section{Introduction}

This paper: compositional reasoning for concurrent programs, even in
the presence of sharing in data structures. Using \colosl, it is
possible to write concise specifications that only talk about the
portion of state actually accessed by the program, with respect to
both the private memory and the region that is shared between several
threads.

This goes beyond the state of the art in concurrent program
verification.

Reasoning about concurrency: major advances since Owicki-Gries:
compositional reasoning, from the seminal paper by Jones~\cite{rg} on
the one hand, and O'Hearn~\cite{csl-orig,csl-tcs} on the other hand,
later fused together in a series of increasingly compositional
formalisms such as RGSep~\cite{viktor-marriage}, Local RG~\cite{lrg},
deny-guarantee~\cite{dg}, and the CAP
family~\cite{cap-ecoop10,icap,tada}, to name a few.

Reasoning about sequential programs: major advances in
compositionality for pointer programs via separation
logic~\cite{seplog}, recently extended to programs manipulating data
structures with intrinsic sharing, such as graphs~\cite{ramification}.

Main ingredients: ability to zoom in on the portion of the shared
region under consideration, and to manipulate the interference
relation between threads so as to project the global effect of
interferences onto the subjective view of the shared state. The
subjective views of concurrently running threads can be merged
together at the end by \emph{overlapping} them.

Contributions:
\begin{enumerate}
\item
  We introduce the program logic \colosl, which achieves fully
  compositional rely/guarantee style reasoning for concurrent
  programs. Our soundness result is parametric in the underlying
  programming language and separation algebra of states.
\item
  We evaluate \colosl on a range of key examples that were challenging
  for previous work. We are able to derive natural and concise specs,
  circumscribed to the part of the shared state accessed by each
  thread, even in the case of threads operating on subparts of data
  structures with unspecified sharing (directed acyclic graphs in our
  example).
\item
  We demonstrate the generality of \colosl by deriving the main
  reasoning principles of existing program logics for concurrency
  within our framework.
\end{enumerate}

\paragraph{Outline}
