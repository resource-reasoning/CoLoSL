\section{Introduction}

Shared memory concurrency bla.  Compositionality is important and
hard. Motivate via our target examples.

[What we don't have]: because not subjective enough.

[What we do]: subjective = good. (view of the concrete state,
``fine-grained'' shared region--versus fine-grained reasoning on
coarse regions).

This paper presents the program logic \colosl, which achieves
compositional reasoning for concurrent programs, even in the presence
of sharing in data structures.  Using \colosl, it is possible to prove
programs with respect to concise specifications that only talk about
the portions of local and shared state actually accessed by each
thread in the program, dubbed their \emph{subjective states}.  One may
then reuse these specifications in the context of any larger local
state (as is standard in separation logic~\cite{rey02}), and,
crucially for compositional reasoning about concurrent programs, any
larger shared state. The subjective states of different threads in a
program are allowed to overlap arbitrarily, ensuring maximum
reusability of proofs.


Achieving a greater degree of compositional reasoning supposes that we
change the way the shared state is described logically.  In \colosl,
shared state assertions are of the form $\shared P I$ where $P$ is a
formula and $I$ a set of \emph{actions} (or interferences). Such an
assertion means that $P$ is true of \emph{some part} of the shared
state, and that this part of the shared state is subject to
interferences in $I$ (which describes those actions mutating the
shared state that are to be expected from the environment and those
that the current thread is allowed to perform).

Four novel reasoning principles allow us to reason about this novel
interpretation of shared state assertions in \colosl. First, since we
only care about some unspecified fraction of the shared state, we can
always  zoom in on a fraction of the subjective state:
\begin{align*}
  \shared{P * Q}{I} &\vdash \shared{P}{I}  \tag{\textsc{Forget}}
\end{align*}
TODO: why this is cool.

Second, the $*$ connective of separation logic splits the shared state
into \emph{overlapping} subjective views (EXPLAIN OVERLAP):
\begin{align*}
  \shared{P}{I_1} * \shared{Q}{I_2} &\vdash \shared{P \sepish Q}{I_1 \cup I_2} \tag{\textsc{Merge}}
\end{align*}
[TODO: new: usually $I$ never changes.]

In combination with the usual law of parallel composition of
separation logic~\cite{csl-tcs}, this yields a powerful mechanism to
distribute the shared state between several threads:
\[
\infrule{Parallel}
        {\hoare{P_1}{\mathbb{P}_1}{Q_1}\\
          \hoare{P_2}{\mathbb{P}_2}{Q_2}}
        {\hoare{P_1 * P_2}{\mathbb{P}_1 || \mathbb{P}_2}{Q_1 * Q_2}}
        {}
\]


Third, a set of actions $I$ can be exchanged for any other $I'$ that
has the same projected effect on the subjective state $P$; when that
is the case, we write $ I \weakenIb{P} I'$ and say that the actions
are \emph{shifted}:
\begin{align*}
  I \weakenIb{P} I'
  &\text{ implies }
  \shared{P}{I} \Vvdash \shared{P}{I'}
  \tag{\textsc{Shift}}
\end{align*}
TODO: new: usually $I$ never changes. Semantic equivalence between $I$
and $I'$ partially axiomatised; enough for our examples.

Finally, private resources can be incorporated into the shared state
together with new actions:
\begin{align}
  P \containI I
  &\text{ implies }
  P \Vvdash
  \exsts{\capAss{1}, \capAss{2}} \capAss{1} * \shared{P *
    \capAss{2}}{I}
  \tag{\textsc{Extend}}
\end{align}
The side condition $P \containI I$ ensures that the mutations
performed by actions in $I$ are confined to $P$.  TODO: why this is
new (new actions may refer to existing shared state).



Contributions:
\begin{enumerate}
\item
  We introduce the program logic \colosl, which achieves fully
  compositional rely/guarantee style reasoning for concurrent
  programs. Our soundness result is parametric in the underlying
  programming language and separation algebra of states.
\item
  We evaluate \colosl on a range of key examples that were challenging
  for previous work. We are able to derive natural and concise specs,
  circumscribed to the part of the shared state accessed by each
  thread, even in the case of threads operating on subparts of data
  structures with unspecified sharing (directed acyclic graphs in our
  example).
\item
  We demonstrate the generality of \colosl by deriving the main
  reasoning principles of existing program logics for concurrency
  within our framework.
\end{enumerate}

\paragraph{Related work}
Reasoning about concurrency: major advances since Owicki-Gries:
compositional reasoning, from the seminal paper by Jones~\cite{rg} on
the one hand, and O'Hearn~\cite{csl-orig,csl-tcs} on the other hand,
later fused together in a series of increasingly compositional
formalisms such as RGSep~\cite{viktor-marriage}, Local RG~\cite{lrg},
deny-guarantee~\cite{dg}, and the CAP
family~\cite{cap-ecoop10,icap,tada}, to name a few.

[have pictures of the shared state throughout the ages, like in the
  York talk?]

In some way, this mirrors the progression w.r.t.\ compositionality of
reasoning about sequential programs with resources/heap: first Hoare
logic~\cite{hoarelogic} carry around the whole of the heap. Major
advances in compositionality for pointer programs via separation
logic~\cite{seplog}, which enables \emph{local} proofs where
irrelevant and disjoint pieces of state are \emph{framed}, recently
extended to programs manipulating data structures with intrinsic
sharing, such as graphs~\cite{ramification}.


\paragraph{Outline}
\S\ref{sec:intuition} motivates the four main mechanisms of \colosl on
an illustrative example; \S\ref{sec:logic} formalises our program
logic and the semantics of state assertions; \S\ref{sec:examples}
sketches the proofs of more examples (TODO: more awesome);
\S\ref{sec:soundness} presents our soundness result, and
\S\ref{sec:conclusion} contrast with related work and concludes.
