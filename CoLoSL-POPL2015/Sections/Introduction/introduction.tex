\section{Introduction}

%% Jules: be more general than just program logics? Model-checking,
%% etc.?


%\pg{This is what we should answer by the introduction: 
%What's the problem with the current state of the art?
%What is our solution?
%Why is the problem we're solving hard?
%How do we solve the challenges?
%Why do we do better than existing work?
%What are the lessons learnt from the paper?}


A key difficulty in verifying properties of shared-memory concurrent
programs is to be able to reason compositionally about each thread in
isolation, even though in reality the correctness of the whole system
is the collaborative result of intricately intertwined actions of the
threads.  Such compositional reasoning is essential for verifying
large concurrent systems, library code and more generally incomplete
programs, and for replicating a programmer's intuition about why their
implementations are correct.

Separation logic reasoning~\cite{seplog,csl-tcs} has
been used to demonstrate the importance of reasoning about only the
portions of the local machine state (the resource) actually accessed
by programs. Recently, several techniques have been built on top of
separation logic, such as those related to CAP
reasoning~\cite{lotsofthem}, which have made substantial advances in
fine-grained compositional reasoning about complex algorithms which
access the global shared state.  However, the rigidity of existing
techniques forces us to work with {\em static} global shared state,
such as a shared data structure for representing sets, even though a
thread might only access just a small part of it.  This global state
must be robust with respect to all possible interferences from the
environment, {even on parts of the shared state not actually accessed
  by the current thread}.  Intuitively, compositionality calls for
reasoning that only refers to the part of the global shared state actually
accessed by each thread.  Exisiting reasoning techniques are too rigid
to achieve this. 



This paper introduces the program logic \colosl, where  threads access one
global shared state and are verified with respect to their
\emph{subjective views} of this state. Each
subjective (personalised) view  is an assertion which provides a thread-specific description
of part of the shared state. It describes 
the partial shared resource necessary
for the thread to run and the  thread-specific interference 
describing how the thread and the environment can effect this partial
shared resource. Subjective views may arbitrarily overlap with each
other, and subtlely expand and contract  to the natural resources and the 
interference required by  the current thread. 
This flexibility
provides truly compositional reasoning  for shared-memory concurrency.


%This paper presents the program logic \colosl, which achieves
%compositional reasoning for concurrent programs by enabling
%\emph{subjective} views of the shared state. Subjective views may be
%composed arbitrarily and manipulated so as to retain only the portions
%of local and shared state actually accessed by each thread in the
%program, and to consider only the interferences relevant to that piece
%of shared state. \colosl achieves a greater degree of compositionality
%than existing work by enabling \emph{finer-grained} sharing: the
%program logic can consider that threads share only what is relevant to
%their function, a key ingredient of compositionality.



%Driven by the ever-increasing need for concurrency in software,
%%  spurred by recent hardware developments, 
%program logics for shared-memory concurrency have progressed towards
%the twin ideals of fine-grain reasoning and compositionality. The
%former enables elegant proof techniques about increasingly subtle
%concurrency idioms~\cite{vv06popl,vv07msc,todo}, while the latter
%allows programs to be proved component-wise and their proofs to be
%reusable as-is in any client
%program~\cite{csl-tcs,cap-ecoop10,icap}.
%%  Central to these compositional verification frameworks is the
%% formalism used to describe both the state shared between program
%% threads and the possible interference on that state.
%%
%% dating back from Owicki~\cite{owicki}, and later Jones who
%% integrated the notion of interference in the logic
%% itself~\cite{rg}.
%% 

%% Consider for instance
%% a program where threads operate on subgraphs of a global graph.


%%  dubbed their \emph{subjective states}.  One may then reuse these
%%   specifications in the context of any larger local state (as is
%%   standard in separation logic~\cite{rey02}), and, crucially for
%%   compositional reasoning about concurrent programs, any larger
%%   shared state. The subjective states of different threads in a
%%   program are allowed to overlap arbitrarily, ensuring maximum
%%   reusability of proofs.


%\colosl assertions comprise standard assertions from the
%separation-logic literature~\cite{rey02,ramification}, plus {\em
%  subjective views} and {\em capabilities}. 

A subjective view
$\shared{P} I$ comprises an  assertion $P$ which gives {\em partial}
information about the global shared state and interference assertion
$I$ which describes how this partial shared state may be changed by
the thread or the environment. The interference assertion $I$ declares
actions of the form $\token a : P' \swap Q$, with capability assertion
$[\token a]$ providing a thread with permission to use this action
$\token a$. Subjective views expand or contract depending on the resource required
by the thread, which has subtle consequences for the logical
reasoning. In particular, it is possible to strengthen actions in $I$
to record  some global information known to the subjective view.
Then, the view can be weakened to just 
the resources and possibly strengthened
actions appropriate to  the thread. The strengthened actions retain  some
global information despite the weakened view. 


We introduce several novel reasoning principles which allow us to expand and
contract subjective views. Subjective views can always be duplicated
using the \copyRule  principle:
\begin{align*}
  \label{eq:split}
  \shared{P}{I} &=> \shared{P}{I} * \shared P I \tag{\copyRule}
\end{align*}
In combination with the usual law of parallel composition of
separation logic~\cite{csl-tcs}, this yields a powerful mechanism to
distribute the shared state between several threads:
\[
\infer[\parRule]
        {\hoare{P_1 * P_2}{\mathbb{P}_1 || \mathbb{P}_2}{Q_1 * Q_2}}
        {\hoare{P_1}{\mathbb{P}_1}{Q_1} &
          \hoare{P_2}{\mathbb{P}_2}{Q_2}}
\]
The subjective views for the individual subthreads will be typically
too strong, describing resource that is not necessary being used by
the thread. However, before  weakening the view, it will sometimes 
be useful to strengthen some of the actions in the interference assertion
to preserve global knowledge. We achieve this using the \shiftRule 
principle:
\begin{align*}
  \label{eq:shift}
  I \weakenIb{P} I'
  &\text{ implies }
  \shared{P}{I} ===> \shared{P}{I'}
  \tag{\shiftRule}
\end{align*}
This principle states that an interference assersion $I$ can be
exchanged for any other interference assertion $I'$ that has the same projected
effect on the subjective state $P$. When this is the case, we write $
I \weakenIb{P} I'$ and say that the actions have been  \emph{shifted}. It can
be used to strengthen actions given by the combined global knowledge
arising form the combination of $I$ and $P$. It can also be used to forget actions
which are not relevant to $P$. This \shiftRule 
principle describes a flexible  behaviour of interference  which is 
in marked  contrast with most existing formalisms, where 
interference is fixed throughout the verification of the whole
program.


Now, with a possibly strengthened interference assertion, we are able 
to use the 
 FORGET principle to zoom  on the resource required by the current thread:
\begin{align*}
  \label{eq:forget}
  \shared{P * Q}{I} &=> \shared{P}{I}  \tag{\forgetRule}
\end{align*}
We  can again use the \shiftRule  principle to, this time, forget some of
the actions not now relevant to the subjective view. Finally, as usual, we must
{\em stabalise} the assertion of the subjective view,  so that the view is
robust with respect to interferences from the environment. If the
knowledge given by the 
combination of the assertion $P$ and the interference assertion $I$ is too weak, then this stabilisation results in
a precondition that is too weak. 


These reasoning principles enable us to provide subjective views for
the subthreads which are just right. 
We can proceed to verify the subthreads, knowing that their
subjective views describe personalised preconditions which only  
describe the resource relevant to the individual threads. The
resulting postconditions naturally describe
overlapping parts of the shared state, which are then joined together
using the disjoint concurrency rule and the \mergeRule principle:
\begin{align*}
  \label{eq:merge}
  \shared{P}{I_1} * \shared{Q}{I_2} &=> \shared{P \sepish Q}{I_1 \cup I_2} \tag{\mergeRule}
\end{align*}
The assertion $P ** Q$ 
describes the overlapping of states arising from $P$ and $Q$, using the sepish
connective introduced in~\cite{gareth,jules}. 
The new
interference relation is simply the union of previous
interferences. Finally, we can use the \shiftRule principle to simplify the
interference assertion now that we are back with a larger subjective view.

\pgcomment{In section 3, we should introduce the sepish conenctive
  carefully, people are likely not to know it.} 


The last principle to mention is how subjective views are
created
using the \extendRule  principle: 

\begin{align}
  \label{eq:extend}
  P \containI I
  &\text{ implies }
  P ===>
  \exsts{\vec{\token{a}}, \vec{\token{b}}}[\vec{\token{a}}] * \shared{P *
   [\vec{\token{b}}]}{I}
  \tag{\extendRule}
\end{align}
\pgcomment{Jules, please fix this notation  for as and bs. I think
  this vec notation is quite good, but we shoud probably define box of
  vec 
  somewhere or do something different using iterative star or
  something.} 
%
The side condition $P \containI I$ ensures that the actions associated
with interference assertion $I$ are confined to $P$. The existential
quantification of capabilities is to ensure the \emph{freshness} of
the generated capabilities. 
The main novelty of this rule is the fact
that new actions may refer to existing shared state, as we shall see  in
our set example discussed below. Notice that we  do not require a corresponding principle for destroying
subjective views since this is achieved by the FORGET principle. 




\pgcomment{Azalea, make sure you highlight 
' The main novelty of this rule is the fact
that new actions may refer to existing shared state, as we will see in
our examples.' in the set example ins ection 5.} 






With these reasoning principles, we are able to expand and contract
subjective views  to provide just the resource required by a thread.
In essence, we provide  frame-like reasoning for  subjective  views. 
The interaction between  these principles is
subtle. We are able to forget resource of the subjective assertion $P$
 whilst at the same time adapting the interference relation $I$  to
capture global constraints associated with that lost resource. In 
\S\ref{sec:intuition}, we 
demonstrate this using 
a variant of the famous  token ring mutual exclusion algorithm [? ],
introduced by Dijkstra to illustrate distributed, global knowledge between
threads. The point is that, as well as local knowledge
about how the threads behave, the programmer also has global knowledge
about how the threads constrain each other. Our reasoning captures the
spirit of reasoning locally about individual threads at the same time
as retaining global knowledge about the overall thread behaviour. It
is a good  introductory example, as it uses all the highlighted \colosl
reasoning principles in a fundamental way. 

\pgcomment{In next para, I wasn't quite sure what to say so I just did
  adaquate `blah'.}

In \S\ref{sec:sectionthree}, we interpret \colosl assertions
 over a particular domain which includes: thread-local
state exclusively visible to the thread; {\em one} global shared state
accessible by all thread (in contrast with the CAP approach); and {\em
  action models} describing how the global state can be updated.  In
\S\ref{sec:sectionfour}, we define the rely and guarantee conditions
of each thread is in terms of their action models, and provide a
sketch of soundness. In particular, the
reasoning principles highlighted above are simple semantic
consequences of our generic model of program states.  The full
technical details can
be found 
in the accompanying technical report~\cite{.}.



In \S\ref{sec:examples}, we 
study two further, more challenging examples. We verify an concurrent algorithm for
computing the spanning tree of a graph. This algorithm uses threads
spawned on arbitrary overlapping graphs. We demonstrate that the flexible, overlapping
subjective views of \colosl are just what we need. 
We also verify a concurrent set module
implemented using a hand-over-hand list-locking
algorithm. 
%Interestingly, this
%example uses the \extendRule principle   to the create new actions which refer to existing
%state in a fundamental way. 
The flexibility of \colosl's subjective views  leads to a
considerably simpler correctness proof   compared to that of CAP. 

\julescomment{Our soundness theorem doesn't include leak-freedom!
  That's a fundamental limitation of \colosl. Could be overcome by
  keeping a handle on all resources, that is only lost upon
  disposal. (comment repeated in soundness section). PG: ok to ignore
  this for now.}

