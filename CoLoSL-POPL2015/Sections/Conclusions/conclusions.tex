\section{Conclusions}
\label{sec:conclusion}

Well, that was awesome.

Future work: because we are able to specify shared state and
interferences locally, there is hope that \colosl can be adapted to
reason about distributed systems as well, as in our motivating example
of \S\ref{sec:intuition}.

\julescomment{This is a brain dump. Doesn't have to make it in the
  submitted version.}

One of the new paradigms of \colosl is the ability to manipulate the
interference relation, via \emph{shifting}. The focus of this paper
has largely been on manipulations of the spatial components of
subjective views, and the treatment of shifting has been left at the
minimum we needed to cover our range of examples. This opens the way
to more fine-grained reasoning about interferences. For instance, a
process who never receives a capability could safely over-approximate
the corresponding interference. Conversely, a process who never gives
a capability away could safely under-approximate the corresponding
interference relation. In \colosl, this is not yet allowed, as
interferences can only be shifted to equivalent ones, the only
exception being the ability to forget actions that have no visible
effect on the current subjective state. One could imagine tracking
which process is allowed to own which capability to be more flexible
in this instance. (if that can be useful at all?)
