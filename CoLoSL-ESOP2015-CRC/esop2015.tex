\documentclass[runningheads,a4paper]{llncs}

\usepackage{azalea}
\setcounter{tocdepth}{3}

\urldef{\emails}\path|{azalea, j.villard, pg}@doc.ic.ac.uk|    
%\newcommand{\keywords}[1]{\par\addvspace\baselineskip
%\noindent\keywordname\enspace\ignorespaces#1}

\begin{document}
%
%\title{CoLoSL: \underline{Co}ncurrent \underline{Lo}cal \underline{S}ubjective \underline{L}ogic}
\title{CoLoSL: Concurrent Local Subjective Logic}
\author{Azalea Raad\and Jules Villard\and Philippa Gardner}
\institute{Imperial College London\\
\emails}

\maketitle

\begin{abstract}
A key difficulty in verifying shared-memory concurrent programs is
reasoning compositionally about each thread in isolation. Existing
verification techniques for fine-grained concurrency require 
reasoning about static global shared resource, impeding compositionality.  This
paper introduces the program logic of \colosl, where each thread is
verified with respect to its subjective view of the global
shared state.
This subjective view describes only that part of the global shared resource accessed by the
thread. Subjective views may arbitrarily overlap with each other, and
expand and contract depending on the resource required by the thread.
This flexibility provides truly compositional proofs for shared-memory
concurrency, which we demonstrate on a range of examples including a
concurrent computation of a spanning tree of a graph.
\end{abstract}

\allowdisplaybreaks
\input Sections/Introduction/introduction.tex
\input Sections/Intuition/intuition.tex
\input Sections/Model/model.tex
\input Sections/Examples/examples.tex
\input Sections/Conclusion/conclusion.tex
\bibliographystyle{abbrv}
\bibliography{biblio}
\appendix{
\appendix{
\chapter{Auxiliary Lemmata}\label{app:lemmata}
	\begin{lemma}[\forgetRule-Closure]\label{lem:forget-closure}
%
\[
\begin{array}{l}
	\for{s_1, s_2, r \in \LStates} \for{\ \lmod, \lmod', \gmod \in \AMods}\for{n \in \Nats}\\
	\hspace*{0.5cm} \extendsAMUpto{\lmod, \gmod}{n}{s_1 \composeL s_2}{r}{\lmod'} \implies 
									\extendsAMUpto{\lmod, \gmod}{n}{s_1}{s_2 \composeL r}{\lmod'}
\end{array}
\]
%
\begin{proof} By induction on number of steps $n$.

%\noindent Pick an arbitrary $s_1, s_2, r \in \LStates, \lmod, \lmod', \gmod \in \AMods$.\\
\noindent\textbf{Base case}\\
\textbf{RTS. }
%
\[
\begin{array}{l}
	\for{s_1, s_2, r \in \LStates} \for{\ \lmod, \lmod', \gmod \in \AMods}\\
	\hspace*{0.5cm} \extendsAMUpto{\lmod, \gmod}{0}{s_1 \composeL s_2}{r}{\lmod'} \implies 
									\extendsAMUpto{\lmod, \gmod}{0}{s_1}{s_2 \composeL r}{\lmod'}
\end{array}
\]
%
Pick an arbitrary $s_1, s_2, r \in \LStates, \lmod, \lmod', \gmod \in \AMods$ such that $\extendsAMUpto{\lmod, \gmod}{0}{s_1 \composeL s_2}{r}{\lmod'}$.
Since $\extendsAMUpto{\lmod, \gmod}{0}{s_1 \composeL s_2}{r}{\lmod'}$, we know $\lmod' \subseteq \lmod$  and consequently by definition of $\downarrow_0$ we have $\extendsAMUpto{\lmod, \gmod}{0}{s_1}{s_2 \composeL r}{\lmod'}$.\\

\noindent\textbf{Inductive Step} Pick an arbitrary $n \in \Nats$, such that
%
\begin{equation}
	\tag{I.H.}
%
	\begin{array}{l}
		\for{s_1, s_2, r \in \LStates} \for{\ \lmod, \lmod', \gmod \in \AMods}\\
		\hspace*{0.5cm} \extendsAMUpto{\lmod, \gmod}{(n-1)}{s_1 \composeL s_2}{r}{\lmod'} \implies 
										\extendsAMUpto{\lmod, \gmod}{(n-1)}{s_1}{s_2 \composeL r}{\lmod'}
	\end{array}
%
\label{LCS:IH}
\end{equation}
%
\textbf{RTS. }
%
\[
\begin{array}{l}
	\for{s_1, s_2, r \in \LStates} \for{\ \lmod, \lmod', \gmod \in \AMods}\\
	\hspace*{0.5cm} \extendsAMUpto{\lmod, \gmod}{n}{s_1 \composeL s_2}{r}{\lmod'} \implies 
									\extendsAMUpto{\lmod, \gmod}{n}{s_1}{s_2 \composeL r}{\lmod'}
\end{array}
\]
%
Pick an arbitrary $s_1, s_2, r \in \LStates, \lmod, \lmod', \gmod \in \AMods$ such that 
%
\begin{align}
	& \extendsAMUpto{\lmod, \gmod}{n}{s_1 \composeL s_2}{r}{\lmod'} \label{LCS:Ass1}
\end{align}
%
Show
%
\begin{align}
	& 
	\V{\ca{}}  \V{a\in \lmod'(\ca{})} \nonumber \\
  &\quad (\m{potential}(a, s_1 \composeL s_2\composeL r) /| \m{visible}(a, s_1)=> \nonumber\\
  & \qquad \qquad\for{(s', r') \in a[s_1, s_2 \composeL r]} \extendsAMUpto{\lmod, \gmod}{(n-1)}{s'}{r'}{\lmod'}) \label{LCS:Goal0}\\
%  
  &\quad (\m{potential}(a, s_1 \composeL s_2\composeL r) /| \neg\m{visible}(a, s_1)=> \nonumber\\
  & \qquad \qquad \extendsAMUpto{\lmod, \gmod}{(n-1)}{s_1}{a[s_1 \composeL s_2 \composeL r] - s_1}{\lmod'}) \label{LCS:Goal1}\\
%    
  &\quad\land \m{enabled}(a,s_1 \composeL s_2 \composeL r)
  => (s_1 \composeL s_2 \composeL r,
  a[s_1 \composeL s_2 \composeL r])\in \gmod(\ca{}))
  /|\null \label{LCS:Goal2}\\
%  
  &\V{\ca{}}\V{a\in \lmod(\ca{})}
  \m{potential}(a,s_1 \composeL s_2 \composeL r) =>\null \nonumber \\
  &\ \m{reflected}(a,s_1 \composeL s_2 \composeL r,\lmod'(\ca{})) |/\null \nonumber \\
%  
  &\ \neg\m{visible}(a,s_1) /| \exsts{r'} a[s_1 \composeL s_2 \composeL r] = s_1 \composeL r' /| \extendsAMUpto{\lmod, \gmod}{(n-1)}{s_1}{r'}{\lmod'}  \label{LCS:Goal3}
%
\end{align}
%
\textbf{RTS. (\ref{LCS:Goal0})}\\
Pick an arbitrary $\ca{}$, $a \in \lmod'(\ca{})$ and $(s', r')$ such that 
\begin{align}
	\m{potential}(a, s_1 \composeL s_2 \composeL r) /| \m{visible}(a, s_1) \label{LCS:Ass2}\\
	(s', r') \in a[s_1, s_2 \composeL r] \label{LCS:Ass3}
\end{align}
%
Then from the definition of $a[s_1, s_2 \composeL r]$, the cross-split property and since $\m{visible}(a, s_1)$, we have:
%
\begin{align}
	\exsts{ps_1 > \unitL, ps_2, pr, s'_1, s'_2, r'}\ & s_1 = ps_1 \composeL  s'_1 \label{LCS:Ass4-1}\\
	& s_2 = ps_2 \composeL s'_2 \label{LCS:Ass4-2}\\
	& r = pr \composeL r' \label{LCS:Ass4-3}\\
	& \m{fst}(\updateFP{a}) = ps_1 \composeL ps_2 \composeL pr \label{LCS:Ass4-4}\\
	& \m{snd}(\updateFP{a}) \compatible s'_1 \composeL s'_2 \composeL r'  \label{LCS:Ass4-5}\\
	& s' = \m{snd}(\updateFP{a}) \composeL s'_1 \label{LCS:Ass4-6}\\
	& r' = s'_2 \composeL r' \label{LCS:Ass4-7}
\end{align}
% 
Then from (\ref{LCS:Ass4-1})-(\ref{LCS:Ass4-5}) and by definition of $a[s_1 \composeL s_2, r]$ we know $\m{visible}(a, s_1 \composeL s_2) /| (\m{snd}(\updateFP{a}) \composeL s'_1 \composeL s'_2, r') \in a[s_1 \composeL s_2, r]$. Consequently from (\ref{LCS:Ass1}) and (\ref{LCS:Ass2}) we have:
%
\begin{equation}
	\extendsAMUpto{\lmod, \gmod}{(n-1)}{\m{snd}(\updateFP{a} \composeL s'_1 \composeL s'_2}{r'}{\lmod'} \nonumber
\end{equation}
%
and thus from (\ref{LCS:IH})
%
\begin{align}
	\extendsAMUpto{\lmod, \gmod}{(n-1)}{\m{snd}(\updateFP{a} \composeL s'_1}{s'_2 \composeL r'}{\lmod'} \label{LCS:Ass5}
\end{align}
% 
Finally, from (\ref{LCS:Ass4-6}), (\ref{LCS:Ass4-7}) and (\ref{LCS:Ass5}) we have: 
%
\begin{align}
	\extendsAMUpto{\lmod, \gmod}{(n-1)}{s'}{r'}{\lmod'} \nonumber
\end{align}
% 
as required.

%
%
%
\noindent\textbf{RTS. (\ref{LCS:Goal1})}\\
Pick an arbitrary $\ca{}$, $a \in \lmod'(\ca{})$ and $s''$ such that 
\begin{align}
	\m{potential}(a, s_1 \composeL s_2 \composeL r) /| \neg\m{visible}(a, s_1) \label{LCS:ass2}\\
	s'' = a[s_1 \composeL s_2 \composeL r] \label{LCS:ass3}
\end{align}
%
There are two cases to consider:\\
\textbf{Case 1. }$\m{visible}(a, s_1 \composeL s_2, r)$\\
Since $a[s_1 \composeL s_2 \composeL r]$ is defined (\ref{LCS:ass2}), we know $\exsts{(s', r')} \in a[s_1 \composeL s_2, r]$. Consequently, from the definition of $a[s_1 \composeL s_2, r]$, the cross-split property and since $\neg\m{visible}(a, s_1)$ (\ref{LCS:ass2}) and $\m{visible}(a, s_2)$ (Assumption of case 1.) and from (\ref{LCS:ass3}) we have:
%
\begin{align}
	\exsts{ps_2, pr, s'_2, r'}\ & s_2 = ps_2 \composeL s'_2 \label{LCS:ass4-1}\\
	& r = pr \composeL r' \label{LCS:ass4-2}\\
	& \m{fst}(\updateFP{a}) = ps_2 \composeL pr \label{LCS:ass4-3}\\
	& \m{snd}(\updateFP{a}) \compatible s_1 \composeL s'_2 \composeL r'  \label{LCS:ass4-4}\\
	& s' = \m{snd}(\updateFP{a}) \composeL s_1 \composeL  s'_2 \label{LCS:ass4-5}\\
	& s'' = \m{snd}(\updateFP{a}) \composeL s_1 \composeL  s'_2 \composeL r' \label{LCS:ass4-6}
\end{align}
% 
Consequently, from (\ref{LCS:ass2}), (\ref{LCS:Ass1}) and assumption of case 1. we have:
%
\begin{align*}
	\extendsAMUpto{\lmod, \gmod}{(n-1)}{s'}{r'}{\lmod'}
\end{align*}
% 
and consequently, from (\ref{LCS:ass4-5}) and (\ref{LCS:IH})
%
\begin{align*}
	\extendsAMUpto{\lmod, \gmod}{(n-1)}{s_1}{\m{snd}(\updateFP{a}) \composeL  s'_2 \composeL r'}{\lmod'}
\end{align*}
%
and finally, from (\ref{LCS:ass4-6}), (\ref{LCS:ass3}) and the definition of $a[s_1 \composeL s_2 \composeL r] - s_1$, we have:
%
\begin{align*}
	\extendsAMUpto{\lmod, \gmod}{(n-1)}{s_1}{a[s_1 \composeL s_2 \composeL r] - s_1}{\lmod'}
\end{align*}
% 
as required.\\
%
\noindent\textbf{Case 2. }$\neg\m{visible}(a, s_1 \composeL s_2, r)$\\
From the assumption of case 2, (\ref{LCS:ass2}) and (\ref{LCS:Ass1}) we have:
%
\begin{align*}
	\extendsAMUpto{\lmod, \gmod}{(n-1)}{s_1 \composeL s_2}{a[s_1 \composeL s_2 \composeL r] - (s_1 \composeL s_2)}{\lmod'}
\end{align*}
% 
and consequently from (\ref{LCS:IH})
%
\begin{align*}
	\extendsAMUpto{\lmod, \gmod}{(n-1)}{s_1}{a[s_1 \composeL s_2 \composeL r] - (s_1 \composeL s_2) \composeL s_2}{\lmod'}
\end{align*}
% 
that is, 
%
\begin{align*}
	\extendsAMUpto{\lmod, \gmod}{(n-1)}{s_1}{a[s_1 \composeL s_2 \composeL r] - s_1}{\lmod'}
\end{align*}
% 
as required.\\
%%
%%
%%



\noindent\textbf{RTS. (\ref{LCS:Goal2})}\\
Pick an arbitrary $\ca{}$ and $a \in \lmod'(\ca{})$ such that 
\begin{align}
	\m{enabled}(a, s_1 \composeL s_2 \composeL r) \label{LCS:Ass9}
\end{align}
%
Then from (\ref{LCS:Ass1}) and (\ref{LCS:Ass9}) we have:
%
\begin{align}
	(s_1 \composeL s_2 \composeL r, a[s_1 \composeL s_2 \composeL r]) \in \gmod(\ca{}) \nonumber
\end{align}
% 
as required.

\noindent\textbf{RTS. (\ref{LCS:Goal3})}\\
Pick an arbitrary $\ca{}$ and $a \in \lmod(\ca{})$ such that 
\begin{align}
	\m{potential}(a, s_1 \composeL s_2 \composeL r) \label{LCS:Ass12}
\end{align}
%(\ref{LCS:Ass})
Then from (\ref{LCS:Ass1}) we have: 
%
\begin{align*}
  &\m{reflected}(a,s_1 \composeL s_2 \composeL r,\lmod'(\ca{})) |/\null \nonumber \\
%  
  &\neg\m{visible}(a,s_1 \composeL s_2) /| \exsts{r'} a[s_1 \composeL s_2 \composeL r] = s_1 \composeL s_2 \composeL r' /| \extendsAMUpto{\lmod, \gmod}{(n-1)}{s_1 \composeL s_2}{r'}{\lmod'}  \nonumber
\end{align*}
%
and consequently from (\ref{LCS:IH})
%
\begin{align*}
  &\m{reflected}(a,s_1 \composeL s_2 \composeL r,\lmod'(\ca{})) |/\null \nonumber \\
%  
  &\neg\m{visible}(a,s_1 \composeL s_2) /| \exsts{r'} a[s_1 \composeL s_2 \composeL r] = s_1 \composeL s_2 \composeL r' /| \extendsAMUpto{\lmod, \gmod}{(n-1)}{s_1}{s_2 \composeL r'}{\lmod'}  \nonumber
\end{align*}
%
%Then from (\ref{LCS:Ass1}), the definition of $a[s_1 \composeL s_2, r]$, the cross-split property and the cancellativity of logical states separation algebras we have:
%%
%\begin{align}
%&\begin{array}{l}
%	\m{reflected}(a, s_1 \composeL s_2 \composeL r, \lmod'(\ca{}) \lor\\
%	(\neg\m{visible}(a, s_1 \composeL s_2) \land \\
%	\begin{array}{@{} l l @{}}
%		\exsts{ps_1, ps_2, s'_1, s'_2, pr, r', s''} & s_1 = ps_1 \composeL s'_1\\
%		& s_2 = ps_2 \composeL s'_2\\
%		& r = pr \composeL r'\\
%		& \m{fst}(\updateFP{a}) = ps_1 \composeL ps_2 \composeL pr\\
%		& \m{snd}(\updateFP{a}) = ps_1 \composeL ps_2 \composeL s'' \land\\
%		& \extendsAMUpto{\lmod, \gmod}{(n-1)}{s_1 \composeL s_2}{s'' \composeL r'}{\lmod'}
%	\end{array}\\
%	)
%\end{array}
%\label{LCS:Ass13}
%\end{align}
%% 
%Consequently from (\ref{LCS:IH}) we have: 
%\begin{align}
%&\begin{array}{l}
%	\m{reflected}(a, s_1 \composeL s_2 \composeL r, \lmod'(\ca{}) \lor\\
%	(\neg\m{visible}(a, s_1 \composeL s_2) \land \\
%	\begin{array}{@{} l l @{}}
%		\exsts{ps_1, ps_2, s'_1, s'_2, pr, r', s''} & s_1 = ps_1 \composeL s'_1\\
%		& s_2 = ps_2 \composeL s'_2\\
%		& r = pr \composeL r'\\
%		& \m{fst}(\updateFP{a}) = ps_1 \composeL ps_2 \composeL pr\\
%		& \m{snd}(\updateFP{a}) \composeL s'_1= s_1 \composeL ps_2 \composeL s'' \land\\
%		& \extendsAMUpto{\lmod, \gmod}{(n-1)}{s_1}{ps_2 \composeL s'' \composeL s'_2 \composeL r'}{\lmod'}
%	\end{array}\\
%	)
%\end{array}
%\nonumber
%\end{align}
%% 
%%
%and by definition of $a[s_1, s_2 \composeL r]$ we have:
%\begin{align}
%&\begin{array}{l}
%	\m{reflected}(a, s_1 \composeL s_2 \composeL r, \lmod'(\ca{}) \lor\\
%	(\neg\m{visible}(a, s_1 \composeL s_2) \land \\
%	\begin{array}{@{} l l @{}}
%		\exsts{s'''} & \m{fst}(a[s_1, s_2 \composeL r]) = s_1 \composeL s''' \land\\
%		& \extendsAMUpto{\lmod, \gmod}{(n-1)}{s_1}{s''' \composeL \m{snd}(a[s_1, s_2 \composeL r])}{\lmod'}
%	\end{array}\\
%	)
%\end{array}
%\label{LCS:Ass14}
%\end{align}
Finally from the definition of $\m{visible}$ and by \lem~\ref{lem:disjointByOrder} we can rewrite the above as: 
%
%
\begin{align*}
  &\m{reflected}(a,s_1 \composeL s_2 \composeL r,\lmod'(\ca{})) |/\null \nonumber \\
%  
  &\neg\m{visible}(a,s_1) /| \exsts{r'} a[s_1 \composeL s_2 \composeL r] = s_1 \composeL r' /| \extendsAMUpto{\lmod, \gmod}{(n-1)}{s_1}{r'}{\lmod'}  \nonumber
\end{align*}
%
%
%\begin{align}
%&\begin{array}{l}
%	\m{reflected}(a, s_1 \composeL s_2 \composeL r, \lmod'(\ca{}) \lor\\
%	(\neg\m{visible}(a, s_1) \land \\
%	\begin{array}{@{} l l @{}}
%		\exsts{s'''} & \m{fst}(a[s_1, s_2 \composeL r]) = s_1 \composeL s''' \land\\
%		& \extendsAMUpto{\lmod, \gmod}{(n-1)}{s_1}{s''' \composeL \m{snd}(a[s_1, s_2 \composeL r])}{\lmod'}
%	\end{array}\\
%	)
%\end{array}
%\nonumber
%\end{align}
%%
as required.
%
\end{proof}
\end{lemma}
%
%
	\begin{lemma}[\mergeRule-Closure]\label{lem:merge-closure}
For all $\lmod, \lmod_1, \lmod_2, \gmod \in \AMods$ and $s_p, s_c, s_q, r \in \LStates$,
%
\[
\begin{array}{l}
	\extendsAM{\lmod, \gmod}{s_p \composeL s_c}{s_q \composeL r}{\lmod_{1}} /| \extendsAM{\lmod, \gmod}{s_q \composeL s_c}{s_p \composeL r}{\lmod_{2}}
	\implies\\
	\hspace*{2cm} \extendsAM{\lmod, \gmod}{s_p \composeL s_c \composeL s_q}{r}{\lmod_{1} \cup \lmod_{2}}
\end{array}
\]
%
\begin{proof} Pick an arbitrary $\lmod, \lmod_1, \lmod_2, \gmod \in \AMods$ and $s_p, s_c, s_q, r \in \LStates$ such that 
%
\begin{align}
	\extendsAM{\lmod, \gmod}{s_p \composeL s_c}{s_q \composeL r}{\lmod_{1}} /| \extendsAM{\lmod, \gmod}{s_q \composeL s_c}{s_p \composeL r}{\lmod_{2}} \label{MC:Ass1}
\end{align} 
%
From the definition of $\downarrow$, it then suffices to show
%
\begin{align}
	& \lmod_1 \cup \lmod_2  \subseteq \lmod \label{MC:Goal1}\\
	& \for {n \in \Nats}  \extendsAMUpto{\lmod, \gmod}{n}{s_p \composeL s_c \composeL s_q}{r}{\lmod_{1} \cup \lmod_{2}} \label{MC:Goal2}
\end{align}
%

\noindent\textbf{RTS. (\ref{MC:Goal1})} \\
Since from (\ref{MC:Ass1}) and the definition of $\downarrow$ we have $\lmod_1 \subseteq \lmod /| \lmod_2 \subseteq \lmod$, we can thus conclude $\lmod_1 \cup \lmod_2 \subseteq \lmod$ as required. \\

\noindent\textbf{RTS. (\ref{MC:Goal2})} \\
Rather than proving (\ref{MC:Goal2}) directly, we first establish the following.
%
\begin{align}
	\for {n \in \Nats} \for{s_p, s_c, s_q, r \in \LStates} & \nonumber\\
	& \hspace{-4cm}\extendsAMUpto{\lmod, \gmod}{n}{s_p \composeL s_c}{s_q \composeL r}{\lmod_1} /| \extendsAMUpto{\lmod, \gmod}{n}{s_c \composeL s_q}{s_p \composeL r}{\lmod_2} \nonumber\\
	& \hspace{-3cm} \implies \extendsAMUpto{\lmod, \gmod}{n}{s_p \composeL s_c \composeL s_q}{r}{\lmod_1 \cup \lmod_2} \label{MC:Goal3}
\end{align}
%
We can then despatch (\ref{MC:Goal2}) from (\ref{MC:Ass1}) and (\ref{MC:Goal3}); since for an arbitrary $n \in \Nats$, from (\ref{MC:Ass1}) and the definition of $\downarrow$ we have $\extendsAMUpto{\lmod, \gmod}{n}{s_p \composeL s_c}{s_q \composeL r}{\lmod_1} /| \extendsAMUpto{\lmod, \gmod}{n}{s_c \composeL s_q}{s_p \composeL r}{\lmod_2}$ and consequently from (\ref{MC:Goal3}) we derive $\extendsAMUpto{\lmod, \gmod}{n}{s_p \composeL s_c \composeL s_q}{r}{\lmod_1 \cup \lmod_2}$ as required. \\

\noindent\textbf{RTS. (\ref{MC:Goal3})} \\
We proceed by induction on the number of steps $n$.\\

%\noindent Pick an arbitrary $s_1, s_2, r \in \LStates, \lmod, \lmod', \gmod \in \AMods$.\\
\noindent\textbf{Base case }$n=0$\\
Pick an arbitrary $s_1, s_2, r \in \LStates$. We are then required to show	$\extendsAMUpto{\lmod, \gmod}{0}{s_1}{s_2 \composeL r}{\lmod'} $ which follows trivially from the definition of $\downarrow_0$.\\

\noindent\textbf{Inductive Step} Pick an arbitrary $s_p, s_q, s_c, r \in \LStates$ and $n \in \Nats$, such that
%
\begin{align}
	& \extendsAMUpto{\lmod, \gmod}{n}{s_p \composeL s_c}{s_q \composeL r}{\lmod_{1}} \label{LMC:Ass1}\\
	& \extendsAMUpto{\lmod, \gmod}{n}{s_q \composeL s_c}{s_p \composeL r}{\lmod_{2}} \label{LMC:Ass2}\\
	& \for{s_p, s_q, s_c, r \in \LStates} \nonumber \\
	&	\quad \extendsAMUpto{\lmod, \gmod}{(n-1)}{s_p \composeL s_c}{s_q \composeL r}{\lmod_{1}} \;\land\; \extendsAMUpto{\lmod, \gmod}{(n-1)}{s_q \composeL s_c}{s_p \composeL r}{\lmod_{2}} \nonumber \\
	&	\tag{I.H.} \qquad\quad\implies  \extendsAMUpto{\lmod, \gmod}{(n-1)}{s_p \composeL s_c \composeL s_q}{r}{\lmod_{1} \cup \lmod_{2}} \label{LMC:IH}
\end{align}
%
%
\noindent\textbf{RTS.}
%
\begin{align}
	& 
	\V{\ca{}}  \V{a \in \left(\lmod_{1} \cup \lmod_{2}\right)(\ca{})} \nonumber \\
  &\quad(\m{potential}(a,s_p \composeL s_c \composeL s_q \composeL r) => \nonumber \\
  &\qquad \for{(s', r') \in a[s_p \composeL s_c \composeL s_q , r]} \extendsAMUpto{\lmod, \gmod}{(n-1)}{s'}{r'}{\lmod_{1} \cup \lmod_{2}}) /| \label{LMC:Goal1}\\
%    
  &\quad(\m{enabled}(a,s_p \composeL s_c \composeL s_q \composeL r) => \nonumber\\
  &\qquad\qquad (s_p \composeL s_c \composeL s_q  \composeL r, a[s_p \composeL s_c \composeL s_q  \composeL r])\in \gmod(\ca{})) \land\label{LMC:Goal2}\\
%  
  &\V{\ca{}}\V{a\in \lmod(\ca{})}
  \m{potential}(a,s_p \composeL s_c \composeL s_q \composeL r) =>\null \nonumber \\
  &\quad \m{reflected}(a,s_p \composeL s_c \composeL s_q \composeL r,\left(\lmod_{1} \cup \lmod_{2}\right)(\ca{})) |/\null \nonumber \\
%  
  &\quad \neg\m{visible}(a,s_p \composeL s_c \composeL s_q) /| \for{(s', r') \in a[s_p \composeL s_c \composeL s_q, r]}\extendsAMUpto{\lmod, \gmod}{(n-1)}{s'}{r'}{\lmod_{1} \cup \lmod_{2}} \label{LMC:Goal3}
\end{align}
%
%

\noindent\textbf{RTS. (\ref{LMC:Goal1})}\\
Pick an arbitrary $\ca{}$, $a = (p, q) \in \left(\lmod_{1} \cup \lmod_{2} \right)(\ca{})$ and $s', r'$ such that
\begin{align}
	& \m{potential}(a, s_p \composeL s_c \composeL s_q \composeL r) \label{LMC:Ass3}\\
	& (s', r') \in a[s_p \composeL s_c \composeL s_q, r] \label{LMC:visible-pcq}
\end{align}
%
Then from the definition of $a[s_p \composeL s_c \composeL s_q]$ and by the cross-split property we know there exists $p_p, p_c, p_q, s_p', s_c', s_q' \in \LStates$ such that :
\begin{align}
	\begin{array}{l}
		s' = s_p \composeL s_c \composeL s_q |/ \\
		\left(
		\begin{array}{l}
			(p_p > \unitL |/ p_c > \unitL |/ p_q > \unitL) /| s' = \snd{\updateFP{a}} \composeL s_p' \composeL s_c' \composeL s_q' 		\\
			/| \fst{\updateFP{a}} = p_p \composeL p_c \composeL p_q \composeL p_r \\
			/| s_p = p_p \composeL s_p' /| 
			s_c = p_c \composeL s_c' /|
			s_q = p_q \composeL s_q' /|
			r = p_r \composeL r' 
		\end{array}
		\right)
	\end{array}
	\label{LMC:ass4}
\end{align}
%
and consequently from the definitions of $a[s_p \composeL s_c, s_q \composeL r]$ and $a[s_c  \composeL s_q, s_p \composeL r]$ we have:
%
\begin{align}
\begin{array}{l}
	\left(
	\begin{array}{@{} l @{}}
		s' = s_p \composeL s_c \composeL s_q /| \\
		(s_p \composeL s_c, s_q \composeL r') \in a[s_p \composeL s_c, s_q \composeL r] /| (s_c \composeL s_q, s_p \composeL r') \in a[s_c  \composeL s_q, s_p \composeL r]
	\end{array}
	\right) \\
	|/ 
	\left(
	\begin{array}{@{} l @{}}
	 	s' = \snd{\updateFP{a}} \composeL s_p' \composeL s_c' \composeL s_q' /| \\
	 	\left(
	 	\begin{array}{@{} l @{}}
	 		\left(
	 		\begin{array}{@{} l @{}}
	 			(( p_c > \unitL |/ (p_c = \unitL /| p_p > \unitL /| p_q > \unitL))) /| \\
	 			(\snd{\updateFP{a}} \composeL s_p' \composeL s_c' 	, s_q' \composeL r') \in a[s_p \composeL s_c, s_q \composeL r] /|\\
	 			(\snd{\updateFP{a}} \composeL s_c' \composeL s_q', s_p' \composeL r') \in a[s_c  \composeL s_q, s_p \composeL r]
	 		\end{array}
	 		\right)\\
	 		|/
	 		\left(
	 		\begin{array}{@{} l@{}}
	 			p_p > \unitL /| p_c = p_q = \unitL /|\\
	 			(\snd{\updateFP{a}} \composeL s_p' \composeL s_c', s_q' \composeL r') \in a[s_p \composeL s_c, s_q \composeL r] /|\\
	 			(s_c' \composeL s_q', \snd{\updateFP{a}} \composeL s_p' \composeL r') \in a[s_c  \composeL s_q, s_p \composeL r] 
	 		\end{array}
	 		\right)\\
	 		|/
	 		\left(
	 		\begin{array}{@{} l@{}}
				p_q > \unitL /| p_c = p_p = \unitL /|\\
				(s_p' \composeL s_c', \snd{\updateFP{a}} \composeL s_q' \composeL r') \in a[s_p \composeL s_c, s_q \composeL r] /| \\
				(\snd{\updateFP{a}} \composeL s_c' \composeL s_q', s_p' \composeL r') \in a[s_c  \composeL s_q, s_p \composeL r]
	 		\end{array}
	 		\right)
	 	\end{array}
	 	\right)
	\end{array}
	\right)
\end{array}
\label{LMC:Ass5}
\end{align}
%
From the definition of $\lmod_{1} \cup \lmod_{2}$ we know:
%
\begin{align}
	a \in \lmod_{1}(\ca{}) \lor a \in \lmod_{2}(\ca{}) \nonumber
\end{align}
%(\ref{LMC:Ass})
There are two cases to consider:\\

\noindent\textbf{Case 1.} $a \in \lmod_{1}(\ca{})$\\
%
From (\ref{LMC:Ass5}), the assumption of case 1, (\ref{LMC:Ass1}) and (\ref{LMC:Ass3}) we have:
%
\begin{align}
\begin{array}{l}
	\left(
	\begin{array}{@{} l @{}}
		s' = s_p \composeL s_c \composeL s_q /| \\
		\extendsAMUpto{\lmod, \gmod}{(n-1)}{s_{p} \composeL s_{c}}{s_q \composeL r'}{\lmod_1} /| 
		(s_c \composeL s_q, s_p \composeL r') \in a[s_c  \composeL s_q, s_p \composeL r]
	\end{array}
	\right) \\
	|/ 
	\left(
	\begin{array}{@{} l @{}}
	 	s' = \snd{\updateFP{a}} \composeL s_p' \composeL s_c' \composeL s_q' /| \\
	 	\left(
	 	\begin{array}{@{} l @{}}
	 		\left(
	 		\begin{array}{@{} l @{}}
	 			(( p_c > \unitL |/ (p_c = \unitL /| p_p > \unitL /| p_q > \unitL))) /| \\
	 			\extendsAMUpto{\lmod, \gmod}{(n-1)}{\snd{\updateFP{a}} \composeL s_{p}' \composeL s_{c}'}{s_q' \composeL r'}{\lmod_1} /|\\
	 			(\snd{\updateFP{a}} \composeL s_c' \composeL s_q', s_p' \composeL r') \in a[s_c  \composeL s_q, s_p \composeL r]
	 		\end{array}
	 		\right)\\
	 		|/
	 		\left(
	 		\begin{array}{@{} l@{}}
	 			p_p > \unitL /| p_c = p_q = \unitL /|\\
	 			\extendsAMUpto{\lmod, \gmod}{(n-1)}{\snd{\updateFP{a}} \composeL s_{p}' \composeL s_{c}'}{s_q' \composeL r'}{\lmod_1} /| \\
	 			(s_c' \composeL s_q', \snd{\updateFP{a}} \composeL s_p' \composeL r') \in a[s_c  \composeL s_q, s_p \composeL r] 
	 		\end{array}
	 		\right)\\
	 		|/
	 		\left(
	 		\begin{array}{@{} l@{}}
				p_q > \unitL /| p_c = p_p = \unitL /|\\
				\extendsAMUpto{\lmod, \gmod}{(n-1)}{s_{p}' \composeL s_{c}'}{\snd{\updateFP{a}} \composeL s_q' \composeL r'}{\lmod_1}  /| \\
				(\snd{\updateFP{a}} \composeL s_c' \composeL s_q', s_p' \composeL r') \in a[s_c  \composeL s_q, s_p \composeL r]
	 		\end{array}
	 		\right)
	 	\end{array}
	 	\right)
	\end{array}
	\right)
\end{array}
\label{LMC:Ass6}
\end{align}
%(\ref{LMC:Ass})
%
Since $\lmod_{1} \subseteq \lmod$, we know $a \in \lmod(\ca{})$. Consequently, from (\ref{LMC:Ass2}) and (\ref{LMC:Ass3}) we have:
%(\ref{LMC:Ass})
\begin{align*}
	&\m{reflected}(a, s_p \composeL s_c \composeL s_q \composeL r, \lmod_{2}(\ca{})) \lor \\
	&\neg\m{visible}(a, s_c \composeL s_q) /| \extendsAMUpto{\lmod, \gmod}{(n-1)}{s_c \composeL s_q}{a[s_p \composeL s_c \composeL s_q \composeL e] - s_c \composeL s_q}{\lmod_2}
\end{align*}
%
There are two cases to consider:\\

\noindent\textbf{Case 1.1.} 
%
\[
\begin{array}{l}
	\neg\m{visible}(a, s_c \composeL s_q) /| \extendsAMUpto{\lmod, \gmod}{(n-1)}{s_c \composeL s_q}{a[s_p \composeL s_c \composeL s_q \composeL e] - s_c \composeL s_q}{\lmod_2}
\end{array}
\]
%(\ref{LMC:Ass})
By definition of $\m{visible}$ and from the assumption of case 1.1 and (\ref{LMC:Ass6}) we have:
%
\begin{align}
\begin{array}{l}
	\left(
	\begin{array}{@{} l @{}}
		s' = s_p \composeL s_c \composeL s_q /| \\
		\extendsAMUpto{\lmod, \gmod}{(n-1)}{s_{p} \composeL s_{c}}{s_q \composeL r'}{\lmod_1} /| 
		\extendsAMUpto{\lmod, \gmod}{(n-1)}{s_c \composeL s_q}{s_p \composeL r'}{\lmod_2}
	\end{array}
	\right) \\
	|/ 
	\left(
	\begin{array}{@{} l @{}}
	 	s' = \snd{\updateFP{a}} \composeL s_p' \composeL s_c' \composeL s_q' /| p_c = p_q = \unitL /| s_c' = s_c /| s_q' = s_q /| \\
	 			\extendsAMUpto{\lmod, \gmod}{(n-1)}{\snd{\updateFP{a}} \composeL s_{p}' \composeL s_{c}'}{s_q' \composeL r'}{\lmod_1} /| \\
	 			\extendsAMUpto{\lmod, \gmod}{(n-1)}{s_c' \composeL s_q'}{\snd{\updateFP{a}} \composeL s_p' \composeL r'}{\lmod_2} 
	\end{array}
	\right)
\end{array}
\label{LMC:Ass7}
\end{align}
%
From (\ref{LMC:Ass7}) and (\ref{LMC:IH}) we have:
\begin{align*}
\begin{array}{l}
	\extendsAMUpto{\lmod, \gmod}{(n-1)}{s'}{r'}{\lmod_1 \cup \lmod_2}
\end{array}
\end{align*}
%
as required.\\
%
%
%
%
%

\noindent\textbf{Case 1.2.}
\[
\begin{array}{l l}
	\m{reflected}(a, s_p \composeL s_c \composeL s_q \composeL r, \lmod_{2}(\ca{})) 
\end{array}
\]
%(\ref{LMC:Ass})
From (\ref{LMC:Ass3}), the definition of $\m{potential}$ and by \lem~\ref{lem:nonEmptyOverlap} we have:
%
\begin{align*}
	& \exsts{l} \fst{a} < s_p \composeL s_c \composeL s_q \composeL r \composeL l /| \null\\
	& \exsts{l} \fst{\updateFP{a}} \composeL l = s_p \composeL s_c \composeL s_q \composeL r /| \snd{\updateFP{a}} \compatible l
\end{align*}
%
and thus from the assumption of case 1.1.2. we know there exists $a' \in \lmod_2(\ca{})$ such that: 
%
\begin{align*}
	& \updateFP{a'} = \updateFP{a} /|\\
	& \exsts{l} \fst{a'} < s_p \composeL s_c \composeL s_q \composeL r \composeL l /| \null\\
	& \exsts{l} \fst{\updateFP{a'}} \composeL l = s_p \composeL s_c \composeL s_q \composeL r /| \snd{\updateFP{a'}} \compatible l
\end{align*}
%
and consequently from the definition of $\m{potential}$ and \lem~\ref{lem:nonEmptyOverlap} we have: 
%
\begin{align}
	\updateFP{a'} = \updateFP{a} /| \m{potential}(a', s_p \composeL s_c \composeL s_q \composeL s_r) \label{LMC:Ass20}
\end{align}
%(\ref{LMC:Ass})
On the other hand, from the definition of $a[s_c \composeL s_q, s_p \composeL r]$ and since $\updateFP{a'} = \updateFP{a}$(\ref{LMC:Ass20}), from (\ref{LMC:Ass6}) we have: 
%
\begin{align}
\begin{array}{l}
	\left(
	\begin{array}{@{} l @{}}
		s' = s_p \composeL s_c \composeL s_q /| \\
		\extendsAMUpto{\lmod, \gmod}{(n-1)}{s_{p} \composeL s_{c}}{s_q \composeL r'}{\lmod_1} /| 
		(s_c \composeL s_q, s_p \composeL r') \in a'[s_c  \composeL s_q, s_p \composeL r]
	\end{array}
	\right) \\
	|/ 
	\left(
	\begin{array}{@{} l @{}}
	 	s' = \snd{\updateFP{a}} \composeL s_p' \composeL s_c' \composeL s_q' /| \\
	 	\left(
	 	\begin{array}{@{} l @{}}
	 		\left(
	 		\begin{array}{@{} l @{}}
	 			(( p_c > \unitL |/ (p_c = \unitL /| p_p > \unitL /| p_q > \unitL))) /| \\
	 			\extendsAMUpto{\lmod, \gmod}{(n-1)}{\snd{\updateFP{a}} \composeL s_{p}' \composeL s_{c}'}{s_q' \composeL r'}{\lmod_1} /|\\
	 			(\snd{\updateFP{a}} \composeL s_c' \composeL s_q', s_p' \composeL r') \in a'[s_c  \composeL s_q, s_p \composeL r]
	 		\end{array}
	 		\right)\\
	 		|/
	 		\left(
	 		\begin{array}{@{} l@{}}
	 			p_p > \unitL /| p_c = p_q = \unitL /|\\
	 			\extendsAMUpto{\lmod, \gmod}{(n-1)}{\snd{\updateFP{a}} \composeL s_{p}' \composeL s_{c}'}{s_q' \composeL r'}{\lmod_1} /| \\
	 			(s_c' \composeL s_q', \snd{\updateFP{a}} \composeL s_p' \composeL r') \in a'[s_c  \composeL s_q, s_p \composeL r] 
	 		\end{array}
	 		\right)\\
	 		|/
	 		\left(
	 		\begin{array}{@{} l@{}}
				p_q > \unitL /| p_c = p_p = \unitL /|\\
				\extendsAMUpto{\lmod, \gmod}{(n-1)}{s_{p}' \composeL s_{c}'}{\snd{\updateFP{a}} \composeL s_q' \composeL r'}{\lmod_1}  /| \\
				(\snd{\updateFP{a}} \composeL s_c' \composeL s_q', s_p' \composeL r') \in a'[s_c  \composeL s_q, s_p \composeL r]
	 		\end{array}
	 		\right)
	 	\end{array}
	 	\right)
	\end{array}
	\right)
\end{array}
\label{LMC:Ass21}
\end{align}
%
and thus from (\ref{LMC:Ass2}) we have:
%
\begin{align}
\begin{array}{l}
	\left(
	\begin{array}{@{} l @{}}
		s' = s_p \composeL s_c \composeL s_q /| \\
		\extendsAMUpto{\lmod, \gmod}{(n-1)}{s_{p} \composeL s_{c}}{s_q \composeL r'}{\lmod_1} /| 
		\extendsAMUpto{\lmod, \gmod}{(n-1)}{s_{c} \composeL s_{q}}{s_p \composeL r'}{\lmod_2}
	\end{array}
	\right) \\
	|/ 
	\left(
	\begin{array}{@{} l @{}}
	 	s' = \snd{\updateFP{a}} \composeL s_p' \composeL s_c' \composeL s_q' /| \\
	 	\left(
	 	\begin{array}{@{} l @{}}
	 		\left(
	 		\begin{array}{@{} l @{}}
	 			(( p_c > \unitL |/ (p_c = \unitL /| p_p > \unitL /| p_q > \unitL))) /| \\
	 			\extendsAMUpto{\lmod, \gmod}{(n-1)}{\snd{\updateFP{a}} \composeL s_{p}' \composeL s_{c}'}{s_q' \composeL r'}{\lmod_1} /|\\
	 			\extendsAMUpto{\lmod, \gmod}{(n-1)}{\snd{\updateFP{a}} \composeL s_{c}' \composeL s_{q}'}{s_p' \composeL r'}{\lmod_2} 
	 		\end{array}
	 		\right)\\
	 		|/
	 		\left(
	 		\begin{array}{@{} l@{}}
	 			p_p > \unitL /| p_c = p_q = \unitL /|\\
	 			\extendsAMUpto{\lmod, \gmod}{(n-1)}{\snd{\updateFP{a}} \composeL s_{p}' \composeL s_{c}'}{s_q' \composeL r'}{\lmod_1} /| \\
	 			\extendsAMUpto{\lmod, \gmod}{(n-1)}{s_{c}' \composeL s_{q}'}{\snd{\updateFP{a}} \composeL s_p' \composeL r'}{\lmod_2} 
	 		\end{array}
	 		\right)\\
	 		|/
	 		\left(
	 		\begin{array}{@{} l@{}}
				p_q > \unitL /| p_c = p_p = \unitL /|\\
				\extendsAMUpto{\lmod, \gmod}{(n-1)}{s_{p}' \composeL s_{c}'}{\snd{\updateFP{a}} \composeL s_q' \composeL r'}{\lmod_1}  /| \\
				\extendsAMUpto{\lmod, \gmod}{(n-1)}{\snd{\updateFP{a}} \composeL s_{c}' \composeL s_{q}'}{s_p' \composeL r'}{\lmod_2}
	 		\end{array}
	 		\right)
	 	\end{array}
	 	\right)
	\end{array}
	\right)
\end{array}
\label{LMC:Ass22}
\end{align}
%
%(\ref{LMC:Ass})
From (\ref{LMC:Ass22}) and (\ref{LMC:IH}) we have: 
%
\begin{align*}
	\extendsAMUpto{\lmod, \gmod}{(n-1)}{s'}{r'}{\lmod_1 \cup \lmod_2} 
\end{align*}
%(\ref{LMC:Ass})
as required.\\
%
%
%
%
%
%

\noindent\textbf{Case 2.} $a \in \lmod_{2}(\ca{})$\\
The proof of this case is analogous to that of previous case and is omitted here.\\
%
%
%
%
%




\noindent\textbf{RTS. (\ref{LMC:Goal2})}\\
Pick an arbitrary $\ca{}$ and $a \in \left(\lmod_{1} \cup \lmod_{2} \right)(\ca{})$ such that
\begin{equation}
	\m{enabled}(a, s_p \composeL s_c \composeL s_q \composeL r) \label{LMC:Ass30}
\end{equation}
%
There are two cases to consider:\\

\noindent\textbf{Case 1.} $a \in \lmod_{1}(\ca{})$\\
From assumption of case 1, (\ref{LMC:Ass1}) and (\ref{LMC:Ass30}) we then have: 
%
\begin{align*}
	(s_p \composeL s_c \composeL s_q \composeL r, a[s_p \composeL s_c \composeL s_q \composeL r]) \in \gmod(\ca{})
\end{align*}
%
as required. \\

\noindent\textbf{Case 2.} $a \in \lmod_{2}(\ca{})$\\
The proof of this case is analogous to that of previous case and is omitted here.\\
%
%
%
%
%

\noindent\textbf{RTS. (\ref{LMC:Goal3})}\\
Pick an arbitrary $\ca{}$ and $a \in\lmod(\ca{})$ such that
\begin{equation}
	\m{potential}(a, s_p \composeL s_c \composeL s_q \composeL r) \label{LMC:Ass40}
\end{equation}
%(\ref{LMC:Ass})
From (\ref{LMC:Ass1}) and (\ref{LMC:Ass40}) we have: 
%
\begin{align*}
	&\m{reflected}(a,s_p \composeL s_c \composeL s_q \composeL r,\lmod_{1}(\ca{})) |/\null \\
%  
  &\neg\m{visible}(a,s_p \composeL s_c) /| \for{(s', r') \in a[s_p \composeL s_c, s_q \composeL r]} \extendsAMUpto{\lmod, \gmod}{(n-1)}{s'}{r'}{\lmod_{1}}
\end{align*}
%
and consequently from the definition of $\lmod_1 \cup \lmod_2$ we have: 
%
\begin{align}
	&\m{reflected}(a,s_p \composeL s_c \composeL s_q \composeL r, \left( \lmod_{1} \cup \lmod_2 \right) (\ca{})) |/\null \nonumber \\
%  
  &\neg\m{visible}(a,s_p \composeL s_c) /| \for{(s', r') \in a[s_p \composeL s_c, s_q \composeL r]} \extendsAMUpto{\lmod, \gmod}{(n-1)}{s'}{r'}{\lmod_{1}} \label{LMC:Ass41}
\end{align}
%(\ref{LMC:Ass})
Similarly, from (\ref{LMC:Ass2}) and (\ref{LMC:Ass40}) we have: 
%
\begin{align}
	&\m{reflected}(a,s_p \composeL s_c \composeL s_q \composeL r, \left(\lmod_1 \cup \lmod_{2}\right)(\ca{})) |/\null \nonumber \\
%  
  &\neg\m{visible}(a, s_c \composeL s_q) /| \for{(s', r') \in a[s_c \composeL s_q, s_p \composeL r]} \extendsAMUpto{\lmod, \gmod}{(n-1)}{s'}{r'}{\lmod_{2}} \label{LMC:Ass42}
\end{align}
%(\ref{LMC:Ass})
From (\ref{LMC:Ass41}) and (\ref{LMC:Ass42}) we have: 
%
\begin{align}
	&\m{reflected}(a,s_p \composeL s_c \composeL s_q \composeL r, \left(\lmod_1 \cup \lmod_{2}\right)(\ca{})) |/\null \nonumber \\
%  
  &\neg\m{visible}(a,s_p \composeL s_c) /| \for{(s', r') \in a[s_p \composeL s_c, s_q \composeL r]} \extendsAMUpto{\lmod, \gmod}{(n-1)}{s'}{r'}{\lmod_{1}} \nonumber\\
  &\neg\m{visible}(a, s_c \composeL s_q) /| \for{(s', r') \in a[s_c \composeL s_q, s_p \composeL r]} \extendsAMUpto{\lmod, \gmod}{(n-1)}{s'}{r'}{\lmod_{2}} \label{LMC:Ass43}
\end{align}
%(\ref{LMC:Ass})
If the first disjunct is the case then the desired result holds trivially. On the other hand, in case of the second disjunct we have:
%
\begin{align}
	& 
	\begin{array}{@{} l @{}}
		\neg\m{visible}(a,s_p \composeL s_c) /| \neg\m{visible}(a, s_c \composeL s_q) \\
		/| \for{(s', r') \in a[s_p \composeL s_c, s_q \composeL r]} \extendsAMUpto{\lmod, \gmod}{(n-1)}{s'}{r'}{\lmod_{1}} \\
  	/| \for{(s', r') \in a[s_c \composeL s_q, s_p \composeL r]} \extendsAMUpto{\lmod, \gmod}{(n-1)}{s'}{r'}{\lmod_{2}} 
	\end{array}
	\label{LMC:Ass44}
\end{align}
%
From (\ref{LMC:Ass44}) and by definition of $\m{visible}$ we have:
%
\begin{align}
	\neg\m{visible}(a, s_p \composeL s_c \composeL s_q)
	\label{LMC:Ass45}
\end{align}
%
Pick an arbitrary $s', r' \in \LStates$ such that
%
\begin{align}
	(s', r') \in a[s_p \composeL s_c \composeL s_q, r]
	\label{LMC:Ass46}
\end{align}
%(\ref{LMC:Ass})
Then from (\ref{LMC:Ass45}), (\ref{LMC:Ass46}) and the definitions of $a[s_p \composeL s_c, s_q \composeL r]$ and $a[s_c \composeL s_q, s_p \composeL r]$ we know 
%
\begin{align}
	& s' = s_p \composeL s_c \composeL s_q \label{LMC:Ass47}\\
	&(s_p \composeL s_c, s_q \composeL r') \in a[s_p \composeL s_c, s_q \composeL r] /| 
	(s_c \composeL s_q, s_p \composeL r') \in a[s_c \composeL s_q, s_p \composeL r]
	\label{LMC:Ass48}
\end{align}
%
Consequently from (\ref{LMC:Ass44}) and (\ref{LMC:Ass48}) we have
%
\begin{align*}
	\extendsAMUpto{\lmod, \gmod}{(n-1)}{s_p \composeL s_c}{s_q \composeL r'}{\lmod_1} /| \extendsAMUpto{\lmod, \gmod}{(n-1)}{s_c \composeL s_q}{s_p \composeL r'}{\lmod_2}
\end{align*}
% 
and thus from (\ref{LMC:IH}) and (\ref{LMC:Ass47})
%
\begin{align}
	\extendsAMUpto{\lmod, \gmod}{(n-1)}{s'}{r'}{\lmod_1 \cup \lmod_2}
	\label{LMC:Ass49}
\end{align}
%(\ref{LMC:Ass})
Finally, from (\ref{LMC:Ass45}), (\ref{LMC:Ass46}) and (\ref{LMC:Ass49}) we have:
%
\begin{align*}
	& \neg\m{visible}(a, s_p \composeL s_c \composeL s_q)  /| \\
	& \for{(s', r') \in a[s_p \composeL s_c \composeL s_q, r]} \extendsAMUpto{\lmod, \gmod}{(n-1)}{s'}{r'}{\lmod_1 \cup \lmod_2}
\end{align*}
%
as required.
%
%
%
%
%

\end{proof}
\end{lemma}
%
%
	\begin{lemma}[\shiftRule-Fence]\label{lem:shift-fence}
For all $\lmod_1, \lmod_2 \in \AMods$, $s, s', r \in \LStates$ and $a \in \m{rg}(\lmod_1)$:
%
\[
	\lmod_1 \weakenI{\{s\}} \lmod_2 /| \left(s' \in a(s) \lor (s', -) \in a[s, r] \right) \implies \lmod_1 \weakenI{\{s'\}} \lmod_2
\]
%
\begin{proof}
Pick an arbitrary $\lmod_1, \lmod_2 \in \AMods$, $s, s', r \in \LStates$ and $a \in \m{rg}(\lmod_1)$ such that:
%
\begin{align}
	\lmod_1 \weakenI{\{s\}} \lmod_2 \label{LMS:Ass1}
\end{align}
%(\ref{LMS:Ass})
\textbf{RTS. } $\lmod_1 \weakenI{\{s'\}} \lmod_2$.\\
There are two cases to consider:\\
\textbf{Case 1. }$s' \in a(s)$\\
From the definition of $\weakenI{\{s\}}$ and (\ref{LMS:Ass1}) we know there exists a fence $\fence{}$ such that:
%
\begin{align}
	& s \in \fence{} \label{LMS:Ass3}\\
	& \fence{} \fences \lmod_1 \label{LMS:Ass4}\\
	\for{l \in \fence{}} \for{\ca{}}& \for{a \in \lmod_2(\ca{})} \m{reflected}(a, l, \lmod_1(\ca{})) /| \null \nonumber\\
	& \for{a \in \lmod_1(\ca{})} a(l) \text{ is defined } /| \m{visible}(a, l) \implies \m{reflected}(a, l, \lmod_2(\ca{})) \label{LMS:Ass5}
\end{align}
%(\ref{LMS:Ass})
By definition of $\fences$ and from (\ref{LMS:Ass3})-(\ref{LMS:Ass4}) and assumption of case 1. we have: 
%
\begin{align}
	& s' \in \fence{} \label{LMS:Ass6}
\end{align}
%
Finally by definition of $\weakenI{\{s'\}}$ and (\ref{LMS:Ass4})-(\ref{LMS:Ass6}) we have
%
\begin{align*}
	\lmod_1 \weakenI{\{s'\}} \lmod_2
\end{align*}
%
as required.\\

\noindent\textbf{Case 2. }$(s', -) \in a[s, r]$\\
%From the definition of $a[s, r]$ we know :
%\begin{align*}
%	(s' = s ) |/ 
%	\left(
%	\begin{array}{@{}l l@{}}
%		\exsts{p_s ? \unitL}\exsts{p_r, s'', r''} & s = p_s \composeL s''\\
%		& r = p_r \composeL r'' \\
%		& \fst{\updateFP{a}} = p_s \composeL p_r \\
%		& s' = \snd{\updateFP{a}} \composeL s'' \\
%		& s' \composeL r'' \text{ is defined}
%	\end{array} 
%	\right)	
%\end{align*}
%
From the definitions of $a[s, r]$ and $a(s)$ and from the assumption of the case we know $s' \in a(s)$. The rest of the proof is identical to that of case 1.
\end{proof}
\end{lemma}
%
%
%
\begin{lemma}[\shiftRule-Closure-1]\label{lem:shift-closure}
%
For all $\lmod_1, \lmod_2, \lmod, \gmod \in \AMods$ and $s, r \in \LStates$,
%
\[
	\extendsAM{\lmod, \gmod}{s}{r}{\lmod_1} /| \lmod_1 \weakenI{\{s\}} \lmod_2 \implies \extendsAM{\lmod \cup \lmod_2, \gmod}{s}{r}{\lmod_2}
\]
%
\begin{proof} Pick an arbitrary $\lmod_1, \lmod_2, \lmod, \gmod \in \AMods$ and $s, r \in \LStates$ such that 
%
\begin{align}
	& \extendsAM{\lmod, \gmod}{s}{r}{\lmod_1} \label{SC:Ass1}\\
	& \lmod_1 \weakenI{\{s\}} \lmod_2 \label{SC:Ass2}
\end{align} 
%
From the definition of $\downarrow$, it then suffices to show
%
\begin{align}
	& \lmod_2 \subseteq \lmod \cup \lmod_2\label{SC:Goal1}\\
	& \for {n \in \Nats}  \extendsAMUpto{\lmod \cup \lmod_2, \gmod}{n}{s}{r}{\lmod_2} \label{SC:Goal2}
\end{align}
%
\noindent\textbf{RTS. (\ref{SC:Goal1})} \\
This holds trivially from the definition of $\lmod \cup \lmod_2$.\\

\noindent\textbf{RTS. (\ref{SC:Goal2})} \\
Rather than proving (\ref{SC:Goal2}) directly, we first establish the following.
%
\begin{align}
	& \for {n \in \Nats} \for{s, r \in \LStates} \nonumber\\
	& \quad \extendsAMUpto{\lmod, \gmod}{n}{s}{r}{\lmod_1} /| \lmod_1 \weakenI{\{s\}} \lmod_2 \implies \extendsAMUpto{\lmod \cup \lmod_2, \gmod}{n}{s}{r}{\lmod_2} \label{SC:Goal3}
\end{align}
%
We can then despatch (\ref{SC:Goal2}) from (\ref{SC:Ass1}), (\ref{SC:Ass2}) and (\ref{SC:Goal3}); since for an arbitrary $n \in \Nats$, from (\ref{SC:Ass1}) and the definition of $\downarrow$ we have $\extendsAMUpto{\lmod, \gmod}{n}{s}{r}{\lmod_1}$ and consequently from (\ref{SC:Ass2}) and (\ref{SC:Goal3}) we derive $\extendsAMUpto{\lmod \cup \lmod_2, \gmod}{n}{s}{r}{\lmod_2} $ as required. \\

\noindent\textbf{RTS. (\ref{SC:Goal3})} \\
We proceed by induction on the number of steps $n$.\\

%\noindent Pick an arbitrary $s_1, s_2, r \in \LStates, \lmod, \lmod', \gmod \in \AMods$.\\
\noindent\textbf{Base case }$n=0$\\
Pick an arbitrary $s, r \in \LStates$. We are then required to show	$\extendsAMUpto{\lmod \cup \lmod_2, \gmod}{0}{s}{r}{\lmod_2} $ which follows trivially from the definition of $\downarrow_0$.\\


\noindent\textbf{Inductive Case}\\
Pick an arbitrary $s, r \in \LStates$ such that:
\begin{align}
	&\extendsAMUpto{\lmod, \gmod}{n}{s}{r}{\lmod_1} \label{LSC:Ass1}\\
	&\lmod_1 \weakenI{\{s\}} \lmod_2 \label{LSC:Ass2}\\
%		
	&	\for{s, r \in \LStates}  \nonumber\\
	& \tag{I.H} 
		\quad \extendsAMUpto{\lmod, \gmod}{(n-1)}{s}{r}{\lmod_1} /| \lmod_1 \weakenI{\{s\}} \lmod_2 \implies \extendsAMUpto{\lmod \cup \lmod_2, \gmod}{(n-1)}{s}{r}{\lmod_2} \label{LSC:IH}
\end{align}
%
\textbf{RTS. } 
%
\begin{align}
	& 
	\V{\ca{}}  \V{a\in \lmod_2(\ca{})} \nonumber \\
  &\quad (\m{potential}(a,s \composeL r) => \nonumber\\
  & \quad\qquad\for{(s', r') \in a[s, r]} \extendsAMUpto{\lmod \cup \lmod_2, \gmod}{(n-1)}{s'}{r'}{\lmod_2}) \label{LSC:Goal1} /| \\
%   
  &\quad (\m{enabled}(a,s \composeL r)
  => (s\composeL r, a[s \composeL r])\in \gmod(\ca{}))
  /|\null \label{LSC:Goal2}\\
%  
  &\V{\ca{}}\V{a\in \left(\lmod \cup \lmod_2 \right) (\ca{})}
  \m{potential}(a,s \composeL r) =>\null \nonumber \\
  &\ \m{reflected}(a,s \composeL r,\lmod_2(\ca{})) |/\null \nonumber \\
%  
  &\ \neg\m{visible}(a,s) /| \for{(s', r') \in a[s, r]} \extendsAMUpto{\lmod \cup \lmod_2, \gmod}{(n-1)}{s'}{r'}{\lmod_2}  \label{LSC:Goal3}
\end{align}
%
\noindent\textbf{RTS. (\ref{LSC:Goal1})} \\
%(\ref{LSC:Ass})
Pick an arbitrary $\ca{}$, $a \in \lmod_2(\ca{})$ and $s', r' \in \LStates$ such that:
\begin{align}
	&\m{potential}(a, s \composeL r) \label{LSC:Ass3}\\
	&(s', r') \in a[s, r] \label{LSC:Ass4}
\end{align}
%%
From (\ref{LSC:Ass3}), the definition of $\m{potential}$ and by \lem~\ref{lem:nonEmptyOverlap} we know there exists $l$ such that 
%
\begin{align*}
	& \fst{a} < s \composeL r \composeL l /| \null\\
	& \exsts{l'} \fst{\updateFP{a}} \composeL l' = s \composeL r /| \snd{\updateFP{a}} \compatible l'
\end{align*}
%
and thus from (\ref{LSC:Ass2}) we know there exists $a' \in \lmod_1(\ca{})$ such that: 
%
\begin{align*}
	& \updateFP{a'} = \updateFP{a} /|\\
	& \fst{a'} < s \composeL r \composeL l /| \null\\
	& \exsts{l'} \fst{\updateFP{a'}} \composeL l' = s \composeL r /| \snd{\updateFP{a'}} \compatible l'
\end{align*}
%
and consequently from the definition of $\m{potential}$ and \lem~\ref{lem:nonEmptyOverlap} we have: 
%
\begin{align}
	\updateFP{a'} = \updateFP{a} /| \m{potential}(a', s \composeL r) \label{LSC:Ass4-1}
\end{align}
%(\ref{LMC:Ass})
Since $\updateFP{a'} = \updateFP{a}$, from the definition of $a[s, r]$ we know $a[s, r] = a'[s, r]$. Thus, from (\ref{LSC:Ass1}), (\ref{LSC:Ass3}), (\ref{LSC:Ass4}) and (\ref{LSC:Ass4-1}) we have:
%
\begin{align}
	\extendsAMUpto{\lmod, \gmod}{(n-1)}{s'}{r'}{\lmod_1} \label{LSC:Ass5}
\end{align}
%
From (\ref{LSC:Ass2}), (\ref{LSC:Ass4}), \lem~\ref{lem:shift-fence} and since $a[s, r] = a'[s, r]$, we have
%
\begin{align}
	\lmod_1 \weakenI{\{s'\}} \lmod_2 \label{LSC:Ass6}
\end{align}
%
Finally, from (\ref{LSC:Ass5}), (\ref{LSC:Ass6}) and (\ref{LSC:IH}) we have:
%
\begin{align*}
	\extendsAMUpto{\lmod \cup \lmod_2, \gmod}{(n-1)}{s'}{r'}{\lmod_2}
\end{align*}
%
as required.\\
%
%
%
%



\noindent\textbf{RTS. (\ref{LSC:Goal2})} \\
%(\ref{LSC:Ass})
Pick an arbitrary $\ca{}$ and $a \in \lmod_2(\ca{})$ such that:
\begin{align}
	\m{enabled}(a, s \composeL r)\label{LSC:Ass7}
\end{align}
%%
From (\ref{LSC:Ass1}) and (\ref{LSC:Ass7}) we then have:
%
\begin{align*}
	(s \composeL r, a[s \composeL r]) \in \gmod(\ca{})
\end{align*}
%
as required.\\
%
%
%


\noindent\textbf{RTS. (\ref{LSC:Goal3})}\\
%(\ref{LSC:Ass})
Pick an arbitrary $\ca{}$ and $a \in \left(\lmod \cup \lmod_2 \right)(\ca{})$ such that:
%
\begin{align}
	\m{potential}(a, s \composeL r)\label{LSC:Ass8}
\end{align}
%
If $a \in \lmod_2 (\ca{})$, then it is trivially the case that $\m{reflected}(a, s \composeL r)$ and thus the desired result (\ref{LSC:Goal3}) holds. On the other hand, if $a \in \lmod(\ca{})$, then from (\ref{LSC:Ass1}) and (\ref{LSC:Ass8}) we have:
%
\begin{align*}
	& \m{reflected}(a, s \composeL r, \lmod_1(\ca{})) |/ \\
	& \neg \m{visible}(a, s) /| \for{(s', r') \in a[s, r]} \extendsAMUpto{\lmod, \gmod}{(n-1)}{s'}{r'}{\lmod_1} 
\end{align*}
%
There are two cases to consider:\\
\noindent\textbf{Case 1.} 
%
\[
\begin{array}{l}
	\neg \m{visible}(a, s) /| \for{(s', r') \in a[s, r]}  \extendsAMUpto{\lmod, \gmod}{(n-1)}{s'}{r'}{\lmod_1}
\end{array}
\]
%(\ref{LSC:Ass})
From the assumption of case 1, (\ref{LSC:Ass2}) and (\ref{LSC:IH}) we have: 
%
\[
\begin{array}{l}
	\neg \m{visible}(a, s) /| \for{(s', r') \in a[s, r]} \extendsAMUpto{\lmod \cup \lmod_2, \gmod}{(n-1)}{s'}{r'}{\lmod_2}
\end{array}
\]
%(\ref{LSC:Ass})
as required.\\

%
%
%
%
\noindent\textbf{Case 2.} 
%
\[
\begin{array}{l}
		\m{reflected}(a, s \composeL r, \lmod_1(\ca{})) 
\end{array}
\]
%(\ref{LSC:Ass})
Pick an arbitrary $l \in \LStates$ such that:
%
\begin{align}
	\fst{a} \leq s \composeL r \composeL l \label{LSC:Ass9}
\end{align}
%
Then from the assumption of case 2 and the definition of $\m{reflected}$ we have:
%(\ref{LSC:Ass})
\begin{align}
	\exsts{a' \in \lmod_1(\ca{})} \fst{a'} \leq s \composeL r \composeL l /| \updateFP{a'} = \updateFP{a} \label{LSC:Ass10}
\end{align}
%(\ref{LSC:Ass})
Since either $\m{visible}(a', s)$ or $\neg\m{visible}(a', s)$, there are two cases to consider:\\
%
\textbf{Case 2.1.} $\m{visible}(a', s)$ \\
From (\ref{LSC:Ass8}) and by definition of $\m{potential}$ we know $a[s \composeL r]$ is defined; from (\ref{LSC:Ass10}), and the definition of $a'[s \composeL r]$ we know that $a'[s \composeL r]$ is also defined. Consequently, from the definition of $a'(s)$, we know $a'(s)$ is also defined. Thus, from the assumptions of case 2.1, (\ref{LSC:Ass2}), (\ref{LSC:Ass10}) and from the definition of $\weakenI{\{s\}}$ we have 
%
\begin{align}
	\exsts{a'' \in \lmod_2(\ca{})} \fst{a''} \leq s \composeL r \composeL l /| \updateFP{a''} = \updateFP{a'} \label{LSC:Ass11}
\end{align} 
%
Finally, from (\ref{LSC:Ass9}), (\ref{LSC:Ass10}), (\ref{LSC:Ass11}) and by definition of $\m{reflected}$ we have:
%
\begin{align*}
	\m{reflected}(a, s \composeL r, \lmod_2(\ca{}))
\end{align*} 
%
as required.\\

\noindent\textbf{Case 2.2.} $\neg\m{visible}(a', s)$ \\
%(\ref{LSC:Ass})
From (\ref{LSC:Ass8}), (\ref{LSC:Ass10}) and by definition of $\m{potential}$ we have:
%
\begin{align}
	\m{potential}(a', s \composeL r) \label{LSC:Ass12}
\end{align}
%
Consequently, from (\ref{LSC:Ass1}), (\ref{LSC:Ass10}), and (\ref{LSC:Ass12}) 
%
\begin{align*}
	\for{(s', r') \in a'[s, r]} \extendsAMUpto{\lmod, \gmod}{(n-1)}{s'}{r'}{\lmod_1}
\end{align*}
% 
and since $\updateFP{a} = \updateFP{a'}$ (\ref{LSC:Ass10}), by definition of $a[s, r]$ we have:
%
\begin{align*}
	\for{(s', r') \in a[s, r]} \extendsAMUpto{\lmod, \gmod}{(n-1)}{s'}{r'}{\lmod_1} 
\end{align*}
%
Consequently, from (\ref{LSC:Ass2}) and (\ref{LSC:IH}) we have: 
%
\begin{align}
	\for{(s', r') \in a[s, r]} \extendsAMUpto{\lmod \cup \lmod_2, \gmod}{(n-1)}{s'}{r'}{\lmod_2} \label{LSC:Ass13}
\end{align}
%
On the other hand, since $\updateFP{a} = \updateFP{a'}$ (\ref{LSC:Ass10}), from the definition of $\m{visible}$ and the assumption of case 2.2. we have:
%
\begin{align}
	\neg\m{visible}(a, s) \label{LSC:Ass14}
\end{align}
%(\ref{LSC:Ass})
Finally, from (\ref{LSC:Ass13}) and (\ref{LSC:Ass14}) we have:
%
\begin{align*}
	\neg\m{visible}(a, s) /| \for{(s', r') \in a[s, r]} \extendsAMUpto{\lmod \cup \lmod_2, \gmod}{(n-1)}{s'}{r'}{\lmod_2} 
\end{align*}
%
as required.

\end{proof}
\end{lemma}
%
%
%
%
%
%
\begin{lemma}[\shiftRule-Closure-2]\label{lem:shift-closure-2}
For all $\lmod_0, \lmod_1, \lmod_2, \lmod, \gmod \in \AMods$ and  $s_1, r_1, s_0, r_0 \in \LStates$
%
\[
\begin{array}{@{} l @{\hspace{-1cm}} l @{} } 
	\extendsAM{\lmod, \gmod}{s_1}{r_1}{\lmod_1} /| \lmod_1 \weakenI{\{s_1\}} \lmod_2 /| \null & \\
	\extendsAM{\lmod, \gmod}{s_0}{r_0}{\lmod_0} /| s_1 \composeL r_1 = s_0 \composeL r_0 \implies & \\
	&\extendsAM{\lmod \cup \lmod_2, \gmod}{s_0}{r_0}{\lmod_0}
\end{array}
\]
%
\begin{proof} Pick an arbitrary $\lmod_0, \lmod_1, \lmod_2, \lmod, \gmod \in \AMods$ and $s_1, r_1, s_0, r_0 \in \LStates$ such that 
%
\begin{align}
	& \extendsAM{\lmod, \gmod}{s_1}{r_1}{\lmod_1} \label{SC2:Ass1}\\
	& \lmod_1 \weakenI{\{s_1\}} \lmod_2 \label{SC2:Ass2}\\
	& \extendsAM{\lmod, \gmod}{s_0}{r_0}{\lmod_0} \label{SC2:Ass3}\\
	& s_1 \composeL r_1 = s_0 \composeL r_0 \label{SC2:Ass4}
\end{align} 
%
From the definition of $\downarrow$, it then suffices to show
%
\begin{align}
	& \lmod_0 \subseteq \lmod \cup \lmod_2\label{SC2:Goal1}\\
	& \for {n \in \Nats}  \extendsAMUpto{\lmod \cup \lmod_2, \gmod}{n}{s_0}{r_0}{\lmod_0} \label{SC2:Goal2}
\end{align}
%
\noindent\textbf{RTS. (\ref{SC2:Goal1})} \\
From (\ref{SC2:Ass3}) and the definition of $\downarrow$ we have $\lmod_0 \subseteq \lmod$ and consequently $\lmod_0 \subseteq \lmod \cup \lmod_2$ as required.\\

\noindent\textbf{RTS. (\ref{SC2:Goal2})} \\
Rather than proving (\ref{SC2:Goal2}) directly, we first establish the following.
%
\begin{align}
	& \for {n \in \Nats} \for{s_1, r_1, s_0, r_0 \in \LStates} \nonumber\\
	& \quad \extendsAMUpto{\lmod, \gmod}{n}{s_1}{r_1}{\lmod_1} /| \lmod_1 \weakenI{\{s_1\}} \lmod_2 /| \nonumber\\
	& \quad \extendsAMUpto{\lmod, \gmod}{n}{s_0}{r_0}{\lmod_0} /| s_1 \composeL r_1 = s_0 \composeL r_0 \implies \nonumber\\
	& & \hspace{-3cm}  \extendsAMUpto{\lmod \cup \lmod_2, \gmod}{n}{s_0}{r_0}{\lmod_0} \label{SC2:Goal3}
\end{align}
%
We can then despatch (\ref{SC2:Goal2}) from (\ref{SC2:Ass1})-(\ref{SC2:Ass4}) and (\ref{SC2:Goal3}); since for an arbitrary $n \in \Nats$, from (\ref{SC2:Ass1}), (\ref{SC2:Ass3}) and the definition of $\downarrow$ we have $\extendsAMUpto{\lmod, \gmod}{n}{s_1}{r_1}{\lmod_1} /| \extendsAMUpto{\lmod, \gmod}{n}{s_0}{r_0}{\lmod_0}$ and consequently from (\ref{SC2:Ass2}), (\ref{SC2:Ass4}) and (\ref{SC2:Goal3}) we derive $\extendsAMUpto{\lmod \cup \lmod_2, \gmod}{n}{s_0}{r_0}{\lmod_0} $ as required. \\

\noindent\textbf{RTS. (\ref{SC2:Goal3})} \\
We proceed by induction on the number of steps $n$.\\

%\noindent Pick an arbitrary $s_1, s_2, r \in \LStates, \lmod, \lmod', \gmod \in \AMods$.\\
\noindent\textbf{Base case }$n=0$\\
Pick an arbitrary $s_1, r_1, s_0, r_0 \in \LStates$. We are then required to show	$\extendsAMUpto{\lmod \cup \lmod_2, \gmod}{0}{s_0}{r_0}{\lmod_0} $ which follows trivially from the definition of $\downarrow_0$.\\


\noindent\textbf{Inductive Case}\\
Pick an arbitrary $n \in \Nats$ and $s_1, r_1, s_0, r_0 \in \LStates$ such that:
\begin{align}
	&\extendsAMUpto{\lmod, \gmod}{n}{s_1}{r_1}{\lmod_1} \label{LSC2:Ass1}\\
	&\lmod_1 \weakenI{\{s_1\}} \lmod_2 \label{LSC2:Ass2}\\
	& \extendsAMUpto{\lmod, \gmod}{n}{s_0}{r_0}{\lmod_0} \label{LSC2:Ass3}\\
	& s_1 \composeL r_1 = s_0 \composeL r_0 \label{LSC2:Ass4}\\
	\tag{I.H}	
	&\begin{array}{@{} l @{\hspace{-3cm}} l @{\hspace*{-2cm}} l @{}}
		\for{s'_1, r'_1, s'_0, r'_0 \in \LStates} &&\\
		& \extendsAMUpto{\lmod, \gmod}{(n-1)}{s'_1}{r'_1}{\lmod_1} /| \lmod_1 \weakenI{\{s'_1\}} \lmod_2 /| & \\
		& \extendsAMUpto{\lmod, \gmod}{(n-1)}{s'_0}{r'_0}{\lmod_0} /| s'_1 \composeL r'_1 = s'_0 \composeL r'_0 \implies & \\
		&& \extendsAMUpto{\lmod \cup \lmod_2, \gmod}{(n-1)}{s'_0}{r'_0}{\lmod_0}
	\end{array} \label{LSC2:IH}
\end{align}
%
\textbf{RTS. } 
%
\begin{align}
	& 
	\V{\ca{}}  \V{a\in \lmod_0(\ca{})} \nonumber \\
  &\quad (\m{potential}(a,s_0 \composeL r_0)  => \nonumber\\
  & \quad\qquad \for{(s', r') \in a[s_0, r_0]} \extendsAMUpto{\lmod \cup \lmod_2, \gmod}{(n-1)}{s'}{r'}{\lmod_0}) \label{LSC2:Goal1}\\
%   
  &\quad\land \m{enabled}(a,s_0 \composeL r_0)
  => (s_0 \composeL r_0, a[s_0 \composeL r_0])\in \gmod(\ca{}))
  /|\null \label{LSC2:Goal2}\\
%  
  &\V{\ca{}}\V{a\in \left(\lmod \cup \lmod_2 \right) (\ca{})}
  \m{potential}(a, s_0 \composeL r_0) =>\null \nonumber \\
  &\ \m{reflected}(a, s_0 \composeL r_0,\lmod_0(\ca{})) |/\null \nonumber \\
%  
  &\ \neg\m{visible}(a, s_0) /| \for{(s', r') \in a[s_0, r_0]} \extendsAMUpto{\lmod \cup \lmod_2, \gmod}{(n-1)}{s'}{r'}{\lmod_0}  \label{LSC2:Goal3}
\end{align}
%
\noindent\textbf{RTS. (\ref{LSC2:Goal1})} \\
Pick an arbitrary $\ca{}$, $a \in \lmod_0(\ca{})$ and $(s', r')$ such that
%
\begin{align}
	& \m{potential}(a, s_0 \composeL r_0) \label{LSC2:Ass5}\\
	& (s', r') \in a[s_0, r_0] \label{LSC2:Ass6}
\end{align}
%(\ref{LSC2:Ass})
Then from (\ref{LSC2:Ass3}) and (\ref{LSC2:Ass5})-(\ref{LSC2:Ass6}) we have:
%
\begin{align}
	\extendsAMUpto{\lmod, \gmod}{(n-1)}{s'}{r'}{\lmod_0} \label{LSC2:Ass7}
\end{align}
%
From (\ref{LSC2:Ass6}) and the definition of $a[s_0, r_0]$, we know that $\fst{\updateFP{a}} \leq s_0 \composeL r_0$ and consequently from (\ref{LSC2:Ass4}) we have $\fst{\updateFP{a}} \leq s_1 \composeL r_1$. Thus from \lem~\ref{lem:divideUpper} we know there exists $p_s, p_r \in \LStates$ such that : 
%
\begin{align*}
	\fst{\updateFP{a}} = p_s \composeL p_r /| p_s \leq s_1 /| p_r \leq r_1
\end{align*}
%
From the definition of $a[s_1, r_1]$ we then have
%
\begin{align*}
	& (p_s = \unitL /| (s_1, \snd{\updateFP{a}} \composeL (r_1 -p_r)) \in a[s_1, r_1])\\
	|/ & (p_s > \unitL /| (\snd{\updateFP{a}} \composeL (s_1 - p_s), r_1 - p_r) \in a[s_1, r_1]) 
\end{align*}
%
That is, there exists $s'', r'' \in \LStates$ such that
%
\begin{align}
	(s'', r'') \in a[s_1, r_1]
	\label{LSC2:Ass8}
\end{align}
%
From (\ref{LSC2:Ass2}), (\ref{LSC2:Ass8}) and \lem~\ref{lem:shift-fence} we have:
%
\begin{align}
	\lmod_1 \weakenI{\{s''\}} \lmod_2 \label{LSC2:Ass9}
\end{align}
%
From (\ref{LSC2:Ass4}), (\ref{LSC2:Ass6}), (\ref{LSC2:Ass8}) and \lem~\ref{lem:action-application} we have:
%
\begin{align}
	s' \composeL r' = s'' \composeL r'' \label{LSC2:Ass10}
\end{align}
%
Since $a \in \lmod_0(\ca{})$ and $\lmod_0 \subseteq \lmod$, we know $a \in \lmod(\ca{})$; consequently, from (\ref{LSC2:Ass8}) and (\ref{LSC2:Ass1}) we have: 
%
\begin{align*}
	& \m{reflected}(a, s_1 \composeL r_1, \lmod_1(\ca{})) |/\\
	& \neg\m{visible}(a, s_1) /| \for{(s'', r'') \in a[s_1, r_1]} \extendsAMUpto{\lmod, \gmod}{(n-1)}{s''}{r''}{\lmod_1} 
\end{align*}
%(\ref{LSC2:Ass})
There are two cases to consider:\\
%
\noindent\textbf{Case 1. }
$\neg\m{visible}(a, s_1) /| \for{(s'', r'') \in a[s_1, r_1]} \extendsAMUpto{\lmod, \gmod}{(n-1)}{s''}{r''}{\lmod_1}$\\
%
From the assumption of the case and (\ref{LSC2:Ass8}) we have
%
\begin{align}
	\extendsAMUpto{\lmod, \gmod}{(n-1)}{s''}{r''}{\lmod_1}
	\label{LSC2:Ass11}
\end{align}
%
Consequently, from (\ref{LSC2:Ass7}), (\ref{LSC2:Ass9}), (\ref{LSC2:Ass10}), (\ref{LSC2:Ass11}) and (\ref{LSC2:IH}) we have: 
%
\begin{align*}
	\extendsAMUpto{\lmod \cup \lmod_2, \gmod}{(n-1)}{s'}{r'}{\lmod_0}
\end{align*}
%
as required.\\

\noindent\textbf{Case 2. }
$\m{reflected}(a, s_1 \composeL r_1, \lmod_1(\ca{}))$\\
%(\ref{LSC2:Ass})
From (\ref{LSC2:Ass4}) and (\ref{LSC2:Ass5}) we have $\m{potential}(a, s_1 \composeL r_1)$ and consequently from the definition of $\m{potential}$ we know there exists $l \in \LStates$ such that $\fst{a} \leq s_1 \composeL r_1 \composeL l$. Thus from the assumption of the case and the definition of $\m{reflected}$ we have:
%
\begin{align}
	\exsts{a' \in \lmod_1(\ca{}) } \updateFP{a} = \updateFP{a'} /| \fst{a'} \leq s_1 \composeL r_1 \composeL l
	\label{LSC2:Ass12}
\end{align}
%
From (\ref{LSC2:Ass4}) and (\ref{LSC2:Ass5}) we have $\m{potential}(a, s_1 \composeL r_1)$. Consequently, from (\ref{LSC2:Ass12}) and the definition of $\m{potential}$ we have:
%
\begin{align}
	\m{potential}(a', s_1 \composeL r_1)
	\label{LSC2:Ass13}
\end{align}
%
On the other hand, from (\ref{LSC2:Ass8}), (\ref{LSC2:Ass12}) and the definition of $a'[s_1, r_1]$ we know $(s'', r'') \in a'[s_1, r_1]$. Thus from (\ref{LSC2:Ass1}), (\ref{LSC2:Ass12}) and (\ref{LSC2:Ass13}) we have:
%
\begin{align}
	\extendsAMUpto{\lmod, \gmod}{(n-1)}{s''}{r''}{\lmod_1}
	\label{LSC2:Ass14}
\end{align}
%
Finally, from (\ref{LSC2:Ass7}), (\ref{LSC2:Ass9}), (\ref{LSC2:Ass10}), (\ref{LSC2:Ass14}) and (\ref{LSC2:IH}) we have: 
%
\begin{align*}
	\extendsAMUpto{\lmod \cup \lmod_2, \gmod}{(n-1)}{s'}{r'}{\lmod_0}
\end{align*}
%
as required.\\
%
%
%
%

\noindent\textbf{RTS. (\ref{LSC2:Goal2})} \\
Pick an arbitrary $\ca{}$ and $a \in \lmod_0(\ca{})$ such that
%
\begin{align*}
	\m{enabled}(a, s_0 \composeL r_0) 
\end{align*}
%(\ref{LSC2:Ass})
Then from (\ref{LSC2:Ass3}) we have:
%
\begin{align*}
	(s_0 \composeL r_0, a[s_0 \composeL r_0]) \in \gmod(\ca{})
\end{align*}
%
as required.\\
%
%
%
%

\noindent\textbf{RTS. (\ref{LSC2:Goal3})} \\
Pick an arbitrary $\ca{}$ and $a \in \left(\lmod \cup \lmod_2 \right)(\ca{})$ such that
%
\begin{align}
	\m{potential}(a, s_0 \composeL r_0) \label{LSC2:Ass25}
\end{align}
%(\ref{LSC2:Ass})
There are two cases to consider:\\

\noindent\textbf{Case 1. } $a \in \lmod(\ca{})$\\
Then from (\ref{LSC2:Ass3}) and assumption of case 1. we have:
%
\begin{align*}
&\begin{array}{l}
	\m{reflected}(a, s_0 \composeL r_0, \lmod_0(\ca{})) |/ \\
	\neg\m{visible}(a, s_0) /| \for{(s', r') \in a[s_0, r_0]} \extendsAMUpto{\lmod, \gmod}{(n-1)}{s'}{r'}{\lmod_0}
\end{array}
\end{align*}
%
In the case of the first disjunct the desired result holds trivially. On the other hand, in the case of the second disjunct we have:
%
\begin{align}
	\neg\m{visible}(a, s_0) /| \for{(s', r') \in a[s_0, r_0]} \extendsAMUpto{\lmod, \gmod}{(n-1)}{s'}{r'}{\lmod_0} \label{LSC2:Ass17}
\end{align}
%
Pick an arbitrary $s', r'$ such that 
%
\begin{align}
	(s', r') \in a[s_0, r_0]
	\label{LSC2:Ass26}
\end{align}
%(\ref{LSC2:Ass})
Then from (\ref{LSC2:Ass3}) and (\ref{LSC2:Ass25})-(\ref{LSC2:Ass26}) we have:
%
\begin{align}
	\extendsAMUpto{\lmod, \gmod}{(n-1)}{s'}{r'}{\lmod_0} \label{LSC2:Ass27}
\end{align}
%
From (\ref{LSC2:Ass26}) and the definition of $a[s_0, r_0]$, we know that $\fst{\updateFP{a}} \leq s_0 \composeL r_0$ and consequently from (\ref{LSC2:Ass4}) we have $\fst{\updateFP{a}} \leq s_1 \composeL r_1$. Thus from \lem~\ref{lem:divideUpper} we know there exists $p_s, p_r \in \LStates$ such that : 
%
\begin{align*}
	\fst{\updateFP{a}} = p_s \composeL p_r /| p_s \leq s_1 /| p_r \leq r_1
\end{align*}
%
From the definition of $a[s_1, r_1]$ we then have
%
\begin{align*}
	& (p_s = \unitL /| (s_1, \snd{\updateFP{a}} \composeL (r_1 -p_r)) \in a[s_1, r_1])\\
	|/ & (p_s > \unitL /| (\snd{\updateFP{a}} \composeL (s_1 - p_s), r_1 - p_r) \in a[s_1, r_1]) 
\end{align*}
%
That is, there exists $s'', r'' \in \LStates$ such that
%
\begin{align}
	(s'', r'') \in a[s_1, r_1]
	\label{LSC2:Ass28}
\end{align}
%
From (\ref{LSC2:Ass2}), (\ref{LSC2:Ass28}) and \lem~\ref{lem:shift-fence} we have:
%
\begin{align}
	\lmod_1 \weakenI{\{s''\}} \lmod_2 \label{LSC2:Ass29}
\end{align}
%
From (\ref{LSC2:Ass4}), (\ref{LSC2:Ass26}), (\ref{LSC2:Ass28}) and \lem~\ref{lem:action-application} we have:
%
\begin{align}
	s' \composeL r' = s'' \composeL r'' \label{LSC2:Ass30}
\end{align}
%
Since $a \in \lmod(\ca{})$ (assumption of case 1), from (\ref{LSC2:Ass28}) and (\ref{LSC2:Ass1}) we have: 
%
\begin{align*}
	& \m{reflected}(a, s_1 \composeL r_1, \lmod_1(\ca{})) |/\\
	& \neg\m{visible}(a, s_1) /| \for{(s'', r'') \in a[s_1, r_1]} \extendsAMUpto{\lmod, \gmod}{(n-1)}{s''}{r''}{\lmod_1} 
\end{align*}
%(\ref{LSC2:Ass})
There are two cases to consider:\\
%
\noindent\textbf{Case 1. }
$\neg\m{visible}(a, s_1) /| \for{(s'', r'') \in a[s_1, r_1]} \extendsAMUpto{\lmod, \gmod}{(n-1)}{s''}{r''}{\lmod_1}$\\
%
From the assumption of the case and (\ref{LSC2:Ass28}) we have
%
\begin{align}
	\extendsAMUpto{\lmod, \gmod}{(n-1)}{s''}{r''}{\lmod_1}
	\label{LSC2:Ass31}
\end{align}
%
Consequently, from (\ref{LSC2:Ass27}), (\ref{LSC2:Ass29}), (\ref{LSC2:Ass30}), (\ref{LSC2:Ass31}) and (\ref{LSC2:IH}) we have: 
%
\begin{align*}
	\extendsAMUpto{\lmod \cup \lmod_2, \gmod}{(n-1)}{s'}{r'}{\lmod_0}
\end{align*}
%
as required.\\

\noindent\textbf{Case 2. }
$\m{reflected}(a, s_1 \composeL r_1, \lmod_1(\ca{}))$\\
%(\ref{LSC2:Ass})
From (\ref{LSC2:Ass4}) and (\ref{LSC2:Ass25}) we have $\m{potential}(a, s_1 \composeL r_1)$ and consequently from the definition of $\m{potential}$ we know there exists $l \in \LStates$ such that $\fst{a} \leq s_1 \composeL r_1 \composeL l$. Thus from the assumption of the case and the definition of $\m{reflected}$ we have:
%
\begin{align}
	\exsts{a' \in \lmod_1(\ca{}) } \updateFP{a} = \updateFP{a'} /| \fst{a'} \leq s_1 \composeL r_1 \composeL l
	\label{LSC2:Ass32}
\end{align}
%
From (\ref{LSC2:Ass4}) and (\ref{LSC2:Ass25}) we have $\m{potential}(a, s_1 \composeL r_1)$. Consequently, from (\ref{LSC2:Ass32}) and the definition of $\m{potential}$ we have:
%
\begin{align}
	\m{potential}(a', s_1 \composeL r_1)
	\label{LSC2:Ass33}
\end{align}
%
On the other hand, from (\ref{LSC2:Ass28}), (\ref{LSC2:Ass32}) and the definition of $a'[s_1, r_1]$ we know $(s'', r'') \in a'[s_1, r_1]$. Thus from (\ref{LSC2:Ass1}), (\ref{LSC2:Ass32}) and (\ref{LSC2:Ass33}) we have:
%
\begin{align}
	\extendsAMUpto{\lmod, \gmod}{(n-1)}{s''}{r''}{\lmod_1}
	\label{LSC2:Ass34}
\end{align}
%
Finally, from (\ref{LSC2:Ass27}), (\ref{LSC2:Ass29}), (\ref{LSC2:Ass30}), (\ref{LSC2:Ass34}) and (\ref{LSC2:IH}) we have: 
%
\begin{align*}
	\extendsAMUpto{\lmod \cup \lmod_2, \gmod}{(n-1)}{s'}{r'}{\lmod_0}
\end{align*}
%
as required.
%
\end{proof}
%
\end{lemma}
%
%
%
\begin{lemma}[action-application]\label{lem:action-application}
%
For all $a \in \LStates \times \LStates$ and $s_1$ , $r_1$, $s_2$, $r_2$, $s'_1$, $r'_1$, $s'_2$, $r'_2 \in \LStates$,
\[
	s_1 \composeL r_1 = s_2 \composeL r_2 /| (s'_1, r'_1) \in a[s_1, r_1] /| (s'_2, r'_2) \in a[s_2, r_2] \implies s'_1 \composeL r'_1 = s'_2 \composeL r'_2 
\]
%
\begin{proof}
Take arbitrary $a \in \LStates \times \LStates$ and $s_1$ , $r_1$, $s_2$, $r_2$, $s'_1$, $r'_1$, $s'_2$, $r'_2 \in \LStates$ such that 
%
\begin{align}
	& s_1 \composeL r_1 = s_2 \composeL r_2 \label{LAA:Ass1}\\
	& (s'_1, r'_1) \in a[s_1, r_1] \label{LAA:Ass2}\\
	& (s'_2, r'_2) \in a[s_2, r_2] \label{LAA:Ass3}
\end{align}
%
Then from (\ref{LAA:Ass2}), and the definitions of $a[s_1, r_1]$ and $a[s_1 \composeL r_1]$ we have:
%
\begin{align}
	a[s_1 \composeL r_1] = s'_1 \composeL r'_1 \label{LAA:Ass4}
\end{align}
%
Similarly, from (\ref{LAA:Ass3}) we have:
%
\begin{align}
	a[s_2 \composeL r_2] = s'_2 \composeL r'_2 \label{LAA:Ass5}
\end{align}
%
Finally, from (\ref{LAA:Ass1}), (\ref{LAA:Ass4}) and (\ref{LAA:Ass5}) we have:
%
\begin{align*}
	s'_1 \composeL r'_1 = s'_2 \composeL r'_2 
\end{align*}
%
as required.
\end{proof}
%
\end{lemma}
%
%

	\newpage	
%
\begin{lemma}[\extendRule-Containment]\label{lem:extend-containment}
Given any two action models $\lmod, \lmod' \in \AMods$, 
\[
\begin{array}{l l}
	\for{n \in \Nats} \for{l \in \LStates} \isContainedAM{\lmod}{n}{l}{\lmod \cup \lmod'}
\end{array}
\]
%
\begin{proof}
Pick an arbitrary $\lmod, \lmod' \in \AMods$. We then proceed by induction on $n$.\\

\noindent\textbf{Base case $n = 0$}\\
Pick an arbitrary $l \in \LStates$. From the definition of $\leq_0$ we then trivially have $\isContainedAM{\lmod}{0}{l}{\lmod \cup \lmod'}$ as required.\\

\noindent\textbf{Inductive case}\\
Pick an arbitrary $n \in \Nats^{+}$ such that
%
\begin{align}
	\for{l' \in \LStates} \isContainedAM{\lmod}{n-1}{l'}{\lmod \cup \lmod'} \label{ECon:IH} \tag{I.H.}
\end{align}
%
Pick an arbitrary $l \in \LStates$. We are then required to show 
%
\begin{align*}
	\for{\ca{}} \for{a \in \lmod(\ca{})} \m{reflected}(a, l, (\lmod \cup \lmod')(\ca{})) \land \isContainedAM{\lmod}{n-1}{a[l]}{\lmod \cup \lmod'} 
\end{align*}
%
Pick an arbitrary $\ca{}$ and $a \in \lmod(\ca{})$. From the definition of $\m{reflected}$ we trivially have:
%
\begin{align*}
	\m{reflected}(a, l, (\lmod \cup \lmod')(\ca{}))
\end{align*}
%
If $a[l]$ is undefined then we vacuously have $\isContainedAM{\lmod}{n-1}{a[l]}{\lmod \cup \lmod'}$. On the other hand, if $a[l]$ is defined then from (\ref{ECon:IH}) we have $\isContainedAM{\lmod}{n-1}{a[l]}{\lmod \cup \lmod'}$ as required.
\end{proof}
%

\end{lemma}
%
%%
%\begin{lemma}[\extendRule-Containment]\label{lem:extend-containment2}
%Given any two action models $\lmod, \lmod' \in \AMods$, 
%\[
%\begin{array}{l l}
%	\for{n \in \Nats} \for{l, r \in \LStates} \isContainedAM{\lmod}{n}{l}{\lmod'} \land l \compatL r \implies \isContainedAM{\lmod}{n}{l \composeL r}{\lmod'}
%\end{array}
%\]
%%
%\begin{proof}
%Pick an arbitrary $\lmod, \lmod' \in \AMods$. We then proceed by induction on $n$.\\
%
%\noindent\textbf{Base case $n = 0$}\\
%Pick an arbitrary $l, r \in \LStates$ such that $ \isContainedAM{\lmod}{0}{l}{\lmod'}$ and $l \compatL r$. From the definition of $\leq_0$ we then trivially have $\isContainedAM{\lmod}{0}{l \composeL r}{\lmod'}$ as required.\\
%
%\noindent\textbf{Inductive case}\\
%Pick an arbitrary$n \in \Nats^{+}$ and  $l, r \in \LStates$ such that
%%
%\begin{align}
%	l \compatL r \land \isContainedAM{\lmod}{n}{l}{\lmod'} \label{ECon2:Ass1}
%	\for{l', r' \in \LStates} \isContainedAM{\lmod}{n-1}{l'}{\lmod'} \land l' \compatL r' \implies \isContainedAM{\lmod}{n-1}{l' \composeL r'}{\lmod'} \label{ECon2:IH} \tag{I.H.}
%\end{align}
%%
%\textbf{RTS.}
%%
%\begin{align*}
%	\for{\ca{}} \for{a \in \lmod(\ca{})} \m{reflected}(a, l \composeL r, \lmod'(\ca{})) \land \isContainedAM{\lmod}{n-1}{a[l \composeL r]}{\lmod'} 
%\end{align*}
%%
%Pick an arbitrary $\ca{}$ and $a = (p, q, c) \in \lmod(\ca{})$. From the definition of $\m{reflected}$ we trivially have:
%%
%\begin{align*}
%	\m{reflected}(a, l, (\lmod \cup \lmod')(\ca{}))
%\end{align*}
%%
%If $a[l]$ is undefined then we vacuously have $\isContainedAM{\lmod}{n-1}{a[l]}{\lmod \cup \lmod'}$. On the other hand, if $a[l]$ is defined then from (\ref{ECon:IH}) we have $\isContainedAM{\lmod}{n-1}{a[l]}{\lmod \cup \lmod'}$ as required.
%\end{proof}
%%
%
%\end{lemma}
%%
%
\begin{lemma}[\extendRule-Closure-1]\label{lem:extend-closure}
%
For all $\lmod, \lmod_{e}, \lmod_{0} \in \AMods$ such that $\for{\ca{} \in \dom{\lmod_0}} \lmod_0(\ca{}) = \emptyset$; and for all $g, s_e \in \LStates$ and $\fence{}, \fence{e} \in \pset{\LStates}$,
\[
\begin{array}{l l}
	g \in \fence{} /| \fence{} \strictfences \lmod /| s_e \in \fence{e} /| \fence{e} \strictfences \lmod_e \cup \lmod_0
	\implies  \extendsAM{\lmod \cup \lmod_{e} \cup \lmod_0}{s_{e}}{g}{\lmod_{e}}
\end{array}
\]
%
%where
%%
%\[
%\begin{array}{@{} l  l @{}}
%	\left( (\lmod_1, \fence{1}) + (\lmod_{2}, \fence{2}) \right)(\ca{}) \eqdef  & 
%	\left\{
%		(f', a[f']) \;\;\middle|
%		\begin{array}{ l @{}}
%			\left(a \in \lmod_{1}(\ca{}) |/ a \in \lmod_{2}(\ca{})  \right) /| \\
%			f' \in \fence{1} \composeL \fence{2} /| \m{enabled}(a, f')	
%		\end{array}		  
%	\right\}\\
%\end{array}
%\]
%%
%and
%%
%\[
%\begin{array}{@{} l l @{}}
%	\fence{1} \composeL \fence{2} \eqdef & 
%	\left\{
%		f_{1} \composeL f_{2} \;\;\middle|\;\; 
%		f_{1} \in \fence{1} /| f_{2} \in \fence{2}
%%		\begin{array}{l @{\hspace{5pt}} l @{}}
%%			\exsts{\fence{}, \fence{e}} & g \in \fence{} /| f \in \fence{} /| \fence{} \strictfences \lmod /| \\
%%			& s_{e} \in \fence{e} /| f_{e} \in \fence{e} /| \fence{e} \strictfences \lmod_{e} 
%%		\end{array}
%	\right\}
%\end{array}
%\]
%%
\begin{proof} Pick an arbitrary $\lmod, \lmod_e, \lmod_0 \in \AMods$, $g, s_e \in \LStates$ and $\fence{}, \fence{e} \in \pset{\LStates}$ such that 
%
\begin{align}
	& \for{\ca{} \in \dom{\lmod_0}} \lmod_0(\ca{}) = \emptyset \label{EC:Ass0}\\
	& g \in \fence{} /|  s_e \in \fence{e}   \label{EC:Ass1}\\
	& \fence{} \strictfences \lmod /| \fence{e} \strictfences \lmod_e \cup \lmod_0 \label{EC:Ass2}
\end{align} 
%
From the definition of $\downarrow$, it then suffices to show
%
\begin{align}
	& \for {n \in \Nats} \isContainedAM{\lmod_e}{n}{s_e \composeL g}{\lmod \cup \lmod_e \cup \lmod_0} \label{EC:Goal1}\\
	& \for {n \in \Nats}  \extendsAMUpto{\lmod \cup \lmod_e}{n}{s_e}{g}{\lmod_e} \label{EC:Goal2}
\end{align}
%
\noindent\textbf{RTS. (\ref{EC:Goal1})} \\
This follows immediately from \lem~\ref{lem:extend-containment}.\\

\noindent\textbf{RTS. (\ref{EC:Goal2})} \\
Rather than proving (\ref{EC:Goal2}) directly, we first establish the following.
%
\begin{align}
	& \for {n \in \Nats} \for{g, s_e \in \LStates} \nonumber\\
	& \;\; g \in \fence{} /| s_e \in \fence{e} \implies \extendsAMUpto{\lmod \cup \lmod_e \cup \lmod_0}{n}{s_e}{g}{\lmod_e} \label{EC:Goal3}
\end{align}
%
We can then despatch (\ref{EC:Goal2}) from (\ref{EC:Ass1}) and (\ref{EC:Goal3}); since for an arbitrary $n \in \Nats$, from (\ref{EC:Ass1}) and (\ref{EC:Goal3}) we have $\extendsAMUpto{\lmod \cup \lmod_e \cup \lmod_0}{n}{s_e}{g}{\lmod_e}$ as required. \\

\noindent\textbf{RTS. (\ref{EC:Goal3})} \\
We proceed by induction on the number of steps $n$.\\

%\noindent Pick an arbitrary $s_1, s_2, r \in \LStates, \lmod, \lmod', \gmod \in \AMods$.\\
\noindent\textbf{Base case }$n=0$\\
Pick an arbitrary $g, s_e \in \LStates$. We are then required to show	$\extendsAMUpto{\lmod \cup \lmod_e \cup \lmod_0}{0}{s_e}{g}{\lmod_e}$ which follows trivially from the definition of $\downarrow_0$.\\


\noindent\textbf{Inductive case }\\
Pick an arbitrary $n \in \Nats$ and $g, s_e \in \LStates$ such that
%
\begin{align}
	& g \in  \fence{} \label{LEC:Ass1}\\
	& s_e \in \fence{e} \label{LEC:Ass2}\\
%	& \gmod = (\lmod, g) + (\lmod_{e}, s_e) \label{LEC:Ass3}\\
	& \tag{I.H} \for{g', s_e'}  g' \in \fence{} /| s'_e \in \fence{e} \implies \extendsAMUpto{\lmod \cup \lmod_{e} \cup \lmod_0}{(n-1)}{s_e'}{g'}{\lmod_{e}} \label{LEC:IH}
\end{align}
%
\textbf{RTS.}
%
\begin{align}
%	& 
%	\V{\ca{}}  \V{a\in \lmod_e(\ca{})} \nonumber \\
%  &\quad (\m{potential}(a, s_e \composeL g)  => \nonumber\\
%  & \quad\qquad\for{(s', r') \in a[s_e, g]} \extendsAMUpto{\lmod \cup \lmod_e \cup \lmod_0, \gmod}{(n-1)}{s'}{r'}{\lmod_e}) \label{LEC:Goal1}\\
%%  
%  &\quad \m{enabled}(a,s_e \composeL g)
%  => (s_e \composeL g, a[s_e \composeL g])\in \gmod(\ca{})) \label{LEC:Goal2}\\
%  
  &\V{\ca{}}\V{a\in \left(\lmod \cup \lmod_e \cup \lmod_0 \right) (\ca{})}
  \m{potential}(a,s_e \composeL g) =>\null \nonumber \\
%  &\ \m{reflected}(a, s_e \composeL g,\lmod_e(\ca{})) |/\null \nonumber \\
%%  
%  &\ \neg\m{visible}(a, s_e) /| \for{(s', r') \in a[s_e, g]} \extendsAMUpto{\lmod \cup \lmod_e \cup \lmod_0, \gmod}{(n-1)}{s'}{r'}{\lmod_e}  
	& \quad
	\begin{array}{@{} l @{}}
		\left(\m{reflected}(a,s_e \composeL g, \lmod_e(\ca{})) |/ \neg\m{visible}(a, s_e) \ \right) /| \\
		\for{(s', r') \in a[s_e, g]} \extendsAMUpto{\lmod \cup \lmod_e \cup \lmod_0}{(n-1)}{s'}{r'}{\lmod_e}
 	\end{array} 
  \label{LEC:Goal3}
\end{align}
%
%

%\noindent\textbf{RTS. \ref{LEC:Goal1}}\\

%Since from (\ref{LEC:Ass5}) and the definition of $\m{potential}$ we have $s_e \composeL g \meetL \fst{a} \not= \emptyset$ and consequently, $s_e \meetL \fst{a} \not= \emptyset$, from (\ref{LEC:Ass2}) we have:
%%
%\begin{align}
%	\fst{\updateFP{a}} \leq s_e /| \fst{\updateFP{a}} \disjoint g\label{LEC:Ass7}
%\end{align}
%% 
%From (\ref{LEC:Ass5}), (\ref{LEC:Ass7}) and the definition of $\m{potential}$ we have:
%%
%\begin{align}
%	\m{potential}(a, s_e) \label{LEC:Ass8}
%\end{align}
%%
%On the other hand, from (\ref{LEC:Ass6}), (\ref{LEC:Ass7}) and the definitions of $a[s_e, g]$ and $\disjoint$, we have: 
%%
%\begin{align}
%	& r' = g \label{LEC:Ass9}\\
%	& a[s_e] = s' \label{LEC:Ass10}
%\end{align}
%%
%Consequently, from (\ref{EC:Ass2}), (\ref{LEC:Ass2}), (\ref{LEC:Ass8}), (\ref{LEC:Ass10}) and the definition of $\strictfences$ we have:
%%
%\begin{align}
%	s' \in  \fence{e}  \label{LEC:Ass11}
%\end{align}
%%
%Finally, from (\ref{LEC:Ass1}), (\ref{LEC:Ass11}) and (\ref{LEC:IH}) we have:
%%
%\begin{align*}
%	\extendsAMUpto{\lmod \cup \lmod_{e} \cup \lmod_0, \gmod}{(n-1)}{s'}{g}{\lmod_e}
%\end{align*}
%%
%and consequently from (\ref{LEC:Ass9})
%%
%\begin{align*}
%	\extendsAMUpto{\lmod \cup \lmod_{e} \cup \lmod_0, \gmod}{(n-1)}{s'}{r'}{\lmod_e}
%\end{align*}
%%
%as required.\\
%%
%%
%%
%%
%%
%%
%%
%%

\noindent\textbf{RTS. \ref{LEC:Goal3}}\\
Pick an arbitrary $\ca{}$, $a = (p, q, c) \in \left(\lmod \cup \lmod_e \cup \lmod_0 \right)(\ca{})$ and $(s', r')$ such that
%
\begin{align}
	& \m{potential}(a, s_e \composeL g)  \label{LEC:ass1}\\
	& (s', r') \in a[s_e, g] \label{LEC:ass2}
\end{align}
%(\ref{LEC:Ass})
Since from (\ref{EC:Ass0}) we know $\lmod_0(\ca{}) = \emptyset$ we know $a \in \lmod_e(\ca{}) |/  a \in\lmod(\ca{})$, and thus there are two cases to consider. \\

%If the first disjunct is the case, the desired result holds trivially since from the definition of $\m{reflected}$ and (\ref{LEC:Ass25}) we have $\m{reflected}(a, s_e \composeL g, \lmod_e(\ca{}))$ as required.
\noindent\textbf{Case 1. } $a \in \lmod(\ca{})$\\ 
From (\ref{LEC:ass1}) and the definition of $\m{potential}$ we have $s_e \composeL g \meetL p \composeL c \not= \emptyset$ and consequently, $g \meetL p \composeL c \not= \emptyset$, from (\ref{LEC:Ass1}) we then have:
%
\begin{align}
	p \leq g /| p \disjoint s_e \label{LEC:ass3}
\end{align}
% 
From (\ref{LEC:ass1}), (\ref{LEC:ass3}) and the definitions of $\m{potential}$ we have:
%
\begin{align}
	\m{potential}(a, g) \label{LEC:ass4}
\end{align}
%
From (\ref{LEC:ass3}) and the definition of $\m{visible}$ we have:
%
\begin{align}
	\neg\m{visible}(a, s_e) 
	\label{LEC:ass5}
\end{align}
%
On the other hand, from (\ref{LEC:ass2}), (\ref{LEC:ass3}) and the definitions of $a[s_e, g]$, $a[g]$ and $\disjoint$, we know: 
%
\begin{align}
	& s' = s_e  \label{LEC:ass5}\\
	& a[g] = r'  \label{LEC:ass6}
\end{align}
%
Consequently, from (\ref{EC:Ass2}), (\ref{LEC:Ass1}), (\ref{LEC:ass4}),  (\ref{LEC:ass6}) and the definition of $\strictfences$ we have:
%
\begin{align}
	r' \in \fence{}  \label{LEC:ass7}
\end{align}
%
Finally, from (\ref{LEC:Ass2}), (\ref{LEC:ass7}), (\ref{LEC:IH}) we have:
%
\begin{align}
	\extendsAMUpto{\lmod \cup \lmod_{e} \cup \lmod_0}{(n-1)}{s_e}{r'}{\lmod_e}
	\label{LEC:ass8}
\end{align}
%
and consequently from (\ref{LEC:ass5}) and (\ref{LEC:ass2})-(\ref{LEC:ass8}) we have
%
\begin{align*}
	& \neg\m{visible}(a, s_e) /| \\
	& \for{(s', r') \in a[s_e, g]} \extendsAMUpto{\lmod \cup \lmod_{e} \cup \lmod_0}{(n-1)}{s_e}{r'}{\lmod_e}
\end{align*}
%
as required.\\
%
%
%
%

\noindent\textbf{Case 2. }$a \in \lmod_e(\ca{})$\\
From the assumption of the case and the definition of $\m{reflected}$ we trivially have
%
\begin{align}
	\m{reflected}(a, s_e \composeL g, \lmod_e(\ca{})
	\label{LEC:ass10}
\end{align}
%
Since from (\ref{LEC:ass1}) and the definition of $\m{potential}$ we have $s_e \composeL g \meetL p \composeL c \not= \emptyset$ and consequently, $s_e \meetL p \composeL c \not= \emptyset$, from (\ref{LEC:Ass2}) we have:
%
\begin{align}
	p \leq s_e /| p \disjoint g\label{LEC:ass11}
\end{align}
% 
From (\ref{LEC:ass1}), (\ref{LEC:ass11}) and the definition of $\m{potential}$ we have:
%
\begin{align}
	\m{potential}(a, s_e) \label{LEC:ass12}
\end{align}
%
On the other hand, from (\ref{LEC:ass2}), (\ref{LEC:ass11}) and the definitions of $a[s_e, g]$ and $\disjoint$, we have: 
%
\begin{align}
	& r' = g \label{LEC:ass13}\\
	& a[s_e] = s' \label{LEC:ass14}
\end{align}
%
Consequently, from (\ref{EC:Ass2}), (\ref{LEC:Ass2}), (\ref{LEC:ass12}), (\ref{LEC:ass14}) and the definition of $\strictfences$ we have:
%
\begin{align}
	s' \in  \fence{e}  \label{LEC:ass15}
\end{align}
%
Finally, from (\ref{LEC:Ass1}), (\ref{LEC:ass15}) and (\ref{LEC:IH}) we have:
%
\begin{align*}
	\extendsAMUpto{\lmod \cup \lmod_{e} \cup \lmod_0}{(n-1)}{s'}{g}{\lmod_e}
\end{align*}
%
and consequently from (\ref{LEC:ass13})
%
\begin{align}
	\extendsAMUpto{\lmod \cup \lmod_{e} \cup \lmod_0}{(n-1)}{s'}{r'}{\lmod_e}
	\label{LEC:ass16}
\end{align}
%
Thus from (\ref{LEC:ass2}) and (\ref{LEC:ass10})-(\ref{LEC:ass16}) we have:
%
\begin{align*}
	& \m{reflected}(a, s_e \composeL g, \lmod_e(\ca{})) /| \\
	& \for{(s', r') \in a[s_e, g]} \extendsAMUpto{\lmod \cup \lmod_e \cup \lmod_0}{(n-1)}{s'}{r'}{\lmod_e}
\end{align*}
%
as required.\\

\end{proof}
\end{lemma}
%
%
%
%
%
%
%
%
%
%
%

\begin{lemma}[\extendRule-closure-2]\label{lem:extend-closure-2}
For all $\lmod_0, \lmod{}, \lmod_{e} \in \AMods$, $s, g, s_e \in \LStates$ and $\fence{}, \fence{e} \in \pset{\LStates}$
%
\[
\begin{array}{l}
	g \in \fence{} /| \fence{} \strictfences \lmod /|
	s_e \in \fence{e} /| \fence{e} \strictfences \lmod_{e} /|
	\extendsAM{\lmod}{s}{g-s}{\lmod_0}\\
	\qquad\implies
	\extendsAM{\lmod \cup \lmod_{e}}{s}{(g-s) \composeL s_e}{\lmod_{0}}
\end{array}
\]
%
\begin{proof} Pick an arbitrary $\lmod_0, \lmod, \lmod_e \in \AMods$, $s, g, s_e \in \LStates$ and $\fence{}, \fence{e} \in \pset{\LStates}$ such that 
%
\begin{align}
	& g \in \fence{} /| s_e \in \fence{e}  \label{EC2:Ass1}\\
	& \fence{} \strictfences \lmod /| \fence{e} \strictfences \lmod{e} \label{EC2:Ass2}\\
	& \extendsAM{\lmod}{s}{g-s}{\lmod_0} \label{EC2:Ass3}
\end{align} 
%
From the definition of $\downarrow$, it then suffices to show
%
\begin{align}
	& \for{n \in \Nats} \isContainedAM{\lmod_0}{n}{g \composeL s_e}{\lmod \cup \lmod_2} \label{EC2:Goal1}\\
	& \for{n \in \Nats} \extendsAMUpto{\lmod \cup \lmod_e}{n}{s}{(g-s) \composeL s_e}{\lmod_0} \label{EC2:Goal2}
\end{align}
%
\noindent\textbf{RTS. (\ref{EC2:Goal1})} \\
%Since from (\ref{EC2:Ass3}) and the definition of $\downarrow$ we have $\lmod_0 \subseteq \lmod$, we can consequently derive $\lmod_0 \subseteq \lmod \cup \lmod_e$ as required.\\
Rather than proving (\ref{EC2:Goal1}) directly, we first establish the following.
%
\begin{align}
	& \for {n \in \Nats} \for{g, s_e \in \LStates} \nonumber\\
	& \quad g \in \fence{} /|  s_e \in \fence{e} /| \isContainedAM{\lmod_0}{n}{g}{\lmod} \implies \nonumber\\
	& \qquad \isContainedAM{\lmod_0}{n}{g \composeL s_e}{\lmod \cup \lmod_e} \label{EC2G1:Goal3}
\end{align}
%
We can then despatch (\ref{EC2:Goal1}) from (\ref{EC2:Ass1}), (\ref{EC2:Ass3}) and (\ref{EC2G1:Goal3}); since for an arbitrary $n \in \Nats$, from (\ref{EC2:Ass3}) and the definition of $\leq_n$ we have $\isContainedAM{\lmod_0}{n}{g}{\lmod}$ and consequently from (\ref{EC2:Ass1}) and (\ref{EC2G1:Goal3}) we derive $\isContainedAM{\lmod_0}{n}{g \composeL s_e}{\lmod \cup \lmod_e} $ as required. \\

\noindent\textbf{RTS. (\ref{EC2G1:Goal3})} \\
We proceed by induction on the number of steps $n$.\\

%\noindent Pick an arbitrary $s_1, s_2, r \in \LStates, \lmod, \lmod', \gmod \in \AMods$.\\
\noindent\textbf{Base case }$n=0$\\
Pick an arbitrary $g, s_e \in \LStates$. We are then required to show	$\isContainedAM{\lmod_0}{0}{g \composeL s_e}{\lmod \cup \lmod_e}$ which follows trivially from the definition of $\leq_0$.\\


\noindent\textbf{Inductive case }\\
Pick an arbitrary $n \in \Nats$ and $g, s_e \in \LStates$ such that
%
\begin{align}
	& g \in \fence{} \label{LEC2G1:Ass1}\\
	& s_e \in \fence{e} \label{LEC2G1:Ass2}\\
	& \isContainedAM{\lmod_0}{n}{g}{\lmod} \label{LEC2G1:Ass4}\\
	& \for{g'', s_e''}  g'' \in \fence{} /| s''_e \in \fence{e} /| \isContainedAM{\lmod_0}{(n-1)}{g''}{\lmod} \nonumber \\
	& \tag{I.H} \qquad \implies \isContainedAM{\lmod_0}{(n-1)}{g''\composeL s_e''}{\lmod \cup \lmod_{e}} \label{LEC2G1:IH}
\end{align}
%
\textbf{RTS.}
%
\begin{align*}  
	& \quad
  \begin{array}{@{} l @{}}
  	\for{\ca{}}\for{a \in \lmod_0(\ca{})}
		\m{reflected}(a,g \composeL s_e, \lmod \cup \lmod_e(\ca{})) \land
		\isContainedAM{\lmod_0}{(n-1)}{a[g \composeL s_e]}{\lmod \cup \lmod_e}
 	\end{array} 
\end{align*}
%
Pick an arbitrary $\ca{}$, $a$ and $p, q, c$ such that
%
\begin{align}
	& a = (p, q, c) \in \lmod_0(\ca{}) \label{LEC2G1:Ass5}
\end{align}
%(\ref{LEC2G1:Ass})
Pick an arbitrary $l$ such that
%
\begin{align}
	& p \composeL c \leq g \composeL s_e \composeL l \label{LEC2G1:Ass6}
\end{align}
%(\ref{LEC2G1:Ass})
Then from (\ref{LEC2G1:Ass4}),(\ref{LEC2G1:Ass6}) we know there exists $a''$ and $c''$ such that
%
\begin{align}
	a'' = (p, q, c'') \in \lmod(\ca{}) \land p \composeL c'' \leq g \composeL s_e \composeL l \label{LEC2G1:Ass7}
\end{align}
%
and thus from (\ref{LEC2G1:Ass6}) and by definitions of $\lmod \cup \lmod_e$ and $\m{reflected}$ we have:
%
\begin{align*}
	\m{reflected}(a, g \composeL s_e, \lmod \cup \lmod_e(\ca{}))
\end{align*}
%
as required. \\
%
On the other hand, from (\ref{LEC2G1:Ass7}) and the definition of action application we have:
%
\begin{align}
	a''[g \composeL s_e] = a[g \composeL s_e] \land a''[g] = a[g] 
	\label{LEC2G1:Ass8}
\end{align}
%
From (\ref{LEC2G1:Ass1}), (\ref{LEC2G1:Ass7}) and (\ref{EC2:Ass2}) we know
%
\begin{align}
	a''[g] \in \fence{} \land p \leq g \label{LEC2G1:Ass9}
\end{align}
%
and consequently from the the definition of action application we have:
%
\begin{align}
	a''[g \composeL s_e] = a''[g] \composeL s_e \label{LEC2G1:Ass10}
\end{align}
%
From (\ref{LEC2G1:Ass4}) and (\ref{LEC2G1:Ass5}) and (\ref{LEC2G1:Ass8}) we have:
%
\begin{align}
	\isContainedAM{\lmod_0}{(n-1)}{a''[g]}{\lmod} \label{LEC2G1:Ass11}
\end{align}
%
Finally from (\ref{LEC2G1:Ass9}), (\ref{LEC2G1:Ass2}), (\ref{LEC2G1:Ass11}) and (\ref{LEC2G1:IH}) we have:
%
\begin{align*}
	\isContainedAM{\lmod_0}{(n-1)}{a''[g] \composeL s_e}{\lmod \cup \lmod_e}
\end{align*}
%
and consequently from (\ref{LEC2G1:Ass10}) and (\ref{LEC2G1:Ass8}) 
%
\begin{align*}
	\isContainedAM{\lmod_0}{(n-1)}{a[g \composeL s_e]}{\lmod \cup \lmod_e}
\end{align*}
%
as required.\\



\noindent\textbf{RTS. (\ref{EC2:Goal2})} \\
Rather than proving (\ref{EC2:Goal2}) directly, we first establish the following.
%
\begin{align}
	& \for {n \in \Nats} \for{s, g, s_e \in \LStates} \nonumber\\
	& \quad g \in \fence{} /|  s_e \in \fence{e} /| \extendsAMUpto{\lmod}{n}{s}{g-s}{\lmod_0} \implies \nonumber\\
	& \qquad \extendsAMUpto{\lmod \cup \lmod_e}{n}{s}{(g-s) \composeL s_e}{\lmod_0} \label{EC2:Goal3}
\end{align}
%
We can then despatch (\ref{EC2:Goal2}) from (\ref{EC2:Ass1})-(\ref{EC2:Ass3}) and (\ref{EC2:Goal3}); since for an arbitrary $n \in \Nats$, from (\ref{EC2:Ass3}) and the definition of $\downarrow$ we have $\extendsAMUpto{\lmod}{n}{s}{g-s}{\lmod_0}$ and consequently from (\ref{EC2:Ass1}) and (\ref{EC2:Goal3}) we derive $\extendsAMUpto{\lmod \cup \lmod_e}{n}{s}{(g-s) \composeL s_e}{\lmod_0} $ as required. \\

\noindent\textbf{RTS. (\ref{EC2:Goal3})} \\
We proceed by induction on the number of steps $n$.\\

%\noindent Pick an arbitrary $s_1, s_2, r \in \LStates, \lmod, \lmod', \gmod \in \AMods$.\\
\noindent\textbf{Base case }$n=0$\\
Pick an arbitrary $s, g, s_e \in \LStates$. We are then required to show	$\extendsAMUpto{\lmod \cup \lmod_e}{0}{s}{(g-s) \composeL s_e}{\lmod_0}$ which follows trivially from the definition of $\downarrow_0$.\\


\noindent\textbf{Inductive case }\\
Pick an arbitrary $n \in \Nats$ and $s, r, g, s_e \in \LStates$ such that
%
\begin{align}
	& g \in \fence{} \label{LEC2:Ass1}\\
	& s_e \in \fence{e} \label{LEC2:Ass2}\\
%	& \gmod' = (\lmod, g) + (\lmod_{e}, s_e) 
	& g = s \composeL r \label{LEC2:Ass3} \\
	& \extendsAMUpto{\lmod}{n}{s}{r}{\lmod_0} \label{LEC2:Ass4}\\
	& \for{s'', g'', s_e''}  g'' \in \fence{} /| s''_e \in \fence{e} /| \extendsAMUpto{\lmod}{(n-1)}{s''}{g''-s''}{\lmod_0} \nonumber \\
	& \tag{I.H} \qquad \implies \extendsAMUpto{\lmod \cup \lmod_{e}}{(n-1)}{s''}{(g''-s'') \composeL s_e''}{\lmod_{0}} \label{LEC2:IH}
\end{align}
%
\textbf{RTS.}
%
\begin{align*}
%	& 
%	\V{\ca{}}  \V{a\in \lmod_0(\ca{})} \nonumber \\
%  &\quad (\m{potential}(a, g \composeL s_e) => \nonumber\\
%  & \quad\qquad\for{(s', r') \in a[s, r \composeL s_e]} \extendsAMUpto{\lmod \cup \lmod_e, \gmod'}{(n-1)}{s'}{r'}{\lmod_0}) \label{LEC2:Goal1}\\
%%   
%  &\quad (\m{enabled}(a,g \composeL s_e)
%  => (g \composeL s_e, a[g \composeL s_e])\in \gmod'(\ca{})) \label{LEC2:Goal2}\\
%  
%  &\V{\ca{}}\V{a\in \left(\lmod \cup \lmod_e \right) (\ca{})}
%  \m{potential}(a,g \composeL s_e) =>\null \nonumber \\
%  &\ \m{reflected}(a, g \composeL s_e, \lmod_0(\ca{})) |/\null \nonumber \\
%%  
%  &\ \neg\m{visible}(a, s) /| \for{(s', r') \in a[s, r \composeL s_e]} \extendsAMUpto{\lmod \cup \lmod_e, \gmod'}{(n-1)}{s'}{r'}{\lmod_0} 
	& \quad
  \begin{array}{@{} l @{}}
		\left(\m{reflected}(a,g \composeL s_e, \lmod_0(\ca{})) |/ \neg\m{visible}(a, s) \ \right) /| \\
		\for{(s', r') \in a[s, r \composeL s_e]} \extendsAMUpto{\lmod \cup \lmod_e}{(n-1)}{s'}{r'}{\lmod_0}
 	\end{array} 
\end{align*}
%
%
Pick an arbitrary $\ca{}$, $a = (p, q, c) \in \left(\lmod \cup \lmod_e \right)(\ca{})$ and $(s', r')$ such that
%
\begin{align}
	& \m{potential}(a, g \composeL s_e) \label{LEC2:Ass5}\\
	& (s', r') \in a[s, r \composeL s_e] \label{LEC2:Ass6}
\end{align}
%(\ref{LEC2:Ass})
Since either $a \in \lmod(\ca{})$ or $a \in \lmod_e(\ca{})$, there are two cases to consider:\\

\noindent\textbf{Case 1. }$a \in \lmod(\ca{})$\\
Since from (\ref{LEC2:Ass5}) and the definition of $\m{potential}$ we have $s_e \composeL g \meetL p \composeL c \not= \emptyset$ and consequently, $g \meetL p \composeL c \not= \emptyset$, from (\ref{LEC2:Ass1}) we have:
%
\begin{align}
	p \leq g /| p \disjoint s_e \label{LEC2:Ass7}
\end{align}
% 
From (\ref{LEC2:Ass5}), (\ref{LEC2:Ass7}) and the definition of $\m{potential}$ we have:
%
\begin{align}
	\m{potential}(a, g) \label{LEC2:Ass8}
\end{align}
%
On the other hand, from (\ref{LEC2:Ass6}), (\ref{LEC2:Ass7}) and the definitions of $a[s, r \composeL s_e]$ and $\disjoint$, we know there exists $r''$: 
%
\begin{align}
	& r' = r'' \composeL s_e \label{LEC2:Ass9}\\
	& (s', r'') \in a[s, r]  \label{LEC2:Ass10}
\end{align}
%
From (\ref{LEC2:Ass10}) and the definitions of $a[s, r]$ and $a[s \composeL r]$, we know $s' \composeL r'' = a[s \composeL r]$. Consequently, from (\ref{EC2:Ass2}), (\ref{LEC2:Ass1}), (\ref{LEC2:Ass8}), the definition of $\strictfences$ and since $g = s \composeL r$ (\ref{LEC2:Ass3}), we have:
%
\begin{align}
	s' \composeL r'' \in \fence{}  \label{LEC2:Ass11}
\end{align}
%(\ref{LEC2:Ass})
On the other hand, from (\ref{LEC2:Ass4}), (\ref{LEC2:Ass8}), (\ref{LEC2:Ass5}) and (\ref{LEC2:Ass10}) we have:
%
\begin{align}
	& \left(\m{reflected}(a, s \composeL r, \lmod_0(\ca{}) ) |/ \neg\m{visible}(a, s) \right) /| \label{LEC2:Ass12}\\
	& \extendsAMUpto{\lmod}{(n-1)}{s'}{r''}{\lmod_0} \label{LEC2:Ass13}
\end{align}
%
From (\ref{LEC2:Ass2}), (\ref{LEC2:Ass11}), (\ref{LEC2:Ass12}) and (\ref{LEC2:IH}) we have:
%
\begin{align*}
	\extendsAMUpto{\lmod \cup \lmod_{e}}{(n-1)}{s'}{r'' \composeL s_e}{\lmod_0}
\end{align*}
%
and thus from (\ref{LEC2:Ass9})
%
\begin{align}
	\extendsAMUpto{\lmod \cup \lmod_{e}}{(n-1)}{s'}{r'}{\lmod_0}
	\label{LEC2:Ass14}
\end{align}
%
Consequently, from (\ref{LEC2:Ass6})-(\ref{LEC2:Ass14}) we have:
%
\begin{align}
	\for{(s', r') \in a[s, r \composeL s_e]} \extendsAMUpto{\lmod \cup \lmod_e}{(n-1)}{s'}{r'}{\lmod_0}
	\label{LEC2:Ass15}
\end{align}
%
From (\ref{LEC2:Ass12}) there are two cases to consider:\\
\textbf{Case 1.1. }$\neg\m{visible}(a, s)$\\
From (\ref{LEC2:Ass15}) and the assumption of the case we have:
%(\ref{LEC2:Ass})
\begin{align*}
	& \neg\m{visible}(a, s) /| \\
	& \for{(s', r') \in a[s, r \composeL s_e]} \extendsAMUpto{\lmod \cup \lmod_e}{(n-1)}{s'}{r'}{\lmod_0}
\end{align*}
% 
as required. \\
%
%
%

\noindent\textbf{Case 1.2. }$\m{reflected}(a, s \composeL r, \lmod_0(\ca{})$\\
Pick an arbitrary $l \in \LStates$ such that 
%(\ref{LEC2:Ass})
\begin{align}
	p \composeL c \leq g \composeL s_e \composeL l \label{LEC2:Ass16}
\end{align} 
%
From the assumption of the case, (\ref{LEC2:Ass3}) and the definition of $\m{reflected}$ we then have
%
\begin{align}
	\exsts{a', c'} a' = (p, q, c') \land a' \in \lmod_0(\ca{}) /| p \composeL c' \leq g \composeL s_e \composeL l \label{LEC2:Ass17}
\end{align}
%
Thus from (\ref{LEC2:Ass16}), (\ref{LEC2:Ass17}) and the definition of $\m{reflected}$ we have:
%
\begin{align}
	\m{reflected}(a, g \composeL s_e, \lmod_0(\ca{}))
	\label{LEC2:Ass18}
\end{align}
%
Thus from (\ref{LEC2:Ass15}) and (\ref{LEC2:Ass18}) we have:
%(\ref{LEC2:Ass})
\begin{align*}
	& \m{reflected}(a, g \composeL s_e, \lmod_0(\ca{})) /| \\
	& \for{(s', r') \in a[s, r \composeL s_e]} \extendsAMUpto{\lmod \cup \lmod_e}{(n-1)}{s'}{r'}{\lmod_0}
\end{align*}
%
as required.\\
%
%
%



%

\noindent\textbf{Case 2. } $a \in \lmod_e(\ca{})$\\
Since from (\ref{LEC2:Ass5}) and the definition of $\m{potential}$ we have $g \composeL s_e \meetL p \composeL c \not= \emptyset$ and consequently, $s_e \meetL p \composeL c \not= \emptyset$, from (\ref{LEC2:Ass2}) and the assumption of the case we have:
%
\begin{align}
	p \leq s_e /| p \disjoint g \label{LEC2:Ass26}
\end{align}
% 
and consequently from the definition of $\m{visible}$ and (\ref{LEC2:Ass3}) we have
%
\begin{align}
	\neg\m{visible}(a, s)
	\label{LEC2:not-visible}
\end{align}
%
From (\ref{LEC2:Ass5}), (\ref{LEC2:Ass26}) and the definition of $\m{potential}$ we have:
%
\begin{align}
	\m{potential}(a, s_e) \label{LEC2:Ass27}
\end{align}
%
From (\ref{LEC2:Ass3}), (\ref{LEC2:Ass26}) and the definitions of $a[s, r \composeL s_e]$ and $\disjoint$, we know there exists $s_e'$ such that: 
%
\begin{align}
	& s' = s /| r' = r \composeL s_e' \label{LEC2:Ass29}\\
	& a[s_e] = s_e'  \label{LEC2:Ass30}
\end{align}
%
Consequently, from (\ref{EC2:Ass2}), (\ref{LEC2:Ass2}), (\ref{LEC2:Ass27}), (\ref{LEC2:Ass30}) and the definition of $\strictfences$ we have:
%
\begin{align}
	s_e' \in \fence{e}  \label{LEC2:Ass31}
\end{align}
%
From (\ref{LEC2:Ass3}), (\ref{LEC2:Ass29}) and \lem~\ref{lem:future-closure} we have:
%
\begin{align}
	\extendsAMUpto{\lmod}{(n-1)}{s'}{r}{\lmod_0}  \label{LEC2:Ass32}
\end{align}
%
From (\ref{LEC2:Ass1}), (\ref{LEC2:Ass31}), (\ref{LEC2:Ass32}), (\ref{LEC:IH}) we have:
%
\begin{align*}
	\extendsAMUpto{\lmod \cup \lmod_{e}}{(n-1)}{s'}{r \composeL s_e'}{\lmod_0}
\end{align*}
%
and thus from (\ref{LEC2:Ass29}) 
%
\begin{align}
	\extendsAMUpto{\lmod \cup \lmod_{e}}{(n-1)}{s'}{r'}{\lmod_0}
	\label{LEC2:Ass33}
\end{align}
%
Finally, from (\ref{LEC2:Ass6}), (\ref{LEC2:not-visible}) and (\ref{LEC2:Ass33}) we have:
%
\begin{align*}
	&\neg\m{visible}(a, s) /|\\
	& \for{(s', r') \in a[s, r \composeL s_e]} \extendsAMUpto{\lmod \cup \lmod_{e}}{(n-1)}{s'}{r'}{\lmod_0}
\end{align*}
%
as required.

\end{proof}
\end{lemma}

	\begin{lemma}[]\label{lem:future-closure}
For all $l, r \in \LStates$, $\amod{}, \amod{L}, \amod{}' \in \AMods$ and $n \in \NatsPlus$
%
\[
	\extendsAMUpto{\amod{}, \amod{L}}{n}{l}{r}{\amod{}'} \implies \extendsAMUpto{\amod{}, \amod{L}}{(n-1)}{l}{r}{\amod{}'}
\]
%
\begin{proof} By induction on number of steps $n$.\\
Pick an arbitrary $l, r \in \LStates$, $\amod{}, \amod{L}, \amod{}' \in \AMods$ and $n \in \NatsPlus$.\\
\textbf{Base case: n = 1}\\
\textbf{RTS.} 
%
\begin{equation}
	\extendsAMUpto{\amod{}, \amod{L}}{0}{l}{r}{\amod{}'} \nonumber
\end{equation}
%
This holds trivially by definition of $\left(\amod{}, \amod{L}\right)\!\downarrow_{0}$\\

\noindent\textbf{Inductive case}\\
\textbf{Assume:}
%
\begin{align}
	\extendsAMUpto{\amod{}, \amod{L}}{(n+1)}{l}{r}{\amod{}'} \label{L9:Ass1}\\
	\extendsAMUpto{\amod{}, \amod{L}}{n}{l}{r}{\amod{}'} \implies \extendsAMUpto{\amod{}}{(n-1)}{l}{r}{\amod{}'} \tag{I.H} \label{L9:I.H}
\end{align}
%
\textbf{Show:}
%
\begin{align}
	&\begin{array}{l}
			\for{\ca{}, c, d, t, l_3, l_4} \for{(l_1 \composeL f, l_2 \composeL f) \in \amod{}'(\ca{})} \\
%			
			\hspace*{0.2cm}\left(\for{l'} l' \leq l_1 \;\land\; l' \leq l_2 \implies l' = \unitL \right) \;\land\; l_1 \composeL f \leq l \composeL r \composeL t \;\land\\
%			
			\hspace*{0.2cm} l_1 = l_3 \composeL l_4 \;\land\; 	l_1 \maxMeetL l = l_3 \;\land\; l = l_3 \composeL c \;\land\; r = l_4 \composeL d \implies \\
%			
			\hspace*{0.5cm}\extendsAMUpto{\amod{}, \amod{L}}{(n-1)}{l_2 \composeL c}{d}{\amod{}'} \;\land\\
			\hspace*{0.5cm} t = \unitL \implies (l_1 \composeL c \composeL d, l_2 \composeL c \composeL d) \in \amod{}(\ca{}) 
			\;\lor\; l_2 \composeL c \composeL d \text{ is undefined} 	
	\end{array}\label{L9:Goal1}\\
%
  &\begin{array}{l}
  	\for{\ca{}} \for{(l_1 \composeL f, l_2 \composeL f) \in \amod{L}(\ca{})} \for{c, d} \\
  \hspace*{0.2cm} l_1 \composeL f \composeL c = l \composeL r \composeL d \;\land\; \left(\for{l'} l' \leq l_0 \land l' \leq l'_0 \implies l' = \unitL\right)  \;\implies\\
  \hspace*{0.4cm}\exsts{f', c'} (l_1 \composeL f', l_2 \composeL f') \in \amod{}'(\ca{}) \;\land\; l_1 \composeL f' \composeL c' =  l \composeL r \composeL d\\
		\hspace*{0.4cm}\lor\;l_2 \composeL f \composeL c \text{ is undefined}\\
		\hspace*{0.4cm}\lor\; l_1 \maxMeetL l = \unitL \;\land\; \exsts{r'} l_2 \composeL f \composeL c = l \composeL r' \composeL d \;\land\; \extendsAMUpto{\amod{}, \amod{L}}{(n-1)}{l}{r'}{\amod{}'}
  \end{array} \label{L9:Goal2}
\end{align}
%

\noindent\textbf{RTS. (\ref{L9:Goal1})}\\
Pick an arbitrary $\ca{}, c, d, t, l_1, l_2, l_3, l_4, f$ such that
%
\begin{equation}
\hspace*{-0.15cm}
\begin{array}{l}
	(l_1 \composeL f, l_2 \composeL f) \in \amod{}'(\ca{}) \;\land\; 
	\left(\for{l'} l' \leq l_1 \land l' \leq l_2 \implies l' = \unitL \right)
	\;\land\; l_1 \composeL f \leq l \composeL r \composeL t\\
%
	\land\; l_1 = l_3 \composeL l_4 \;\land\; 	l_1 \maxMeetL l = l_3 \;\land\; l = l_3 \composeL c \;\land\; r = l_4 \composeL d
\end{array} \label{L9:Ass2}
\end{equation}
%
From (\ref{L9:Ass1}) and (\ref{L9:Ass2}) we have:
%
\begin{align}
	& \extendsAMUpto{\amod{}, \amod{L}}{n}{l_2 \composeL c}{d}{\amod{}'} \nonumber \;\land\\
	& t = \unitL \implies (l_1 \composeL c \composeL d, l_2 \composeL c \composeL d) \in \amod{}(\ca{}) \;\lor\; l_2 \composeL c \composeL d \text{ is undefined} \nonumber
\end{align}
%
From (\ref{L9:I.H}) we can rewrite the above as 
%
\begin{align}
	& \extendsAMUpto{\amod{}, \amod{L}}{(n-1)}{l_2 \composeL c}{d}{\amod{}'} \nonumber \;\land\\
	& t = \unitL \implies (l_1 \composeL c \composeL d, l_2 \composeL c \composeL d) \in \amod{}(\ca{}) \;\lor\; l_2 \composeL c \composeL d \text{ is undefined} \label{L9:Ass3}
\end{align}
%
as required.\\

\noindent\textbf{RTS. (\ref{L9:Goal2})}\\
Pick an arbitrary $\ca{}, l_1, l_2, f, c, d$ such that
%
\begin{align}
	\begin{array}{l}
		(l_1 \composeL f, l_2 \composeL f) \in \amod{L}(\ca{})\\
  	l_1 \composeL f \composeL c = l \composeL r \composeL d\\
  	\for{l'} l' \leq l_0 \land l' \leq l'_0 \implies l' = \unitL
	\end{array} \label{L9:Ass4}
\end{align}
%
From (\ref{L9:Ass1}) and (\ref{L9:Ass4}) we have:
%
\begin{align}
  \begin{array}{l}
  	\exsts{f', c'} (l_1 \composeL f', l_2 \composeL f') \in \amod{}'(\ca{}) \;\land\; l_1 \composeL f' \composeL c' =  l \composeL r \composeL d\\
		\lor\;l_2 \composeL f \composeL c \text{ is undefined}\\
		\lor\; l_1 \maxMeetL l = \unitL \;\land\; \exsts{r'} l_2 \composeL f \composeL c = l \composeL r' \composeL d \;\land\; \extendsAMUpto{\amod{}, \amod{L}}{n}{l}{r'}{\amod{}'}
  \end{array}\label{L9:Ass5}
\end{align}
%
From (\ref{L9:I.H}) we can rewrite (\ref{L9:Ass5}) as
%
\begin{align}
	\begin{array}{l}
  	\exsts{f', c'} (l_1 \composeL f', l_2 \composeL f') \in \amod{}'(\ca{}) \;\land\; l_1 \composeL f' \composeL c' =  l \composeL r \composeL d\\
		\lor\;l_2 \composeL f \composeL c \text{ is undefined}\\
		\lor\; l_1 \maxMeetL l = \unitL \;\land\; \exsts{r'} l_2 \composeL f \composeL c = l \composeL r' \composeL d \;\land\; \extendsAMUpto{\amod{}, \amod{L}}{n}{l}{r'}{\amod{}'}
  \end{array} \nonumber
\end{align}
%
as required.
\end{proof}
\end{lemma}
%
%

	\begin{lemma}[Sequential command soundness]\label{lem:seqSoundness}
For all $\seq{} \in \Seqs$, $\left(\Hp{1}, \seq{}, \Hp{2}\right) \in \AxiomsSeq$ and $\h{} \in \Heaps$:
%
\[
	\opSemSeq{\seq{}}{\reifyH{\Hp{1} \composeH \{\h{}\}}} \subseteq \reifyH{\Hp{2} \composeH \{\h{}\}}
\]
%
\begin{proof}
By induction over the structure of $\seq{}$. Pick an arbitrary $\h{} \in \Heaps$.\\

\noindent\textbf{Case \hspace*{0.3cm}}\bc{}\\
This follows immediately from parameter \ref{param:elementary-soundness}.\\


\noindent\textbf{Case \hspace*{0.3cm}\texttt{skip}}\\
\textbf{RTS.}
%
\[
	\opSemSeq{\seq{}}{\reifyH{\Hp{} \composeH \{\h{}\}}} 
	\subseteq \reifyH{\Hp{} \composeH \{\h{}\}}
\]
%
\begin{proof}
%
\[
\begin{array}{r l}
	\opSemSeq{\seq{}}{\reifyH{\Hp{} \composeH \{\h{}\}}} 
	= &
	\reifyH{\Hp{} \composeH \{\h{}\}}\\

	\subseteq & \reifyH{\Hp{} \composeH \{\h{}\}}
\end{array}
\]
%
as required.
\renewcommand{\qed}{}
\end{proof}
%
%

\noindent\textbf{Case \hspace*{0.3cm}}$\seq{1} ; \seq{2}$\\
\textbf{RTS.}
%
\[
	\opSemSeq{\seq{1}; \seq{2}}{\reifyH{\Hp{} \composeH \{\h{}\}}} 
	\subseteq \reifyH{\Hp{}' \composeH \{\h{}\}}
\]
%
where $\left(\Hp{}, \seq{1}, \Hp{}'' \right), \left(\Hp{}'', \seq{2}, \Hp{}' \right)  \in \AxiomsSeq$.
\begin{proof}
%
\[
\begin{array}{r l}
	
	\opSemSeq{\seq{1}; \seq{2}}{\reifyH{\Hp{} \composeH \{\h{}\}}} 
	= &  
	\opSemSeq{\seq{2}}{ \opSemSeq{\seq{1}}{\reifyH{\Hp{} \composeH \{\h{}\}}}}\\

	\text{(I.H.) \hspace*{0.5cm}}
	\subseteq &
	\opSemSeq{\seq{2}}{\reifyH{\Hp{}'' \composeH \{\h{}\}}}\\
	
	\text{(I.H.) \hspace*{0.5cm}}
	\subseteq &
	\reifyH{\Hp{}' \composeH \{\h{}\}}
	
\end{array}
\]
%
as required.
\renewcommand{\qed}{}
\end{proof}
%
%

\noindent\textbf{Case \hspace*{0.3cm}}$\seq{1} + \seq{2}$\\
\textbf{RTS.}
%
\[
	\opSemSeq{\seq{1} + \seq{2}}{\reifyH{\Hp{} \composeH \{\h{}\}}} 
	\subseteq \reifyH{\Hp{}' \composeH \{\h{}\}}
\]
%
where $\left(\Hp{}, \seq{1}, \Hp{}' \right), \left(\Hp{}, \seq{2}, \Hp{}' \right)  \in \AxiomsSeq$.
\begin{proof}
%
\[
\hspace*{-0.2cm}
\begin{array}{r l}
	
	\opSemSeq{\seq{1} + \seq{2}}{\reifyH{\Hp{} \composeH \{\h{}\}}} 
	= &  
	\opSemSeq{\seq{1}}{\reifyH{\Hp{} \composeH \{\h{}\}}} \;\cup\; \opSemSeq{\seq{2}}{\reifyH{\Hp{} \composeH \{\h{}\}}}\\

	\text{(I.H.) \hspace*{0.5cm}}
	\subseteq &
	\reifyH{\Hp{}' \composeH \{\h{}\}} \;\cup\; \reifyH{\Hp{}' \composeH \{\h{}\}}\\
	
	\subseteq &
	\reifyH{\Hp{}' \composeH \{\h{}\}}
	
\end{array}
\]
%
as required.
\renewcommand{\qed}{}
\end{proof}
%
%

\noindent\textbf{Case \hspace*{0.3cm}}$\seq{}^{*}$\\
\textbf{RTS.}
%
\[
	\opSemSeq{\seq{}^{*}}{\reifyH{\Hp{} \composeH \{\h{}\}}} 
	\subseteq \reifyH{\Hp{} \composeH \{\h{}\}}
\]
%
where $\left(\Hp{}, \seq{}, \Hp{} \right)  \in \AxiomsSeq$.
\begin{proof}
%
\[
\begin{array}{r l}
	
	\opSemSeq{\seq{}^{*}}{\reifyH{\Hp{} \composeH \{\h{}\}}} 
	= &  
	\opSemSeq{\skipC + \seq{}; \seq{}^{*}}{\reifyH{\Hp{} \composeH \{\h{}\}}} \\
	
	= & \opSemSeq{\skipC}{\reifyH{\Hp{} \composeH \{\h{}\}}} 
		\cup \opSemSeq{\seq{}; \seq{}^{*}}{\reifyH{\Hp{} \composeH \{\h{}\}}} \\

	\text{(I.H.) \hspace*{0.5cm}}
	\subseteq &
	\reifyH{\Hp{} \composeH \{\h{}\}} \cup \reifyH{\Hp{} \composeH \{\h{}\}}\\
	
	\subseteq &
	\reifyH{\Hp{} \composeH \{\h{}\}}
	
\end{array}
\]
%
as required.
\renewcommand{\qed}{}
\end{proof}
%
%
\end{proof}
\end{lemma}
%%
	%
\begin{lemma}[] \label{lem:guaranteeContainment}
%
For all $w_1, w_2, w, w' = (l', g', \lmod') \in \Worlds$,
\[
\begin{array}{l}
	w_1 \composeW w_2 = w /| (l', g', \lmod{}') \in \guarantee(w_1) \implies (\localPart{(w_2)}, g', \lmod') \in \rely(w_2)
\end{array}
\]
%
\begin{proof} Pick an arbitrary $w_1, w_2, w$ such that:
%
\begin{align}
	& w_1 \composeW w_2 = w \label{L13:Ass1}\\
	& (l', g', \lmod') \in \guarantee(w_1) \label{L13:Ass2}
\end{align}
%
\textbf{RTS.}
%
\begin{align*}
	(\localPart{(w_2)}, g', \lmod') \in \rely(w_2) 
\end{align*}
%
From (\ref{L13:Ass2}) and by definition of \guarantee\ we know:
%
\begin{align}
	(l', g', \lmod') \in \left(\updateG \cup \extendG\right)^{*}(w_2) \label{L13:Ass3}
\end{align}
%
From (\ref{L13:Ass1}), (\ref{L13:Ass3}) and by Lemmata \ref{lem:updateGContainment} and \ref{lem:extendGContainment} we have:
%
\begin{align}
	(\localPart{(w_2)}, g', \lmod') \in \left(\updateR \cup \extendR \right)^{*}(w_2) \nonumber
\end{align}
%
and consequently 
%
\begin{align}
	(\localPart{(w_2)}, g', \lmod') \in \rely(w_2) \nonumber
\end{align}
%
as required.
\end{proof}
\end{lemma}
%
%
\begin{lemma}[] \label{lem:updateGContainment}
%
For all $w_1, w_2, w, w' = (l', g', \lmod') \in \Worlds$,
%
\[
\begin{array}{l}
	w_1 \composeW w_2 = w /| (l', g', \lmod') \in \updateG(w_1) \implies (\localPart{(w_2)}, g', \lmod') \in \updateR(w_2)
\end{array}
\]
%
where we write $\updateR(w)$ to denote $\left\{w' \mid (w, w') \in \updateR(w) \right\}$.
%
\begin{proof} Pick an arbitrary $W_1 = (l_1, g_1, \lmod_{1})$, $w_2 = (l_2, g_2, \lmod_{2}), w$ and $(l', g', \lmod')$ such that:
%
\begin{align}
	& w_1 \composeW w_2 = w \label{L11:Ass1}\\
	& (l', g', \lmod') \in \updateG(w_1) \label{L11:Ass2}
\end{align}
%
\textbf{RTS.}
%
\begin{align*}
	(\localPart{(w_2)}, g', \lmod') \in \updateR(w_2) 
\end{align*}
*
From (\ref{L11:Ass1}) we know:
%
\begin{align}
	&g_1 = g_2 \label{L11:Ass3}\\
	&\lmod_{1} = \lmod_{2} \label{L11:Ass4}
\end{align}
%
By definition of \updateG and from (\ref{L11:Ass2}) and (\ref{L11:Ass4}) we know:
%
\begin{align}
	& \lmod' = \lmod_{1} = \lmod_{2}  \label{L11:Ass5} \\
	& \ortCap{\capPart{l_1 \composeL g_1)}} = \ortCap{\heapPart{(l' \composeL g')}} \label{L11:Perms} \\
	& g' = g_1 \lor 
	\left(\begin{array}{l}
		\exsts{\ca{} \leq \capPart{(l_1)}}  (g_1, g') \in \padAM{\lmod_{1}}(\ca{}) \;\land \\
		\ortH{\heapPart{(l_1 \composeL g_1)}} = \ortH{\heapPart{(l' \composeL g')}}
	\end{array} \right) \nonumber
\end{align}
%
There are two cases to consider:\\

\noindent\textbf{Case 1.} $g_1 = g'$\\
From (\ref{L11:Ass3}) and the assumption of the case we know $g' = g_2$. Consequently, from (\ref{L11:Ass5}) we have:

\begin{equation}
	(\localPart{(w_2)}, g', \lmod') = (l_2, g_2, \lmod_{2}) \label{L11:Ass6}
\end{equation}
%
By definition of \updateR\ and from (\ref{L11:Ass6}) we can conclude:
%
\begin{equation}
	(\localPart{(w_2)}, g', \lmod') \in \updateR(l_2, g_2, \lmod_{2}) \label{L11:Ass7}
\end{equation}
%
as required.\\

\noindent\textbf{Case 2.} 
%
\begin{align}
	& \exsts{\ca{} \leq \capPart{(l_1)}} (g_1, g') \in \padAM{\lmod_{1}}(\ca{}) \label{L11:Ass8} \\
	&\ortH{\heapPart{(l_1 \composeL g_1)}} = \ortH{\heapPart{(l' \composeL g')}} \label{L11:Ass9}
\end{align}
%
From (\ref{L11:Ass1}), (\ref{L11:Ass3}) and (\ref{L11:Ass4}) we know that 
%
\begin{equation}
	w = (l_1 \composeL l_2, g_2, \lmod_{2}) \label{L11:Ass10}
\end{equation}
%
Since $\wf{w}$ (by definition of \Worlds) and from (\ref{L11:Ass3}) we know:
\begin{align}
	&\capPart{(l_1 \composeL l_2 \composeL g_2)} = \capPart{(l_1)} \composeCap \capPart{(l_2)} \composeCap {(g_2)} = \capPart{(l_1 \composeL g_1)} \composeCap \capPart{(l_2)}\hspace*{0.2cm} \text{is defined} \label{L11:Ass11}\\
%	
	&\heapPart{(l_1 \composeL l_2 \composeL g_1)} = \heapPart{(l_1 \composeH g_1)} \composeH \heapPart{(l_2)}  \hspace*{0.2cm} \text{is defined} \label{L11:Ass12}
\end{align}
%
Since $\ca{1} \leq \capPart{(l_1)}$ (\ref{L11:Ass8}), from (\ref{L11:Ass11}) and Lemma \ref{lem:disjointByOrder}, we know:
%
\begin{equation}
	\ca{} \compatible\ \capPart{(l_2)} \composeCap \capPart{(g_2)} \label{L11:Ass13}
\end{equation}
%
From (\ref{L11:Ass3}), (\ref{L11:Ass9}) and (\ref{L11:Ass12}) we know
%
\begin{equation}
	\heapPart{(l' \composeL g')} \composeH \heapPart{(l_2)} = \heapPart{(l' \composeL l_2 \composeL g')}\hspace*{0.2cm}\text{ is defined} \label{L11:Ass14}
\end{equation}
%
From (\ref{L11:Perms}) and (\ref{L11:Ass11}) we know
%
\begin{equation}
	\capPart{(l' \composeL g')} \composeCap \capPart{(l_2)} = \capPart{(l' \composeL l_2 \composeL g')}\hspace*{0.2cm}\text{ is defined} \label{L11:Ass15}
\end{equation}
%
From (\ref{L11:Ass14}) and (\ref{L11:Ass15}) we know $l'_1 \composeL l_2 \composeL g'$ is defined and consequently:
%
\begin{equation}
	l_2 \composeL g' \hspace*{0.2cm} \text{ is defined} \label{L11:Ass16}
\end{equation}
%
From (\ref{L11:Ass4}), (\ref{L11:Ass8}), (\ref{L11:Ass13}), (\ref{L11:Ass16}) and by definition of \updateR, we have:
%
\begin{equation*}
	(\localPart{(w_2)}, g', \lmod') = (l_2, g_2, \lmod_{2}) \in \updateR(l_2, s_2, \lmod_{2})  
\end{equation*}
%
as required.
\end{proof}
%
\end{lemma}
%
%
\begin{lemma}[]\label{lem:extendGContainment}
%
For all $w_1, w_2, w, w' = (l', g', \lmod') \in \Worlds$,
\[
\begin{array}{l}
	w_1 \composeW w_2 = w /| w' \in \extendG(w_1) \implies (\localPart{(w_2)}, g', \lmod') \in \extendR(w_2)
\end{array}
\]
%
\begin{proof} Pick an arbitrary $w_1 = (l_1, g_1, \lmod_{1})$, $w_2 = (l_2, g_2, \lmod_{2}), w$ and $ w' = (l', g', \lmod')$ such that:
%
\begin{align}
	& w_1 \composeW  w_2 = w \label{L12:Ass1}\\
	& w' \in \extendG(w_1) \label{L12:Ass2}
\end{align}
%
\textbf{RTS.}
%
\begin{align*}
	(\localPart{(w_2)}, g', \lmod') \in \extendR(w_2) 
\end{align*}
%
From (\ref{L12:Ass1}) we know:
%
\begin{align}
	& g_1 = g_2 /| \lmod_{1} = \lmod_{2} \label{L12:Ass3}
\end{align}
%\lmod', \gmod'
By definition of \extendG\ and from (\ref{L12:Ass2}) and (\ref{L12:Ass3}) we know there exists $l_3, l_4, g'' \in \LStates$, $\ca{1}, \ca{2} \in \Caps$ and $\lmod''$ such that
%
\begin{align}
& \begin{array}{l}
	l_1 = l_3 \composeL l_4 /| l'_1 = l_3 \composeL (\unitH, \ca{1})  \\
	/| g'' = l_4 \composeL (\unitH, \ca{2}) /| g' = g_2 \composeL g'' \\
	/| \ca{1} \composeCap \ca{2} \containedIn \dom{\lmod''} /|  \ca{1} \composeCap \ca{2} \disjoint \dom{\lmod_{2}} \\
	/| \lmod' = \lmod_{2} \cup \lmod'' 
	/| \expandsAM{\lmod''}{g''}{g_2}{\lmod_{2}} 
	/| g'' \containI \lmod'' 
%	/| \extendsAM{\lmod', \gmod'}{g''}{g_2}{\lmod''} 
\end{array}
\label{L12:Ass4}
\end{align}
%
From (\ref{L12:Ass4}) and the definition of $\extendR$ we have:
%
\begin{equation*}
	(\localPart{(w_2)}, g', \lmod{}') \in \extendR(w_2) 
\end{equation*}
% 
as required.
\end{proof}
%
%
\end{lemma}
%
%%
%\begin{lemma}[]\label{lem:shiftGContainment}
%%
%For all $w_1, w_2, w, w' = (l', g', \lmod') \in \Worlds$, 
%\[
%\begin{array}{l}
%	w_1 \composeW w_2 = w /| w' \in \shiftG(w_1) \implies (\localPart{(w_2)}, g', \lmod') \in \shiftR(w_2)
%\end{array}
%\]
%%
%\begin{proof} Pick an arbitrary $w_1 = (l_1, g_1, \lmod_{1})$, $w_2 = (l_2, g_2, \lmod_{2}), w$ and $w' = (l', g', \lmod')$ such that:
%%
%\begin{align}
%	& w_1 \composeW w_2 = w \label{LSGC:Ass1}\\
%	& w' \in \shiftG(w_1) \label{LSGC:Ass2}
%\end{align}
%%
%\textbf{RTS.}
%%
%\begin{align*}
%	(\localPart{(w_2)}, g', \lmod') \in \shiftR(w_2) 
%\end{align*}
%%
%From (\ref{LSGC:Ass1}) we know:
%%
%\begin{align}
%	& g_1 = g_2 /| \lmod_{1} = \lmod_{2}  \label{LSGC:Ass3}
%\end{align}
%%
%By definition of \shiftG and from (\ref{LSGC:Ass2}), and (\ref{LSGC:Ass3}) we know there exists $\lmod'' \in \AMods$ such that
%%
%\begin{align}
%%
%%	& g' = g_2 /| \gmod' = \gmod_2 /| \lmod' = \lmod_{2} \cup \lmod'' /| \for{\ca{}} \for{a \in \lmod'' (\ca{})} \m{reflected}(a, g_2, \lmod_2) \label{LSGC:Ass7}
%	& g' = g_2 /| \lmod' = \lmod_{2} \cup \lmod'' /| \expandsAM{\lmod''}{\unitL}{g_2}{\lmod_{2}} \label{LSGC:Ass7}
%%
%\end{align}
%%
%From (\ref{LSGC:Ass7}) and the definition of $\shiftR$ we have:
%%
%\begin{equation}
%	(\localPart{(w_2)}, g', \lmod') \in \shiftR(w_2) \nonumber
%\end{equation}
%% 
%as required.
%\end{proof}
%%
%%
%\end{lemma}
%%

	\begin{lemma}[]\label{lem:nonEmptyOverlap}Given any separation algebra $(\mathbb{A}, \compose{\mathbb{A}}, \unit{\mathbb{A}})$ with the cross-split property, for any $a, b \in \mathbb{A}$:
%
\[
	\exsts{c \in \mathbb{A}} a \leq b \compose{\mathbb{A}} c \iff a \meet{\mathbb{A}} b \not= \emptyset
\]
%
$\m{Proof} \implies$. We proceed with proof by contradiction.
Take arbitrary $a, b, c \in \mathbb{A}$ such that 
%
\begin{equation}
	a \leq b \composeL c \label{L5:Ass1}
\end{equation}
%
and assume
%
\begin{equation}
	a \meet{\mathbb{A}} b = \emptyset \label{L5:Ass2}
\end{equation}
%
From (\ref{L5:Ass2}) and by definition of $\meet{\mathbb{A}}$, we have:
%
\begin{equation}
	\neg\exsts{d, e, f, g} a = d \compose{\mathbb{A}} e /| b = e \compose{\mathbb{A}} f /| d \compose{\mathbb{A}} e \compose{\mathbb{A}} f = g \label{L5:Ass3}
\end{equation}
%
From (\ref{L5:Ass1}) we have $\exsts{h} a \compose{\mathbb{A}} h = b \compose{\mathbb{A}} c$ and consequently by the cross-split property we have:
%
\begin{align}
	\exsts{ab, ac, hb, hc, t} &
	ab \compose{\mathbb{A}} ac = a 	\label{L5:Ass4}\\
	& hb \compose{\mathbb{A}} hc = h \label{L5:Ass5}\\
	& ab \compose{\mathbb{A}} hb = b \label{L5:Ass6}\\
	& ac \compose{\mathbb{A}} hc = c \label{L5:Ass7}\\
	& t = ab \compose{\mathbb{A}} ac \compose{\mathbb{A}} hb \compose{\mathbb{A}} hc \label{L5:Ass8}
\end{align}
%
From (\ref{L5:Ass8}) we have:
%
\begin{equation}
	\exsts{s} ab \compose{\mathbb{A}} ac \compose{\mathbb{A}} hb = s \label{L5:Ass9}
\end{equation}
%
From (\ref{L5:Ass4}), (\ref{L5:Ass6}) and (\ref{L5:Ass9}) we have: 
%
\begin{equation}
	\exsts{d, e, f, g} a = d \compose{\mathbb{A}} e /| b = e \compose{\mathbb{A}} f /| d \compose{\mathbb{A}} e \compose{\mathbb{A}} f = g \label{L5:Ass10}
\end{equation}
%
From (\ref{L5:Ass3}) and (\ref{L5:Ass10}) we derive a contradiction and can hence deduce:
%
\begin{equation}
	a \meet{\mathbb{A}} b \not= \emptyset \nonumber
\end{equation}
%
as required.\\


\noindent$\m{Proof} \Leftarrow$. Take arbitrary $a, b \in \mathbb{A}$ such that 
%
\begin{align*}
	a \meet{\mathbb{A}}	b \not= \emptyset
\end{align*}
%
Then by definition of $\meet{\mathbb{A}}$ we have: 
%
\begin{align*}
	\exsts{d, e, f \in \mathbb{A}} & a = d \compose{\mathbb{A}} e\\
	& b = e \compose{\mathbb{A}} f \\
	& d \compose{\mathbb{A}} e \compose{\mathbb{A}} f \text{ is defined}
\end{align*}
% 
and thus by definition of $\leq$ we have $a \leq b \compose{\mathbb{A}} d$ and consequently
%
\begin{align*}
	\exsts{c} a \leq b \compose{\mathbb{A}} c
\end{align*}
%
as required.
\qed
\end{lemma}
%
%
\begin{lemma}[]\label{lem:divideUpper}
%
Given any separation algebra ($\mathcal{M}, \compose{\mathcal{M}}, \unit{\mathcal{M}}$) with the cross-split property:
\[
	\for{a, b, c \in \mathcal{M}} a \leq b \compose{\mathcal{M}} c \implies \exsts{m, n} a = m \compose{\mathcal{M}} n \;\land\; m \leq b \;\land\; n \leq c
\]
%
\begin{proof}
Pick an arbitrary $a, b, c \in \mathcal{M}$. Since $a \leq b \compose{\mathcal{M}} c$, we know $\exsts{d \in \mathcal{M}}$ such that:
%
\begin{equation}
	a \compose{\mathcal{M}} d = b \compose{\mathcal{M}} c \label{L6:Ass1}
\end{equation}
%
By the cross-split property of $\mathcal{M}$, We can deduce: $\exsts{ab, ac, db, dc \in \mathcal{M}}$ such that:
%
\begin{align}
	a = ab \compose{\mathcal{M}} ac \label{L6:Ass2}\\
	b = ab \compose{\mathcal{M}} db \label{L6:Ass3}\\
	c = ac \compose{\mathcal{M}} dc \label{L6:Ass4}\\
	d = db \compose{\mathcal{M}} dc \nonumber 
\end{align}
%
Since $ab \leq b$ (\ref{L6:Ass3}) and $ac \leq c$ (\ref{L6:Ass4}), from (\ref{L6:Ass2}) we can deduce:
%
\begin{equation}
	\exsts{m, n \in \mathcal{M}} a = m \compose{\mathcal{M}} n \;\land\; m \leq b \;\land\; n \leq c \nonumber
\end{equation}
%
as required.
\end{proof}
\end{lemma}
%
%
\begin{lemma}[]\label{lem:disjointByOrder}
Given any separation algebra ($\mathcal{M}, \compose{\mathcal{M}}, \unit{\mathcal{M}}$),
\[
	\for{a, b, c, d \in \mathcal{M}} a \compose{\mathcal{M}} b = d \;\land\; c \leq b \implies 
	\exsts{f \in \mathcal{M}} a \compose{\mathcal{M}} c = f
\]
%
\begin{proof}
Pick an arbitrary $a, b, c, d \in \mathcal{M}$ such that:
%
\begin{align}
	a \compose{\mathcal{M}} b = d \label{L7:Ass1}\\
	c \leq b \label{L7:Ass2}
\end{align}
%
From (\ref{L7:Ass2}), we have: 
%
\begin{equation}
	\exsts{e \in \mathcal{M}} c \compose{\mathcal{M}} e = b \label{L7:Ass3}
\end{equation}
%
and consequently from (\ref{L7:Ass1}) we have:
%
\begin{equation}
	a \compose{\mathcal{M}} c \compose{\mathcal{M}} e = d \label{L7:Ass4}
\end{equation}
%
Since $e \leq d$ (\ref{L7:Ass4}), we have: 
%
\begin{equation}
	\exsts{f \in \mathcal{M}} e \compose{\mathcal{M}} f = d \label{L7:Ass5}
\end{equation} 
%
From (\ref{L7:Ass4}), (\ref{L7:Ass5}) and cancellativity of separation algebras we have:
%
\begin{equation}
	a \compose{\mathcal{M}} c = f
\end{equation}
%
and thus
%
\begin{equation}
	\exsts{f \in \mathcal{M}} a \compose{\mathcal{M}} c = f
\end{equation}
%
as required.
\end{proof}
\end{lemma}
%
%



}
%\documentclass[12pt]{article}
\usepackage{azalea}
%\usepackage{wasysym}
\title{Ode to $Q$}\date{}
\begin{document}
\maketitle
\begin{tabular}{l @{\hspace*{1cm}} c}
Dear $Q$, this is $P$, let's call off this fight
& $\shared{P}{I_1} \hspace*{1.5cm} \shared{Q}{I_2}$\vspace*{2pt}\\

Though we may differ, we're both in the right 
&$\shared{P}{I_1} * \shared{Q}{I_2} = \happyFace$\vspace*{2pt}\\

Contend we may yet consistent our ways 
&$I_1 = \{a : P \swap P'\} \;\; I_2 = \{a: Q \swap Q'\}$\vspace*{2pt}\\

In concert our shared fates unfold with sleight
& $I_1 \cup I_2 = \happyFace$\\\\\\


I'm lost in the dark, you see all that's light
& $\shared{P}{I_1} \sadFace \hspace*{0.5cm} * \hspace*{0.5cm}  \shared{Q}{I_2} \happyFace$\vspace*{2pt}\\

Your wisdom I crave, what say we unite? 
& $\shared{P \sepish Q}{I_1 \cup I_2} \happyFace$\vspace*{2pt}\\

In tandem evolving we gain what we yearned 
& $\shared{P' \sepish Q'}{I_1 \cup I_2} $\vspace*{2pt}\\

Though strong our liaison we part with delight
&$\shared{P'}{I_1 \cup I_2} \;\; *\;\;  \shared{Q'}{I_1 \cup I_2} $\\\\\\


That which was complete is now marred and trite
& $\shared{P'}{I_1 \cup I_2 \frownie}  \;\;\; * \;\;\;  \shared{Q'}{I_1 \cup I_2 \frownie} $\vspace*{2pt}\\

We seek out that simple yet stable rite
&$\shared{P'}{I_1 \smiley{}}  \;\;\; * \;\;\;  \shared{Q'}{I_2 \smiley{}} $\vspace*{2pt}\\

%Dim may your memory, though cherish I do
%& $\shared{P'}{I_1}  $\\
Your memory now dimmed,  your spirit I bear
& $\shared{P'}{I_1}  $\\

In mine for always and this I recite:
&$\shared{P'}{I_1}  \;*\;  \shade{? \shared{Q'}{I_2}\lor \shared{Q''}{I_3}\lor  \shared{Q'''}{I_3} \lor \cdots} $\\

%Your spirit in mine and this I recite:
%&$\shared{P'}{I_1}  \;\;\;\;\;\;  \shade{? \shared{Q'}{I_2}}\;\; \shade{? \shared{Q''}{I_3}}\;\; \shade{? \shared{Q'''}{I_3}} $\\
%Your memory now dimmed, $Q$ dear I note
%& $\shared{P'}{I_1}  $\\
%
%Your spirit for always and this I recite
%&$\shared{P'}{I_1}  \;\;\;\;\;\;  \shade{? \shared{Q'}{I_2}}\;\; \shade{? \shared{Q''}{I_3}}\;\; \shade{? \shared{Q'''}{I_3}} $\\
\\\\



Our discord our strength, conciseness our might
&\textbf{Co}ncurrent\vspace*{4pt}\\

Tangled at heart though disjointed at sight
&\textbf{Lo}cal\hspace*{28pt}\vspace*{4pt}\\

CoLoSL our feat, our effort but slight
&\textbf{S}ubjective\hspace*{3pt}\vspace*{4pt}\\

Sound our ideas, our paper pray cite! & \textbf{L}ogic\hspace*{28pt}


\end{tabular}
\end{document}
}





\end{document}
