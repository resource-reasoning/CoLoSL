\section{Introduction}
\label{sec:introduction}


%\pg{This is what we should answer by the introduction: 
%What's the problem with the current state of the art?
%What is our solution?
%Why is the problem we're solving hard?
%How do we solve the challenges?
%Why do we do better than existing work?
%What are the lessons learnt from the paper?}

A key difficulty in verifying properties of fine-grained concurrent
programs is being able to reason compositionally about each thread in
isolation, even though in reality the correctness of the whole system
is the collaborative result of intricately intertwined actions of the
threads.  Such compositional reasoning is essential for: verifying
large concurrent systems, since it allows them to be verified
component-wise;  verifying library code and incomplete programs,
since one does not need to know about the context of execution; and 
replicating a programmer's intuition about why their implementations
are correct, since their informal arguments are typically  kept local and do
not bring the whole system into the reasoning.


Rely-guarantee~\cite{rg} and various combinations of
rely-guarantee and separation-logic
reasoning~\cite{viktor-marriage,dg,lrg,cap-ecoop10,icap,tada} achieve
compositional reasoning for increasingly difficult inter-thread
interactions. However, we believe that, despite substantial progress, there are many
examples where the specifications and proofs are not as concise as
they might be; they  are either too coarse or
too 
contrived. 
We explore a different approach, introducing the
program logic \colosl, using which we can give small
specifications and proofs which correspond to the  programmer's intuition of what shared
resource
is required by the thread. 




Small specifications were emphasised in the work of O'Hearn, Reynolds
and Yang on separation logic~\cite{o2001local}.  The original
separation logic~\cite{rey02,seplog} achieves local, compositional reasoning
for sequential heap-manipulating programs by splitting the heap into
disjoint heaplets for describing the local resources required by a
program. Compositionality then rests on two powerful mechanisms: a program
can be specified using only those resources that it actually accesses;
and this specification can be simply reused in any context that
contains these resources.  By making the specification as small as
possible, we can ensure that it can be reused in a large set of
possible contexts using the {\em frame rule}. This plays an important
role in achieving scalable {compositional reasoning}, as each
block of code can be proved in isolation and its small specification
reused in larger contexts.


Concurrent separation logic extends this compositional reasoning to
concurrent programs using the {\em disjoint concurrency rule}, where
individual threads use both local resources private to the
thread as well as static shared resources, which can be accessed by
all threads through critical sections. Since then, there have been many
extensions combining rely-guarantee reasoning with ideas from
concurrent separation logic to reason about fine-grained concurrency:
RGSep~\cite{viktor-marriage} 
%and then deny-guarantee reasoning~\cite{dg} 
introduced local resource and shared disjoint
regions which are stable with respect to a static interference
relation stating how the current thread and the environment can affect
the region; CAP~\cite{cap-ecoop10} and its
extensions~\cite{hocap,icap,tada} in addition abstract the
regions. This work has achieved substantial success; see the related
work section for more details. Yet, we believe these approaches
are not always able to provide small
specifications for concurrent programs, impeding 
compositionality.



The problem is due to the rigid nature of the shared disjoint regions and
their static interference, which   limits how we can  work
with a
concurrent data structure. A disjoint region typically either describes 
the entire data structure such as a linked list,  or contains
individual 
components of the data structure such as the nodes of a
linked list. However, threads may have shared access to arbitrary 
parts of the data structure, which cannot be directly expressed in the reasoning.
%and we would like directly express this
%access  in
%the reasoning. 
The interference associated with the region is static in that it is defined when
the region is created and is fixed throughout its entire lifetime. However, 
threads having access to parts of a shared region need only know about the interference on those parts. In addition, 
it is not always possible
to predict all future interactions  associated with  a shared
region at the time  of creation. Just as the original separation logic  uses the frame rule to
obtain small specifications for the \emph{local} state, we seek an
analogous framing mechanism on both the {shared} resource  {and} its
interference to obtain small
specifications of the \emph{shared} state. 


For example, consider a linked list consisting of $n+1$ nodes accessed
concurrently by $n$ threads where the $i$th
thread requires access to the $i$th and $(i+1)$st nodes. Current
approaches cannot provide a small specification for each thread which
captures  just these two shared nodes and their interference. Now
consider a program whose threads manipulate subgraphs of a graph, such
as a recursive
concurrent spanning tree algorithm. Current approaches cannot give small specifications capturing just the subgraph manipulated by each thread due to intrinsic, unspecified sharing  between subgraphs.
Finally, consider a concurrent set implementation. In \colosl it is
possible to describe the interference associated with a new element as it is added to the set. The interference on new elements need not be the same as before and may only be known at the time they are added. Existing approaches cannot accommodate such dynamic interference relations.
% Finally, consider a concurrent set
%library initially endowed with just an insert function. Current
%approaches cannot extend their reasoning to include a remove function,
%without having to prove the whole library again.

%. As such, upon its creation, provision
%has to be made for all future extensions of it and it is generally not
%possible to arbitrarily grow it with extra state under new
%interference rules.  As we discuss in \S\ref{sec:examples}, this
%results in unnatural specifications when dynamically-allocated
%resources become shared during execution, such as in the concurrent
%set module example of~\cite{cap-ecoop10}.


%\pgcomment{Not sure should be in:These issues, in particular the first one, not only affect the
%locality of the reasoning, but also impede \emph{modularity} and proof
%reuse. When the implementation of a module $M_1$ is verified against
%its specification, its proof can only be reused from within another
%module $M_2$ if either the parts of the shared state accessed by $M_2$
%are disjoint from those of $M_1$, or the shared state is partitioned
%identically for both modules. The probability that several modules
%agree on the exact partition and shape of the shared resources is slim.}

This paper introduces the program logic \colosl, which stands for
\underline{Co}ncurrent \underline{Lo}cal \underline{S}ubjective
\underline{L}ogic. \colosl's semantic model is based on one global
shared state, and each thread is verified with respect to its partial
{\em subjective view} of this state. Each subjective (personalised)
view comprises an assertion which describes \emph{parts} of the shared
global resource used by a thread, and an interference relation which
describes how the thread and the environment may affect these parts.
Subjective views may arbitrarily overlap with each other, with both
their resources and interference expanding and contracting in
accordance with what is required by the thread. Interestingly, this
sometimes requires rewriting the interference relation so that the
interference on the smaller state captures the same information as the
interference on the bigger state. This expansion and contraction of
subjective views provides small specifications for individual threads
and local reasoning about a thread's shared state.


%\pgcomment{Not for here, section3:We give the main ideas behind the semantic model and the soundness of
%\colosl in \S\ref{sec:colosl}. In this section, we also provide proof
%rules to discharge \colosl-specific proof obligations by reducing them
%to regular separation logic validity checks. Thus one can use \colosl
%without having to reason directly in the model.}

We demonstrate \colosl reasoning on a range of examples.  The first
example in \S\ref{sec:intuition} is Dijkstra's token ring mutual
exclusion algorithm~\cite{dijkstra74}. Regardless of the size of the
ring, we are able to give a small specification to  each thread in
isolation such that each proof only mentions resources
associated with two of its neighbours. This means that the proof can be largely reused when the
implementation of the ring is changed to allow dynamic spawning of new
participants.  In \S\ref{sec:examples}, we study two further
examples. The first is a concurrent spanning tree algorithm for
graphs, where threads are recursively spawned on potentially
overlapping subgraphs. We demonstrate that the flexible, overlapping
subjective views of \colosl are just what we need. 
%In particular, to
%the best of our knowledge,
%we provide the first small specification of a concurrent graph algorithm.
The second is a concurrent set module implemented 
using a hand-over-hand
list-locking algorithm. Our \colosl reasoning  for this set module  improves
on CAP reasoning~\cite{cap-ecoop10} in that our small
specifications can be dynamically extended to include other behaviour
in future
whereas the  static CAP interference  must predict all behaviour from the start. 



Most of the technical details have been left out due to space
constraints, and are provided in the accompanying technical report~\cite{colosl-tr14}.

%This paper presents the program logic \colosl, which achieves
%compositional reasoning for concurrent programs by enabling
%\emph{subjective} views of the shared state. Subjective views may be
%composed arbitrarily and manipulated so as to retain only the portions
%of local and shared state actually accessed by each thread in the
%program, and to consider only the interferences relevant to that piece
%of shared state. \colosl achieves a greater degree of compositionality
%than existing work by enabling \emph{finer-grained} sharing: the
%program logic can consider that threads share only what is relevant to
%their function, a key ingredient of compositionality.
%
%
%
%Driven by the ever-increasing need for concurrency in software,
%%  spurred by recent hardware developments, 
%program logics for shared-memory concurrency have progressed towards
%the twin ideals of fine-grain reasoning and compositionality. The
%former enables elegant proof techniques about increasingly subtle
%concurrency idioms~\cite{vv06popl,vv07msc,todo}, while the latter
%allows programs to be proved component-wise and their proofs to be
%reusable as-is in any client
%program~\cite{csl-tcs,cap-ecoop10,icap}.
%%  Central to these compositional verification frameworks is the
%% formalism used to describe both the state shared between program
%% threads and the possible interference on that state.
%%
%% dating back from Owicki~\cite{owicki}, and later Jones who
%% integrated the notion of interference in the logic
%% itself~\cite{rg}.
%% 
%
%% Consider for instance
%% a program where threads operate on subgraphs of a global graph.
%
%
%%  dubbed their \emph{subjective states}.  One may then reuse these
%%   specifications in the context of any larger local state (as is
%%   standard in separation logic~\cite{rey02}), and, crucially for
%%   compositional reasoning about concurrent programs, any larger
%%   shared state. The subjective states of different threads in a
%%   program are allowed to overlap arbitrarily, ensuring maximum
%%   reusability of proofs.
%
%
%
%
% Subjective views can expand or contract depending on the resources required by the thread, which has subtle consequences for the logical reasoning. In particular, it is possible to strengthen actions in $I$ to record  some global information known to the subjective view.
%Then, the view can be weakened to just the resources and possibly strengthened actions appropriate to  the thread. The strengthened actions retain  some global information despite the weakened view. 



\paragraph{Related work}
Jones' rely-guarantee reasoning~\cite{rg} provided a breakthrough in
compositional reasoning about concurrent programs. It described
permitted interferences between threads using global rely and
guarantee relations on one  global shared state. This work is
compositional in the sense that the guarantee of one thread is the
rely of another. However, O'Hearn~\cite{csl-tcs} demonstrated that,
for  concurrent reasoning to scale,
 it is important to work with small specifications
based on local  state and shared state rather than working with one
global state. In particular, he introduced concurrent
separation logic based on thread-local state and static invariants for
the shared state.

\pgcomment{Care: deny-guarantee reasoning v CAP.} 

Since then, there has been much recent work on compositional
reasoning about fine-grained concurrent algorithms. Vafeiadis and
Parkinson~\cite{viktor-marriage} combined rely-guarantee reasoning with
separation logic, with the state split into thread-local state and
disjoint shared regions, and global rely-guarantee relations providing
the interference on these regions. 
CAP reasoning~\cite{cap-ecoop10} increased  compositionality by, instead of the
global
rely-guarantee relations, having static interference relations
associated with the disjoint shared regions and capabilities in the local
state for dynamically controlling the permitted  interference, and concurrent
abstract predicates for hiding implementation details. Several
program logics have adapted this work to incorporate more
abstraction~\cite{caresl}, higher-order features~\cite{icap} and more flexible
capabilities~\cite{tada}. We believe that all  this work has  limited compositional
reasoning in the sense that it is only possible to frame off local
state and unused shared regions. It is not possible to 
frame within regions nor frame inside the interference
relations.  (It is, of course, possible to weaken the region assertion
by the logical implication $P * Q \Rightarrow P * \texttt{true}$, but this is
not the same as being able to shrink regions \emph{and} their interference to
just work with $P$.) 

Meanwhile, a different breakthrough in compositionality came from
Feng's local rely-guarantee (LRG) reasoning~\cite{lrg}. The LRG model
comprises one global shared state, with assertions describing
{disjoint} but flexible parts of this state and rely-guarantee
relations determining how the thread and environment can affect these
partial states. With LRG, it is possible to frame off disjoint parts of
shared state and their associated disjoint rely-guarantee relations, but, as noted
by its author, the strong disjointness restrictions make this approach
only applicable for disjoint modules in the code. We took significant
inspiration from LRG, combining the flexibility of the LRG approach
with our subjective views to reason locally about overlapping shared
states.

%\pgcomment{Have I grabbed all the points about LRG?}



Finally, Fine-grained Concurrent Separation Logic
(FCSL)~\cite{scsl-esop14} of Nanevski {\em et al.} explores a
different notion of region called {\em concurroids}. Like the regions
in CAP, concurroids describe disjoint pieces of shared state together
with their static interference. It is therefore not possible to frame
within concurroids. The term ``subjectivity'' in FCSL refers to the
fact that, unlike CAP, concurroids have three parts: the ``joint''
part; and the disjoint ``self'' and ``other'' parts.  Although FCSL
did not influence \colosl, it does highlight an interesting point
regarding resource transfer between regions: in CAP, communication
between regions is achieved indirectly via the local state; in FCSL,
communication between concurroids is achieved directly through
dangling external transitions; and, in \colosl, communication between
compatible subjective views can be achieved by merging the two views.





%% \subsection{Related work bullet points} 
%% \begin{itemize}
%% 	\item Rely-Guarantee
%% 		\begin{enumerate}
%% 			\item State = one global shared state (no thread-local state)
%% 			\item Interference defined as the union of the actions of the thread (guarantee) and the environment (rely)
%% 			\item No composition on the state (state is always global - can't frame anything)
%% 			\item No composition on the interference (R, G are global interference is always global)
%% 			\item Interference is static: you can't temporarily block (enable) certain threads from doing certain actions
%% 		\end{enumerate}
%% %
%% %

%% 		\item RGSep
%% 			\begin{enumerate}
%% 				\item Combines RG with separation logic
%% 				\item State = thread local state + shared state. Shared state divided into regions
%% 				\item Composition on the thread-private resource (using *)
%% 				\item Limited composition on the shared resource and its interference: You can frame at the region level; granularity of frame is pre-determined and is at region level; you can frame off regions and their interference, but you can't frame within region resources or within region interferences (you can of course weaken the resource by rewriting $\shared{P*Q}{}$ as $\shared{P*true}{}$) 
%% 				\item Threads can't have arbitrary compatible but overlapping (in the $**$ sense) views of the shared resources and their interference (rely/guarantee)
%% 				\item Interference is static: you can't temporarily block (enable) certain threads from doing certain actions
%% 			\end{enumerate}
%% %
%% %

%% 		\item LRG
%% 			\begin{enumerate}
%% 				\item State = one shared resource
%% 				\item Limited composition on the resource and interference (using *): one can always de-compose and frame off \emph{disjoint} resources and their disjoint interference (the R and the G); more flexible than RGSep in that there is no notion of regions and thus frames are not pre-determined; you can frame off as much as you like as long as they are disjoint; this is achieved by precise fences (invariants)
%% 				\item Threads can't have arbitrary compatible but overlapping (in the $**$ sense) views of the shared resources and their interference (rely/guarantee)
%% 				\item Interference is static: you can't temporarily block (enable) certain threads from doing certain actions
%% 			\end{enumerate}
%% %
%% %

%% 		\item CAP Family
%% 			\begin{enumerate}
%% 				\item State = thread-local resource + bunch of regions; region = resource + interference; attached to each region is an interference relation
%% 				\item Composition on the thread-private resource (using *)
%% 				\item Limited composition on the shared resource and its interference: You can frame at the region level; granularity of frame is pre-determined and is at region level; you can frame off regions and their interference, but you can't frame within region resources or within region interferences (you can of course weaken the resource by rewriting $\shared{P*Q}{}$ as $\shared{P*true}{}$) 
%% 				\item Threads can't have arbitrary compatible but overlapping (in the $**$ sense) views of the shared resources and their interference
%% 				\item Dynamic interference;  interference defined in terms of actions associated with capabilities; borrowing techniques from deny-guarantee reasoning, it uses a combination of capabilities and the interference to temporarily block/enable actions
%% 				\item Added abstraction through abstract predicates
%% 			\end{enumerate}
%% %
%% %

%% 		\item FCSL (concurroids)
%% 		\begin{enumerate}
%% 			\item State = a bunch of concurroids associated with labels (like regions in cap); e.g. a lock concurroid; a designated  concurroid labelled \textsf{priv} for thread local resources
%% 			\item Concurroid = resource + transitions
%% 			\item The resources held by concurroids are disjoint from one another (like regions of CAP)
%% 			\item Each concurroid has an associated PCM type (Partial commutative monoid)  that describe the type of concurroid resource; e.g. a heap + ghost resource
%% 			\item A concurroid has three parts: the joint part (j), the self part(s) and the other part (o) and this is what subjectivity refers to
%% 			\item Concurroids are very much like regions of CAP; so everything we said about resource and interference composition is also valid here:		
%% 			\item  Limited composition on the shared resource and its interference like in CAP: You can frame at the concurroid level; granularity of frame is pre-determined and is at concurroid level; you can frame off concurroids and their transitions, but you can't frame within concurroid resources or within concurroid transitions 
%% 			\item Threads can't have arbitrary compatible but overlapping (in the $**$ sense) views of the shared resources and their transitions
%% 			\item Direct ``communication" (resource transfer) between concurroids (through dangling external transitions); unlike CAP where communication between regions was done through the thread-local state; in \colosl there is no regions, you can communicate between two subjective views by combining them using the \mergeRule principle
%% 			\item Subjectivity in this world refers to the fact that each concurroid describes what parts of the concurroid resources are held by both the thread and its environment (the self and other component)
%% 			\item In \colosl subjectivity refers to the fact that different threads may see different (but compatible) parts of the shared state subject to different (but compatible) interference according to their needs. 
%% 			\item Subjective views in \colosl can overlap; in FCSL, the self and other component (the subjective parts) are disjoint
%% 		\end{enumerate}
%% \end{itemize}

