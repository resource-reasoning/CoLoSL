\section{Assertions and Judgements (Continued)}\label{sec:assertions-continued}
In this section, we present additional rules for despatching \colosl proof obligations of fencing, shifting and stability. 
%We present a number of judgements for the elimination of ($**$) connective as well as rewriting non-intersection judgements of the form $\separate{P}{Q}$. 
We provide a mechanism for \emph{flattening} assertions with nested subjective views such as $\shared{P * \shared{Q}{I'}}{I}$ into an equivalent assertions with no nested subjective views. Moreover, as some of the judgements presented in \S\ref{sec:colosl} are limited to local assertions, we provide a method for localising assertions (weakening assertions to local assertions where subjective views are removed) and weakening interference assertions to contain actions with whose pre- and post conditions are local assertions. 
%However, since manipulating assertions including negation is not straightforward (as is typical), we limit the techniques introduced in this section to \emph{standard} assertions: the fragment of \colosl assertions with standard connectives of separation logic excluding negation ($\neg P$) and septraction ($P \septraction Q$). We proceed with the grammar of standard assertions. 
%
%
%\begin{definition}[Standard assertions]
%The \emph{standard \colosl assertions}, $\SAssertions$, is the fragment of assertions that excludes negation and septraction defined by the following grammar. 
%%
%\[
%	\SAssertions \ni P, Q ::= A \mid \exsts{x} P \mid P \lor Q \mid  P * Q \mid P \sepish Q \mid \shared{P}{I}
%\]
%%
%\end{definition}
\subsection{Assertion Rewriting and Weakening}
Although \colosl allows for nested subjective views (e.g. $\shared{P * \shared{Q}{I'}}{I}$), it is often useful to \emph{flatten} them into equivalent assertions with no nested boxes. 
We introduce 
%\emph{local assertions}, $\LAssertions \subset \Assertions$, a subset of assertions that do not contain any subjective views; as well as \emph{flat assertions}, 
\emph{flat assertions}, $\FAssertions \subset \Assertions$, to capture a subset of assertions with no nested subjected views and provide a \emph{flattening mechanism} to rewrite standard \colosl assertions as equivalent flat assertions. 
%
%
\begin{definition}[Flattening]
%The set of \emph{local assertions} $\LAssertions$, and 
The set of \emph{flat assertions} $\FAssertions$ is defined by the following grammar where $\lass{p} \in \LAssertions$.
%
\begin{align*}
%	\LAssertions \ni \lass{P}, \lass{Q} & ::= A \mid \exsts{x} P \mid \lass{P} \lor \lass{Q} \mid \lass{P} * \lass{Q} \mid \lass{P} \sepish \lass{Q}\\
%
	\FAssertions \ni \fass{P}, \fass{Q} & ::= \lass{p} \mid \shared{\lass{p}}{I} \mid \fass{P} \land \fass{Q} \mid \fass{P} \lor \fass{Q} \mid \exsts{x} \fass{P} \mid \fass{P} * \fass{Q} \mid \fass{P} \sepish \fass{Q}
\end{align*}
%
The \emph{flattening} function $\un{.} : \Assertions \rightarrow \FAssertions$ is defined inductively as follows where $x$ does not appear free in $I$ and $\odot \in \{\lor, *, **\}$. 
%
\begin{align*}
	& \un{\lass{p}} \eqdef  \lass{p} \quad &&  
	\un{P \odot Q} \eqdef  \un{P} \odot \un{Q}  \\%\quad \text{for } \odot \in \{\land, \lor, *, \sepish\}\\
%
%
	& \un{\exsts{x} P} \eqdef  \exsts{x} \un{P} &&
	\un{\shared{\shared{P}{I} * Q}{I'}} \eqdef  \un{\shared{P}{I}} * \un{\shared{Q}{I'}}\\
%
%
	& \un{\shared{\lass{p}}{I}} \eqdef  \shared{\lass{p}}{I} \quad &&
	\un{\shared{\shared{P}{I} \sepish Q}{I'}} \eqdef  \un{\shared{P}{I}} * \un{\shared{Q}{I'}}\\
%
%
	& \un{\shared{\exsts{x} P}{I}} \eqdef  \exsts{x} \un{\shared{P}{I}} &&
	\un{\shared{P \lor Q}{I}} \eqdef \un{\shared{P}{I}} \lor \un{\shared{Q}{I}}
%
%
%	& \un{\shared{\shared{P}{I} \land L}{I'}} \eqdef \m{false} &&
%	\un{\shared{\shared{P}{I} \land ((L \lor Q) \odot R)}{I'}} \eqdef 
%	\un{\shared{\shared{P}{I} \land (Q \odot R)}{I'}}\\
%%
%%
%	&&&
%	\un{\shared{\shared{P}{I} \land B}{I'}} \eqdef
%	\un{\shared{P}{I}} * \un{\shared{B}{I'}}
\end{align*}
%
%
%
%\begin{mathpar}
%	B ::= \emp|\, \shared{P}{I}|\, B \odot B
%	
%	L ::= \m{false} \mid \cell{x}{v} \mid [\token{a}] \mid L \ominus Q \quad \text{ for }\ominus \in \{\land, *, \sepish\}
%\end{mathpar}
%
\end{definition}
%
%

It is often useful to weaken an assertion into a local assertion where the resultant assertion contains no subjective views. We thus introduce an \emph{erasure} mechanism for weakening \colosl assertions to local ones. 
%
%
\begin{definition}[Erasure]
The \emph{erasure} of an assertion $\erase{(.)}: \Assertions \rightarrow \LAssertions$ is defined inductively over the structure of standard assertions as follows with $\odot\in\{\lor, *, \sepish \}$
%
\begin{mathpar}
	\erase{A} \eqdef A  

	\erase{(\exsts{x} P)} \eqdef \exsts{x}\erase{P} 

	\erase{(P \odot Q)} \eqdef  \erase{P} \odot \erase{Q}
	
		\erase{\left(\shared{P}{I}\right)} \eqdef \emp 
\end{mathpar}
%
%Observe that $P \implies \erase{P}$ holds for any assertion $P$.
%
\end{definition}
%
%
%%
%\begin{definition}[Interference semantics]\label{def:interference-semantics}
%The \emph{semantics of interference assertions}, $\semI[(.)]{.}: \LEnv --> \AMods$, is defined as follows for $\set{A} \in \pset{\set{Token}}$.
%\begin{align*}
%  \semI[\lenv]{I}(\token{A}) &==
%  \left\{
%  \begin{array}{@{}l@{\ }|@{\ }r@{}}
%    (p \composeL r, q \composeL r)&
%    \begin{array}{@{}l@{}}
%      \interAss{\overline{[\token{a}]}^{\token{A}}}{\bar{y}}{P}{Q} \in I /|  \exsts{\bar{v}, \lmod} \lenv' = [\lenv|||\bar y:\bar v] \\
%      /| (p \composeL r), \lenv'\slsat \sumA{P} 
%      /| (q \composeL r), \lenv' \slsat \sumA{Q} \\
%      /|\ p, \lenv' |=_{p \composeL r, \lmod} P 
%      /| q, \lenv' |=_{q \composeL r, \lmod} Q 
%    \end{array}
%  \end{array}
%  \right\}
%  \end{align*}
%\end{definition}
%%
%
\begin{lemma}\label{lem:flattening-and-erasure}
The following entailments are valid.
%
\begin{mathpar}
	\infer{
		P <=> \un{P}
	}
	{}
	
	\infer{
		P \entails \erase{P}
	}
	{}	
%	
%	\infer{
%		\shared{P}{I} * \shared{Q}{I'} \entails \shared{R}{I} * \shared{Q}{I'}
%	}
%	{
%		R \entails P
%		&
%		P \sepish Q \entails R \sepish Q
%	}
\end{mathpar}
%\hrule
\end{lemma}
%
%
Observe that when despatching a local fencing obligation, using the top left rule in \fig~\ref{fig:local-fencing-rules} one can always weaken an assertion to a local one by erasing it using the second entailment above. Similarly, when despatching shifting obligations, one can use the top left rule in \fig~\ref{fig:shifting-rules} to weaken an assertion by erasing it into a local one.

When despatching proof obligations involving interference assertions such as local fencing ($\fenceAss{} \strictfences I$), fencing ($\fenceAss{} \fences I$) and shifting ($I \weakenI{P} I'$) conditions, it is often easier to do so with respect to interference assertions with actions whose pre- and post-conditions are local assertions. We thus provide a few judgements for weakening the actions of interference assertions to those with local action footprints. 

The intuitive interpretation of an action of the form $\lass{p} * \shared{\lass{q}}{I} \swap \lass{p'} * \shared{\lass{q}}{I}$ is: whenever parts of the shared state satisfies $\lass{P}$ and parts of it satisfies $\lass{Q}$, update the subpart captured by $\lass{P}$ to one described by $\lass{P'}$. As such, given the semantics of interference assertions, the interpretation of this action is contained in that of the action of the form $\lass{P} ** \lass{Q} \swap \lass{P'} ** \lass{Q}$. We define a \emph{merging} operator ($\sumA{}$) on standard assertions for removing boxes and combining their contents using the ($**$) connective. We then appeal to this operator for weakening interference assertions. When defining the merging operator for an assertion $P$, we assume that $P$ is in the \emph{prenex normal form}, i.e. all its existential quantifiers are stringed at the front of the assertion such that $P \equiv \exsts{\bar{x}} P'$ where there are no bound variables in $P'$. 
%We thus provide a procedure for rewriting standard \colosl assertions in the prenex normal form.
%
%%
%\begin{definition}[Prenex normal form]
%Given a standard assertion $P \in \SAssertions$, the \emph{prenex normal form} function, $\promote{.}: \SAssertions \rightarrow \SAssertions$, is defined as follows:
%%
%\begin{mathpar}
%	\promote{P} \eqdef \exists{\bar{x}^{\in S}}.\; P'
%	
%	\text{where } (S, P') = \ep{P}
%\end{mathpar}
%%
%%
%with the auxiliary function $\ep{.}: \Assertions \rightarrow (\pset{\textsf{LVar}} \times \Assertions)$ defined inductively over the structure of assertions as follows where $\odot \in \{\lor, *, \sepish\}$.
%We assume that the bound logical variables of $P$ are pairwise distinct; and that the bound variables of $P$ in $\shared{P}{I}$ do not appear free in $I$\footnote{Note that it is always possible to achieve these stipulations by renaming the bound variables in a capture-avoiding manner.}.
%In what follows, $\ep{P} = (S_1, P')$ and $\ep{Q} = (S_2, Q')$ where applicable.
%%
%\begin{mathpar}
%	\ep{A} \!\!\eqdef\! (\emptyset, A) 
%	
%	\ep{\shared{P}{I}} \!\!\eqdef\!  (S_1, \shared{P'}{I})
%	
%	\ep{P \odot Q} \!\!\eqdef\! \left(S_1 \uplus S_2, P' \odot Q' \right)
%
%	\ep{\exsts{x}P} \!\!\eqdef\! \left({x} \uplus S_1, P'\right)
%\end{mathpar}
%%
%\end{definition}
%%
%%
%\begin{lemma}
%Given any assertion $P \in \SAssertions$, the following entailment is valid.
%%
%\[
%\infer{
%	P <=> \promote{P}
%}
%{}
%\]
%%
%\begin{proof}
%The proof is by induction over the structure of \colosl assertions and follows from their semantics. 
%%The full proof is provided in~\cite{colosl-tr14}.
%\end{proof}
%\end{lemma}
%%
%%
\begin{definition}[Merge]\label{def:assertion-merge}
Given a standard assertion $P \equiv \exsts{\bar{x}} P'$ in the prenex normal form with no bound variables in $P'$, the \emph{merge} operation, $\sumA{(.)}: \Assertions \rightarrow \LAssertions$, is defined as follows:
%
\begin{mathpar}
	\sumA{P} \eqdef  \exists{\bar{x}}.\; \lass{P} \sepish \lass{Q}
	
	\text{where } (\lass{P}, \lass{Q}) = \ub{\un{P'}} 
\end{mathpar}
%
%
with the auxiliary function $\ub{.}: \FAssertions \rightarrow (\LAssertions \times \LAssertions)$ defined inductively over the structure of local assertions as follows where $\lass{p}, \lass{q}, \lass{p'}, \lass{q'} \in \LAssertions$ and $\odot \in \{*, \sepish\}$.
% We assume that the bound logical variables of the left-hand assertion are pairwise distinct (and are otherwise renamed to be distinct in a capture-avoiding manner). 
 In what follows, $\ub{\fass{P}} = (\lass{p}, \lass{p'})$ and $\ub{\fass{Q}} = (\lass{q}, \lass{q'})$ where applicable.
%
\begin{mathpar}
	\ub{\lass{p}} \!\!\eqdef\! (\lass{p}, \emp) 
	
	\ub{\shared{\lass{p}}{I}} \!\!\eqdef\!  (\emp, \lass{p})
	
	\ub{\fass{P} \odot \fass{Q}} \!\!\eqdef\! \left(\lass{p} \odot \lass{q}, \lass{p'} \sepish \lass{q'} \right)

	\ub{\fass{P} \lor \fass{Q}} \!\!\eqdef\! \left(\lass{p} \lor \lass{q}, \lass{p'} \lor \lass{q'} \right)
\end{mathpar}
%
%
\end{definition}
%
%
\begin{definition}[Interference entailment]
An interference assertion $I$ \emph{entails} interference assertion $I'$, written $I \entailsI I'$, if
%
\[
	\for{\lenv \in \LEnv} \semI[\lenv]{I} \subseteq \semI[\lenv]{I'}
\]
%
\end{definition}
\begin{lemma}[Interference weakening]\label{lem:interference-weakening}
The following entailments are valid.
%
\begin{mathpar}
%	\infer{
%		P \confines I	
%	}
%	{
%		\erase{P} \strictfences I	
%	}
	\infer{
		I \cup I_1 \entailsI I \cup I_2
	}
	{
		I_1 \entailsI I_2
	}	
	
	\infer{
		\left\{ \interAss{[\token{A}]}{\bar{y}}{P}{Q} \right\}	
		\entailsI
		\left\{ \interAss{[\token{A}]}{\bar{y}}{\sumA{P}}{\sumA{Q}} \right\}	
	}{}
		
	\infer{
		\left\{ \interAss{[\token{A}]}{\bar{y}}{P}{Q} \right\}	
		\entailsI
		\left\{ \interAss{[\token{A}]}{\bar{y}}{P'}{Q'} \right\}	
	}
	{
		P \entails\! P'
		&
		Q \entails Q'	
	}
%	
\end{mathpar}
%
\end{lemma}
%
Note that, since for any given standard assertion $P$, we have $P |- \erase{P}$ (\lem~\ref{lem:flattening-and-erasure}), it is always possible to weaken interference assertions such that its actions are expressed in terms of local assertions, using the last judgement. 


%
%
\subsection{Extended Judgements}
%
When establishing the stability of a subjective view $\shared{P}{I}$, we need to ascertain that the assertion is robust with respect to the actions in $I$ that may be carried out by the environment. An action $a$ in $I$ may only be executed by the environment if it holds the associated capability \emph{and} $a$'s pre-condition is compatible with the shared state. Consequently, if the capability associated with $a$ is unavailable to the environment (either because it is in the shared state or is held locally by the thread), \emph{or} the pre-condition of $a$ is not compatible with the shared state, the action cannot be performed by the environment and thus the subjective view is stable against $a$. As such, when establishing the stability of an assertion $P$, we often take into account the \emph{context} of the assertion. That is, we combine our knowledge of the thread's local state and other parts of the shared state to rule out those actions that cannot be performed by the environment. For instance, given the assertion $\shared{P}{I} * [\token{a}] * R$ where $I \eqdef \left\{[\token{a}]: P \swap P'\right\}$, we can determine that given the context $[\token{a}] * R$, the subjective view is stable against the action of $[\token{a}]$ since it is held locally and thus cannot be utilised by the environment. Similarly, given the assertion $\shared{P}{I} * \shared{\cell{\li x}{1}}{I'}$ where $I \eqdef \left\{[\token{a}]: P * \cell{\li x}{2}\swap P' * \cell{\li x}{2} \right\}$, we can infer that given the context $\shared{\cell{\li x}{1}}{I'}$, the subjective view is stable against the action of $[\token{a}]$ since $\cell{\li x}{2}$ is incompatible with $\cell{\li x}{1}$. 

Similar to the merging operator of \defin~\ref{def:assertion-merge}, given a standard assertion $R$ that describes the context within which stability is determined, we define a \emph{gathering} operator ($\prodA{}$) that weakens $R$ into a local assertion. We then define a combination operator $\combine{.}{.}$ that combines the stability context with the action pre-condition. 
%
%
\begin{definition}[Gather]
Given a standard assertion $P \equiv \exsts{\bar{x}} P'$ in the prenex normal form with no bound variables in $P'$, the \emph{gather} operation, $\prodA{(.)}: \Assertions \rightarrow \LAssertions$, is defined as follows:
%
\begin{mathpar}
	\prodA{P} \eqdef \exists{\bar{x}}.\; \lass{P} * \lass{Q}
	
	\text{where } (\lass{P}, \lass{Q}) = \ub{\un{P'}} 
\end{mathpar}
%
%
with the auxiliary function $\ub{.}: \FAssertions \rightarrow (\LAssertions \times \LAssertions)$ defined as in \defin~\ref{def:assertion-merge},
\end{definition}
%
%
%
\begin{definition}[Combination]
Given an assertion $P \eqdef \exsts{\bar{x}}P'$ describing the current state, and a standard assertion $Q \eqdef \exsts{\bar{y}}Q'$ describing an action precondition, their \emph{combination} $\combine{.}{.}: \Assertions \times \Assertions \rightarrow \LAssertions$, is defined as follows. We assume that $P$ and $Q$ are in their prenex normal form with no bound variables in $P'$ and $Q'$.
%
\[
\begin{array}{l l}
	\combine{P}{Q} \eqdef & \exsts{\bar{x}, \bar{y}} \lass{P} * (\lass{P}' \sepish \lass{Q} \sepish \lass{Q}')\\
	& \quad \text{where } (\lass{P}, \lass{P}') = \ub{\un{P'}} \text{ and }  (\lass{Q}, \lass{Q}') = \ub{\un{Q'}} 
\end{array}
\]
%
\end{definition}
%
%
\fig~\ref{fig:additional-judgements} presents additional judgements for fencing and shifting. Recall that the rules presented in \fig~\ref{fig:local-fencing-rules} and \fig~\ref{fig:shifting-rules} were tailored for those interference assertions whose pre- and post-conditions were local assertions. However, using the rightmost judgements on the first two rows of \fig~\ref{fig:additional-judgements} combined with the last two entailments of \lem~\ref{lem:interference-weakening}, on can first weaken interference assertions as to contain only local assertions and then appeal to the judgements of \fig~\ref{fig:local-fencing-rules} and \fig~\ref{fig:shifting-rules} to despatch the relevant obligations. \emph{Mutatis mutandis} for stability judgements. Finally, although the stability rules of \fig~\ref{fig:additional-judgements} are defined for flat assertions, as shown in \lem~\ref{lem:flattening-and-erasure}, given any standard assertion $P$, one can always rewrite it as an equivalent flat assertion. 
%
%
\begin{figure}
\hrule\vspace{5pt}
\begin{mathpar}
	\infer{
		\fenceAss{} \fences \left\{[\token{A}]: \exsts{\bar{y}} \lass{P} \swap \lass{Q}\right\}
	}
	{
		\fenceAss{} ** (\exsts{\bar{y}} \lass{P}) \slentails \m{false}	
	}
	
	\infer{
		\fenceAss{} \fences I
	}
	{
	 I \entailsI I' 
	 &
	 \fenceAss{} \fences I'
	}\vspace{20pt}
\end{mathpar}
%
%
\begin{mathpar}
%	\infer{
%	 	I  \weakenI{\fenceAss{}} \emptyset
%	 }
%	 {
%	 	I  \entailsI I'
%	 	&
%	 	I' \weakenI{\fenceAss{}} \emptyset
%	 }
	\infer{
		\left\{[\token{A}]: \exsts{\bar{y}} \lass{P} \swap \lass{Q}\right\}	 \weakenI{\fenceAss{}}
		\emptyset 
	}
	{
		\fenceAss{} ** (\exsts{\bar{y}} \lass{P}) \slentails \m{false}	
	}
	
	\infer{
		I \cup I' \weakenI{\fenceAss{}} I
	}
	{
		I \cup I' \entailsI I''
		&
		I'' \weakenI{\fenceAss{}} I
	}
%		
%	\infer{
%		\left\{ [\token{A}]: P \swap Q \right\} \weakenI{\fenceAss{}} I'
%	}
%	{
%		\left\{ [\token{A}]: \erase{P} \swap \erase{Q} \right\} \weakenI{\fenceAss{}} I'
%	}
		
	\infer
 %	[\proofRule{Exist}]
 	{
 		\left\{[\token{A}]: \exists\overline{v_i \in S_i}^{i \in I}.\, P \swap Q \right\} 
 		\approx^{\m{true}}
 		\bigcup\limits_{\overline{w_i \in S_i}^{i \in I}} \left\{[\token{A}]: P \overline{[w_i /v_i]}^{i \in I} \swap  Q \overline{[w_i /v_i]}^{i \in I} \right\} 
 	}
 	{
 	}	\vspace{25pt}
\end{mathpar}
%
%
\begin{mathpar}
	 \infer{
	 	\stable{P \odot Q}	
	 }
	 {
	 	\stable{P}
	 	&
	 	\stable{Q}	
	 }

	\infer{
	 	\stable{P * Q}	
	 }
	 {
	 	\stableTo{P}{Q}
	 	&
	 	\stableTo{Q}{P}	
	 }

	\infer{
	 	\stableTo{P}{R}	
	 }
	 {
	 	\stable{P}
	 }	

	\infer{
	 	\stableTo{P * Q}{R}
	 }
	 {
	 	\stableTo{P}{Q * R}
	 	&
	 	\stableTo{Q}{P * R}
	 }	
	
	\infer{
	 	\stableTo{P}{R * R'}
	 }
	 {
	 	\stableTo{P}{R}
	 }	
		
	\infer{
	 	\stableTo{P \odot Q}{R}
	 }
	 {
	 	\stableTo{P}{R}
	 	&
	 	\stableTo{Q}{R}
	 }	
	
	\infer{
	 	\stableTo{\exsts{x} P}{R}
	 }
	 {
	 	\stableTo{P}{R}
	 }
	
	\infer{
	 	\stableTo{\shared{\lass{P}}{I}}{R}
	 }
	 {
	 	\stableTo{\shared{\lass{P}}{I'}}{R}
	 	&
	 	I' \weakenI{\lass{P}} I
	 }
		
	 \infer{
	 	\stableTo{\shared{\lass{P}}{I}}{R}
	 }
	 {
	 	\stableIn{\lass{P}}{I}{R * \shared{\lass{P}}{I}}
	 }

	\infer{
	 	\stableIn{\lass{P}}{I}{R}	
	 }
	 {
	 	I \entailsI I'
	 	&
	 	\stableIn{\lass{P}}{I'}{R}	
	 }	
	
	\infer{
	 	\stableIn{\lass{P}}{\left\{\interAss{[\token{A}]}{\bar{y}}{\lass{Q}_1}{\lass{Q}_2}\right\}}{R}	
	 }
	 {
	 	[\token{A}] * \prodA{R} \slentails \m{false}
	 }
	
	
	 \infer{
	 	\stableIn{\lass{P}}{\left\{[\token{A}]: \lass{Q}_1 \swap \lass{Q}_2 \right\}} {R}	
	 }
	 {
	 	\combine{R}{\lass{Q}_1} \slentails \m{false}
	 }
	
	
	 \infer{
	 	\stableIn{\lass{P}}{\left\{[\token{A}]: \lass{Q}_1 \swap \lass{Q}_2 \right\}}{R}	
	 }
	 {
	 	\left(\lass{Q}_1 \septraction \combine{R}{\lass{Q}_1} \,\right) * \lass{Q}_2 \slentails \lass{P} * \m{true}
	 }
\end{mathpar}
\hrule
\caption{Additional judgements for fencing, shifting and stability where $\odot \in \{\lor, *, **\}$, $P, Q, R, R' \in \FAssertions$ and $\lass{P}, \lass{Q}, \lass{Q}_1, \lass{Q}_2, \fenceAss{} \in \LAssertions$.}
\label{fig:additional-judgements}
\end{figure}
%
%

%\fig~\ref{fig:stability-rules} presents a number of judgements that reduce stability checks to logical entailments. These judgements are defined for flat assertions. However, as shown in \lem~\ref{lem:flattening-and-erasure}, given any standard assertion $P$, it is always possible to rewrite it as an equivalent flat assertion. 
%%
%\begin{figure*}
%\hrule\vspace*{5pt}
%\begin{mathpar}
%	\infer{
%		\stable{\lass{p}}	
%	}{}
%	
%%	\infer{
%%		\stable{P}	
%%	}
%%	{
%%		\stable{\un{P}}	
%%	}
%%
%	\infer{
%		\stable{P \odot Q}	
%	}
%	{
%		\stable{P}
%		&
%		\stable{Q}	
%	}
%
%	\infer{
%		\stable{\exsts{x} P}	
%	}
%	{
%		\stable{P}
%	}
%
%%	\infer{
%%		\stable{\shared{\lass{P}}{I}}	
%%	}
%%	{
%%		\lass{P} \fences I	
%%	}	
%%	
%%	\infer[?]{
%%		\stable{\shared{\lass{P}}{I}}	
%%	}
%%	{
%%		\stable{\shared{\lass{P}}{I'}} 
%%		&
%%		I' \weakenI{\lass{P}} I
%%	}
%%	
%%	
%	\infer{
%		\stable{P * Q}	
%	}
%	{
%		\stableTo{P}{Q}
%		&
%		\stableTo{Q}{P}	
%	}
%\end{mathpar}\vspace{5pt}\\
%%
%%
%\begin{mathpar}
%	\infer{
%		\stableTo{P}{R}	
%	}
%	{
%		\stable{P}
%	}	
%	
%	\infer{
%		\stableTo{P * Q}{R}
%	}
%	{
%		\stableTo{P}{Q * R}
%		&
%		\stableTo{Q}{P * R}
%	}	
%	
%	\infer{
%		\stableTo{P}{R * R'}
%	}
%	{
%		\stableTo{P}{R}
%	}	
%		
%	\infer{
%		\stableTo{P \odot Q}{R}
%	}
%	{
%		\stableTo{P}{R}
%		&
%		\stableTo{Q}{R}
%	}	
%	
%	\infer{
%		\stableTo{\exsts{x} P}{R}
%	}
%	{
%		\stableTo{P}{R}
%	}
%	
%	\infer{
%		\stableTo{\shared{\lass{P}}{I}}{R}
%	}
%	{
%		\stableTo{\shared{\lass{P}}{I'}}{R}
%		&
%		I' \weakenI{\lass{P}} I
%	}
%	
%	
%	\infer{
%		\stableTo{\shared{\lass{P}}{I}}{R}
%	}
%	{
%		\stableIn{\lass{P}}{I}{R * \shared{\lass{P}}{I}}
%	}
%\end{mathpar}\vspace{5pt}\\
%%
%%
%\begin{mathpar}
%	\infer{
%		\stableIn{\lass{P}}{I}{R * Q}
%	}
%	{
%		\stableIn{\lass{P}}{I}{R}
%	}
%%	
%%	
%%	\infer{
%%		\stableIn{\lass{P}}{I}{\fass{R}}	
%%	}
%%	{
%%		\stableIn{\lass{P}}{I'}{\fass{R}}
%%		&
%%		I' \weakenI{\lass{P}} I
%%	}	
%	
%	\infer{
%		\stableIn{\lass{P}}{I_1 \cup I_2}{R}	
%	}
%	{
%		\stableIn{\lass{P}}{I_1}{R}
%		&
%		\stableIn{\lass{P}}{I_2}{R}
%	}	
%	
%	\infer{
%		\stableIn{\lass{P}}{I}{R}	
%	}
%	{
%		I \entailsI I'
%		&
%		\stableIn{\lass{P}}{I'}{R}	
%	}	
%	
%	\infer{
%		\stableIn{\lass{P}}{\left\{[\token{A}]: \lass{Q}_1 \swap \lass{Q}_2 \right\}} {R}	
%	}
%	{
%		\combine{R}{\lass{Q}_1} \slentails \m{false}
%	}
%	
%	\infer{
%		\stableIn{\lass{P}}{\left\{\interAss{[\token{A}]}{\bar{y}}{\lass{Q}_1}{\lass{Q}_2}\right\}}{R}	
%	}
%	{
%		[\token{A}] * \prodA{R} \slentails \m{false}
%	}
%	
%	\infer{
%		\stableIn{\lass{P}}{\left\{[\token{A}]: \lass{Q}_1 \swap \lass{Q}_2 \right\}}{R}	
%	}
%	{
%		\left(\lass{Q}_1 \septraction \combine{R}{\lass{Q}_1} \,\right) * \lass{Q}_2 \slentails \lass{P} * \m{true}
%	}
%%	
%%	
%%	
%\end{mathpar}
%\hrule
%%\vspace*{5pt}
%\caption{Stability judgements where $P, Q, Q_1, Q_2, R, R' \in \FAssertions$; $\lass{P}, \lass{Q}_1, \lass{Q}_2 \in \LAssertions$ and $\odot \in \{\lor, *, **\}$.}
%\label{fig:stability-rules}
%\end{figure*}
%%
%
%\subsection{Reasoning about $**$ and $\separate{}{}$ for standard assertions}
%\label{sec:elimination-judgements}
%
%\emph{Pure} assertions are those which describe empty heaps.
%
%\begin{definition}[Pure assertions]
%  The \emph{pure} local assertions are those given by the following
%  grammar:
%  \[
%  \PAssertions \ni p,q ::= emp \mid \m{false}\mid E_1 = E_2 \mid \exsts{x} p \mid p \lor q \mid  p * q \mid p \sepish q
%  \]
%\end{definition}
%
%
%\fig~\ref{fig:sepish-rules} presents a number of rules for eliminating the overlapping conjunction ($**$) connective in standard assertions. 
%%
%\begin{figure}[h!]
%\hrule\vspace{5pt}
%\begin{mathpar}
%	\infer{
%		p ** \lass{Q} <=> p * \lass{Q}
%	}
%	{p\in\PAssertions}
%		
%	
%	\infer{
%		\cell{x}{v} ** [y] <=> \cell{x}{v} * [y]
%	}
%	{}
%
%	\infer{
%		\cell{x}{v} ** \cell{y}{v'} <=> \left(x \not= y \land \cell{x}{v} * \cell{y}{v'}\right) \lor \left(x = y \land v = v' \land \cell{x}{v}\right)
%	}
%	{}
%	
%	\infer{
%		[x] ** [y] <=> \left( x \not= y \land [x] * [y]\right) \lor \left( x = y \land [x]\right)
%	}
%	{}
%					
%	\infer{
%		(\lass{P} \lor \lass{Q}) ** \lass{R} <=> (\lass{P} \sepish \lass{R}) \lor (\lass{Q} ** \lass{R})
%	}
%	{}
%	
%	\infer{
%		(\exsts{x} \lass{P} 	) ** \lass{Q} <=> \exsts{x} \lass{P} ** \lass{Q}
%	}
%	{
%		\m{fv}(\lass{Q}) \cap \{x\} = \emptyset	
%	}
%%
%%	
%\end{mathpar}
%\hrule
%\caption{Rules for eliminating the overlapping conjunction ($**$) connective where $\lass{P}, \lass{Q}, \lass{R} \in \LAssertions$.}
%\label{fig:sepish-rules}
%\end{figure}
%%
%% 
%
%\noindent\fig~\ref{fig:intersection-rules} presents a number of judgements for establishing that the states described by two standard assertions do not intersect.
%%
%\begin{figure}[h!]
%\hrule\vspace{5pt}
%\begin{mathpar}
%	\infer{
%		\separate{\emp}{\lass{P}}
%	}
%	{}
%		
%	\infer{
%		\separate{\m{false}}{\lass{P}}
%	}
%	{}
%	
%	\infer{
%		\separate{[y]}{\cell{x}{v}}
%	}
%	{}	
%
%	\infer={
%		\separate{\lass{P}}{\lass{Q}}
%	}
%	{
%		\separate{\lass{Q}}{\lass{P}}
%	}
%	
%	\infer{
%		\separate{[x]}{[y]} <=> x \not= y
%	}
%	{}
%		
%	\infer{
%		\separate{\cell{x}{v}}{\cell{y}{v'}} <=> \left( x \not= y \lor (x = y \land v \not= v') \right)
%	}
%	{
%	}
%
%%	\infer{
%%		\separate{(\lass{p} \septraction \lass{q})}{\lass{r}}
%%	}
%%	{
%%		\separate{\lass{q}}{\lass{r}}
%%	}
%	\infer{
%		\separate{(\lass{P} \odot \lass{Q})}{R}
%	}
%	{	
%		\separate{\lass{P}}{\lass{R}}
%		&
%		\separate{\lass{Q}}{\lass{R}}
%	}
%	
%	\infer={
%		\separate{(\exsts{x} \lass{P})}{R}		
%	}
%	{
%		\separate{(\lass{P}[v/x])}{R}
%		&
%		\text{for }v \in \set{Val}
%	}
%%	
%%	\infer={
%%		\separate{(\for{x} \lass{P})}{R}		
%%	}
%%	{
%%		\separate{(\lass{P}[v/x])}{R}
%%		&
%%		\text{for }v \in \set{Val}
%%	}
%%
%%	
%\end{mathpar}
%\hrule
%\caption{Rules for empty intersection judgements where $\odot \in \{\lor, *, ** \}$ and $\lass{P}, \lass{Q}, \lass{R} \in \LAssertions$.}
%\label{fig:intersection-rules}
%\end{figure}
%%
%
