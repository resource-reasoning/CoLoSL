\paragraph{Concluding Remarks}
%
We have introduced \colosl, a new program logic for reasoning locally
about the shared state. We focus on 
subjective views,  which expand and contract to 
provide a flexible
treatment of both the shared resource and its interference. 
However, \colosl is still young, and
lacks many features of its various cousins.  There are many
interesting ideas present in the literature: e.g.\ abstract states
governed by state transition systems~\cite{caresl}; higher-order
reasoning~\cite{icap}; and abstract atomicity~\cite{tada}. All these
ideas require further investigation. Here, our aim was to simply
introduce subjective views as a fundamental new way of underpinning such reasoning.

%In another direction, \colosl makes heavy use of a richer fragment
%of separation logic than is typically considered (in particular by
%automatic tools), specifically due to the inclusion of the $**$ and
%$--o$ connectives. Improved ways of reasoning about these connectives
%would greatly facilitate more complex proofs, be it in \colosl or
%other program logics based on separation logic.


\paragraph{Acknowledgements}
We are grateful to Aquinas Hobor for providing us with the example of
\S\ref{sec:intuition}, and to Matthew Parkinson for
suggesting to record catalysts in the model. We would also like to thank
Pedro da Rocha Pinto for his continuous feedback on earlier versions
of this paper. This research was funded by EPSRC grants K008528/1 and
H008373/2.
