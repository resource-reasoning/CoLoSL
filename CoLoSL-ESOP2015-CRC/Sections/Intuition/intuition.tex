\section{Informal Development}
\label{sec:intuition}

We illustrate the main \colosl reasoning principles by sketching a
proof of an implementation of Dijkstra's token ring mutual exclusion
algorithm, which pioneered \emph{self-stabilising} distributed
algorithms~\cite{dijkstra74}.

The token ring algorithm, %presented in \fig\ref{fig:concurrentInc},
assumes a network of $n{+}1$ machines, arranged in a ring with a designated \emph{master} machine and$n$ slave machines. Each machine maintains a local counter and has
access to the value of its left neighbour's counter; the state of the
system consists of all $n{+}1$ counters. Starting in an arbitrary state,
the network eventually stabilises to legitimate
states~\cite{dijkstra-proof}, with the following global property: 
in the $i$th legitimate state, all machines in the range $0\cdots i{-}1$ (with machine $0$ being the master) have some value $v{+}1$, and all others in the $i \cdots n$ range have value $v$.  In the $i$th legitimate state, only the $i$th machine can make progress: it increments its counter by $1$ and takes the system to the next legitimate state ($i{+}1 \,\text{mod}\, n$). For a ring at address $x$, the $i$th legitimate state is depicted below (for both $i>0$ and $i=0$).  \\ \null\hfill
%a number of 
%successive machines in the ring have the same value $v+1$ (up to say,
%machine $i$, with machine $0$ being the master), then the others have
%value $v$.  In a legitimate state, only one machine at a time is able
%to make progress: either the $i+1$st machine if $i\neq 0$, or the
%master machine if $i=0$ (i.e.\ all machines in the ring hold the same
%value). Both situations are depicted below. Thus, this token ring
%provides mutual exclusion that can be used to mitigate access to a
%shared resource.\\ \null\hfill
  \begin{tabular}{lllllll}
    $x$&$x{+}1$&&$x{+}i{-}1$&$x{+}i$&&$x{+}n$
    \\
    \hline
    \multicolumn{1}{|c|}{v+1} &
    \multicolumn{1}{|c|}{v+1} &
    \multicolumn{1}{|c|}{~$\cdots$~} &
    \multicolumn{1}{|c|}{v+1} &
    \multicolumn{1}{|c|}{~~v~\raisebox{1ex}{$\bullet$}} &
    \multicolumn{1}{|c|}{~$\cdots$~} &
    \multicolumn{1}{|c|}{v}\\
    \hline
  \end{tabular}\hfill
  \begin{tabular}{lllllll}
    $x$&$x{+}1$&&&&&$x{+}n$\\
    \hline
    \multicolumn{1}{|c|}{~~v~\raisebox{1ex}{$\bullet$}} &
    \multicolumn{1}{|c|}{v} &
    \multicolumn{4}{|c|}{~$\cdots$~} &
    \multicolumn{1}{|c|}{v}\\
    \hline
  \end{tabular}
  \hfill\null\\


In the accompanying technical report ~\cite{colosl-tr14}, we outline a proof sketch of token ring's
self-stabilisation phase in \colosl. We show that,
starting from an arbitrary state, the system converges to a legitimate
state. We also provide a proof of the token ring being used for mutual
exclusion (in a legitimate state, only one machine can make progress,
indicated by the $\bullet$ above).  Here, we
focus on the part of the algorithm that makes interesting use of \colosl. We assume that the token ring is in  a legitimate state and all counters hold value $0$. Our proof will make use of the \colosl
principles laid out in \fig\ref{fig:principles}, together with the
usual concurrency rule of separation logic~\cite{csl-tcs} and the rule
of semantic consequence from the Views framework~\cite{views}.  We introduce
them informally as needed, and present them in more detail in
\S\ref{sec:colosl}.

%% \begin{figure}
%%   \centering
%%   \caption{Two legitimate configurations of the token ring.}
%%   \label{fig:ring}
%% \end{figure}

\begin{figure}
\centering
\noindent\hrule
\begin{align*}
  \shared{P * Q}{I} &=> \shared{P}{I}  &\mathllap{(\forgetRule)}&
  &
  \shared{P}{I_1}\!\! * \shared{Q}{I_2} &=> \shared{P \sepish Q}{I_1
    \cup I_2}
  \tag{\mergeRule}
  \\
  (P => Q)
  &=>
  \shared{P}{I} => \shared{Q}{I}
  &(\weakenRule)&
  &
  I \weakenI{P} I'
  \text{ implies }&
  \shared{P}{I} => \shared{P}{I'}
  &\tag{\shiftRule}
  \\
  \shared{P}{I} &
  => \shared{P}{I} * \shared P I
  &(\copyRule)&
  &
  P \containI I 
%  \text{ and }
%  \fresh{\vec{x}, \capAss{1} * \capAss{2}}
  \text{ implies }&
  P ===>
  \exsts{\token A, \token A'} [\token A] * \shared{P * [\token A']}{I}
  \tag{\extendRule}
\end{align*}

\vspace{-15pt}
\begin{mathpar}
\infer[\parRule]
	{
		\{P_1 * P_2\} \;\mathbb{C}_1 || \mathbb{C}_2\; \{Q_1 * Q_2\}		
	}
	{
		\{P_1\} \;\mathbb{C}_1\; \{Q_1\}
		\\
		\{P_2\} \;\mathbb{C}_2\; \{Q_2\}
	}

\infer[\conseqRule]
	{
	  \hoare{P}{\mathbb{C}}{Q}
	}
	{
          P ===> P'&
	  \hoare{P'}{\mathbb{C}}{Q'}&
          Q' ===> Q
	}
\end{mathpar}
\hrule
\caption{Main reasoning principles and rules of \colosl.}
\label{fig:principles}
\end{figure}

\colosl introduces a new assertion $\shared{P} I$ called a
\emph{subjective view}, which comprises an assertion $P$ describing
a \emph{part} of the global shared state and an interference assertion
$I$ which characterises how this partial shared state may be changed
by the thread or the environment. Similar to interference assertions
of CAP~\cite{cap-ecoop10} and permissions of deny-guarantee~\cite{dg},
$I$ declares transitions of the form $[\token a] : Q \swap R$, where a
thread in possession of the $[\token a]$ capability (in its local state) may carry out its transition and update parts of the shared
state satisfying $Q$ to those satisfying $R$. Assertions in Hoare
triples must be {\em stable} with respect to the interference from the environment: that is, robust with respect to the interference assertion $I$. The $*$
(resp.\ $**$) connective is that of separation logic and means that
the current state is a disjoint (resp.\ potentially overlapping)
composition of two states satisfying each sub-formula. The overlapping
conjunction $**$ has been used in the past to reason about sharing in
data structures~\cite{rey-slnotes,js-popl12,ramification}. The
\emph{view shift}~\cite{views} (or
\emph{repartitioning}~\cite{cap-ecoop10}) $P ===> Q$ means that an
(instrumented) state satisfying $P$ may be manipulated without
changing the machine states it denotes. In particular, $(P => Q) => (P
===> Q)$.

Consider the program $\li{ring(x)}$ defined in
\fig\ref{fig:concurrentInc} representing a token ring with $n{+}1$ machines.
%ignoring the assertions (lines starting with \li{//}). 
%We represent each machine in the ring as a heap address that records the value of the counter 
It is written in pseudo-code resembling C with additional constructs
for concurrency: atomic sections $\atomic \_$ which declare that code
behaves atomically; and parallel composition $\_ ||\_ $ which spawns
threads then waits until they complete. In our example, $n{+}1$ threads
are spawned (corresponding to the master thread and $n$ slave threads)
and are run in parallel until all counters reach value $10$.
%(in the fashion described above).

%Note that this implementation can easily be generalised to an
%$n$-place ring. We focus on a 3-place ring for concision.
While the implementation of all slave threads are identical, we shall see that the proof of the
first slave in the ring (at $x{+}1$) is slightly different from the others.
We henceforth refer to the first slave thread as the
\emph{foreman}.  Let us proceed with the proof outline of the other
slave thread, which would apply as-is to any further slaves added to
the program.

%
\begin{figure}
\centering
\noindent\hrule\vspace{10pt}
\vspace{-3ex}
{\begin{lstlisting}[numbers=left,numbersep=5pt]
ring(x)
//$\color{blue}\left\{\begin{array}{@{}l@{}}\tx{x}|-< x * [\token m_{x}] * [\token s_{x{+}1}] * \cdots * [\token s_{x{+}n}]* \shared{\begin{array}{@{}l@{}}\tx{n}|-<n * x|->0 * x{+}1|->0 * \cdots * x{+}n|->0\end{array}}{I}\end{array} \right\}$
$\{$ master(x) $\mid\mid$ slave(x+1) $\mid\mid \cdots \mid\mid$ slave(x+n);
$\}$ //$\color{blue}\left\{\begin{array}{@{}l@{}}\tx{x}|-< x * [\token m_{x}] * [\token s_{x{+}1}] * \cdots * [\token s_{x{+}n}]* \shared{\begin{array}{@{}l@{}}\tx{n}|-<n * x|->10 * x{+}1|->10 * \cdots * x{+}n|->10\end{array}}{I}\end{array} \right\}$
\end{lstlisting}}
%
\begin{tabular}{@{} l @{\hspace{2ex}} l@{}}
{\begin{lstlisting}
master(x)
//$\color{blue} \left\{\begin{array}{@{}l@{}}\varcell{\tx x}{x} * [\token m_{x}]* \shared{\varcell{\tx n}{n} * \cell{x}{0} * \cell{x{+}n}{0}}{M_{x}'}\end{array}\right\}$
$\{$ while(*x != 10)
//$\color{blue} \left\{\begin{array}{@{}l@{}}\varcell{\tx x}{x} *[\token m_{x}]*\null\\\shared{\begin{array}{@{}l@{}} \varcell{\tx n}{n} * \exsts{v}\cell{x}{v} * \cell{x{+}n}{v} \end{array}}{M_{x}'}\end{array} \right\}$
    $\{\ \langle$if (*x == *(x+n))
        *x = *x + 1;$\rangle\ \}$
$\}$ //$\color{blue} \left\{\begin{array}{@{}l@{}}\varcell{\tx x}{x} * [\token m_{x}]*\null\\\shared{\begin{array}{@{}l@{}}\varcell{\tx n}{n} * \cell{x}{10} * (\cell{x{+}n}{10} \lor \cell{x{+}n}{9})\end{array}}{M_{x}'}\end{array} \right\}$
\end{lstlisting}}
&
\begin{lstlisting}
slave(x)
//$\color{blue} \left\{\begin{array}{@{}l@{}}\varcell{\tx x}{x} * [\token s_{x}]*\null\\\shared{\begin{array}{@{}l@{}}\cell{x}{0} * (\cell{x{-}1}{0} \lor \cell{x{-}1}{1})\end{array}}{S_{x}'}\end{array} \right\}$
$\{$  while(*x != 10)
//$\color{blue} \left\{\begin{array}{@{}l@{}}\varcell{\tx x}{x} * [\token s_{x}]*\null\\\shared{\begin{array}{@{}l@{}}\exsts{v}\cell{x}{v} *\null\\ (\cell{x{-}1}{v} \lor \cell{x{-}1}{v{+}1})\end{array}}{S_{x}'}\end{array} \right\}$
    $\{\ \langle$if (*x != *(x-1))
       *x = *(x-1);$\rangle\ \}$
$\}$ //$\color{blue} \left\{\begin{array}{@{}l@{}}\varcell{\tx x}{x} * [\token s_{x}]*\null\\ \shared{\begin{array}{@{}l@{}}\cell{x}{10} *\\ (\cell{x{-}1}{10} \lor \cell{x{-}1}{11})\end{array}}{S_{x}'}\end{array} \right\}$
\end{lstlisting}
\end{tabular}
\[
\begin{array}{@{}r@{}l@{}l@{}}
  s_x &\null ==
  [\token s_x]{:}\, \exsts{v} \cell{x}{v} * \cell{x{-}1}{v{+}1} &\null\swap
  \cell{x}{v{+}1} * \cell{x{-}1}{v{+}1}
  \\
  m_{x} &\null ==
  [\token m_{x}]{:}\, \exsts{v,n}\varcell{\tx n}{n} * \cell{x}{v} * \cell{x{+}n}{v} &\null\swap
  \varcell{\tx n}{n} * \cell{x}{v{+}1} * \cell{x{+}n}{v}
  \\
  s'_x &\null ==
  [\token s_x]{:}\, \exsts{v} \cell{x{+}1}{v} * \cell{x}{v} * \cell{x{-}1}{v{{+}}1} &\null\swap
  \cell{x{+}1}{v} * \cell{x}{v{+}1} * \cell{x{-}1}{v{+}1}
\end{array}
\]
\vspace{-8pt}
\[
\begin{array}{@{}r@{}l@{}}
  m'_{x} &\null ==
  [\token m_{x}]{:}\, \exsts{v,n}
  \begin{array}[t]{@{}l@{}}
    \varcell{\tx n}{n} * \cell{x{+}1}{v} * \cell{x}{v} * \cell{x{+}n}{v} \swap\null\\
    \qquad \varcell{\tx n}{n} * \cell{x{+}1}{v} * \cell{x}{v{+}1} *
    \cell{x{+}n}{v}
  \end{array}
  \\
  l'_{x}  &\null ==
  [\token s_{x}]{:}\, \exsts{v,n}
  \begin{array}[t]{@{}l@{}}
    \varcell{\tx n}{n} * \cell{x}{v{+}1} * \cell{x{+}n{-}1}{v{+}1} * \cell{x{+}n}{v}
    \swap\null\\
    \qquad
    \varcell{\tx n}{n} *
    \cell{x}{v{+}1} * \cell{x{+}n{-}1}{v{+}1} * \cell{x{+}n}{v{+}1}
  \end{array}
\end{array}
\]
\vspace{-5pt}
\begin{mathpar}
  I \eqdef \{m_{x}, s_{x{+}1}, \cdots, s_{x{+}n} \}

  M_{x}' \eqdef \{m_x, l_{x}'\}

  F_{x}' \eqdef \{s_x, m_{x{-}1}'\}

  S_x' \eqdef \{s_x, s_{x{-}1}'\}
\end{mathpar}
\vspace{-5pt}\hrule
\caption{A proof sketch of the ring program together in \colosl. Lines
  starting with $\color{blue}{//}$ contain assertions that describe the
  local state and the subjective shared state at the relevant program
  point. The proof of \li{slave} is still valid when replacing $S_x'$
  with $F_x'$, which is needed to prove the first slave of the ring
  (the foreman).}
\label{fig:concurrentInc}
\end{figure}
%
%% In our reasoning, we represent the resources associated with each
%% machine in the ring as two consecutive cells in the heap recording
%% the current value of the associated counter; and b)
%% the address of the previous machine in the ring (at address $x+1$).
%% We now proceed with the proof outline of slave threads. The proof of
%% the master and foreman threads are similar and can be described
%% analogously.

\paragraph{Proof of slave threads}
Let us forget for a moment about the proof outline of
\fig\ref{fig:concurrentInc} and attempt to prove \li{slave} in
isolation, in the spirit of local reasoning. Since
\li{slave(x)} inspects the value of its counter pointed to by \li{x}
and compares it against the counter at \li{x-1} (its left neighbour in
the ring), a tempting precondition for $\li{slave(x)}$ would be one describing
just these two locations, e.g.
\vspace{-3pt}
\[
\li{x} \harpoonything x * [\token{s}_x] * \shared{\cell{x}{0} * (\cell{x{-}1}{0} \lor \cell{x{-}1}{1})}{S_x}
\qquad
S_x = \{s_{x}, s_{x{-}1}\}
\vspace{-3pt}
\]
%
The above assertion comprises: a) a variable assertion stating that the thread
locally owns  variable \li{x} with value $x$ (using the variables-as-resource model~\cite{variablesAsResource}); b) a
\emph{capability} $[\token{s}_x]$ that allows it to perform the
associated $s_x$ action (see below and \fig\ref{fig:concurrentInc}); and c) a subjective view of the shared state: $x$
points to $0$ and $x{-}1$ (its left neighbour) points to either $0$ or $1$, since its left
neighbour might have already incremented its own counter. The
interference assertion associated with the subjective view is captured by $S_x$ and consists of two
actions: $s_x$ and $s_{x{-}1}$, where $s_{x}$ represent an increment
of the contents of $x$ under the condition that its value is one less
than the value at address $x{-}1$. Since it owns $[\token s_x]$ locally, the
current thread is the only one that can perform $s_x$. On the other
hand, the capability $[\token s_{x{-}1}]$ is not locally owned, thus
the environment could potentially perform the
associated action whenever its precondition (on the left-hand side of $\swap$) is satisfied.
%that performed by the current machine (enabled by $[\token{x}]$); and that of the previous machine in the ring (enabled by $[\token{s}_p]$). Each action allows the value of the corresponding machine to be incremented by $1$ so long is it is the one that can make a move (As described above). 
Upon closer inspection, since the subjective view says nothing of the
value of the cell at address $x{-}2$, $s_{x{-}1}$ can always fire in this
situation. This assertion is thus not \emph{stable}. If we were to
weaken it to a stable assertion, we would get the assertion below,
which is now too imprecise for our purposes:\vspace{-3pt}
\[
\li{x} \harpoonything x * [\token{s}_x] * \shared{\exsts{v}
  \cell{x}{0} * \cell{x{-}1}{v}}{S_x}
\vspace{-3pt}
\]

To remedy this, we have to step back and think again at the level of
the whole algorithm. As the programmer knows, the situation above
cannot happen as $x{-}2$ can only be at most one ahead of $x$ itself.
We can thus replace $S_x$ by $S'_x$ and give a stronger stable
precondition that captures just what we want, as in
\fig\ref{fig:concurrentInc}. The proof of $\li{slave(x)}$ is now
relatively straightforward. By inspection (or using the rules of
\S\ref{subsec:prules}), the invariant of the while loop and the
postcondition are also stable. The atomic section performs action
$s_x$ (temporarily ``opening the box'', i.e.\ transferring shared
resources into the local state to perform the update) and preserves
the invariant.  The final postcondition of $\li{slave(x)}$ follows
from the invariant and the boolean expression of the loop.

\paragraph{Proof of master and foreman threads}
The proof sketches of the master and foreman (first slave) threads are
analogous, following \fig\ref{fig:concurrentInc}. As the first slave,
the foreman has to account for interference from the master thread
instead of another slave thread. Note that the master has access to
variable \li{n} holding the current size of the ring, as
does its associated action $m_x$, since whether it can fire or not is
decided by the value of the counter at address $x+n$. 
%Let us now detail the proof of \li{ring(x)}.

\paragraph{Proof of the ring}
The precondition of \li{ring} states that it owns all capabilities,
which will be distributed amongst the $n+1$ threads; the global
variable \li{n} is shared, as are all three counters, initialised
to 0. The interference $I$ associated with the subjective view
consists of the actions of the three threads. Because at this stage we have a global view
of the state of the ring (and moreover all capabilities are held
locally), the $s_{x+i}$ actions are enough to guarantee
stability.

Let us write $\shared P I$ for this initial subjective
view. This assertion may be freely duplicated using the \copyRule
principle of \fig\ref{fig:principles} and each thread is given a copy together with the
appropriate capability using the \parRule rule. For instance, the
thread running \li{slave(x+$i$)} (for $i>1$) gets $[\token{s}_{x{+}i}] *
\shared{P}{I}$.  This assertion does not match the precondition of
\li{slave(x+i)} just yet. Using the principles given
in \fig\ref{fig:principles}, we can weaken the assertion as such:
%
\begin{align*}
  \shared{P}{I} 
  &\stackrel{(\shiftRule)}{=>}
  \shared{P}{\{m_x\} \cup \{s_{x{+}j} \mid 1 \leq j < i-1\} \cup \{s'_{x{+}i{-}1}\} \cup \{s_{x{+} j} \mid  i \leq j \leq n\}}\\[-5pt]
%  
  &\stackrel{(\forgetRule)}{=>}
  \shared{\cell{x{+}i{-}1}{0} * \cell{x{+}i}{0}}{\{m_x, \cdots, s'_{x{+}i{-}1}, s_{x{+}i, \cdots, s_{x{+}n}}\}} \\[-5pt]
%  
  &\stackrel{(\shiftRule)}{=>}
  \shared{\cell{x{+}i{-}1}{0} * \cell{x{+}i}{0}}{S'_{x{+}i}}\\[-5pt]
% 
  &\stackrel{(\weakenRule)}{=>}
  \shared{\cell{x{+}2}{0} * (\cell{x{+}1}{0} \lor \cell{x{+}1}{1})}{S'_{x{+}2}}
\end{align*}
%
We begin by rewriting the $s_{x{+}1}$ action of $I$ into the stronger
action of $s'_{x{+}1}$ using \shiftRule. In general, \shiftRule allows
us to replace $I$ with any interference relation $I'$ that has the
same observable effect as far as the subjective assertion $P$ of the
subjective view is concerned (written $I \weakenI{P} I'$). In this
instance, actions $s_{x{+}2}$ and $s'_{x{+}2}$ have the same effect
according to $P$, as discussed in the proof of \li{slave(x)}. As such,
rewriting $s_{x{+}1}$ as $s'_{x{+}2}$ does not alter its behaviour and
merely reflects stronger knowledge about how $x{+}2$ and $x$ are
related through $x{+}1$. In particular, $I\weakenI{P} \{m_{x},
s'_{x{+}2}, s_{x{+}2}\}$ as required.

Next, because subjective views only describe \emph{parts} of the
shared state, we can use the \forgetRule principle to obtain a weaker
view of the shared state, in this case a view that ignores the cell at
address $x$.  With $x$ out of scope, the action $m_{x}$ no longer has
observable effects on the assertion, since it mutates only the cell at
address $x$. We can thus apply the \shiftRule principle again to rid
the interference of $m_{x}$ and get $S'_{x{+}2}$.

Finally, we weaken the resulting subjective view so that it is stable
with respect to $S'_{x{+}2}$, i.e.\ preserved by those of its actions
that the environment may perform (here, $s'_{x{+}2}$). This
yields the precondition of \li{slave(x+2)} as in
\fig\ref{fig:concurrentInc}. The preconditions of the master and
foreman threads can be derived analogously.

Once all threads have completed their operations, we join up their
postconditions using the \mergeRule principle, which embodies a
crucial feature of \colosl: different subjective views
\emph{overlap}. Since $|/$ distributes over $**$, the subjective view
simplifies to $\shared{\cell{x}{10} * \cell{x{+}1}{10} *
  \cell{x{+}2}{10}}{S'_{x} \cup S'_{x{+}1} \cup
  S'_{x{+}2}}$. Finally, since $S'_{x} \cup S'_{x{+}1} \cup
S'_{x{+}2}$ $\weakenI{\li{n}|-< n * \cell{x}{10} * \cell{x{+}1}{10} *
  \cell{x{+}2}{10}} I $, by the \shiftRule principle, we get
the postcondition of \li{ring(x)}. This concludes our \colosl proof of \li{ring(x)}. 

We complete our semi-formal overview by noting that unlike
CAP~\cite{cap-ecoop10} and as in iCAP~\cite{icap}, we do not provide
an explicit \emph{unsharing} mechanism to claim shared resources and
render them thread-local. Instead, this behaviour can be simply
encoded for a resource described by $P$ and associated with tokens
$[\token{a}_1]$, \ldots , $[\token{a}_{n}]$ as an additional action
of the form \vspace{0pt}
\[
[\token{a}_1] * \cdots * [\token{a}_{n}]{:}\, P \swap [\token{a}_1] * \cdots * [\token{a}_{n}]\vspace{0pt}
\]
That is, a thread holding all the tokens can remove the resource $P$
from the shared state and move them into its local state. In return,
it must transfer the associated capabilities into the shared
state. Borrowed from CAP~\cite{cap-ecoop10}, this pattern of resource
transfer is common in \colosl actions. In general, resources that only
appear on the left hand side of $\swap$ indicate their removal from
the shared state (and their addition to the thread's local state);
dually, resources that only appear on the right hand side of $\swap$
denote their transfer to the shared state (from the thread's local
state).

\paragraph{Small specifications and proof reuse}
Our expansion and
contraction of subjective views, in particular with shifting of
interference assertions in key places, enables us to confine the
specification and verification of each thread to just the resources
they need. Such small specifications make proofs robust against changes to each thread's
environment, and provide more opportunities for proof reuse.
% For
%instance, our proof of the ring straightforwardly generalises to $n$
%slave threads, without needing to reprove \li{master} or \li{slave} at
%all. 

For instance, let us now modify the code of \li{ring(x)} in line~3 by
forking an additional thread that dynamically grows the ring by
spawning more slave threads (we leave the details to the accompanying
technical report). When the ring has size
$n$, we use an instance of the \extendRule principle as follows in
order to add a new slave (at $x{+}n{+}1$) to the shared state. In
doing so, we also define the new slave's associated interference and
accompanying capability.
\begin{align*}
  \cell{(x{+}n{+}1)}{v} \semimplies \exsts{\token{s}_{x{+}n{+}1}}[\token{s}_{x{+}n{+}1}] * \shared{\cell{x{+}n{+}1}{v}}{\{s_{x{+}n{+}1}\}}
\end{align*}
Our proof changes only minimally to accommodate the updated program: the proof
of \li{master(x)} now has to account for the new
interference on the value of $n$ since the environment can grow the
ring and thus increment the value of $n$. However, the proofs of
other threads can be reused straightforwardly.

%% \begin{figure}
%% \centering
%% \noindent\hrule
%% \begin{tabular}{@{} l @{\hspace{4ex}} l@{}}
%% {\begin{lstlisting}
%% spawn(x)
%% //$\color{blue} \left\{\begin{array}{@{}l@{}}\exsts{x, s, n} \tx{x}|-< x * \tx{size}|-< s * [\token e_{x}]*\null\\ \bigstar_{i=n{+}1}^{s} (\cell{x{+}i}{-})\;\; * \\\shared{\begin{array}{@{}l@{}}\exsts{v}  \tx{n}|-< n * \cell{x}{v} *\\ \quad(\cell{x{+}n}{v} \lor\cell{x{+}n}{v{-}1})\end{array}}{E_x}\end{array} \right\}$
%% $\{$ if(n $<$ size) then
%% //$\color{blue} \left\{\begin{array}{@{}l@{}}\exsts{x, s, n} n{<}s \land \tx{x}|-< x * \tx{size}|-< s * [\token e_{x}]*\null\\ \bigstar_{i=n+1}^{s} (\cell{x{+}i}{-})\;\; * \\\shared{\begin{array}{@{}l@{}}\exsts{v}  \tx{n}|-< n * \cell{x}{v} *\\ \quad(\cell{x{+}n}{v} \lor\cell{x{+}n}{v{-}1})\end{array}}{E_x}\end{array} \right\}$
%%     $\langle$ *x + n + 1 = *x;
%%        n = n+1; $\rangle$
%% //$\color{blue} \left\{\begin{array}{@{}l@{}}\exsts{x, s, n} \tx{x}|-< x * \tx{size}|-< s * [\token e_{x}]*\null\\ \bigstar_{i=n+1}^{s} (\cell{x{+}i}{-})\;\; * [\token{s}_{x{+}n}] * \\\shared{\begin{array}{@{}l@{}}\exsts{v} \tx{n}|-< n *\cell{x}{v} * \\(\cell{x{+}n{-}1}{v} * \cell{x{+}n}{v{-}1}\lor\null\\ \cell{x{+}n{-}1}{v{-}1} * \cell{x{+}n}{v{-}1})\end{array}}{E_x \cup S'_{x+n}}\end{array} \right\}$
%% //$\color{blue} \left\{\begin{array}{@{}l@{}}\exsts{x, s, n} \tx{x}|-< x * \tx{size}|-< s * [\token e_{x}]*\null\\ \bigstar_{i=n{+}1}^{s} (\cell{x{+}i}{-})\;\; * \\ \shared{\begin{array}{@{}l@{}}\exsts{v}  \tx{n}|-< n * \cell{x}{v} *\\ \quad(\cell{x{+}n}{v} \lor\cell{x{+}n}{v{-}1})\end{array}}{E_x} *\\       \![\token{s}_{x+n}] *\shared{\begin{array}{@{}l@{}}\exsts{v} \cell{x{+}n{-}1}{v} * \\ (\cell{x{+}n}{v}\lor \cell{x{+}n}{v{-}1} )\end{array}}{S'_{x+n}}  \end{array} \right\}$
%%   slave(x+n) $\mid\mid$ spawn(x);
%% $\}$ //$\color{blue} \{\m{false}\}$
%% \end{lstlisting}}
%% &
%% \begin{lstlisting}
%% master(x)
%% //$\color{blue} \left\{\begin{array}{@{}l@{}}\varcell{\tx x}{x} * [\token m_{x}]* \\ \shared{\exsts{n} \varcell{\tx n}{n} * \cell{x}{0} * \cell{x{+}n}{0}}{E_{x}}\end{array}\right\}$
%% $\{$ while(*x != 10)
%% //$\color{blue} \left\{\begin{array}{@{}l@{}}\varcell{\tx x}{x} *[\token m_{x}]*\null\\\shared{\begin{array}{@{}l@{}} \exsts{n, v} \varcell{\tx n}{n} * \cell{x}{v} * \\ \;\cell{x+n}{v} \lor \cell{x{+}n}{v-1}\end{array}}{E_{x}}\end{array} \right\}$
%%     $\{\ \langle$if (*x == *(x+n))
%%         *x = *x + 1;$\rangle\ \}$
%% $\}$ //$\color{blue} \left\{\begin{array}{@{}l@{}}\varcell{\tx x}{x} * [\token m_{x}]*\null\\\shared{\begin{array}{@{}l@{}} \exsts{n}\varcell{\tx n}{n} * \cell{x}{10} * \\
%% \cell{x{+}n}{10} \lor \cell{x{+}n}{9})\end{array}}{E_{x}}\end{array} \right\}$
%% \end{lstlisting}
%% \end{tabular}
%% \begin{mathpar}
%%   e_x ==
%%   [\token e_x]{:}\, \exsts{n} \li{n} \harpoonything n \swap \li{n} \harpoonything n+1

%%   E_x \eqdef \{e_x\} \cup M'_x
%% \end{mathpar}
%% \hrule
%% \caption{Growing the ring.}
%% \label{fig:spawner}
%% \end{figure}

In contrast, in existing approaches such as CAP~\cite{cap-ecoop10}, both $n$ and the global interference relation are observed by all threads.  As such, with the above extension, the global interference relation needs to change (to include the interference on $n$) and the proofs of \emph{both} master and slave threads need to be adapted. 
%
%
%% . For instance, in order to
%% remove the \li{x}, \li{y} and \li{z} variables from the shared state
%% at the end of the $\mathbb{INC}$ program, one can add the following
%% action to all of the above interference assertions ($I$,
%% $I_{\small\li{x}}'$,
%% $I_{\small\li{y}}'$, and $I_{\small\li{z}}'$):
%% %
%% \vspace{-5pt}
%% \[
%% 	a_{\textsf{rem}} \eqdef [\{\token{a}_{\small\li{x}}, \token{a}_{\small\li{y}}, \token{a}_{\small\li{z}}\}]: \cell{\li x}{10} * \cell{\li y}{10} * \cell{\li z}{10} \swap [\token{a}_{\small\li{x}}] * [\token{a}_{\small\li{y}}] * [\token{a}_{\small\li{z}}]
%% \]
%% %
%% That is, when \li{x}, \li{y} and \li{z} all hold value $10$, any thread in possession of all three $[\token{a}_{\small\li{x}}]$, $[\token{a}_{\small\li{y}}]$, $[\token{a}_{\small\li{z}}]$ capabilities in its local state can carry out $a_{\textsf{rem}}$. In doing so, the acting thread will remove the three cells from the shared state (and move them into its local state). In return, it must transfer the three locally held capabilities into the shared state. Note that after adding $a_{\textsf{rem}}$, the subjective views of all three threads remain stable since each thread locally holds one of the capabilities in $\{\token{a}_{\small\li{x}}, \token{a}_{\small\li{y}}, \token{a}_{\small\li{z}}\}$ and thus knows that the environment cannot carry out $a_{\textsf{rem}}$ (as it is missing at least one of the required capabilities). Borrowed from CAP~\cite{cap-ecoop10}, this pattern of resource transfer is common in \colosl actions. In general, resources that only appear on the left hand side of $\swap$ indicate their removal from the shared state (and their addition to the thread's local state); dually, resources that only appear on the right hand side of $\swap$ denote their transfer to the shared state (from the thread's local state).
