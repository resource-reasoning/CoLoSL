%
%
\begin{lemma}[]\label{lem:amodMerge}
%
\[
\begin{array}{l}
	\for{p, q, c, r \in \LState} \for{\amod{}, \amod{L}, \amod{1}, \amod{2} \in \AMods} \for{n \in \Nats}\\
	\hspace*{0.5cm} \extendsAMUpto{\amod{}, \amod{L}}{n}{p \composeL c}{q \composeL r}{\amod{1}} \;\land\; \extendsAMUpto{\amod{}, \amod{L}}{n}{q \composeL c}{p \composeL r}{\amod{2}}
	\implies\\
	\hspace*{2cm} \extendsAMUpto{\amod{}, \amod{L}}{n}{p \composeL c \composeL q}{r}{\amod{1} \cup \amod{2}}
\end{array}
\]
%
\begin{proof} By induction on number of steps $n$.\\
\noindent Pick an arbitrary $p, a, q, b \in \LState, \amod{}, \amod{L}, \amod{1}, \amod{2} \in \AMods$.\\
\noindent\textbf{Base case}\\
\textbf{RTS. }\hspace*{0.5cm}$\extendsAMUpto{\amod{}, \amod{L}}{0}{p \composeL a}{r}{\amod{1} \cup \amod{2}}$\\
This holds trivially by definition of $\extendsAMUpto{\amod{}, \amod{L}}{0}{p \composeL a}{r}{\amod{1} \cup \amod{2}}$\\

\noindent
\textbf{}
\textbf{Inductive Step} Pick an arbitrary $n \in \Nats$, then
%
\begin{equation}
	\tag{I.H.}
	\begin{array}{l}
		\for{p, q, c, r \in \LState} \for{\amod{}, \amod{L}, \amod{1}, \amod{2} \in \AMods} \for{n \in \Nats}\\
%		\hspace*{0.5cm} p \composeL c \in F_1 \;\land\; F_1 \fences \amod{1} \;\land\; c \composeL q \in F_2 \;\land\; F_2 \fences \amod{2} \;\land\; (F_1, \amod{1}) \agrees (F_2, \amod{2})\\
		\hspace*{0.5cm} \extendsAMUpto{\amod{}, \amod{L}}{(n-1)}{p \composeL c}{q \composeL r}{\amod{1}} \;\land\; \extendsAMUpto{\amod{}, \amod{L}}{(n-1)}{q \composeL c}{p \composeL r}{\amod{2}}
		\implies\\
		\hspace*{2cm} \extendsAMUpto{\amod{}, \amod{L}}{(n-1)}{p \composeL c \composeL q}{r}{\amod{1} \cup \amod{2}}
	\end{array}
\label{LM:I.H}
\end{equation}
%
Assume:
%
\begin{align}
%	& p \composeL c \in F_1 \label{LM:Ass1}\\
%	& F_1 \fences \amod{1} \label{LM:Ass2}\\
%	& c \composeL q \in F_2 \label{LM:Ass3}\\
%	& F_2 \fences \amod{2} \label{LM:Ass4}\\
%	& (F_1, \amod{1}) \agrees (F_2, \amod{2}) \label{LM:Ass5}\\
	& \extendsAMUpto{\amod{}, \amod{L}}{n}{p \composeL c}{q \composeL r}{\amod{1}} \label{LM:Ass6}\\
	& \extendsAMUpto{\amod{}, \amod{L}}{n}{q \composeL c}{p \composeL r}{\amod{2}} \label{LM:Ass7}
\end{align}
%
Show
%
\begin{align}
	&\for{\ca{}, d, e, t, l_3, l_4}\for{(l_1 \composeL f, l_2 \composeL f) \in (\amod{1} \cup \amod{2})(\ca{})} \nonumber\\
	& \left(\for{l'} l' \leq l_1 \land l' \leq l_2 \implies l' = \unitL\right) \;\land\; l_1 \composeL f \leq  p  \composeL c \composeL q \composeL r \composeL t \;\land \nonumber\\
	& l_1 = l_3 \composeL l_4 \;\land\; l_1 \maxMeetL p \composeL c \composeL q = l_3 \;\land\; p \composeL c \composeL q = l_3 \composeL d \;\land\; r = l_4 \composeL e \implies \nonumber\\
	&\hspace*{0.5cm} \extendsAMUpto{\amod{}, \amod{L}}{(n-1)}{l_2 \composeL d}{e}{\amod{1} \cup \amod{2}} \;\land \nonumber\\
	& \hspace*{0.5cm} t = \unitL \implies (l_1 \composeL d \composeL e, l_2 \composeL d \composeL e) \in \amod{}(\ca{}) 
\;\lor\; l_2 \composeL d \composeL e \text{ is undefined} \label{LM:Goal1}\\
%
	&\begin{array}{l}
		\for{\ca{}} \for{(l_1 \composeL f, l_2 \composeL f) \in \amod{L}(\ca{})} \for{e, d} \\
  \hspace*{0.2cm} l_1 \composeL f \composeL e = p \composeL c \composeL q \composeL r \composeL d \;\land\; \left(\for{l'} l' \leq l_0 \land l' \leq l'_0 \implies l' = \unitL\right)  \;\implies\\
  \hspace*{0.4cm}\exsts{f', e'} (l_1 \composeL f', l_2 \composeL f') \in \amod{1} \cup \amod{2}(\ca{}) \;\land\; l_1 \composeL f' \composeL e' =  p \composeL c \composeL q \composeL r \composeL d\\
		\hspace*{0.4cm}\lor\;l_2 \composeL f \composeL e \text{ is undefined}\\
		\hspace*{0.4cm}\lor\; l_1 \maxMeetL p \composeL c \composeL q  = \unitL \;\land\; \exsts{r'} l_2 \composeL f \composeL e = p \composeL c \composeL q \composeL r' \composeL d \;\land\\
		\hspace*{0.8cm} \extendsAMUpto{\amod{}, \amod{L}}{(m-1)}{p \composeL c \composeL q}{r'}{\amod{1} \cup \amod{2}}
	\end{array}\label{LM:Goal2}\\\nonumber
\end{align}
%

\noindent\textbf{RTS. (\ref{LM:Goal2})}\\
Pick an arbitrary $\ca{}, l_1, l_2, f, d, e$ such that:
%
\begin{align}
	\begin{array}{l}
		(l_1 \composeL f, l_2 \composeL f) \in \amod{L}(\ca{})\\
		l_1 \composeL f \composeL e =  p \composeL c \composeL q \composeL r \composeL d\\
  	\left(\for{l'} l' \leq l_1 \land l' \leq l_2 \implies l' = \unitL\right)\\
	\end{array} \label{LAM:Ass1}
\end{align} 
%
%Since $p \composeL a = q \composeL b$, we know $p \composeL a \meetL q \composeL b \not= \emptyset$. Let:
%%
%\begin{align}
%	\begin{array}{l}
%		c = p \maxMeetL q\\
%		p = p' \composeL c\\
%		q = q' \composeL c
%	\end{array} \label{LM:max}
%\end{align}
%Pick an arbitrary $\ca{}, (l_1, l'_1) \in \amod{\ca{}}$ such that 
%%
%\begin{equation}
%	l_1 = p \composeL c \composeL q \composeL r \label{LM:Ass28}
%\end{equation}
%%(\ref{LM:Ass})
Then from (\ref{LAM:Ass1}) and (\ref{LM:Ass6}) we have:
%
\begin{align}
	\begin{array}{l}
		\exsts{f', e'} (l_1 \composeL f', l_2 \composeL f') \in \amod{1}(\ca{}) \;\land\; l_1 \composeL f' \composeL e' =  p \composeL c \composeL q \composeL r \composeL d\\
		\lor\;l_2 \composeL f \composeL e \text{ is undefined}\\
		\lor\; l_1 \maxMeetL p \composeL c = \unitL \;\land\; \exsts{r_1} l_2 \composeL f \composeL e = p \composeL c \composeL r_1 \composeL d \;\land\\
		\hspace*{0.4cm} \extendsAMUpto{\amod{}, \amod{L}}{(m-1)}{p \composeL c}{r_1}{\amod{1} }
	\end{array} \label{LM:Ass29}
\end{align}
%
Similarly from (\ref{LM:Ass7}) we have:
%
\begin{align}
	\begin{array}{l}
	\exsts{f', e'} (l_1 \composeL f', l_2 \composeL f') \in \amod{2}(\ca{}) \;\land\; l_1 \composeL f' \composeL e' =  p \composeL c \composeL q \composeL r \composeL d\\
		\lor\;l_2 \composeL f \composeL e \text{ is undefined}\\
		\lor\; l_1 \maxMeetL c \composeL q  = \unitL \;\land\;  \exsts{r_2} l_2 \composeL f \composeL e = c \composeL q \composeL r_2 \composeL d \;\land\\
		\hspace*{0.4cm} \extendsAMUpto{\amod{}, \amod{L}}{(m-1)}{c \composeL q}{r_2}{\amod{2} }
	\end{array} \label{LM:Ass30}
\end{align}
%(\ref{LM:Ass})
From (\ref{LM:Ass29}) and (\ref{LM:Ass30}) we have:
%
\begin{align}
	\begin{array}{l}
		\exsts{f', e'} (l_1 \composeL f', l_2 \composeL f') \in \amod{1} \cup \amod{2}(\ca{}) \;\land\; l_1 \composeL f' \composeL e' =  p \composeL c \composeL q \composeL r \composeL d\\
		\lor\;l_2 \composeL f \composeL e \text{ is undefined}\\
		\lor\; l_1 \maxMeetL p \composeL c \composeL q  = \unitL \;\land\; \exsts{r_1, r_2}\\
		\hspace*{0.2cm} l_2 \composeL f \composeL e = p \composeL c \composeL r_1 \composeL d \;\land\;
		\extendsAMUpto{\amod{}, \amod{L}}{(m-1)}{p \composeL c}{r_1}{\amod{1} }\;\land\\
		\hspace*{0.2cm} l_2 \composeL f \composeL e = c \composeL q \composeL r_2 \composeL d \;\land\;
		\extendsAMUpto{\amod{}, \amod{L}}{(m-1)}{c \composeL q}{r_2}{\amod{2} }
	\end{array}
	\label{LM:Ass31}
\end{align}
%(\ref{LM:Ass})
In the case of the third disjunct where $l_1 \maxMeetL p \composeL c \composeL q  = \unitL $, since $l_1 \leq p \composeL c \composeL q \composeL r \composeL d$, we can deduce $l_1 \leq r \composeL d$, that is:
%
\begin{align}
	\begin{array}{l l}
		\exsts{g} l_1 \composeL g = r \composeL d
	\end{array} \nonumber
\end{align}
%
and consequently $l_1 \composeL f \composeL e = p \composeL c \composeL q \composeL r \composeL d = p \composeL c \composeL q \composeL l_1 \composeL g$
%
and thus by cancellativity of separation algebras we have: 
\begin{align}
	f \composeL e = p \composeL c \composeL q \composeL g \label{LM:Ass32}
\end{align}
%
From (\ref{LM:Ass32}) we can rewrite (\ref{LM:Ass31}) as:
%
\begin{align}
	\begin{array}{l}
		\exsts{f', e'} (l_1 \composeL f', l_2 \composeL f') \in \amod{1} \cup \amod{2}(\ca{}) \;\land\; l_1 \composeL f' \composeL e' =  p \composeL c \composeL q \composeL r \composeL d\\
		\lor\;l_2 \composeL f \composeL e \text{ is undefined}\\
		\lor\; l_1 \maxMeetL p \composeL c \composeL q  = \unitL\\
		\hspace*{0.2cm} l_2 \composeL f \composeL e = l_2 \composeL p \composeL c \composeL q \composeL g\;\land\\
		\hspace*{0.4cm}\extendsAMUpto{\amod{}, \amod{L}}{(m-1)}{p \composeL c}{q \composeL g \composeL l_2}{\amod{1} } \;\land\; \extendsAMUpto{\amod{}, \amod{L}}{(m-1)}{c \composeL q}{l_2 \composeL p \composeL g}{\amod{2} }
	\end{array}
	\label{LM:Ass33}
\end{align}
%(\ref{LM:Ass})
From (\ref{LM:Ass33}) and (\ref{LM:I.H}) we can rewrite (\ref{LM:Ass33}) as:
%
\begin{align}
	\begin{array}{l}
		\exsts{f', e'} (l_1 \composeL f', l_2 \composeL f') \in \left(\amod{1} \cup \amod{2}\right)(\ca{}) \;\land\; l_1 \composeL f' \composeL e'  =  p \composeL c \composeL q \composeL r \composeL d\\
		\lor\;l_2 \composeL f \composeL e \text{ is undefined}\\
		\lor\; l_2 \composeL f \composeL e = l_2 \composeL p \composeL c \composeL q  \composeL g \;\land\\
		\hspace*{0.4cm} \extendsAMUpto{\amod{}, \amod{L}}{(m-1)}{p \composeL c \composeL q}{g \composeL l_2}{\amod{1} \cup \amod{2}}
	\end{array}  \label{LM:Ass34}
\end{align}
%
that is:
%
\begin{align}
	\begin{array}{l}
		\exsts{f', e'} (l_1 \composeL f', l_2 \composeL f') \in \left(\amod{1} \cup \amod{2}\right)(\ca{}) \;\land\; l_1 \composeL f' \composeL e'  =  p \composeL c \composeL q \composeL r \composeL d\\
		\lor\;l_2 \composeL f \composeL e \text{ is undefined}\\
		\lor\; \exsts{r'} l_2 \composeL f \composeL e = p \composeL c \composeL q  \composeL r' \;\land\\
		\hspace*{0.4cm} \extendsAMUpto{\amod{}, \amod{L}}{(m-1)}{p \composeL c \composeL q}{r'}{\amod{1} \cup \amod{2}}
	\end{array}  \label{LM:Ass34}
\end{align}
%
%(\ref{LM:Ass})
%
%
%
%
%

\noindent\textbf{RTS. (\ref{LM:Goal1})}\\
Pick an arbitrary $\ca{}, d, e, t, l_4, l_3, (l_1 \composeL f, l_2 \composeL f) \in (\amod{1} \cup \amod{2})(\ca{})$ such that
\begin{align}
	& \left(\for{l'} l' \leq l_1 \land l' \leq l_2 \implies l' = \unitL\right)\label{LM:Ass47}\\
	& l_1 \maxMeetL  p \composeL c \composeL q = l_3 \;\land\; l_1 = l_3 \composeL l_4 \label{LM:Ass48}\\
	& p \composeL c \composeL q = l_3 \composeL d \label{LM:Ass49}\\
	& r = l_4 \composeL e \label{LM:Ass50}\\
	& l_1 \composeL f \leq p \composeL c \composeL q \composeL r \composeL t\label{LM:Ass51}
%	& (l_1 \composeL d \composeL e, l_2 \composeL d \composeL e) \in \amod{}(\ca{}) \label{LM:Ass52}
\end{align}
%(\ref{LM:Ass})
%
Then by definition of $\amod{1} \cup \amod{2}$ we know:
%
\begin{align}
	(l_1 \composeL f, l_2 \composeL f) \in \amod{1}(\ca{}) \;\lor\; (l_1 \composeL f, l_2 \composeL f) \in \amod{2}(\ca{}) \label{LM:Ass61}
\end{align}
%
\textbf{Case 1.} $(l_1 \composeL f, l_2 \composeL f) \in \amod{1}(\ca{}) \;\land\; (l_1 \composeL f, l_2 \composeL f) \not\in \amod{2}(\ca{})$\\
%
Then from (\ref{LM:Ass6}) and (\ref{LM:Ass47})-(\ref{LM:Ass50}) we know
%
\begin{align}
\begin{array}{l}
\begin{array}{l l}
	\exsts{s_1, s_2, g, h} & l_3 = s_1 \composeL s_2\\
	& s_1 = l_3 \maxMeetL p \composeL c\\
	& p \composeL c = s_1 \composeL g\\
	& q = s_2 \composeL h\\
	& d = g \composeL h\\
	& \extendsAMUpto{\amod{}, \amod{L}}{(n-1)}{l_2 \composeL g}{h \composeL e}{\amod{1}} \;\land \\
	& t = \unitL \implies (l_1 \composeL d \composeL e, l_2 \composeL d \composeL e) \in \amod{}(\ca{}) \;\lor\; l_2 \composeL d \composeL e \text{ is undefined}
\end{array}\\
\end{array}	\label{LMN:Ass1}
\end{align}
%
From (\ref{LM:Ass7}), (\ref{LMN:Ass1} and (\ref{LM:Ass47})-(\ref{LM:Ass51}) and assumption of case 1 we have:
%
\begin{align}
	&\begin{array}{l}
		\exsts{f', j}(l_1 \composeL f', l_2 \composeL f') \in \amod{2}(\ca{}) \;\land\; l_1 \composeL f' \composeL j = l_1 \composeL d \composeL e \composeL t \\
		\lor\; l_2 \composeL d \composeL e \text{ is undefined}\\
		\lor\; l_1 \maxMeetL c \composeL q = \unitL \;\land\; \exsts{r'} l_2 \composeL d \composeL e = q \composeL c \composeL r' \composeL t \;\land\; \extendsAMUpto{\amod{}, \amod{L}}{(n-1)}{q\composeL c}{r'}{\amod{2}}
	\end{array} \nonumber
\end{align}
%
Either $l_2 \composeL d \composeL e $ is undefined in which case the desired result holds trivially. On the other hand if $l_2 \composeL d \composeL e$ is defined, then there are two cases to consider:\\
\textbf{Case 1.1.} 
\[
\begin{array}{l}
		l_1 \maxMeetL c \composeL q = \unitL \;\land\;  \exsts{r'} l_2 \composeL d \composeL e = q \composeL c \composeL r' \composeL t \;\land\; \extendsAMUpto{\amod{}, \amod{L}}{(n-1)}{q\composeL c}{r'}{\amod{2}}
\end{array}
\]
%
From  (\ref{LMN:Ass1}) and since $l_1 \maxMeetL c \composeL q = \unitL$ (assumption of case 1.1), from Lemma \ref{lem:divideUpper} we have:
%
\begin{align}
	\begin{array}{l l}
		& s_2 = \unitL\\
		& q = h\\
		& l_3 = s_1\\
		\exsts{i} & l_3 \composeL i = p\\
		& g =  i \composeL c\\
		& d = i \composeL c \composeL q\\
		& \extendsAMUpto{\amod{}, \amod{L}}{(n-1)}{l_2 \composeL i \composeL c}{q \composeL e}{\amod{1}}\\
		& t = \unitL \implies (l_1 \composeL d \composeL e, l_2 \composeL d \composeL e) \in \amod{}(\ca{})
	\end{array}\label{LM:Ass64}
\end{align}
%
From (\ref{LM:Ass64}) we have $l_2 \composeL d \composeL e = l_2 \composeL i \composeL c \composeL q \composeL e$, and thus the assumption of case 1.1. and by cancellativity of separation algebras we have: $r' = l_2 \composeL i \composeL e$ and thus from the assumption of case 1.1. we have:
%
\begin{align}
	\extendsAMUpto{\amod{}, \amod{L}}{(n-1)}{q\composeL c}{l_2 \composeL i \composeL e}{\amod{2}} \label{LM:Ass65}
\end{align}
%
Consequently, from (\ref{LM:Ass64}) and (\ref{LM:Ass65}) and (\ref{LM:I.H}) we have:
%
\begin{align}
	\extendsAMUpto{\amod{}, \amod{L}}{(n-1)}{l_2 \composeL i \composeL c \composeL q}{e}{\amod{1} \cup \amod{2}} \nonumber
\end{align}
%(\ref{LM:Ass})
and thus from (\ref{LM:Ass64})
%
\begin{align}
	\extendsAMUpto{\amod{}, \amod{L}}{(n-1)}{l_2 \composeL d}{e}{\amod{1} \cup \amod{2}} \label{LM:Ass66}
\end{align}
%(\ref{LM:Ass})
From (\ref{LM:Ass64})  and (\ref{LM:Ass66}) we have:
\begin{align}
	\extendsAMUpto{\amod{}, \amod{L}}{(n-1)}{l_2 \composeL d}{e}{\amod{1} \cup \amod{2}} \;\land\; t = \unitL \implies (l_1 \composeL d \composeL e, l_2 \composeL d \composeL e) \in \amod{}(\ca{})   \nonumber
\end{align}
%
as required.\\
%
%
%
%
%

\noindent\textbf{Case 1.2.}
\[
\begin{array}{l l}
	\exsts{f', j}& (l_1 \composeL f', l_2 \composeL f') \in \amod{2}(\ca{}) \;\land\; l_1 \composeL f' \composeL j = p \composeL c \composeL q \composeL r \composeL t\\
%	&\land\; l_1 \leq (c \composeL q \maxMeetL p \composeL c)
\end{array}
\]
%
%
From (\ref{LMN:Ass1}) and Lemma \ref{lem:divideUpper} we have:
%
\begin{align}
	\begin{array}{l l}
		\exsts{l_5, l_6, i, k} \hspace*{0.2cm} & s_1 = l_5 \composeL l_6 \\
		& p = l_5 \composeL i\\
		& c = l_6 \composeL k\\
		& g = i \composeL k
	\end{array} \label{LM:Ass70}
\end{align} 
%
From (\ref{LM:Ass70}), assumption of case 1, (\ref{LM:Ass6}), (\ref{LM:Ass50}) and (\ref{LMN:Ass1}) we have:
%
%
\begin{align}
	\begin{array}{l}
		\extendsAMUpto{\amod{}, \amod{L}}{(n-1)}{l_2 \composeL i \composeL k}{h \composeL e}{\amod{1}} \;\land		\\
		t = \unitL \implies (l_1 \composeL d \composeL e, l_2 \composeL d \composeL e) \in \amod{}(\ca{}) \;\lor\; l_2 \composeL d \composeL e \hspace*{0.2cm}\text{ is undefined}
	\end{array} \label{LM:Ass71}
\end{align}
%(\ref{LM:Ass})
%(\ref{LM:Ass})
Similarly, from (\ref{LM:Ass70}), assumption of case 1.2, (\ref{LM:Ass7}), (\ref{LM:Ass50}) and (\ref{LMN:Ass1}) we have:
%
\begin{align}
	\begin{array}{l}
		\extendsAMUpto{\amod{}, \amod{L}}{(n-1)}{l_2 \composeL k \composeL h}{i \composeL e}{\amod{2}} \;\land\\
		t = \unitL \implies (l_1 \composeL d \composeL e, l_2 \composeL d \composeL e) \in \amod{}(\ca{}) \;\lor\; l_2 \composeL d \composeL e \hspace*{0.2cm}\text{ is undefined}
	\end{array} \label{LM:Ass72}
\end{align}
%(\ref{LM:Ass})
%(\ref{LM:Ass})
From (\ref{LM:Ass71}) and (\ref{LM:Ass72}) we have:
%
\begin{align}
	\begin{array}{l}
		\extendsAMUpto{\amod{}, \amod{L}}{(n-1)}{l_2 \composeL i \composeL k}{h \composeL e}{\amod{1}} \;\land\;
		\extendsAMUpto{\amod{}, \amod{L}}{(n-1)}{l_2 \composeL k \composeL h}{i \composeL e}{\amod{2}}\\
		\land\; t = \unitL \implies (l_1 \composeL d \composeL e, l_2 \composeL d \composeL e) \in \amod{}(\ca{}) \;\lor\; l_2 \composeL d \composeL e \hspace*{0.2cm}\text{ is undefined}
	\end{array} \label{LM:Ass73}
\end{align}
%(\ref{LM:Ass})
From (\ref{LM:Ass73}) and (\ref{LM:I.H}) we can rewrite (\ref{LM:Ass73}) as:
%
\begin{align}
	\begin{array}{l}
		\extendsAMUpto{\amod{}, \amod{L}}{(n-1)}{l_2 \composeL i \composeL k \composeL h}{e}{\amod{1} \cup \amod{2}} \;\land\\
		t = \unitL \implies (l_1 \composeL d \composeL e, l_2 \composeL d \composeL e) \in \amod{}(\ca{}) \;\lor\; l_2 \composeL d \composeL e \hspace*{0.2cm}\text{ is undefined}
	\end{array} \label{LM:Ass77}
\end{align}
%(\ref{LM:Ass})
From (\ref{LM:Ass70}) and (\ref{LMN:Ass1}) we can rewrite (\ref{LM:Ass77}) as:
%
\begin{align}
	\begin{array}{l}
		\extendsAMUpto{\amod{}, \amod{L}}{(n-1)}{l_2 \composeL d}{e}{\amod{1} \cup \amod{2}} \;\land\\
		t = \unitL \implies (l_1 \composeL d \composeL e, l_2 \composeL d \composeL e) \in \amod{}(\ca{}) \;\lor\; l_2 \composeL d \composeL e \hspace*{0.2cm}\text{ is undefined}
	\end{array}
	 \label{LM:Case3.1.2}
\end{align}
%(\ref{LM:Ass})
as required.\\
%Proof of this case is analogous to that of steps (\ref{LM:Ass21})-(\ref{LM:Case1.2}) and is omitted here.
%
%
%
%
%

\noindent\textbf{Case 2.} $(l_1 \composeL f, l_2 \composeL f) \in \amod{2}(\ca{}) \;\land\; (l_1 \composeL f, l_2 \composeL f) \not\in \amod{1}(\ca{})$\\
The proof of this case is analogous o that of previous case and is omitted here.\\

\noindent\textbf{Case 3.} $(l_1 \composeL f, l_1 \composeL f) \in \amod{2}(\ca{}) \;\land\; (l_1 \composeL f, l_2 \composeL f) \in \amod{2}(\ca{})$\\
This case follows trivially from case 1 and case 2. \\


\end{proof}
\end{lemma}
%
%