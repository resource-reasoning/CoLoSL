%
%
\begin{lemma}[]\label{lem:amodMerge}
%
\[
\begin{array}{l}
	\for{l, r \in \LState} \for{\amod{}, \amod{1}, \amod{2} \in \AMods} \for{n \in \Nats}\\
%	\hspace*{0.5cm} p \composeL c \in F_1 \;\land\; F_1 \fences \amod{1} \;\land\; c \composeL q \in F_2 \;\land\; F_2 \fences \amod{2} \;\land\; (F_1, \amod{1}) \agrees (F_2, \amod{2})\\
	\hspace*{0.5cm} \land\; \extendsAMUpto{\amod{}, L}{n}{p \composeL c}{q \composeL r}{\amod{1}} \;\land\; \extendsAMUpto{\amod{}, L}{n}{c \composeL q}{p \composeL r}{\amod{2}} \\
	\hspace*{0.5cm}\land\;\left( \for{s} s \leq p \land s \leq q \implies s = \unitL \right)
	\implies\\
	\hspace*{2cm} \extendsAMUpto{\amod{}, L}{n}{p \composeL c \composeL q}{r}{\amod{1} \cup \amod{2}}
\end{array}
\]
%
\begin{proof} By induction on number of steps $n$.\\
\noindent Pick an arbitrary $l, r \in \LState, \amod{}, \amod{1}, \amod{2} \in \AMods$.\\
\noindent\textbf{Base case}\\
\textbf{RTS. }\hspace*{0.5cm}$\extendsAMUpto{\amod{}, L}{0}{p \composeL c \composeL q}{r}{\amod{1} \cup \amod{2}}$\\
This holds trivially by definition of $\extendsAMUpto{\amod{}, L}{0}{p \composeL c \composeL q}{r}{\amod{1} \cup \amod{2}}$\\

\noindent
\textbf{}
\textbf{Inductive Step} Pick an arbitrary $n \in \Nats$, then
%
\begin{equation}
	\tag{I.H.}
	\begin{array}{l}
		\for{l, r \in \LState} \for{\amod{}, \amod{1}, \amod{2} \in \AMods} \for{n \in \Nats}\\
		\hspace*{0.5cm} p \composeL c \in F_1 \;\land\; F_1 \fences \amod{1} \;\land\; c \composeL q \in F_2 \;\land\; F_2 \fences \amod{2} \;\land\; (F_1, \amod{1}) \agrees (F_2, \amod{2})\\
		\hspace*{0.5cm} \land\; \extendsAMUpto{\amod{}, L}{(n-1)}{p \composeL c}{q \composeL r}{\amod{1}} \;\land\; \extendsAMUpto{\amod{}, L}{(n-1)}{c \composeL q}{p \composeL r}{\amod{2}}\\
		\hspace*{0.5cm}\land\;\left( \for{s} s \leq p \land s \leq q \implies s = \unitL \right)
		\implies\\
		\hspace*{2cm} \extendsAMUpto{\amod{}, L}{(n-1)}{p \composeL c \composeL q}{r}{\amod{1} \cup \amod{2}}
	\end{array}
\label{LM:I.H}
\end{equation}
%
Assume:
%
\begin{align}
	& p \composeL c \in F_1 \label{LM:Ass1}\\
	& F_1 \fences \amod{1} \label{LM:Ass2}\\
	& c \composeL q \in F_2 \label{LM:Ass3}\\
	& F_2 \fences \amod{2} \label{LM:Ass4}\\
	& (F_1, \amod{1}) \agrees (F_2, \amod{2}) \label{LM:Ass5}\\
	& \extendsAMUpto{\amod{}}{n}{p \composeL c}{q \composeL r}{\amod{1}} \label{LM:Ass6}\\
	& \extendsAMUpto{\amod{}}{n}{c \composeL q}{p \composeL r}{\amod{2}} \label{LM:Ass7}\\
	& \for{s} s \leq p \;\land\; s \leq q \implies s = \unitL \label{LM:Ass0}
\end{align}
%
Show
%
\begin{align}
	&\for{\ca{}, d, e, l_3, l_4}\for{(l_1 \composeL f, l_2 \composeL f) \in (\amod{1} \cup \amod{2})(\ca{})} \nonumber\\
	& \left(\for{l'} l' \leq l_1 \land l' \leq l_2 \implies l' = \unitL\right) \;\land\; l_1 \composeL f \leq  p \composeL c \composeL q \composeL r  \implies\nonumber\\
	& \hspace*{0.2cm} l_1 = l_3 \composeL l_4 \;\land\; l_1 \maxMeetL p \composeL c \composeL q = l_3 \;\land\; p \composeL c \composeL q = l_3 \composeL d \;\land\; r = l_4 \composeL e \implies \nonumber\\
	&\hspace*{1cm} \left((l_1 \composeL d \composeL e, l_2 \composeL d \composeL e) \in \amod{}(\ca{}) \land
	\extendsAMUpto{\amod{}, L}{(n-1)}{l_2 \composeL d}{e}{\amod{}'}\right) \lor l_2 \composeL d \composeL e \text{ is undefined} \label{LM:Goal1}\\
%
	&\begin{array}{l}
		\for{\amod{L} \in L} \for{\ca{}} \for{(l_1 \composeL f, l_2 \composeL f) \in \amod{L}(\ca{})} \\
  \hspace*{0.2cm} l_1 \composeL f \meetL p \composeL c \composeL q \composeL r \not= \emptyset \;\land\; \left(\for{l'} l' \leq l_0 \land l' \leq l'_0 \implies l' = \unitL\right)  \;\implies\\
  \hspace*{0.4cm}\exsts{f'} (l_1 \composeL f', l_2 \composeL f') \in \left(\amod{1} \cup \amod{2}\right)(\ca{}) \;\land\; l_1 \composeL f' \meetL p \composeL c \composeL q \composeL r \not= \emptyset\\
		\hspace*{0.4cm}\lor\;\for{d, e}  l_1 \composeL f \composeL e = p \composeL c \composeL q \composeL r \composeL d \implies\\
		\hspace*{1cm}
		\begin{array}{l l}
			l_2 \composeL f \composeL e \text{ is undefined}\\
			\lor\; \exsts{r'} l_2 \composeL f \composeL e = p \composeL c \composeL q \composeL r' \composeL d \;\land\\ \left( d = \unitL \implies \extendsAMUpto{\amod{G}, L}{(m-1)}{p \composeL c \composeL q}{r'}{\amod{}'}\right)
		\end{array}
	\end{array}\label{LM:Goal2}\\\nonumber
\end{align}
%

\noindent\textbf{RTS. (\ref{LM:Goal2})}\\
Pick an arbitrary $\amod{L} \in L, \ca{}, l_1, l_2, f, d$ such that:
%
\begin{align}
	\begin{array}{l}
		(l_1 \composeL f, l_2 \composeL f) \in \amod{L}(\ca{})\\
		l_1 \composeL f \meetL p \composeL c \composeL q \composeL r \not= \emptyset\\
  	(l_1 \composeL f, l_2 \composeL f) \in \amod{}''(\ca{}) \\
  	\left(\for{l'} l' \leq l_1 \land l' \leq l_2 \implies l' = \unitL\right)\\
	\end{array} \label{LAM: Ass1}
\end{align} 
%
%Pick an arbitrary $\ca{}, (l_1, l'_1) \in \amod{\ca{}}$ such that 
%%
%\begin{equation}
%	l_1 = p \composeL c \composeL q \composeL r \label{LM:Ass28}
%\end{equation}
%%(\ref{LM:Ass})
Then from (\ref{LM:Ass6}) we have:
%
\begin{align}
	\begin{array}{l}
		\exsts{f'} (l_1 \composeL f', l_2 \composeL f') \in \amod{1}(\ca{}) \;\land\; l_1 \composeL f' \meetL p \composeL c \composeL q \composeL r \not= \emptyset\\
		\lor\;\for{d, e}  l_1 \composeL f \composeL e = p \composeL c \composeL q \composeL r \composeL d \implies\\
		\hspace*{0.5cm}
		\begin{array}{l l}
			l_2 \composeL f \composeL e \text{ is undefined}\\
			\lor\; \exsts{r_1} l_2 \composeL f \composeL e = p \composeL c \composeL r_1 \composeL d \;\land\;\left( d = \unitL \implies  \extendsAMUpto{\amod{G}, L}{(m-1)}{p \composeL c}{r_1}{\amod{1}}\right)
		\end{array}
	\end{array} \label{LM:Ass29}
\end{align}
%
Similarly from (\ref{LM:Ass7}) we have:
%
\begin{align}
	\begin{array}{l}
	\exsts{f'} (l_1 \composeL f', l_2 \composeL f') \in \amod{2}(\ca{}) \;\land\; l_1 \composeL f' \meetL p \composeL c \composeL q \composeL r \not= \emptyset\\
	\lor\;\for{d, e}  l_1 \composeL f \composeL e = p \composeL c \composeL q \composeL r \composeL d \implies\\
		\hspace*{0.5cm}
		\begin{array}{l l}
			l_2 \composeL f \composeL e \text{ is undefined}\\
			\lor\; \exsts{r_2} l_2 \composeL f \composeL e = c \composeL q \composeL r_2 \composeL d \;\land\; \left( d = \unitL \implies \extendsAMUpto{\amod{G}, L}{(m-1)}{c \composeL q}{r_2}{\amod{2}}\right)
		\end{array}
	\end{array} \label{LM:Ass30}
\end{align}
%(\ref{LM:Ass})
From (\ref{LM:Ass29}) and (\ref{LM:Ass30}) we have:
%
\begin{align}
	\begin{array}{l}
		\exsts{f'} (l_1 \composeL f', l_2 \composeL f') \in \left(\amod{1} \cup \amod{2}\right)(\ca{}) \;\land\; l_1 \composeL f' \meetL p \composeL c \composeL q \composeL r \not= \emptyset\\
		\lor\;\for{d, e}  l_1 \composeL f \composeL e = p \composeL c \composeL q \composeL r \composeL d \implies\\
		\hspace*{0.5cm}
		\begin{array}{l l}
			l_2 \composeL f \composeL e \text{ is undefined} \lor\;\\
			\exsts{r_1} l_2 \composeL f \composeL e = p \composeL c \composeL r_1 \composeL d \;\land\; \left( d = \unitL \implies \extendsAMUpto{\amod{G}, L}{(m-1)}{p \composeL c}{r_1}{\amod{1}} \right)\;\land\\
			\exsts{r_2} l_2 \composeL f \composeL e = c \composeL q \composeL r_2 \composeL d \;\land\; \left( d = \unitL \implies \extendsAMUpto{\amod{G}, L}{(m-1)}{c \composeL q}{r_2}{\amod{2}}\right)
		\end{array}
	\end{array}
	\label{LM:Ass31}
\end{align}
%(\ref{LM:Ass})
From (\ref{LM:Ass31}) we have $p \composeL c \composeL r_1 \composeL d = q \composeL c \composeL r_2 \composeL d$ and consequently $p \composeL r_1 = q \composeL r_2$. By the cross split property we then have:
%
\begin{align}
	\exsts{pq, pr_2, r_1q, r_1r_2} \hspace*{0.2cm}& pq  \composeL pr_2 = p \label{LM:Ass32}\\
	& r_1q \composeL r_1r_2 = r_1 \label{LM:Ass33}\\
	& pq \composeL r_1q = q \label{LM:Ass34}\\
	& pr_2 \composeL r_1r_2 = r_2 \label{LM:Ass35}
\end{align}
%(\ref{LM:Ass})
Since from (\ref{LM:Ass32}), (\ref{LM:Ass33}) we have $pq \leq p \;\land\; pq \leq q$, from (\ref{LM:Ass0}) we can deduce $pq = \unitL$. Consequently, we can rewrite (\ref{LM:Ass31}) as:
%
\begin{align}
	\begin{array}{l}
		\exsts{f'} (l_1 \composeL f', l_2 \composeL f') \in \left(\amod{1} \cup \amod{2}\right)(\ca{}) \;\land\; l_1 \composeL f' \meetL p \composeL c \composeL q \composeL r \not= \emptyset\\
		\lor\;\for{d, e}  l_1 \composeL f \composeL e = p \composeL c \composeL q \composeL r \composeL d \implies\\
		\hspace*{0.5cm}
		\begin{array}{l l}
			l_2 \composeL f \composeL e \text{ is undefined} \lor\;\\
			\exsts{r_1r_2} l_2 \composeL f \composeL e = p \composeL c \composeL q  \composeL r_1r_2 \composeL d \;\land\\ 			\left( d = \unitL \implies \extendsAMUpto{\amod{G}, L}{(m-1)}{p \composeL c}{q \composeL r_1r_2}{\amod{1}} \land\; \extendsAMUpto{\amod{G}, L}{(m-1)}{c \composeL q}{p \composeL r_1r_2}{\amod{2}}\right)
		\end{array}
	\end{array} \label{LM:Ass36}
\end{align}
%(\ref{LM:Ass})
From (\ref{LM:Ass1})-(\ref{LM:Ass5}), (\ref{LM:Ass36}) and (\ref{LM:I.H}) we can rewrite (\ref{LM:Ass36}) as:
%
\begin{align}
	\begin{array}{l}
		\exsts{f'} (l_1 \composeL f', l_2 \composeL f') \in \left(\amod{1} \cup \amod{2}\right)(\ca{}) \;\land\; l_1 \composeL f' \meetL p \composeL c \composeL q \composeL r \not= \emptyset\\
		\lor\;\for{d, e}  l_1 \composeL f \composeL e = p \composeL c \composeL q \composeL r \composeL d \implies\\
		\hspace*{0.5cm}
		\begin{array}{l l}
			l_2 \composeL f \composeL e \text{ is undefined} \lor\;\\
			\exsts{r'} l_2 \composeL f \composeL e = p \composeL c \composeL q  \composeL r' \composeL d \;\land\; \left( d = \unitL \implies \extendsAMUpto{\amod{G}, L}{(m-1)}{p \composeL c \composeL q}{r'}{\amod{1} \cup \amod{2}} \right)
		\end{array}
	\end{array}  \label{LM:Ass37}
\end{align}
%(\ref{LM:Ass})
%
%
%
%
%

\noindent\textbf{RTS. (\ref{LM:Goal1})}\\
Pick an arbitrary $\ca{}, d, e, l_4, l_3, (l_1 \composeL f, l_2 \composeL f) \in (\amod{1} \cup \amod{2})(\ca{})$ such that
\begin{align}
	& \left(\for{l'} l' \leq l_1 \land l' \leq l_2 \implies l' = \unitL\right)\label{LM:Ass47}\\
	& l_1 \maxMeetL  p \composeL c \composeL q = l_3 \;\land\; l_1 = l_3 \composeL l_4 \label{LM:Ass48}\\
	& p \composeL c \composeL q = l_3 \composeL d \label{LM:Ass49}\\
	& r = l_4 \composeL e \label{LM:Ass50}\\
	& l_1 \composeL f \leq p \composeL c \composeL q \composeL r \label{LM:Ass51}
%	& (l_1 \composeL d \composeL e, l_2 \composeL d \composeL e) \in \amod{}(\ca{}) \label{LM:Ass52}
\end{align}
%(\ref{LM:Ass})
%
Then by definition of $\amod{1} \cup \amod{2}$ we know:
%
\begin{align}
	(l_1 \composeL f, l_2 \composeL f) \in \amod{1}(\ca{}) \;\lor\; (l_1 \composeL f, l_2 \composeL f) \in \amod{2}(\ca{}) \label{LM:Ass61}
\end{align}
%
\textbf{Case 1.} $(l_1 \composeL f, l_2 \composeL f) \in \amod{1}(\ca{}) \;\land\; (l_1 \composeL f, l_2 \composeL f) \not\in \amod{2}(\ca{})$\\
%
Then from (\ref{LM:Ass6}) we know
%
\begin{align}
\begin{array}{l l}
	\exsts{s_1, s_2, g, h} & l_1 = s_1 \composeL s_2\\
	& s_1 = l_1 \maxMeetL p \composeL c\\
	& p \composeL c = s_1 \composeL g\\
	& q\composeL r = s_2 \composeL h\\
	& (l_1 \composeL g \composeL h, l_2 \composeL g \composeL h) \in \amod{}(\ca{}) \;\land\; \extendsAMUpto{\amod{}}{(n-1)}{l_2 \composeL g}{h}{\amod{1}}
\end{array}	\label{LMN:Ass1}
\end{align}
%
From (\ref{LM:Ass7})  and (\ref{LMN:Ass1}) we have:
%
\begin{align}
	& \begin{array}{l l}
		\exsts{f', r'}& (l_1 \composeL f', l_2 \composeL f') \in \amod{2}(\ca{}) \;\land\; l_1 \composeL g \composeL h = l_1 \composeL f' \composeL r' \;\land\; l_2 \composeL g \composeL h = l_2 \composeL f' \composeL r'\\
%		&\land\;  l_1 \leq (c \composeL q \maxMeetL p \composeL c) 
	\end{array}\label{LM:Ass62}\\
	& \hspace*{3cm}\lor\; \nonumber\\
	&\begin{array}{l l}
		\exsts{r'} l_2 \composeL g \composeL h = q \composeL c \composeL r' \;\land\; \extendsAMUpto{\amod{}}{(n-1)}{q\composeL c}{r'}{\amod{2}}
	\end{array}  \label{LM:Ass63}
\end{align}
%
There are two cases to consider:\\
\textbf{Case 1.1.} 
\[
\begin{array}{l}
		\exsts{r'} l_2 \composeL g \composeL h = q \composeL c \composeL r' \;\land\; \extendsAMUpto{\amod{}}{(n-1)}{q\composeL c}{r'}{\amod{2}}
\end{array}
\]
%
Since $l_1 \leq p \composeL c \composeL q \composeL r$ (\ref{LM:Ass63}), from Lemma \ref{lem:divideUpper} we have:
%
\begin{align}
	\begin{array}{l l}
	\exsts{s_1, s_2, a, b}& l_3 = s_1 \composeL s_2 \\
	& s_1 \composeL a =  p\\
	& s_2 \composeL b = c \composeL q\\
	&a \composeL  b = d 
	\end{array} \label{LM:Ass64}
\end{align}
%
Since $p \composeL c \composeL q  = s_1 \composeL a \composeL s_2 \composeL b $ and consequently $l_1 \composeL a \composeL b$ is defined, by definition of $\meetL$ we have:
%
\begin{align}
	s_2 \in (l_1 \meetL c \composeL q) \nonumber
\end{align} 
%
and thus from the assumption of case 1.1. we have $s_2 \leq l_2$. Since $s_2 \leq l_1$ (\ref{LM:Ass64}, \ref{LM:Ass48}), from (\ref{LM:Ass47}) we have $s_2 = \unitL$; thus from (\ref{LM:Ass64}):
%
\begin{align}
	\begin{array}{l}
	l_1 = s_1 \composeL l_4\\
	b = c \composeL q
	\end{array} \label{LM:Ass644}
\end{align} 
% 
and consequently from (\ref{LM:Ass64}) we have:
%
\begin{align}
	l_3 \composeL a = p \nonumber\\
	l_1 \composeL a \composeL e = p \composeL r \label{LM:Ass65}
\end{align}
%
From assumption of case 1, (\ref{LM:Ass6}), (\ref{LM:Ass47})-(\ref{LM:Ass51}) and (\ref{LM:Ass64}), (\ref{LM:Ass65}) we have:
%
\begin{align}
	\begin{array}{l l}
%	\for{l_6, l_7} & l_2 = l_6 \composeL l_7\;\implies\\
	& (l_1 \composeL d \composeL e, l_2 \composeL d \composeL e) \in \amod{}(\ca{}) \;\land\; \extendsAMUpto{\amod{}}{(n-1)}{l_2 \composeL a \composeL c}{q \composeL e}{\amod{1}}\\
	& \;\;\lor (l_2 \composeL d \composeL e)\hspace*{0.2cm}\text{is undefined}
	\end{array} \label{LM:Ass66}
\end{align}
%
On the other hand, from (\ref{LM:Ass63}), (\ref{LM:Ass64}), (\ref{LM:Ass66}) and (\ref{LM:Ass7}) we can deduce:
%
\begin{align}
	\begin{array}{l l}
		&(l_1 \composeL d \composeL e, l_2 \composeL d \composeL e) \in \amod{}(\ca{}) \;\land\;
		\extendsAMUpto{\amod{}}{(n-1)}{l_2 \composeL a \composeL c}{q \composeL e}{\amod{1}}\\
		& \hspace*{1cm}\land\; \extendsAMUpto{\amod{}}{(n-1)}{q \composeL c}{l_2 \composeL a \composeL e}{\amod{2}}\\
		& \;\lor (l_2 \composeL d \composeL e)\hspace*{0.2cm}\text{is undefined}
	\end{array} \label{LM:Ass67}
\end{align}
%(\ref{LM:Ass})
From (\ref{LM:Ass1}), (\ref{LM:Ass2}), assumption of case 1, (\ref{LM:Ass47})-(\ref{LM:Ass51}), (\ref{LM:Ass65}), (\ref{LM:Ass66})  and definition of $\fences$ we have:
%
\begin{equation}
	l_2 \composeL a \composeL c \in F_1 \label{LM:Ass68}
\end{equation}
%
Assume 
%
\begin{align}
	\begin{array}{l l}
		\exsts{t > \unitL} & t \leq l_2 \composeL a \;\land\; t \leq q
	\end{array} \label{LM:Con1}
\end{align}
%
Now also assume 
\begin{equation}
	t \leq l_1 \label{LM:Con2}
\end{equation}
%
From (\ref{LM:Con2}), (\ref{LM:Ass48}) and Lemma \ref{lem:divideUpper} we have:
%
\begin{align}
	\begin{array}{l l}
		\exsts{t_1, t_2} & t = t_1 \composeL t_2\\
		& t_1 \leq l_3\\
		& t_2 \leq l_4
	\end{array} \label{LM:Con3}
\end{align}
%
From (\ref{LM:Con3}) and (\ref{LM:Ass48}) and by definition of $\maxMeetL$, I can deduce: 
%
\begin{equation}
	t_2 = \unitL \label{LM:Con4}
\end{equation}
%
On the other hand from (\ref{LM:Con1}), (\ref{LM:Con3}), (\ref{LM:Ass65}) and (\ref{LM:Ass0}) I have: 
%
\begin{equation}
	t_1 = \unitL \label{LM:Con5}
\end{equation}
% 
From (\ref{LM:Con3}), (\ref{LM:Con4}) and (\ref{LM:Con5}) we have $t = \unitL$ which contradicts our assumption of (\ref{LM:Con1}). Therefore, we can deduce that our assumption of (\ref{LM:Con2}) was wrong and that:
%
\begin{equation}
	t \not\leq l_1 \label{LM:Con6}
\end{equation}
%
Consequently from (\ref{LM:Con1}) and (\ref{LM:Con6}) we can conclude
%
\begin{equation}
	(c \composeL q) \meetL l_1 \not= (c \composeL q) \meetL l_2 \nonumber
\end{equation}
%
This is however a contradiction to the assumption of case 1.1 and therefore, we can conclude that our assumption of (\ref{LM:Con1}) was wrong and that we have:
%
\begin{equation}
	\for{s} t\leq l_2 \composeL a \land t \leq  q \implies t = \unitL \label{LM:Ass60}
\end{equation}
%
From (\ref{LM:Ass68}), (\ref{LM:Ass2}), (\ref{LM:Ass3}), (\ref{LM:Ass4}), (\ref{LM:Ass5}), (\ref{LM:Ass67}), (\ref{LM:Ass60}) and (\ref{LM:I.H}), we can rewrite (\ref{LM:Ass66}) and (\ref{LM:Ass67}) as:
%
\begin{align}
	\begin{array}{l l}
		& (l_1 \composeL d \composeL e, l_2 \composeL d \composeL e) \in \amod{}(\ca{}) \;\land\;
		\extendsAMUpto{\amod{}}{(n-1)}{l_2 \composeL a \composeL c \composeL q}{e}{\amod{1} \cup \amod{2}}\\
		& \;\lor (l_2 \composeL d \composeL e)\hspace*{0.2cm}\text{is undefined}
	\end{array} \label{LM:Ass69}
\end{align}
%(\ref{LM:Ass})
From (\ref{LM:Ass64}) and (\ref{LM:Ass644}) we have: 
%
\begin{align}
	\begin{array}{l l}
		& (l_1 \composeL d \composeL e, l_2 \composeL d \composeL e) \in \amod{}(\ca{}) \;\land\;
		\extendsAMUpto{\amod{}}{(n-1)}{l_2 \composeL d}{e}{\amod{1} \cup \amod{2}}\\
		& \;\lor (l_2 \composeL d \composeL e)\hspace*{0.2cm}\text{is undefined}
	\end{array} \nonumber
\end{align}
%
%
%

\noindent\textbf{Case 1.2.}
\[
\begin{array}{l l}
	\exsts{f'}& (l_1 \composeL f', l_2 \composeL f') \in \amod{2}(\ca{}) \;\land\; l_1 \composeL f' \leq p \composeL c \composeL q \composeL r \\
%	&\land\; l_1 \leq (c \composeL q \maxMeetL p \composeL c)
\end{array}
\]
%
%
Since $l_3 \leq p \composeL c \composeL q$ (\ref{LM:Ass49}), from Lemma \ref{lem:divideUpper} we have:
%
\begin{align}
	\begin{array}{l l}
		\exsts{l_5, l_6, l_7, e, g, h} \hspace*{0.2cm} & l_3 = l_5 \composeL l_6 \composeL l_7\\
		& p = l_5 \composeL i\\
		& c = l_6 \composeL g\\
		& q = l_7 \composeL h\\
		& d = i \composeL g \composeL h
	\end{array} \label{LM:Ass70}
\end{align} 
%
From (\ref{LM:Ass70}), assumption of case 1, (\ref{LM:Ass6}) and (\ref{LM:Ass50}) we have:
%
%
\begin{align}
	\begin{array}{l}
		(l_1 \composeL d \composeL e, l_2 \composeL d \composeL e) \in \amod{}(\ca{}) \;\land\;
		\extendsAMUpto{\amod{}}{(n-1)}{l_2 \composeL i \composeL g}{h \composeL e}{\amod{1}}\\
		\lor\; l_2 \composeL d \composeL e \hspace*{0.2cm}\text{ is undefined}
	\end{array} \label{LM:Ass71}
\end{align}
%(\ref{LM:Ass})
%(\ref{LM:Ass})
Similarly, from (\ref{LM:Ass70}), assumption of case 1.2, (\ref{LM:Ass7}) and (\ref{LM:Ass50}) we have:
%
\begin{align}
	\begin{array}{l}
		(l_1 \composeL d \composeL e, l_2 \composeL d \composeL e) \in \amod{}(\ca{}) \;\land\;
		\extendsAMUpto{\amod{}}{(n-1)}{l_2 \composeL g \composeL h}{i \composeL e}{\amod{2}}\\
		\lor\; l_2 \composeL d \composeL e \hspace*{0.2cm}\text{ is undefined}
	\end{array} \label{LM:Ass72}
\end{align}
%(\ref{LM:Ass})
%(\ref{LM:Ass})
From (\ref{LM:Ass71}) and (\ref{LM:Ass72}) we have:
%
\begin{align}
	\begin{array}{l}
		(l_1 \composeL d \composeL e, l_2 \composeL d \composeL e) \in \amod{}(\ca{}) \;\land\;\\
		\hspace*{0.6cm}\extendsAMUpto{\amod{}}{(n-1)}{l_2 \composeL i \composeL g}{h \composeL e}{\amod{1}} \;\land\;
		\extendsAMUpto{\amod{}}{(n-1)}{l_2 \composeL g \composeL h}{i \composeL e}{\amod{2}}\\
		\lor\; l_2 \composeL d \composeL e \hspace*{0.2cm}\text{ is undefined}
	\end{array} \label{LM:Ass73}
\end{align}
%(\ref{LM:Ass})
%
From (\ref{LM:Ass1}), (\ref{LM:Ass2}), (\ref{LM:Ass70}), (\ref{LM:Ass47}), (\ref{LM:Ass50}), assumption of case 1 and definition of $\fences$ we have:
%
\begin{equation}
	l_2 \composeL i \composeL g \in F_1 \label{LM:Ass74}
\end{equation}
%
Similarly, from (\ref{LM:Ass3}), (\ref{LM:Ass4}), (\ref{LM:Ass70}), (\ref{LM:Ass47}), (\ref{LM:Ass50}), assumption of case 1.2, and definition of $\fences$ we have:
%
\begin{equation}
	l_2 \composeL g \composeL h \in F_2 \label{LM:Ass75}
\end{equation}
%(\ref{LM:Ass})
From (\ref{LM:Ass0}) and (\ref{LM:Ass70}) we have:
\begin{align}
\begin{array}{l}
	\for{l'} l' \leq i \;\land\; l' \leq h \implies l' = \unitL
\end{array} \label{LM:Ass76}
\end{align}
From (\ref{LM:Ass74}), (\ref{LM:Ass2}), (\ref{LM:Ass75}), (\ref{LM:Ass4}), (\ref{LM:Ass5}), (\ref{LM:Ass0}),   (\ref{LM:Ass73}), (\ref{LM:Ass76}) and (\ref{LM:I.H}) we can rewrite (\ref{LM:Ass73}) as:
%
\begin{align}
	\begin{array}{l}
		(l_1 \composeL d \composeL e, l_2 \composeL d \composeL e) \in \amod{}(\ca{}) \;\land\;\extendsAMUpto{\amod{}}{(n-1)}{l_2 \composeL i \composeL g \composeL h}{e}{\amod{1} \cup \amod{2}} \\
		\lor\; l_2 \composeL d \composeL e \hspace*{0.2cm}\text{ is undefined}
	\end{array} \label{LM:Ass77}
\end{align}
%(\ref{LM:Ass})
From (\ref{LM:Ass70}) we can rewrite (\ref{LM:Ass77}) as:
%
\begin{align}
	\begin{array}{l}
		(l_1 \composeL d \composeL e, l_2 \composeL d \composeL e) \in \amod{}(\ca{}) \;\land\;\extendsAMUpto{\amod{}}{(n-1)}{l_2 \composeL d}{e}{\amod{1} \cup \amod{2}} \\
		\lor\; l_2 \composeL d \composeL e \hspace*{0.2cm}\text{ is undefined}
	\end{array}
	 \label{LM:Case3.1.2}
\end{align}
%(\ref{LM:Ass})

%Proof of this case is analogous to that of steps (\ref{LM:Ass21})-(\ref{LM:Case1.2}) and is omitted here.
%
%
%
%
%

\noindent\textbf{Case 2.} $(l_1 \composeL f, l_2 \composeL f) \in \amod{2}(\ca{}) \;\land\; (l_1 \composeL f, l_2 \composeL f) \not\in \amod{1}(\ca{})$\\
The proof of this case is analogous o that of previous case and is omitted here.\\

\noindent\textbf{Case 3.} $(l_1 \composeL f, l_1 \composeL f) \in \amod{2}(\ca{}) \;\land\; (l_1 \composeL f, l_2 \composeL f) \in \amod{2}(\ca{})$\\
This case follows trivially from case 1 and case 2. \\


\end{proof}
\end{lemma}
%
%