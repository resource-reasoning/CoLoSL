\section*{Auxiliary Lemmata}
\begin{lemma}[Sequential Command Soundness]\label{lem:seqSoundness}
For all $\seq{1} \in \Seqs$, $\left(H_1, \seq{1}, H_2\right) \in \AxiomsSeq$ and $h \in \Heaps$:
%
\[
	\opSemSeq{\seq{1}}{\reifyH{H_1 \composeH \{h\}}} \subseteq \reifyH{H_2 \composeH \{h\}}
\]
%
\begin{proof}
By induction over the structure of $\seq{}$. Pick an arbitrary $h \in \Heaps$.\\

\noindent\textbf{Case \hspace*{0.3cm}}\bc{}\\
This follows from parameter \ref{par:basicSoundness}.\\


\noindent\textbf{Case \hspace*{0.3cm}\skipC}\\
\textbf{RTS.}
%
\[
	\opSemSeq{\seq{1}}{\reifyH{H \composeH \{h\}}} 
	\subseteq \reifyH{H \composeH \{h\}}
\]
%
\begin{proof}
%
\[
\begin{array}{r l}
	\opSemSeq{\seq{1}}{\reifyH{H \composeH \{h\}}} 
	= &
	\reifyH{H \composeH \{h\}}\\

	\subseteq & \reifyH{H \composeH \{h\}}
\end{array}
\]
%
as required.
\renewcommand{\qed}{}
\end{proof}
%
%

\noindent\textbf{Case \hspace*{0.3cm}}$\seq{1} ; \seq{2}$\\
\textbf{RTS.}
%
\[
	\opSemSeq{\seq{1}; \seq{2}}{\reifyH{H \composeH \{h\}}} 
	\subseteq \reifyH{H' \composeH \{h\}}
\]
%
where $\left(H, \seq{1}, H'' \right), \left(H'', \seq{2}, H' \right)  \in \AxiomsSeq$.
\begin{proof}
%
\[
\begin{array}{r l}
	
	\opSemSeq{\seq{1}; \seq{2}}{\reifyH{H \composeH \{h\}}} 
	= &  
	\opSemSeq{\seq{2}}{ \opSemSeq{\seq{1}}{\reifyH{H \composeH \{h\}}}}\\

	\text{(I.H.) \hspace*{0.5cm}}
	\subseteq &
	\opSemSeq{\seq{2}}{\reifyH{H'' \composeH \{h\}}}\\
	
	\text{(I.H.) \hspace*{0.5cm}}
	\subseteq &
	\reifyH{H' \composeH \{h\}}
	
\end{array}
\]
%
as required.
\renewcommand{\qed}{}
\end{proof}
%
%

\noindent\textbf{Case \hspace*{0.3cm}}$\seq{1} + \seq{2}$\\
\textbf{RTS.}
%
\[
	\opSemSeq{\seq{1} + \seq{2}}{\reifyH{H \composeH \{h\}}} 
	\subseteq \reifyH{H' \composeH \{h\}}
\]
%
where $\left(H, \seq{1}, H' \right), \left(H, \seq{2}, H' \right)  \in \AxiomsSeq$.
\begin{proof}
%
\[
\hspace*{-0.2cm}
\begin{array}{r l}
	
	\opSemSeq{\seq{1} + \seq{2}}{\reifyH{H \composeH \{h\}}} 
	= &  
	\opSemSeq{\seq{1}}{\reifyH{H \composeH \{h\}}} \;\cup\; \opSemSeq{\seq{2}}{\reifyH{H \composeH \{h\}}}\\

	\text{(I.H.) \hspace*{0.5cm}}
	\subseteq &
	\reifyH{H' \composeH \{h\}} \;\cup\; \reifyH{H' \composeH \{h\}}\\
	
	\subseteq &
	\reifyH{H' \composeH \{h\}}
	
\end{array}
\]
%
as required.
\renewcommand{\qed}{}
\end{proof}
%
%

\noindent\textbf{Case \hspace*{0.3cm}}$\seq{}^{*}$\\
\textbf{RTS.}
%
\[
	\opSemSeq{\seq{}^{*}}{\reifyH{H \composeH \{h\}}} 
	\subseteq \reifyH{H \composeH \{h\}}
\]
%
where $\left(H, \seq{}, H \right)  \in \AxiomsSeq$.
\begin{proof}
%
\[
\begin{array}{r l}
	
	\opSemSeq{\seq{}^{*}}{\reifyH{H \composeH \{h\}}} 
	= &  
	\opSemSeq{\skipC + \seq{}; \seq{}^{*}}{\reifyH{H \composeH \{h\}}} \\
	
	= & \opSemSeq{\skipC}{\reifyH{H \composeH \{h\}}} 
		\cup \opSemSeq{\seq{}; \seq{}^{*}}{\reifyH{H \composeH \{h\}}} \\

	\text{(I.H.) \hspace*{0.5cm}}
	\subseteq &
	\reifyH{H \composeH \{h\}} \cup \reifyH{H \composeH \{h\}}\\
	
	\subseteq &
	\reifyH{H \composeH \{h\}}
	
\end{array}
\]
%
as required.
\renewcommand{\qed}{}
\end{proof}
%
%
\end{proof}
\end{lemma}
%
%
\begin{lemma}[] \label{lem:updateGContainment}
%
\[
\begin{array}{l}
	\for{w_1, w_2, w \in \Worlds} w_1 \composeW w_2 = w \implies\\
	\hspace*{1cm}(l'_1, s', \amod{}') \in \updateG(w_1) \implies (\localPart{(w_2)}, s', \amod{}') \in \updateR(w_2)
\end{array}
\]
%
where given any relation $S \in (A \times A)$ we write
%
\[
\begin{array}{l}
	S(w) \eqdef \left\{w' \;|\; (w, w') \in S \right\}\\
\end{array}
\]
%
\begin{proof} Pick an arbitrary $(l_1, s_1, \amod{1}), (l_2, s_2, \amod{2}), w$ and $(l'_1, s', \amod{}')$ such that:
%
\begin{align}
	(l_1, s_1, \amod{1}) \composeW (l_2, s_2, \amod{2}) = w \label{L11:Ass1}\\
	(l'_1, s', \amod{}') \in \updateG(l_1, s_1, \amod{1}) \label{L11:Ass2}
\end{align}
%
\textbf{RTS.}
%
\begin{align}
	(\localPart{(w_2)}, s', \amod{}') \in \updateR(l_2, s_2, \amod{2}) \label{L11:Goal}
\end{align}
From (\ref{L11:Ass1}) we know:
%
\begin{align}
	s_1 = s_2 \label{L11:Ass3}\\
	\amod{1} = \amod{2} \label{L11:Ass4}
\end{align}
%
By definition of \updateG and from (\ref{L11:Ass2}) and (\ref{L11:Ass4}) we know:
%
\begin{align}
	& \amod{1} = \amod{2} = \amod{}' \label{L11:Ass5} \\
	& \capSize{\capPart{l_1 \composeL s_1)}} \subseteq \capSize{\heapPart{(l'_1 \composeL s')}} \label{L11:Perms} \\
	& s' = s_1 \lor 
	\left(\begin{array}{l}
		\exsts{\ca{} \leq \capPart{(l_1)}}  (s_1, s') \in \amod{1}(\ca{}) \;\land \\
		\heapSize{\heapPart{l_1 \composeL s_1)}} = \heapSize{\heapPart{(l'_1 \composeL s')}}
	\end{array} \right) \nonumber
\end{align}
%
There are two cases to consider:\\

\noindent\textbf{Case 1.} $s_1 = s'$\\
From (\ref{L11:Ass3}) and the assumption of the case we know $s' = s_2$. Consequently, from (\ref{L11:Ass5}) we have:

\begin{equation}
	(\localPart{(w_2)}, s', \amod{}') = (l_2, s_2, \amod{2}) \label{L11:Ass6}
\end{equation}
%
By definition of \updateR\ and from (\ref{L11:Ass6}) we can conclude:
%
\begin{equation}
	(\localPart{(w_2)}, s', \amod{}') \in \updateR(l_2, s_2, \amod{2}) \label{L11:Ass7}
\end{equation}
%
\textbf{Case 2.} 
%
\begin{align}
	\exsts{\ca{} \leq \capPart{(l_1)}} & (s_1, s') \in \amod{1}(\ca{}) \label{L11:Ass8} \;\land\\
	&\heapSize{\heapPart{l_1 \composeL s_1)}} = \heapSize{\heapPart{(l'_1 \composeL s')}} \label{L11:Ass9}
\end{align}
%
From (\ref{L11:Ass1}), (\ref{L11:Ass3}) and (\ref{L11:Ass4}) we know that 
%
\begin{equation}
	w = (l_1 \composeL l_2, s_2, \amod{2}) \label{L11:Ass10}
\end{equation}
%
Since $\wf{w}$ (by definition of \Worlds) and from (\ref{L11:Ass3}) we know:
\begin{align}
	&\capPart{(l_1 \composeL l_2 \composeL s_2)} = \capPart{(l_1)} \composeCap \capPart{(l_2)} \composeCap {(s_2)} = \capPart{(l_1 \composeL s_1)} \composeCap \capPart{(l_2)}\hspace*{0.2cm} \text{is defined} \label{L11:Ass11}\\
%	
	&\heapPart{(l_1 \composeL l_2 \composeL s_1)} = \heapPart{(l_1 \composeH s_1)} \composeH \heapPart{(l_2)}  \hspace*{0.2cm} \text{is defined} \label{L11:Ass12}
\end{align}
%
Since $\ca{1} \leq \capPart{(l_1)}$ (\ref{L11:Ass8}), from (\ref{L11:Ass11}) and Lemma \ref{lem:disjointByOrder}, we know:
%
\begin{equation}
	\ca{} \disjoint\ \ \capPart{(l_2)} \composeCap \capPart{(s_2)} \label{L11:Ass13}
\end{equation}
%
From (\ref{L11:Ass3}), (\ref{L11:Ass9}) and (\ref{L11:Ass12}) we know
%
\begin{equation}
	\heapPart{(l'_1 \composeL s')} \composeH \heapPart{(l_2)} = \heapPart{(l'_1 \composeL l_2 \composeL s')}\hspace*{0.2cm}\text{ is defined} \label{L11:Ass14}
\end{equation}
%
From (\ref{L11:Perms}) and (\ref{L11:Ass11}) we know
%
\begin{equation}
	\capPart{(l'_1 \composeL s')} \composeCap \capPart{(l_2)} = \capPart{(l'_1 \composeL l_2 \composeL s')}\hspace*{0.2cm}\text{ is defined} \label{L11:Ass15}
\end{equation}
%
From (\ref{L11:Ass14}) and (\ref{L11:Ass15}) we know $l'_1 \composeL l_2 \composeL s'$ is defined and consequently:
%
\begin{equation}
	l_2 \composeL s' \hspace*{0.2cm} \text{ is defined} \label{L11:Ass16}
\end{equation}
%
From (\ref{L11:Ass4}), (\ref{L11:Ass8}), (\ref{L11:Ass13}), (\ref{L11:Ass16}) and by definition of \updateR, we have:
%
\begin{equation}
	(\localPart{(w_2)}, s', \amod{}') = (l_2, s', \amod{}) \in \updateR(l_2, s_2, \amod{})  \label{L11:Ass17}
\end{equation}
%
From (\ref{L11:Ass7}) and (\ref{L11:Ass17}) we can dismiss (\ref{L11:Goal}).
\end{proof}
%
\end{lemma}
%
%
\begin{lemma}[]\label{lem:extendGContainment}
%
\[
\begin{array}{l}
	\for{w_1, w_2, w \in \Worlds} w_1 \composeW w_2 = w \implies\\
	\hspace*{1cm}(l'_1, s', \amod{}') \in \extendG(w_1) \implies (\localPart{(w_2)}, s', \amod{}') \in \extendR(w_2)
\end{array}
\]
%
\begin{proof} Pick an arbitrary $(l_1, s_1, \amod{1}), (l_2, s_2, \amod{2}), w$ and $(l'_1, s', \amod{}')$ such that:
%
\begin{align}
	(l_1, s_1, \amod{1}) \composeW (l_2, s_2, \amod{2}) = w \label{L12:Ass1}\\
	(l'_1, s', \amod{}') \in \extendG(l_1, s_1, \amod{1}) \label{L12:Ass2}
\end{align}
%
\textbf{RTS.}
%
\begin{align}
	(\localPart{(w_2)}, s', \amod{}') \in \extendR(l_2, s_2, \amod{2}) \label{L12:Goal}
\end{align}
From (\ref{L12:Ass1}) we know:
%
\begin{align}
	s_1 = s_2 \label{L12:Ass3}\\
	\amod{1} = \amod{2} \label{L12:Ass4}
\end{align}
%
By definition of \extendG and from (\ref{L12:Ass2}), (\ref{L12:Ass3}) and (\ref{L12:Ass4}) we know:
%
\begin{align}
	\exsts{l_3, l_4, \ca{1}, \ca{2}, s''} \hspace*{0.2cm}& l_1 = l_3 \composeL l_4 \;\land\; l'_1 = l_3 \composeL (\unitH, \ca{1}) \;\land \label{L12:Ass5}\\
	& s'' = l_4 \composeL (\unitH, \ca{2}) \;\land\; s' = s_2 \composeL s'' \label{L12:Ass6}\\
%
	\exsts{F, \amod{0}} \hspace*{0.2cm} & \for{\ca{} \in \left(\{\ca{1}, \ca{2}\} \cup \dom{\amod{0}}\right)} \for{\ca{}' \in \dom{\amod{}}}\;\; \ca{} \disjoint \ca{} \label{L12:Ass7}\\
%
	& s'' \in \fence{} \;\land\; \fence{} \strictfences \amod{0} \label{L12:fence}\\
%	
	&\for{l_7, r_7, \amod{7}} l_7 \composeL r_7 = s_2 \land \extendsAM{\amod{}}{l_7}{r_7}{\amod{7}} \implies \nonumber\\ 
	&\extendsAM{\amod{}'}{l_7}{r_7 \composeL s''}{\amod{7}}  \label{L12:Ass9}\\
%	& \extendsAM{\amod{}'}{s_2}{s''}{\amod{2}} \label{L12:Ass9}\\
%
	& \extendsAM{\amod{}'}{s''}{s_2}{\amod{0}}  \label{L12:Ass10}
\end{align}
%
%Since $\wf{l_1, s_1, \amod{1}}$, we know 
%\begin{equation}
%	\contains{\dom{\amod{1}}}{\capPart{(l_1 \composeL s_1)}} \nonumber
%\end{equation}
%%
%and consequently from (\ref{L12:Ass4}) and (\ref{L12:Ass5})
%\begin{equation}
%	\contains{\dom{\amod{2}}}{\capPart{(l_4)}} \label{L12:Ass11}
%\end{equation}
%%
%From (\ref{L12:Ass5}), (\ref{L12:Ass6}) and (\ref{L12:Ass11}), we have
%%
%\begin{equation}
%	\contains{K \cup \dom{\amod{2}}}{\capPart{s''}} \label{L12:Ass12}
%\end{equation}
%%
%From (\ref{L12:Ass5})-(\ref{L12:Ass10}) and (\ref{L12:Ass12}) we have:
From (\ref{L12:Ass6}) and (\ref{L12:Ass9}) we have:
%
\begin{equation}
	(\localPart{(w_2)}, s', \amod{}') = (l_2, s', \amod{}') \in \extendR(l_2, s_2, \amod{2}) \nonumber
\end{equation}
% 
%and consequently by definition of \rely, 
%%
%\begin{equation}
%	(\localPart{(w_2)}, s', \amod{}') \in \rely(l_2, s_2, \amod{2}) \nonumber
%\end{equation}
%% 
as required.
\end{proof}
%
%
\end{lemma}
%
%
\begin{lemma}[] \label{lem:guaranteeContainment}
%
\[
\begin{array}{l}
	\for{w_1, w_2, w \in \Worlds} w_1 \composeW w_2 = w \implies\\
	\hspace*{1cm}(l', s', \amod{}') \in \guarantee(w_1) \implies (\localPart{(w_2)}, s', \amod{}') \in \rely(w_2)
\end{array}
\]
%
\begin{proof} Pick an arbitrary $w_1, w_2, w$ and $(l'_1, s', \amod{}')$ such that:
%
\begin{align}
	w_1 \composeW w_2 = w \label{L13:Ass1}\\
	(l'_1, s', \amod{}') \in \guarantee(w_1) \label{L13:Ass2}
\end{align}
%
\textbf{RTS.}
%
\begin{align}
	(\localPart{(w_2)}, s', \amod{}') \in \rely(w_2) \label{L13:Goal}
\end{align}
From (\ref{L13:Ass2}) and by definition of \guarantee\ we know:
%
\begin{align}
	(l'_1, s', \amod{}') \in \left(\updateG \cup \extendG \right)^{*}(w_2) \label{L13:Ass3}
\end{align}
%
From (\ref{L13:Ass1}), (\ref{L13:Ass3}) and by Lemmata \ref{lem:updateGContainment} and \ref{lem:extendGContainment} we have:
%
\begin{align}
	(\localPart{(w_2)}, s', \amod{}') \in \left(\updateR \cup \extendR \right)^{*}(w_2) \nonumber
\end{align}
%
and consequently 
%
\begin{align}
	(\localPart{(w_2)}, s', \amod{}') \in \rely(w_2) \nonumber
\end{align}
%
as required.
\end{proof}
\end{lemma}

%
%
\begin{lemma}\label{lemma:contextSwitch}
%
\[
\begin{array}{l}
	\for{p, q, r \in \LState} \for{\amod{}, \amod{}' \in \AMods} \for{n \in \Nats}\\
	\hspace*{0.5cm} \extendsAMUpto{\amod}{n}{p \composeL q}{r}{\amod{}'} \implies 
									\extendsAMUpto{\amod}{n}{p}{q \composeL r}{\amod{}'}
\end{array}
\]
%
\begin{proof} By induction on number of steps $n$.

\noindent Pick an arbitrary $p, q, r \in \LState, \amod{}, \amod{}' \in \AMods$.\\

\noindent\textbf{Base case}\\
\textbf{RTS. }\hspace*{0.5cm}$\extendsAMUpto{\amod}{0}{p \composeL q}{r}{\amod{}'} \implies \extendsAMUpto{\amod}{0}{p}{q \composeL r}{\amod{}'}$\\
This holds trivially by definition of $\extendsAMUpto{\amod}{0}{p}{q \composeL r}{\amod{}'}$\\

\noindent\textbf{Inductive Step} Pick an arbitrary $n \in \Nats$, then
%
\begin{equation}
	\tag{I.H.}
	\extendsAMUpto{\amod}{(n-1)}{p \composeL q}{r}{\amod{}'} \implies 
	\extendsAMUpto{\amod}{(n-1)}{p}{q \composeL r}{\amod{}'}
\label{L1:IH}
\end{equation}
%
Assume:
%
\begin{align}
	& \extendsAMUpto{\amod{}}{n}{p \composeL q}{r}{\amod{}'} \label{L1:Ass1}
\end{align}
%
Show
%
\begin{align}
	&\for{\ca{}, c, d, l_3, l_4} \for{(l_1 \composeL f, l_2 \composeL f) \in \amod{}'(\ca{})} \nonumber\\
	&\hspace*{0.2cm}\left(\for{l'} l' \leq l_1 \land l' \leq l_2 \implies l' = \unitL\right)\;\land\; l_1 \composeL f \leq  p \composeL q \composeL r \implies \nonumber\\
	&\hspace*{0.4cm}  l_1 = l_3 \composeL l_4 \;\land\; l_1 \maxMeetL p \composeL q = l_3 \;\land\; p \composeL q = l_3 \composeL c \;\land\; r = l_4 \composeL d\implies \nonumber\\
	&\hspace*{0.7cm} \left((l_1 \composeL c \composeL d, l_2 \composeL c \composeL d) \in \amod{}(\ca{}) \land
	\extendsAMUpto{\amod}{(n-1)}{l_2 \composeL c}{d}{\amod{}'}\right) \lor l_2 \composeL c \composeL d \text{ is undefined} \label{L1:Goal1}\\ \nonumber\\
%
%	&\hspace*{0.4cm} \land\;  q \composeL r = l_1 \composeL d \;\land\;  (p \composeL l_1 \composeL d, p \composeL l_2 \composeL d) \in \amod{}(\ca{}) \implies \nonumber\\
%	&\hspace*{0.6cm} \extendsAMUpto{\amod}{(n-1)}{p}{l_2 \composeL d}{\amod{}'} \label{L1:Goal3}\\
%	
%	&\hspace*{0.4cm} \land\; l_1 \not\leq p \;\land\; l_1 = l_3 \composeL l_4 \;\land\; p = l_3 \composeL c \;\land\; q \composeL r = l_4 \composeL d  \implies \nonumber\\
%	&\hspace*{0.6cm} \for{l_5, l_6} l_2 = l_5 \composeL l_6  
%%	\;\land\;  (\for{l'} l' \leq l_5 \land l' \leq l_6 \implies l' = \unitL) 
%	\implies \nonumber\\
%	&\hspace*{1cm} (l_1 \composeL c \composeL d, l_2 \composeL c \composeL d) \in \amod{}(\ca{}) \;\land\; \extendsAMUpto{\amod}{(n-1)}{l_5 \composeL c}{l_6 \composeL d}{\amod{}'} \nonumber\\
%	& \hspace*{1cm}\lor\;  (l_2 \composeL c \composeL d) \hspace*{0.2cm} \text{ is undefined} \label{L1:Goal4}\\ \nonumber\\
%	\
	&\for{\ca{}} \for{(l_1, l'_1) \in \amod{}(\ca{})} l_1 = p \composeL q \composeL r \implies \nonumber\\
	&\exsts{l_0, l'_0, f, f_0} \nonumber\\
  &\hspace*{0.5cm}
  \begin{array}{l}
  	\left(\for{l'} l' \leq l_0 \land l' \leq l'_0 \implies l' = \unitL\right)\;\land\\
  	l_1 = l_0 \composeL f_0 \composeL f \;\land\; l'_1 = l'_0 \composeL f_0 \composeL f  \;\land\\
  	(l_0 \composeL f,\ l'_0 \composeL f) \in \amod{}'(\ca{}) 
  \end{array} \nonumber\\
%
	&\hspace*{2cm}\lor  \nonumber\\
	&\exsts{r'} l'_1 = p \composeL r' \;\land\; \extendsAMUpto{\amod}{(n-1)}{p}{r'}{\amod{}'}\label{L1:Goal2}\\\nonumber
\end{align}
%
\textbf{RTS. (\ref{L1:Goal1})}\\
Pick an arbitrary $\ca{}, c, d, l_3, l_4, (l_1 \composeL f, l_2 \composeL f) \in \amod{}'(\ca{})$ such that
%
\begin{align}
	&\for{l'} l' \leq l_1 \land l' \leq l_2 \implies l' = \unitL \label{L1:Ass10}\\
%	\exsts{l_3, l_4} \hspace*{0.5cm} 
	& l_1 = l_3 \composeL l_4 \label{L1:Ass11}\\
	& p = l_3 \composeL c \label{L1:Ass12}\\
	& q \composeL r = l_4 \composeL d \label{L1:Ass13}
%	& (l_3 \composeL c \composeL l_4 \composeL d, l_2 \composeL c \composeL d) \in \amod{}(\ca{}) \label{L1:Ass14}
\end{align}
%(\ref{L1:Ass})
Since $l_4 \leq q \composeL r$ (\ref{L1:Ass13}), from Lemma \ref{lem:divideUpper} we have:
%
\begin{align}
	\exsts{l_5, l_6, e, g} \hspace*{0.5cm} & l_4 = l_5 \composeL l_6 \label{L1:Ass15}\\
	& q = l_5 \composeL e \label{L1:Ass16}\\
	& r = l_6 \composeL g \label{L1:Ass17}
\end{align}
%
From (\ref{L1:Ass13}), (\ref{L1:Ass15})-(\ref{L1:Ass17}) and cancellativity of separation algebra of \LState\ we have:
%
\begin{align}
	e \composeL g = d \label{L1:Ass18}
\end{align}
%
%There are two cases to consider:\\
%
%\noindent\textbf{Case 1.} $l_6 > \unitL$\\
From (\ref{L1:Ass1}), (\ref{L1:Ass10})-(\ref{L1:Ass17}) and assumption of case 1 we have:
%
\begin{align}
%	\;\land\;  (\for{l'} l' \leq l_7 \land l' \leq l_8 \implies l' = \unitL) 
	&(l_1 \composeL c \composeL  d, l_2 \composeL c \composeL d) \in \amod{}(\ca{}) \;\land\;\extendsAMUpto{\amod{}}{(n-1)}{l_2 \composeL c \composeL e}{g}{\amod{}'}  \nonumber\\
	&\lor\; (l_2 \composeL c \composeL d) \hspace*{0.2cm}\text{is undefined} \nonumber
\end{align}
%(\ref{L1:Ass})
From (\ref{L1:IH}) and (\ref{L1:Ass18}) we can rewrite the above as 
%
\begin{align}
	&(l_1 \composeL c \composeL  d, l_2 \composeL c \composeL d) \in \amod{}(\ca{}) \;\land\;\extendsAMUpto{\amod{}}{(n-1)}{l_2 \composeL c}{d}{\amod{}'}  \nonumber\\
	&\lor\; (l_2 \composeL c \composeL d) \hspace*{0.2cm}\text{is undefined} \nonumber
\end{align}\\
%(\ref{L1:Ass})
%
%
%
%

\noindent\textbf{RTS. (\ref{L1:Goal2})}\\
Pick an arbitrary $\ca{}, l_1, l'_1$ such that $(l_1, l'_1) \in \amod{}(\ca{})$ and $l_1 = p \composeL q \composeL r$. From (\ref{L1:Ass1}) we have:
%
\[
\begin{array}{l}
	\exsts{l_0, l'_0, f, f_0} \nonumber\\
  \hspace*{0.5cm}
  \begin{array}{l}
  	\left(\for{l'} l' \leq l_0 \land l' \leq l'_0 \implies l' = \unitL\right)\;\land\\
  	l_1 = l_0 \composeL f_0 \composeL f \;\land\; l'_1 = l'_0 \composeL f_0 \composeL f \;\land\\
  	(l_0 \composeL f,\ l'_0 \composeL f) \in \amod{}'(\ca{}) 
  \end{array} \\
%
	\hspace*{2cm}\lor  \\
	\exsts{r'} l'_1 = p \composeL q\composeL r' \;\land\; \extendsAMUpto{\amod}{(n-1)}{p \composeL q}{r'}{\amod{}'}
\end{array}
\]
%
From (\ref{L1:IH}) we can rewrite the above as: 
%
\[
\begin{array}{l}
	\exsts{l_0, l'_0, f, f_0} \nonumber\\
  \hspace*{0.5cm}
  \begin{array}{l}
  	\left(\for{l'} l' \leq l_0 \land l' \leq l'_0 \implies l' = \unitL\right)\;\land\\
  	l_1 = l_0 \composeL f_0 \composeL f \;\land\; l'_1 = l'_0 \composeL f_0 \composeL f \;\land\\
  	(l_0 \composeL f,\ l'_0 \composeL f) \in \amod{}'(\ca{}) 
  \end{array} \\
%
	\hspace*{2cm}\lor  \\
	\exsts{r'} l'_1 = p \composeL q\composeL r' \;\land\; \extendsAMUpto{\amod}{(n-1)}{p}{q \composeL r'}{\amod{}'}
\end{array}
\]
%
and consequently,
%
\[
\begin{array}{l}
	\exsts{l_0, l'_0, f, f_0} \nonumber\\
  \hspace*{0.5cm}
  \begin{array}{l}
  	l_1 = l_0 \composeL f_0 \composeL f \;\land\; l'_1 = l'_0 \composeL f_0 \composeL f  \;\land\\
  	(l_0 \composeL f,\ l'_0 \composeL f) \in \amod{}'(\ca{}) 
  \end{array} \\
%
	\hspace*{2cm}\lor  \\
	\exsts{r'} l'_1 = p \composeL r' \;\land\; \extendsAMUpto{\amod}{(n-1)}{p}{r'}{\amod{}'}
\end{array}
\]
%
as required.
\end{proof}
\end{lemma}
%
%
\begin{lemma}[] \label{lem:amodHiding}
%
\[
\begin{array}{l}
	\for{l, r \in \LState} \for{\amod{0}, \amod{1}, \amod{} \in \AMods} \for{ F \in \pset{\LState}} \for{n \in \Nats} \\
	
	\hspace*{0.5cm} 
	l \in F \;\land\; 
	F \fences \amod{0} \;\land\; 
	\amod{0} \weakenI{F} \amod{1}\;\land\;
	\extendsAMUpto{\amod{}}{n}{l}{r}{\amod{0}}\\
	
	\hspace*{1.5cm} \implies\\
	
	
	\hspace*{0.5cm}
	\extendsAMUpto{\amod{}}{n}{l}{r}{\amod{1}}\\
\end{array}
\]
%
\begin{proof}
By induction on $n$. \\
Pick an arbitrary $l, r \in \LState,\ \amod{0}, \amod{1}, \amod{} \in \AMods,\  F \in \pset{\LState}$;\\
\noindent\textbf{Base case} $n = 0$\\
\textbf{RTS. } $\extendsAMUpto{\amod{}}{0}{l}{r}{\amod{1}}$\\
This case holds trivially from the definition of $\extendsAMUpto{\amod{}}{0}{l}{r}{\amod{1}}$.\\

\noindent\textbf{Inductive Case}\\
\textbf{Assume}
\begin{align}
	&l \in F \label{L2:Ass1}\\
	&F \fences \amod{0} \label{L2:Ass2}\\
%	&F \fences \amod{1} \label{L2:Ass3}\\
	&\amod{0} \weakenI{F} \amod{1}\label{L2:Ass4}\\
	& \extendsAMUpto{\amod{}}{n}{l}{r}{\amod{0}} \label{L2:Ass5}\\
%
	&\for{l, r \in \LState} l \in F \land F \fences \amod{0} \land \amod{0} \weakenI{F} \amod{1} \land \extendsAMUpto{\amod{}}{(n-1)}{l}{r}{\amod{0}}
	\implies \nonumber\\
	&\hspace*{1cm}\extendsAMUpto{\amod{}}{(n-1)}{l}{r}{\amod{1}} \tag{I.H}\label{L2:I.H}\\\nonumber
\end{align}

\noindent\textbf{Show}
\begin{align}
	&\for{\ca{}, c, d, l_3, l_4}\for{(l_1 \composeL f, l_2 \composeL f) \in \amod{1}(\ca{})} \nonumber\\
	&\hspace*{0.2cm} \left(\for{l'} l' \leq l_1 \land l' \leq l_2 \implies l' = \unitL\right)\;\land\; l_1 \composeL f \leq  l \composeL r \implies \nonumber\\
	&\hspace*{0.4cm}   l_1 = l_3 \composeL l_4 \;\land\; l_1 \maxMeetL l = l_3 \;\land\; l = l_3 \composeL c \;\land\; r = l_4 \composeL d\implies \nonumber\\
	&\hspace*{1cm} \left((l_1 \composeL c \composeL d, l_2 \composeL c \composeL d) \in \amod{}(\ca{}) \land
	\extendsAMUpto{\amod}{(n-1)}{l_2 \composeL c}{d}{\amod{}'}\right) \lor l_2 \composeL c \composeL d \text{ is undefined}\label{L2:Goal1}\\
%	
%	&\hspace*{0.4cm} \land\;  l_1 \not\leq l \;\land\; l_1 = l_3 \composeL l_4 \;\land\; l = l_3 \composeL c \;\land\; r = l_4 \composeL d  \implies \nonumber\\
%	&\hspace*{0.6cm} \for{l_5, l_6} l_2 = l_5 \composeL l_6 
%%	\;\land\;  (\for{l'} l' \leq l_5 \land l' \leq l_6 \implies l' = \unitL) 
%	\implies  \nonumber\\
%	& \hspace*{0.7cm} (l_3 \composeL c \composeL l_4 \composeL d, l_2 \composeL c \composeL d) \in \amod{} \;\land\; \extendsAMUpto{\amod}{(n-1)}{l_5 \composeL c}{l_6 \composeL d}{\amod{}'} \nonumber\\
%	& \hspace*{0.7cm}\lor\; l_2 \composeL c \composeL d \hspace*{0.2cm}\text{ is undefined} \label{L2:Goal4}\\
%	
%	
  &\for{\ca{}} \for{(l_1, l'_1) \in \amod{}(\ca{})} l_1 = l \composeL r \implies \nonumber\\
	&\exsts{l_0, l'_0, f, f_0} \nonumber\\
  &\hspace*{0.5cm}
  \begin{array}{l}
  	\left(\for{l'} l' \leq l_0 \land l' \leq l'_0 \implies l' = \unitL\right)\;\land\\
  	l_1 = l_0 \composeL f_0 \composeL f \;\land\; l'_1 = l'_0 \composeL f_0 \composeL f \;\land\; l_0 \leq l \;\land\\
  	(l_0 \composeL f,\ l'_0 \composeL f) \in \amod{1}(\ca{}) 
  \end{array} \nonumber\\
%
	&\hspace*{2cm}\lor  \nonumber\\
	&\exsts{r'} l'_1 = l \composeL r' \;\land\; \extendsAMUpto{\amod}{(n-1)}{l}{r'}{\amod{1}}\label{L2:Goal2}
% 
\end{align}
%
%%
%
%
%
%
%
\textbf{RTS. (\ref{L2:Goal1})}\\
Pick an arbitrary $\ca{}, c, d, l_1, l_2, f, l_3, l_4$ such that 
\begin{align}
	&\left(\for{l'} l' \leq l_1 \land l' \leq l_2 \implies l' = \unitL\right)\;\land
	(l_1 \composeL f, l_2 \composeL f) \in \amod{1}(\ca{}) \label{L2:Ass19}\\
	&l_3 = l_1 \maxMeetL l \;\land\; l_1 = l_3 \composeL l_4 \;\land\; l = l_3 \composeL c \;\land\; r = l_4 \composeL d\label{L2:Ass20}\\
	&l_1 \composeL f \leq l \composeL r \label{L2:Ass21}
%	& (l_3 \composeL c \composeL l_4 \composeL d, l_2 \composeL c \composeL d) \in \amod{}(\ca{}) \label{L2:Ass22}
\end{align}
%(\ref{L2:Ass})
From (\ref{L2:Ass1}), (\ref{L2:Ass4}), (\ref{L2:Ass19}), (\ref{L2:Ass21}) we know that 
%
\begin{align}
	\exsts{f'} (l_1 \composeL f', l_2 \composeL f') \in \amod{0}(\ca{}) \;\land\; l_1 \composeL f' \leq l \composeL r \label{L2:Ass23}
\end{align}
%
From (\ref{L2:Ass5}), (\ref{L2:Ass20}) and (\ref{L2:Ass23}) we have:
%
\begin{align}
	\begin{array}{l l}
		&(l_1 \composeL c \composeL d, l_2 \composeL c \composeL d) \in \amod{}(\ca{})\;\land\;  \extendsAMUpto{\amod{}}{(n-1)}{l_2 \composeL c}{d}{\amod{0}} \\
		& \lor\; l_2 \composeL c \composeL d \hspace*{0.2cm}\text{ is undefined}
	\end{array} \label{L2:Ass24}
\end{align}
%(\ref{L2:Ass})
From (\ref{L2:Ass1}), (\ref{L2:Ass2}), (\ref{L2:Ass19})-(\ref{L2:Ass21}), (\ref{L2:Ass23}) and by definition of $\fences$ we have:
\begin{align}
	l_2 \composeL c \in F \label{L2:Ass25}
\end{align}
From (\ref{L2:Ass1}), (\ref{L2:Ass2}), (\ref{L2:Ass4}), (\ref{L2:Ass25}) and (\ref{L2:I.H}), we can rewrite (\ref{L2:Ass24}) as:
%
\begin{align}
	\begin{array}{l l}
		&(l_1 \composeL c \composeL d, l_2 \composeL c \composeL d) \in \amod{}(\ca{})\;\land\;  \extendsAMUpto{\amod{}}{(n-1)}{l_2 \composeL c}{d}{\amod{1}} \\
		& \lor\; l_2 \composeL c \composeL d \hspace*{0.2cm}\text{ is undefined}
	\end{array} \label{L2:DismissedGoal4}
\end{align}\\\\
%
%
%
%
%
\noindent\textbf{RTS. (\ref{L2:Goal2})}\\
Pick an arbitrary $\ca{}, l_1, l'_1$ such that
\begin{align}
	&(l_1, l'_1) \in \amod{}(\ca{}) \;\land\; l_1 = l \composeL r \label{L2:Ass40}
\end{align}
%
Since $(l_1, l'_1) \in \amod{}(\ca{})$, from (\ref{L2:Ass5}) we have:
\begin{align}
&\exsts{l_0, l'_0, f, f_0} \nonumber\\
&  \hspace*{0.5cm}
  \begin{array}{l}
  	\left(\for{l'} l' \leq l_0 \land l' \leq l'_0 \implies l' = \unitL\right)\;\land\\
  	l_1 = l_0 \composeL f_0 \composeL f \;\land\; l'_1 = l'_0 \composeL f_0 \composeL f  \;\land\\
  	(l_0 \composeL f,\ l'_0 \composeL f) \in \amod{0}(\ca{}) 
  \end{array} \nonumber\\
%
	& \hspace*{2cm}\lor  \nonumber\\
	& \exsts{r'} l'_1 = l \composeL r' \;\land\; \extendsAMUpto{\amod}{(n-1)}{l}{r'}{\amod{0}} \nonumber
\end{align}
There are two cases to consider.\\
\textbf{Case 1.} 
%
\[
\begin{array}{l}
	\exsts{r'} l'_1 = l \composeL r' \;\land\; \extendsAMUpto{\amod}{(n-1)}{l}{r'}{\amod{0}}
\end{array}
\]
%
From (\ref{L2:Ass1}), (\ref{L2:Ass2}), (\ref{L2:Ass4}) and (\ref{L2:I.H}) we can rewrite the above as:
%
\begin{equation}
 \exsts{r'} l'_1 = l \composeL r' \;\land\; \extendsAMUpto{\amod{}}{(n-1)}{l}{r'}{\amod{1}} \label{L2:Case1}
\end{equation}\\

%
%
%
%
\noindent\textbf{Case 2.} 
%
\[
\begin{array}{l}
	\exsts{l_0, l'_0, f, f_0} \\
  \hspace*{0.5cm}
  \begin{array}{l}
  	\left(\for{l'} l' \leq l_0 \land l' \leq l'_0 \implies l' = \unitL\right)\;\land\\
  	l_1 = l_0 \composeL f_0 \composeL f \;\land\; l'_1 = l'_0 \composeL f_0 \composeL f \;\land\;
  	(l_0 \composeL f,\ l'_0 \composeL f) \in \amod{0}(\ca{}) 
  \end{array} 
\end{array}
\]
%
Since $l_1 = l_0 \composeL f_0 \composeL f$, we know $l_0 \composeL f \leq l_1$ and consequently from (\ref{L2:Ass40})
%
\begin{equation}
	l_0 \composeL f \leq l \composeL r \label{L2:Ass41}
\end{equation}
%
From (\ref{L2:Ass1}), (\ref{L2:Ass4}), assumption of case 2 ($(l_0 \composeL f, l'_0 \composeL f) \in \amod{0}(\ca{})$) and (\ref{L2:Ass41}) we can deduce:
%(\ref{L2:Ass})
\begin{align}
\begin{array}{l}
	\exsts{f'} (l_0 \composeL f', l'_0 \composeL f') \in \amod{1}(\ca{}) \;\land\; l_0 \composeL f' \leq l \composeL r\\
	\lor\; \for{w \in (l \meetL l_0)} w \leq l'_0
\end{array} \nonumber
\end{align}
%
There are two cases to consider:\\

\noindent\textbf{Case 2.1} 
%
%
\[
\begin{array}{l}
	\for{w \in \left(l \meetL l_0 \right)} w \leq l'_0 
\end{array}
\]
%(\ref{L2:Ass})
Since $l_0 \leq l \composeL r$ (\ref{L2:Ass41}), we know from lemma \ref{lem:nonEmptyOverlap} that $l \meetL l_0 \not= \emptyset$. Let $m = l_0 \maxMeetL l$. Then from the assumption of case 2.1. we know that $m \leq l'_0$. On the other hand since $m \leq l_0$ and $m \leq l'_0$, from the assumption of case 2. we have:
%
\begin{equation}
	m = l_0 \maxMeetL l = \unitL \label{L2:noOverlap}
\end{equation} 
%
On the other hand, from (\ref{L2:noOverlap}), lemma \ref{lem:divideUpper} and since $l_0 \leq l \composeL r$ (\ref{L2:Ass41}) we have:
%
\begin{align}
	\exsts{d} r = l_0 \composeL d \label{L2:Ass42}
\end{align}
%
%From (\ref{L2:Ass40}) and the assumption of case 2 we have:
%%
%\begin{align}
%	l_0 \composeL f \composeL f_0 = l \composeL r \label{L2:Ass42}
%\end{align}
%%
%By the cross-split property we then have:
%%
%\begin{align}
%	\exsts{l_0l, l_0r, ff_0l, ff_0r} \hspace*{0.5cm} & l_0 = l_0l \composeL l_0r \label{L2:Ass43}\\
%	& l = l_0l \composeL ff_0l \label{L2:Ass44}\\
%	& r = l_0r \composeL ff_0r \label{L2:Ass45}\\
%	& f \composeL f_0 = ff_0l \composeL ff_0r \label{L2:Ass46}
%\end{align}
%%
%Since $l_0l \composeL l_0r \composeL ff_0l$ is defined, by definition of $\meetL$ and from (\ref{L2:Ass43}) and (\ref{L2:Ass44}) we have:
%%
%\begin{align}
%	l_0l \in (l_0 \meetL l) \label{L2:Ass47}
%\end{align}
%%
%and thus from assumption of case 2 we have $l_0l \leq l'_0$. On the other hand since $l_0l \leq l'_0$ and $l_0l \leq l_0$ (\ref{L2:Ass43}), from assumption of case 2 we have 
%%
%\begin{align}
%	l_0l = \unitL \label{L2:Ass48}
%\end{align}
%%(\ref{L2:Ass})
From (\ref{L2:Ass40}), (\ref{L2:Ass41}), (\ref{L2:Ass42}) and assumption of case 2 we have:
%
\begin{align}
	& (l \composeL l_0 \composeL d, l \composeL l'_0 \composeL d) \in \amod{}(\ca{}) \;\land\; \extendsAMUpto{\amod{}}{(n-1)}{l \composeL l'_0}{d}{\amod{0}} \nonumber\\
	& \lor\; (l \composeL l'_0 \composeL d) \text{ is undefined} \label{L2:Ass49}
\end{align} 
%
On the other hand, from (\ref{L2:Ass40}), assumption of case 2, (\ref{L2:Ass42})and cancellativity of separation algebras we have: 
%
\begin{equation}
	f_0 \composeL f = l \composeL d \label{L2:Ass43}
\end{equation}
% 
and consequently from (\ref{L2:Ass40}), assumption of case 2 and (\ref{L2:Ass43}) we know $l \composeL l'_0 \composeL d$ is defined. Thus, from (\ref{L2:Ass49}) we have:
%
\begin{align}
	& (l \composeL l_0 \composeL d, l \composeL l'_0 \composeL d) \in \amod{}(\ca{}) \;\land\; \extendsAMUpto{\amod{}}{(n-1)}{l \composeL l'_0}{d}{\amod{0}} \label{L2:Ass44}
\end{align} 
%
From (\ref{L2:Ass44}) and lemma \ref{lemma:contextSwitch} we have:
%
\begin{align}
	& (l \composeL l_0 \composeL d, l \composeL l'_0 \composeL d) \in \amod{}(\ca{}) \;\land\; \extendsAMUpto{\amod{}}{(n-1)}{l}{l'_0 \composeL d}{\amod{0}} \label{L2:Ass45}
\end{align} 
%
%(\ref{L2:Ass})
From (\ref{L2:Ass1})-(\ref{L2:Ass4}), (\ref{L2:Ass45}) and (\ref{L2:I.H}) we can rewrite (\ref{L2:Ass45}) as:
%(\ref{L2:Ass})
\begin{align}
	\extendsAMUpto{\amod{}}{(n-1)}{l}{l'_0 \composeL d}{\amod{1}}\label{L2:Ass52}
\end{align}
%
From assumption of case 2 and (\ref{L2:Ass43}) we have:
%
\begin{align}
	l'_1 = l'_0 \composeL l \composeL d \label{L2:Ass53}
\end{align}
%
From (\ref{L2:Ass52}) and (\ref{L2:Ass53}) we have:
%
\begin{equation}
	\exsts{r'} l'_1 = l \composeL r' \;\land\; \extendsAMUpto{\amod{}}{(n-1)}{l}{r'}{\amod{1}}\label{L2:Case2.1}
\end{equation}
%
%
%
%
%
\noindent\textbf{Case 2.2} 
%
\[
\begin{array}{l}
	\exsts{f'} (l_0 \composeL f', l'_0 \composeL f') \in \amod{1}(\ca{}) \;\land\; l_0 \composeL f' \leq l \composeL r 
\end{array}
\]
%(\ref{L2:Ass})
From (\ref{L2:Ass40}) and assumption of case 2 we have $l_0 \composeL f' \leq l_0 \composeL f \composeL f_0$, that is:
%
\begin{equation}
	\exsts{f'_0} l_0 \composeL f' \composeL f'_0 = l_0 \composeL f \composeL f_0 \label{L2:Ass54}
\end{equation}
% 
and thus by cancellativity of separation algebra of \LState, we have:
%
\begin{align}
	f' \composeL f'_0 = f \composeL f_0 \label{L2:Ass55}
\end{align}
%
From assumptions of case 2 and case 2.2 and by (\ref{L2:Ass55}) we have:
%
\begin{align}
\begin{array}{l}
	\exsts{l_0, l'_0, f', f'_0} \\
  \hspace*{0.5cm}
  \begin{array}{l}
  	\left(\for{l'} l' \leq l_0 \land l' \leq l'_0 \implies l' = \unitL\right)\;\land\\
  	l_1 = l_0 \composeL f'_0 \composeL f' \;\land\; l'_1 = l'_0 \composeL f'_0 \composeL f' \;\land\;
  	(l_0 \composeL f',\ l'_0 \composeL f') \in \amod{1}(\ca{}) 
  \end{array} 
\end{array} \label{L2:Case2.2}
\end{align}
%
as required.
\end{proof}
\end{lemma}
%
%
\begin{lemma}[]
%
\[
		\fence{} \fences I \;\land\; I \weakenI{\fence{}} I_1 \implies \fence{} \fences I_1
\]
%
\begin{proof}
\todo
\end{proof}
\end{lemma}
%
%
\begin{lemma}[] \label{lem:extension}
%
\[
		P \;\land\; \exsts{\fence{}} \fence{} \strictfences (I, P)  \Rrightarrow \exsts{\capAss{1}} \;\; \capAss{1} * \shared{P}{I}
\]
%
\begin{proof}
\textbf{RTS. }
%
\[
\begin{array}{l}
	\for{\lenv} \for{w_1 \in \sem[\lenv]{P}} \exsts{w_2 \in \sem[\lenv]{\exsts{\capAss{1}} \;\; \capAss{1} * \shared{P }{I}}}\\
	\hspace*{1cm} (w_1, w_2) \in \guarantee \;\land\; \heapPart{\left(\collapseW{w_1}\right)} = \heapPart{\left(\collapseW{w_2}\right)}
\end{array}
\]
%
Pick an arbitrary $\lenv$, $w_1 = (l, s, \amod{}) \in \sem[\lenv]{P}$ and set $K$ such that:
\begin{align}
	K = 
	\semK[\lenv]{\left( \bigvee\limits_{\lambda \vec{x}. \capAss{}: a \in \dom{I}} \capAss{}\right) \;\lor\; \capAss{1} }
	\;\land\; K \disjoint \dom{\amod{}} \label{LE:Ass1}
\end{align}	 
%
Pick arbitrary $l_1, \cdots, l_n \in \LState$, $\amod{1} , \cdots, \amod{n} \in \AMods$, $F_1, \cdots, F_n \in \pset{\LState}$ such that 
%
\begin{align}
\begin{array}{l}
	l_1 \composeL \cdots \composeL l_n = s \\
	\for{\amod{i}, \amod{j} \in \left(\amod{1} \cup \cdots \cup \amod{n}\right)} \dom{\amod{i}} \disjoint \dom{\amod{j}}\\
	\dom{\amod{}} = \dom{\amod{1}} \cup \cdots \cup \dom{\amod{n}}\\
	\for{i \in 1, \cdots, n} l_i \in F_i \;\land\; F_i \strictfences \amod{i}\;\land\;\\
	\hspace*{1cm}\extendsAM{\amod{}}{l_i}{l_1 \composeL \cdots \composeL l_{i-1} \composeL l_{i+1} \composeL \cdots \composeL l_n}{\amod{i}} \label{LE:Ass2}
\end{array}
\end{align}
%
From the premise of the lemma we know 
%
\begin{align}
	l \in F \;\land\; F \strictfences \intermediateSemI{I} \label{LE:Ass3}
\end{align}
%(\ref{LE:Ass})
Since $\wf{l, s, \amod{}}$, we know $l \composeL s \text{ is defined}$ and thus from (\ref{LE:Ass2}), (\ref{LE:Ass3}) and lemma \ref{lem:amodWitness} we know:
%
\begin{align}
	\exsts{\amod{}'} \bigwedge\limits_{i \in \{1...n\}} \extendsAM{\amod{}'}{l_i}{l_1\composeL \cdots \composeL l_{i-1} \composeL l_{i+1}\composeL \cdots \composeL l_{n} \composeL l}{\amod{i}} \;\land\; \extendsAM{\amod{}'}{l}{s}{\intermediateSemI{I}} \label{LE:Ass4}
\end{align}
%
Pick $\amod{}'$ such that (\ref{LE:Ass4}) is satisfied. Then from (\ref{LE:Ass2}) and (\ref{LE:Ass4}) we know
%
\begin{align}
	\expandsAM{\amod{}'}{s}{l}{\amod{}} \;\land\; \extendsAM{\amod{}'}{l}{s}{\intermediateSemI{I}} \label{LE:Ass5}
\end{align}
%
Pick $\ca{1} \in \semK[\lenv]{\capAss{1}}$ and $w_2 = \left( (\unitH, \ca{1}), l \composeL s, \amod{}'\right)$. Then from (\ref{LE:Ass1}), (\ref{LE:Ass3}), (\ref{LE:Ass5}) we know
%
\begin{align}
	(w_1, w_2) \in \guarantee \label{LE:Ass6}
\end{align}
%
On the other hand, from the definition of $\collapseW{.}$ we know:
%(\ref{LE:Ass})
\begin{align}
	\heapPart{\left(\collapseW{w_1}\right)} = \heapPart{\left(\collapseW{w_2}\right)} \label{LE:Ass7}
\end{align}
%
Finally since $w_2 \in \sem[\lenv]{\capAss{1} * \shared{P}{I}}$, from (\ref{LE:Ass6}) and (\ref{LE:Ass7}) we have
%
\begin{align}
	(w_1, w_2) \in \guarantee \;\land\; \heapPart{\left(\collapseW{w_1}\right)} = \heapPart{\left(\collapseW{w_2}\right)}
\end{align}
%
as required.
\end{proof}
\end{lemma}
%
%
\begin{lemma}[]\label{lem:amodMerge}
%
\[
\begin{array}{l}
	\for{l, r \in \LState} \for{\amod{}, \amod{1}, \amod{2} \in \AMods} \for{n \in \Nats}\\
	\hspace*{0.5cm} p \composeL c \in F_1 \;\land\; F_1 \fences \amod{1} \;\land\; c \composeL q \in F_2 \;\land\; F_2 \fences \amod{2} \;\land\; (F_1, \amod{1}) \agrees (F_2, \amod{2})\\
	\hspace*{0.5cm} \land\; \extendsAMUpto{\amod{}}{n}{p \composeL c}{q \composeL r}{\amod{1}} \;\land\; \extendsAMUpto{\amod{}}{n}{c \composeL q}{p \composeL r}{\amod{2}} \\
	\hspace*{0.5cm}\land\;\left( \for{s} s \leq p \land s \leq q \implies s = \unitL \right)
	\implies\\
	\hspace*{2cm} \extendsAMUpto{\amod{}}{n}{p \composeL c \composeL q}{r}{\amod{1} \cup \amod{2}}
\end{array}
\]
%
\begin{proof} By induction on number of steps $n$.\\
\noindent Pick an arbitrary $l, r \in \LState, \amod{}, \amod{1}, \amod{2} \in \AMods$.\\
\noindent\textbf{Base case}\\
\textbf{RTS. }\hspace*{0.5cm}$\extendsAMUpto{\amod}{0}{p \composeL c \composeL q}{r}{\amod{1} \cup \amod{2}}$\\
This holds trivially by definition of $\extendsAMUpto{\amod}{0}{p \composeL c \composeL q}{r}{\amod{1} \cup \amod{2}}$\\

\noindent
\textbf{}
\textbf{Inductive Step} Pick an arbitrary $n \in \Nats$, then
%
\begin{equation}
	\tag{I.H.}
	\begin{array}{l}
		\for{l, r \in \LState} \for{\amod{}, \amod{1}, \amod{2} \in \AMods} \for{n \in \Nats}\\
		\hspace*{0.5cm} p \composeL c \in F_1 \;\land\; F_1 \fences \amod{1} \;\land\; c \composeL q \in F_2 \;\land\; F_2 \fences \amod{2} \;\land\; (F_1, \amod{1}) \agrees (F_2, \amod{2})\\
		\hspace*{0.5cm} \land\; \extendsAMUpto{\amod{}}{(n-1)}{p \composeL c}{q \composeL r}{\amod{1}} \;\land\; \extendsAMUpto{\amod{}}{(n-1)}{c \composeL q}{p \composeL r}{\amod{2}}\\
		\hspace*{0.5cm}\land\;\left( \for{s} s \leq p \land s \leq q \implies s = \unitL \right)
		\implies\\
		\hspace*{2cm} \extendsAMUpto{\amod{}}{(n-1)}{p \composeL c \composeL q}{r}{\amod{1} \cup \amod{2}}
	\end{array}
\label{LM:I.H}
\end{equation}
%
Assume:
%
\begin{align}
	& p \composeL c \in F_1 \label{LM:Ass1}\\
	& F_1 \fences \amod{1} \label{LM:Ass2}\\
	& c \composeL q \in F_2 \label{LM:Ass3}\\
	& F_2 \fences \amod{2} \label{LM:Ass4}\\
	& (F_1, \amod{1}) \agrees (F_2, \amod{2}) \label{LM:Ass5}\\
	& \extendsAMUpto{\amod{}}{n}{p \composeL c}{q \composeL r}{\amod{1}} \label{LM:Ass6}\\
	& \extendsAMUpto{\amod{}}{n}{c \composeL q}{p \composeL r}{\amod{2}} \label{LM:Ass7}\\
	& \for{s} s \leq p \;\land\; s \leq q \implies s = \unitL \label{LM:Ass0}
\end{align}
%
Show
%
\begin{align}
	&\for{\ca{}, d, e, l_3, l_4}\for{(l_1 \composeL f, l_2 \composeL f) \in (\amod{1} \cup \amod{2})(\ca{})} \nonumber\\
	& \left(\for{l'} l' \leq l_1 \land l' \leq l_2 \implies l' = \unitL\right) \;\land\; l_1 \composeL f \leq  p \composeL c \composeL q \composeL r  \implies\nonumber\\
	& \hspace*{0.2cm} l_1 = l_3 \composeL l_4 \;\land\; l_1 \maxMeetL p \composeL c \composeL q = l_3 \;\land\; p \composeL c \composeL q = l_3 \composeL d \;\land\; r = l_4 \composeL e \implies \nonumber\\
	&\hspace*{1cm} \left((l_1 \composeL d \composeL e, l_2 \composeL d \composeL e) \in \amod{}(\ca{}) \land
	\extendsAMUpto{\amod}{(n-1)}{l_2 \composeL d}{e}{\amod{}'}\right) \lor l_2 \composeL d \composeL e \text{ is undefined} \label{LM:Goal1}\\
%	& \hspace*{0.2cm} p \composeL c \composeL q = l_1 \composeL d  \implies \nonumber\\
%	&\hspace*{0.5cm}\left((l_1 \composeL d \composeL r, l_2 \composeL d \composeL r) \in \amod{}(\ca{}) \land
%	\extendsAMUpto{\amod}{(n-1)}{l_2 \composeL d}{r}{\amod{1} \cup \amod{2}}\right) \nonumber\\
%  &\hspace*{0.5cm}\lor l_2 \composeL d \composeL r \text{ is undefined} \label{LM:Goal1}\\ \nonumber\\
%
	&\for{\ca{}} \for{(l_1, l'_1) \in \amod{}(\ca{})} l_1 = p \composeL c \composeL q \composeL r \implies \nonumber\\
	&\exsts{l_0, l'_0, f, f_0} \nonumber\\
  &\hspace*{0.5cm}
  \begin{array}{l}
  	\left(\for{l'} l' \leq l_0 \land l' \leq l'_0 \implies l' = \unitL\right)\;\land\\
  	l_1 = l_0 \composeL f_0 \composeL f \;\land\; l'_1 = l'_0 \composeL f_0 \composeL f \;\land\; l_0 \leq l \;\land\\
  	(l_0 \composeL f,\ l'_0 \composeL f) \in (\amod{1} \cup \amod{2})(\ca{}) 
  \end{array} \nonumber\\
%
	&\hspace*{2cm}\lor  \nonumber\\
	&\exsts{r'} l'_1 = l \composeL r' \;\land\; \extendsAMUpto{\amod}{(n-1)}{l}{r'}{\amod{1} \cup \amod{2}}\label{LM:Goal2}\\\nonumber
\end{align}
%

\noindent\textbf{RTS. (\ref{LM:Goal2})}\\
Pick an arbitrary $\ca{}, (l_1, l'_1) \in \amod{\ca{}}$ such that 
%
\begin{equation}
	l_1 = p \composeL c \composeL q \composeL r \label{LM:Ass28}
\end{equation}
%(\ref{LM:Ass})
Then from (\ref{LM:Ass6}) we have:
%
\begin{align}
	\begin{array}{l}
		\exsts{l_0, l'_0, f, f_0} \\
	  \hspace*{0.5cm}
	  \begin{array}{l}
	  	\left(\for{l'} l' \leq l_0 \land l' \leq l'_0 \implies l' = \unitL\right)\;\land\\
	  	l_1 = l_0 \composeL f_0 \composeL f \;\land\; l'_1 = l'_0 \composeL f_0 \composeL f   \;\land\\
	  	(l_0 \composeL f,\ l'_0 \composeL f) \in (\amod{1} \cup \amod{2})(\ca{}) 
	  \end{array} \\
%
		\hspace*{2cm}\lor  \\
		\exsts{r_1} l'_1 = p \composeL c \composeL r_1 \;\land\; \extendsAMUpto{\amod}{(n-1)}{p \composeL c}{r_1}{\amod{1}}
	\end{array} \label{LM:Ass29}
\end{align}
%
Similarly from (\ref{LM:Ass7}) we have:
%
\begin{align}
	\begin{array}{l}
		\exsts{l_0, l'_0, f, f_0} \\
	  \hspace*{0.5cm}
	  \begin{array}{l}
	  	\left(\for{l'} l' \leq l_0 \land l' \leq l'_0 \implies l' = \unitL\right)\;\land\\
	  	l_1 = l_0 \composeL f_0 \composeL f \;\land\; l'_1 = l'_0 \composeL f_0 \composeL f   \;\land\\
	  	(l_0 \composeL f,\ l'_0 \composeL f) \in (\amod{1} \cup \amod{2})(\ca{}) 
	  \end{array} \\
%
		\hspace*{2cm}\lor  \\
		\exsts{r_2} l'_1 = q \composeL c \composeL r_2 \;\land\; \extendsAMUpto{\amod}{(n-1)}{q \composeL c}{r_2}{\amod{2}}
	\end{array} \label{LM:Ass30}
\end{align}
%(\ref{LM:Ass})
From (\ref{LM:Ass29}) and (\ref{LM:Ass30}) we have:
%
\begin{align}
	\begin{array}{l}
		\exsts{l_0, l'_0, f, f_0} \\
	  \hspace*{0.5cm}
	  \begin{array}{l}
	  	\left(\for{l'} l' \leq l_0 \land l' \leq l'_0 \implies l' = \unitL\right)\;\land\\
	  	l_1 = l_0 \composeL f_0 \composeL f \;\land\; l'_1 = l'_0 \composeL f_0 \composeL f  \;\land\\
	  	(l_0 \composeL f,\ l'_0 \composeL f) \in (\amod{1} \cup \amod{2})(\ca{}) 
	  \end{array} \\
%
		\hspace*{2cm}\lor  \\
		\exsts{r_1} l'_1 = p \composeL c \composeL r_1 \;\land\; \extendsAMUpto{\amod}{(n-1)}{p \composeL c}{r_1}{\amod{1}} \;\land\\
%
		\exsts{r_2} l'_1 = q \composeL c \composeL r_2 \;\land\; \extendsAMUpto{\amod}{(n-1)}{q \composeL c}{r_2}{\amod{2}}
	\end{array} \label{LM:Ass31}
\end{align}
%(\ref{LM:Ass})
From (\ref{LM:Ass31}) we have $p \composeL c \composeL r_1 = q \composeL c \composeL r_2$ and consequently $p \composeL r_1 = q \composeL r_2$. By the cross split property we then have:
%
\begin{align}
	\exsts{pq, pr_2, r_1q, r_1r_2} \hspace*{0.2cm}& pq  \composeL pr_2 = p \label{LM:Ass32}\\
	& r_1q \composeL r_1r_2 = r_1 \label{LM:Ass33}\\
	& pq \composeL r_1q = q \label{LM:Ass34}\\
	& pr_2 \composeL r_1r_2 = r_2 \label{LM:Ass35}
\end{align}
%(\ref{LM:Ass})
Since from (\ref{LM:Ass32}), (\ref{LM:Ass33}) we have $pq \leq p \;\land\; pq \leq q$, from (\ref{LM:Ass0}) we can deduce $pq = \unitL$. Consequently, we can rewrite (\ref{LM:Ass31}) as:
%
\begin{align}
	\begin{array}{l}
		\exsts{l_0, l'_0, f, f_0} \\
	  \hspace*{0.5cm}
	  \begin{array}{l}
	  	\left(\for{l'} l' \leq l_0 \land l' \leq l'_0 \implies l' = \unitL\right)\;\land\\
	  	l_1 = l_0 \composeL f_0 \composeL f \;\land\; l'_1 = l'_0 \composeL f_0 \composeL f  \;\land\\
	  	(l_0 \composeL f,\ l'_0 \composeL f) \in (\amod{1} \cup \amod{2})(\ca{}) 
	  \end{array} \\
%
		\hspace*{2cm}\lor  \\
		\exsts{r_1r_2} l'_1 = p \composeL c \composeL q \composeL r_1r_2 \;\land\; \extendsAMUpto{\amod}{(n-1)}{p \composeL c}{q \composeL r_1r_2}{\amod{1}} \;\land\\
%
		\extendsAMUpto{\amod}{(n-1)}{q \composeL c}{p \composeL r_1r_2}{\amod{2}}
	\end{array} \label{LM:Ass36}
\end{align}
%(\ref{LM:Ass})
From (\ref{LM:Ass1})-(\ref{LM:Ass5}), (\ref{LM:Ass36}) and (\ref{LM:I.H}) we can rewrite (\ref{LM:Ass36}) as:
%
\begin{align}
	\begin{array}{l}
		\exsts{l_0, l'_0, f, f_0} \\
	  \hspace*{0.5cm}
	  \begin{array}{l}
	  	\left(\for{l'} l' \leq l_0 \land l' \leq l'_0 \implies l' = \unitL\right)\;\land\\
	  	l_1 = l_0 \composeL f_0 \composeL f \;\land\; l'_1 = l'_0 \composeL f_0 \composeL f  \;\land\\
	  	(l_0 \composeL f,\ l'_0 \composeL f) \in (\amod{1} \cup \amod{2})(\ca{}) 
	  \end{array} \\
%
		\hspace*{2cm}\lor  \\
		\exsts{r'} l'_1 = p \composeL c \composeL q \composeL r' \;\land\; \extendsAMUpto{\amod}{(n-1)}{p \composeL c \composeL q}{r'}{\amod{1} \cup \amod{2}}
	\end{array} \label{LM:Ass37}
\end{align}
%(\ref{LM:Ass})
%
%
%
%
%

\noindent\textbf{RTS. (\ref{LM:Goal1})}\\
Pick an arbitrary $\ca{}, d, e, l_4, l_3, (l_1 \composeL f, l_2 \composeL f) \in (\amod{1} \cup \amod{2})(\ca{})$ such that
\begin{align}
	& \left(\for{l'} l' \leq l_1 \land l' \leq l_2 \implies l' = \unitL\right)\label{LM:Ass47}\\
	& l_1 \maxMeetL  p \composeL c \composeL q = l_3 \;\land\; l_1 = l_3 \composeL l_4 \label{LM:Ass48}\\
	& p \composeL c \composeL q = l_3 \composeL d \label{LM:Ass49}\\
	& r = l_4 \composeL e \label{LM:Ass50}\\
	& l_1 \composeL f \leq p \composeL c \composeL q \composeL r \label{LM:Ass51}
%	& (l_1 \composeL d \composeL e, l_2 \composeL d \composeL e) \in \amod{}(\ca{}) \label{LM:Ass52}
\end{align}
%(\ref{LM:Ass})
%
Then by definition of $\amod{1} \cup \amod{2}$ we know:
%
\begin{align}
	(l_1 \composeL f, l_2 \composeL f) \in \amod{1}(\ca{}) \;\lor\; (l_1 \composeL f, l_2 \composeL f) \in \amod{2}(\ca{}) \label{LM:Ass61}
\end{align}
%
\textbf{Case 1.} $(l_1 \composeL f, l_2 \composeL f) \in \amod{1}(\ca{}) \;\land\; (l_1 \composeL f, l_2 \composeL f) \not\in \amod{2}(\ca{})$\\
%
By definition of (\ref{LM:Ass5}) and from (\ref{LM:Ass3}) we have:
%
\begin{align}
	& \begin{array}{l l}
		\exsts{f'}& (l_1 \composeL f', l_2 \composeL f') \in \amod{2}(\ca{}) \;\land\; l_1 \composeL f' \leq p \composeL c \composeL q \composeL r \\
%		&\land\;  l_1 \leq (c \composeL q \maxMeetL p \composeL c) 
	\end{array}\label{LM:Ass62}\\
	& \hspace*{3cm}\lor\; \nonumber\\
	&\begin{array}{l l}
		\neg\exsts{f'} (l_1 \composeL f', l_2 \composeL f') \in \amod{2}(\ca{}) \;\land\; l_1 \composeL f' \leq p \composeL c \composeL q \composeL r \\
		\land\; (c \composeL q) \meetL l_1 = (c \composeL q) \meetL l_2
	\end{array}  \label{LM:Ass63}
\end{align}
%
There are two cases to consider:\\
\textbf{Case 1.1.} 
\[
\begin{array}{l}
		\neg\exsts{f'} (l_1 \composeL f', l_2 \composeL f') \in \amod{2}(\ca{}) \;\land\; l_1 \composeL f' \leq p \composeL c \composeL q \composeL r \\

		\land\; (c \composeL q) \meetL l_1 = (c \composeL q) \meetL l_2
\end{array}
\]
%
Since $l_1 \leq p \composeL c \composeL q \composeL r$ (\ref{LM:Ass63}), from Lemma \ref{lem:divideUpper} we have:
%
\begin{align}
	\begin{array}{l l}
	\exsts{s_1, s_2, a, b}& l_3 = s_1 \composeL s_2 \\
	& s_1 \composeL a =  p\\
	& s_2 \composeL b = c \composeL q\\
	&a \composeL  b = d 
	\end{array} \label{LM:Ass64}
\end{align}
%
Since $p \composeL c \composeL q  = s_1 \composeL a \composeL s_2 \composeL b $ and consequently $l_1 \composeL a \composeL b$ is defined, by definition of $\meetL$ we have:
%
\begin{align}
	s_2 \in (l_1 \meetL c \composeL q) \nonumber
\end{align} 
%
and thus from the assumption of case 1.1. we have $s_2 \leq l_2$. Since $s_2 \leq l_1$ (\ref{LM:Ass64}, \ref{LM:Ass48}), from (\ref{LM:Ass47}) we have $s_2 = \unitL$; thus from (\ref{LM:Ass64}):
%
\begin{align}
	\begin{array}{l}
	l_1 = s_1 \composeL l_4\\
	b = c \composeL q
	\end{array} \label{LM:Ass644}
\end{align} 
% 
and consequently from (\ref{LM:Ass64}) we have:
%
\begin{align}
	l_3 \composeL a = p \nonumber\\
	l_1 \composeL a \composeL e = p \composeL r \label{LM:Ass65}
\end{align}
%
From assumption of case 1, (\ref{LM:Ass6}), (\ref{LM:Ass47})-(\ref{LM:Ass51}) and (\ref{LM:Ass64}), (\ref{LM:Ass65}) we have:
%
\begin{align}
	\begin{array}{l l}
%	\for{l_6, l_7} & l_2 = l_6 \composeL l_7\;\implies\\
	& (l_1 \composeL d \composeL e, l_2 \composeL d \composeL e) \in \amod{}(\ca{}) \;\land\; \extendsAMUpto{\amod{}}{(n-1)}{l_2 \composeL a \composeL c}{q \composeL e}{\amod{1}}\\
	& \;\;\lor (l_2 \composeL d \composeL e)\hspace*{0.2cm}\text{is undefined}
	\end{array} \label{LM:Ass66}
\end{align}
%
On the other hand, from (\ref{LM:Ass63}), (\ref{LM:Ass64}), (\ref{LM:Ass66}) and (\ref{LM:Ass7}) we can deduce:
%
\begin{align}
	\begin{array}{l l}
		&(l_1 \composeL d \composeL e, l_2 \composeL d \composeL e) \in \amod{}(\ca{}) \;\land\;
		\extendsAMUpto{\amod{}}{(n-1)}{l_2 \composeL a \composeL c}{q \composeL e}{\amod{1}}\\
		& \hspace*{1cm}\land\; \extendsAMUpto{\amod{}}{(n-1)}{q \composeL c}{l_2 \composeL a \composeL e}{\amod{2}}\\
		& \;\lor (l_2 \composeL d \composeL e)\hspace*{0.2cm}\text{is undefined}
	\end{array} \label{LM:Ass67}
\end{align}
%(\ref{LM:Ass})
From (\ref{LM:Ass1}), (\ref{LM:Ass2}), assumption of case 1, (\ref{LM:Ass47})-(\ref{LM:Ass51}), (\ref{LM:Ass65}), (\ref{LM:Ass66})  and definition of $\fences$ we have:
%
\begin{equation}
	l_2 \composeL a \composeL c \in F_1 \label{LM:Ass68}
\end{equation}
%
Assume 
%
\begin{align}
	\begin{array}{l l}
		\exsts{t > \unitL} & t \leq l_2 \composeL a \;\land\; t \leq q
	\end{array} \label{LM:Con1}
\end{align}
%
Now also assume 
\begin{equation}
	t \leq l_1 \label{LM:Con2}
\end{equation}
%
From (\ref{LM:Con2}), (\ref{LM:Ass48}) and Lemma \ref{lem:divideUpper} we have:
%
\begin{align}
	\begin{array}{l l}
		\exsts{t_1, t_2} & t = t_1 \composeL t_2\\
		& t_1 \leq l_3\\
		& t_2 \leq l_4
	\end{array} \label{LM:Con3}
\end{align}
%
From (\ref{LM:Con3}) and (\ref{LM:Ass48}) and by definition of $\maxMeetL$, I can deduce: 
%
\begin{equation}
	t_2 = \unitL \label{LM:Con4}
\end{equation}
%
On the other hand from (\ref{LM:Con1}), (\ref{LM:Con3}), (\ref{LM:Ass65}) and (\ref{LM:Ass0}) I have: 
%
\begin{equation}
	t_1 = \unitL \label{LM:Con5}
\end{equation}
% 
From (\ref{LM:Con3}), (\ref{LM:Con4}) and (\ref{LM:Con5}) we have $t = \unitL$ which contradicts our assumption of (\ref{LM:Con1}). Therefore, we can deduce that our assumption of (\ref{LM:Con2}) was wrong and that:
%
\begin{equation}
	t \not\leq l_1 \label{LM:Con6}
\end{equation}
%
Consequently from (\ref{LM:Con1}) and (\ref{LM:Con6}) we can conclude
%
\begin{equation}
	(c \composeL q) \meetL l_1 \not= (c \composeL q) \meetL l_2 \nonumber
\end{equation}
%
This is however a contradiction to the assumption of case 1.1 and therefore, we can conclude that our assumption of (\ref{LM:Con1}) was wrong and that we have:
%
\begin{equation}
	\for{s} t\leq l_2 \composeL a \land t \leq  q \implies t = \unitL \label{LM:Ass60}
\end{equation}
%
From (\ref{LM:Ass68}), (\ref{LM:Ass2}), (\ref{LM:Ass3}), (\ref{LM:Ass4}), (\ref{LM:Ass5}), (\ref{LM:Ass67}), (\ref{LM:Ass60}) and (\ref{LM:I.H}), we can rewrite (\ref{LM:Ass66}) and (\ref{LM:Ass67}) as:
%
\begin{align}
	\begin{array}{l l}
		& (l_1 \composeL d \composeL e, l_2 \composeL d \composeL e) \in \amod{}(\ca{}) \;\land\;
		\extendsAMUpto{\amod{}}{(n-1)}{l_2 \composeL a \composeL c \composeL q}{e}{\amod{1} \cup \amod{2}}\\
		& \;\lor (l_2 \composeL d \composeL e)\hspace*{0.2cm}\text{is undefined}
	\end{array} \label{LM:Ass69}
\end{align}
%(\ref{LM:Ass})
From (\ref{LM:Ass64}) and (\ref{LM:Ass644}) we have: 
%
\begin{align}
	\begin{array}{l l}
		& (l_1 \composeL d \composeL e, l_2 \composeL d \composeL e) \in \amod{}(\ca{}) \;\land\;
		\extendsAMUpto{\amod{}}{(n-1)}{l_2 \composeL d}{e}{\amod{1} \cup \amod{2}}\\
		& \;\lor (l_2 \composeL d \composeL e)\hspace*{0.2cm}\text{is undefined}
	\end{array} \nonumber
\end{align}
%
%
%

\noindent\textbf{Case 1.2.}
\[
\begin{array}{l l}
	\exsts{f'}& (l_1 \composeL f', l_2 \composeL f') \in \amod{2}(\ca{}) \;\land\; l_1 \composeL f' \leq p \composeL c \composeL q \composeL r \\
%	&\land\; l_1 \leq (c \composeL q \maxMeetL p \composeL c)
\end{array}
\]
%
%
Since $l_3 \leq p \composeL c \composeL q$ (\ref{LM:Ass49}), from Lemma \ref{lem:divideUpper} we have:
%
\begin{align}
	\begin{array}{l l}
		\exsts{l_5, l_6, l_7, e, g, h} \hspace*{0.2cm} & l_3 = l_5 \composeL l_6 \composeL l_7\\
		& p = l_5 \composeL i\\
		& c = l_6 \composeL g\\
		& q = l_7 \composeL h\\
		& d = i \composeL g \composeL h
	\end{array} \label{LM:Ass70}
\end{align} 
%
From (\ref{LM:Ass70}), assumption of case 1, (\ref{LM:Ass6}) and (\ref{LM:Ass50}) we have:
%
%
\begin{align}
	\begin{array}{l}
		(l_1 \composeL d \composeL e, l_2 \composeL d \composeL e) \in \amod{}(\ca{}) \;\land\;
		\extendsAMUpto{\amod{}}{(n-1)}{l_2 \composeL i \composeL g}{h \composeL e}{\amod{1}}\\
		\lor\; l_2 \composeL d \composeL e \hspace*{0.2cm}\text{ is undefined}
	\end{array} \label{LM:Ass71}
\end{align}
%(\ref{LM:Ass})
%(\ref{LM:Ass})
Similarly, from (\ref{LM:Ass70}), assumption of case 1.2, (\ref{LM:Ass7}) and (\ref{LM:Ass50}) we have:
%
\begin{align}
	\begin{array}{l}
		(l_1 \composeL d \composeL e, l_2 \composeL d \composeL e) \in \amod{}(\ca{}) \;\land\;
		\extendsAMUpto{\amod{}}{(n-1)}{l_2 \composeL g \composeL h}{i \composeL e}{\amod{2}}\\
		\lor\; l_2 \composeL d \composeL e \hspace*{0.2cm}\text{ is undefined}
	\end{array} \label{LM:Ass72}
\end{align}
%(\ref{LM:Ass})
%(\ref{LM:Ass})
From (\ref{LM:Ass71}) and (\ref{LM:Ass72}) we have:
%
\begin{align}
	\begin{array}{l}
		(l_1 \composeL d \composeL e, l_2 \composeL d \composeL e) \in \amod{}(\ca{}) \;\land\;\\
		\hspace*{0.6cm}\extendsAMUpto{\amod{}}{(n-1)}{l_2 \composeL i \composeL g}{h \composeL e}{\amod{1}} \;\land\;
		\extendsAMUpto{\amod{}}{(n-1)}{l_2 \composeL g \composeL h}{i \composeL e}{\amod{2}}\\
		\lor\; l_2 \composeL d \composeL e \hspace*{0.2cm}\text{ is undefined}
	\end{array} \label{LM:Ass73}
\end{align}
%(\ref{LM:Ass})
%
From (\ref{LM:Ass1}), (\ref{LM:Ass2}), (\ref{LM:Ass70}), (\ref{LM:Ass47}), (\ref{LM:Ass50}), assumption of case 1 and definition of $\fences$ we have:
%
\begin{equation}
	l_2 \composeL i \composeL g \in F_1 \label{LM:Ass74}
\end{equation}
%
Similarly, from (\ref{LM:Ass3}), (\ref{LM:Ass4}), (\ref{LM:Ass70}), (\ref{LM:Ass47}), (\ref{LM:Ass50}), assumption of case 1.2, and definition of $\fences$ we have:
%
\begin{equation}
	l_2 \composeL g \composeL h \in F_2 \label{LM:Ass75}
\end{equation}
%(\ref{LM:Ass})
From (\ref{LM:Ass0}) and (\ref{LM:Ass70}) we have:
\begin{align}
\begin{array}{l}
	\for{l'} l' \leq i \;\land\; l' \leq h \implies l' = \unitL
\end{array} \label{LM:Ass76}
\end{align}
From (\ref{LM:Ass74}), (\ref{LM:Ass2}), (\ref{LM:Ass75}), (\ref{LM:Ass4}), (\ref{LM:Ass5}), (\ref{LM:Ass0}),   (\ref{LM:Ass73}), (\ref{LM:Ass76}) and (\ref{LM:I.H}) we can rewrite (\ref{LM:Ass73}) as:
%
\begin{align}
	\begin{array}{l}
		(l_1 \composeL d \composeL e, l_2 \composeL d \composeL e) \in \amod{}(\ca{}) \;\land\;\extendsAMUpto{\amod{}}{(n-1)}{l_2 \composeL i \composeL g \composeL h}{e}{\amod{1} \cup \amod{2}} \\
		\lor\; l_2 \composeL d \composeL e \hspace*{0.2cm}\text{ is undefined}
	\end{array} \label{LM:Ass77}
\end{align}
%(\ref{LM:Ass})
From (\ref{LM:Ass70}) we can rewrite (\ref{LM:Ass77}) as:
%
\begin{align}
	\begin{array}{l}
		(l_1 \composeL d \composeL e, l_2 \composeL d \composeL e) \in \amod{}(\ca{}) \;\land\;\extendsAMUpto{\amod{}}{(n-1)}{l_2 \composeL d}{e}{\amod{1} \cup \amod{2}} \\
		\lor\; l_2 \composeL d \composeL e \hspace*{0.2cm}\text{ is undefined}
	\end{array}
	 \label{LM:Case3.1.2}
\end{align}
%(\ref{LM:Ass})

%Proof of this case is analogous to that of steps (\ref{LM:Ass21})-(\ref{LM:Case1.2}) and is omitted here.
%
%
%
%
%

\noindent\textbf{Case 2.} $(l_1 \composeL f, l_2 \composeL f) \in \amod{2}(\ca{}) \;\land\; (l_1 \composeL f, l_2 \composeL f) \not\in \amod{1}(\ca{})$\\
The proof of this case is analogous o that of previous case and is omitted here.\\

\noindent\textbf{Case 3.} $(l_1 \composeL f, l_1 \composeL f) \in \amod{2}(\ca{}) \;\land\; (l_1 \composeL f, l_2 \composeL f) \in \amod{2}(\ca{})$\\
This case follows trivially from case 1 and case 2. \\


\end{proof}
\end{lemma}
%
%
%\begin{lemma}[]\label{lem:amodExtension}
%%
%\[
%\begin{array}{l}
%	\for{s, s_e \in \LState} \for{\amod{}', \amod{}, \amod{e} \in \AMods} \for{n \in \Nats} \for{F, F_e \in \pset{\LState}}\\
%	\hspace*{0.5cm} s \in F \;\land\; F \strictfences \amod{} \;\land\; s_e \in F_e \;\land\; F_e \strictfences \amod{e} \;\land\; \dom{\amod{}} \cap \dom{\amod{e}} = \emptyset \\
%	\hspace*{0.5cm} \land\; \amod{}' = \extendAM{\amod{}}{F_e} \uplus \extendAM{\amod{e}}{F}
%	\implies\\
%	\hspace*{2cm} \extendsAMUpto{\amod{}'}{n}{s}{s_e}{\amod{}} \;\land\; \extendsAMUpto{\amod{}'}{n}{s_e}{s}{\amod{n}}
%\end{array}
%\]
%%
%\begin{proof} By induction on number of steps $n$.\\
%\noindent Pick an arbitrary $s, s_e \in \LState, \amod{}', \amod{}, \amod{e} \in \AMods, F, F_e \in \pset{\LState}$.\\
%\noindent\textbf{Base case}\\
%\textbf{RTS. }\hspace*{0.5cm}$\extendsAMUpto{\amod{}'}{0}{s}{s_e}{\amod{}} \;\land\; \extendsAMUpto{\amod{}'}{0}{s_e}{s}{\amod{n}}$\\
%This holds trivially by definition of $\amod{}' \downarrow_{0}$\\
%
%\textbf{Inductive Step} Pick an arbitrary $n \in \Nats$, then
%%
%\begin{equation}
%	\tag{I.H.}
%	\begin{array}{l}
%		\for{s, s_e \in \LState} \for{\amod{}', \amod{}, \amod{e} \in \AMods} \for{n \in \Nats} \for{F, F_e \in \pset{\LState}}\\
%		s \in F \;\land\; F \strictfences \amod{} \;\land\; s_e \in F_e \;\land\; F_e \strictfences \amod{e} \;\land\; \dom{\amod{}} \cap \dom{\amod{e}} = \emptyset \\
%		\land\; \amod{}' = \extendAM{\amod{}}{F_e} \uplus \extendAM{\amod{e}}{F}
%	\implies\\
%	\hspace*{1cm} \extendsAMUpto{\amod{}'}{(n-1)}{s}{s_e}{\amod{}} \;\land\; \extendsAMUpto{\amod{}'}{(n-1)}{s_e}{s}{\amod{n}}
%	\end{array}
%\label{LAE:I.H}
%\end{equation}
%%
%Assume:
%%
%\begin{align}
%	& s \in F \label{LAE:Ass1}\\
%	& F \strictfences \amod{} \label{LAE:Ass2}\\
%	& s_e \in F_e \label{LAE:Ass3}\\
%	& F_e \strictfences \amod{e} \label{LAE:Ass4}\\
%	& \dom{\amod{}} \cap \dom{\amod{e}} = \emptyset \label{LAE:Ass5}\\
%	& \amod{}' = \extendAM{\amod{}}{F_e} \uplus \extendAM{\amod{e}}{F} \label{LAE:Ass6}
%\end{align}
%%
%Show
%%
%\begin{align}
%	&\extendsAMUpto{\amod{}'}{(n)}{s}{s_e}{\amod{}}  \label{LAE:Goal1}\\
%	&\extendsAMUpto{\amod{}'}{(n)}{s_e}{s}{\amod{n}} \label{LAE:Goal2}\\\nonumber
%\end{align}
%%
%
%\noindent\textbf{RTS. (\ref{LAE:Goal1})}\\
%Pick an arbitrary $\ca{}, (l_1, l'_1) \in \amod{}'(\ca{})$ such that 
%%
%\begin{equation}
%	l_1 = s \composeL s_e \label{LAE:Ass7}
%\end{equation}
%%(\ref{LAE:Ass})
%From (\ref{LAE:Ass6}) we know that $\ca{} \in \dom{\amod{}} \uplus \dom{\amod{e}}$. There are two cases to consider. \\
%
%\noindent\textbf{Case 1.} $\ca{} \in \dom{\amod{}}$\\
%From the assumption of the case and (\ref{LAE:Ass6}) we know:
%%
%\begin{align}
%\begin{array}{l l}
%	\exsts{(s_1 \composeL d, s_2 \composeL d) \in \amod{}(\ca{})} \exsts{s_3 \in F_e} \exsts{s_0, d_0, m} &
%	m = d \maxMeetL s_3 \\
%	& s_3 = m \composeL s_0\\
%	& d = m \composeL d_0\\
%	& l_1 = s_1 \composeL m \composeL s_0 \composeL d_0\\
%	& l'_1 = s_2 \composeL m \composeL s_0 \composeL d_0
%\end{array} \label{LAE:Ass8}
%\end{align}
%%
%On the other hand, from (\ref{LAE:Ass2}), (\ref{LAE:Ass7}) and (\ref{LAE:Ass8}) we know that $s_1 \leq s$. Consequently, from (\ref{LAE:Ass6}), (\ref{LAE:Ass7}) and (\ref{LAE:Ass8}) we have:
%%
%\begin{align}
%	\begin{array}{l l}
%		\exsts{c} & m \composeL s_0 \composeL d_0 = c \composeL s_e\\
%		& s = s_1 \composeL c\\
%		& l_1 = s_1 \composeL c \composeL s_e\\
%		& l'_1 = s_2 \composeL c \composeL s_e
%	\end{array} \label{LAE:Ass9}
%\end{align}
%%
%Pick $f_0, f, l_0, l'_0$ such that $f_0 = c \composeL s_e$, $l_0 \composeL f = s_1$ and $l'_0 \composeL f = s_2$. Then from (\ref{LAE:Ass8}) and (\ref{LAE:Ass9}) we have:
%%
%\begin{align}
%	\begin{array}{l}
%		\exsts{l_0, l'_0, f, f_0} \\
%	  \hspace*{0.5cm}
%	  \begin{array}{l}
%	  	\left(\for{l'} l' \leq l_0 \land l' \leq l'_0 \implies l' = \unitL\right)\;\land\\
%	  	l_1 = l_0 \composeL f_0 \composeL f \;\land\; l'_1 = l'_0 \composeL f_0 \composeL f   \;\land\\
%	  	(l_0 \composeL f,\ l'_0 \composeL f) \in \amod{} (\ca{}) 
%	  \end{array}
%	\end{array} \label{LAE:Ass10}
%\end{align}\\
%%
%
%\noindent\textbf{Case 2. }$\ca{} \in \dom{\amod{e}}$\\
%From the assumption of the case and (\ref{LAE:Ass6}) we know:
%%
%\begin{align}
%\begin{array}{l l}
%	\exsts{(s_1 \composeL d, s_2 \composeL d) \in \amod{e}(\ca{})} \exsts{s_3 \in F} \exsts{s_0, d_0, m} &
%	m = d \maxMeetL s_3 \\
%	& s_3 = m \composeL s_0\\
%	& d = m \composeL d_0\\
%	& l_1 = s_1 \composeL m \composeL s_0 \composeL d_0\\
%	& l'_1 = s_2 \composeL m \composeL s_0 \composeL d_0
%\end{array} \label{LAE:Ass11}
%\end{align}
%%
%On the other hand, from (\ref{LAE:Ass4}), (\ref{LAE:Ass7}) and (\ref{LAE:Ass11}) we know that $s_1 \leq s_e$. Consequently, from (\ref{LAE:Ass6}), (\ref{LAE:Ass7}) and (\ref{LAE:Ass11}) we have:
%%
%\begin{align}
%	\begin{array}{l l}
%		\exsts{c} & m \composeL s_0 \composeL d_0 = c \composeL s\\
%		& s_e = s_1 \composeL c\\
%		& l_1 = s_1 \composeL c \composeL s\\
%		& l'_1 = s_2 \composeL c \composeL s
%	\end{array} \label{LAE:Ass12}
%\end{align}
%%(\ref{LAE:Ass})
%On the other hand from (\ref{LAE:Ass3}), (\ref{LAE:Ass4}), (\ref{LAE:Ass11}), (\ref{LAE:Ass12}) and by definition of $\strictfences$, we have:
%%
%\begin{align}
%	s_2 \composeL c \in F_e \label{LAE:Ass13}
%\end{align}
%%
%From (\ref{LAE:Ass1}), (\ref{LAE:Ass2}), (\ref{LAE:Ass13}), (\ref{LAE:Ass4})-(\ref{LAE:Ass6}), (\ref{LAE:I.H}) we have:
%%
%\begin{align}
%	\extendsAMUpto{\amod{}'}{(n-1)}{s}{s_2 \composeL c}{\amod{}} \label{LAE:Ass14}
%\end{align}
%Pick $r'$ such that $r' = s_2 \composeL c$. Then from (\ref{LAE:Ass11}), (\ref{LAE:Ass12}) and (\ref{LAE:Ass14}) we have:
%%
%\begin{align}
%	\begin{array}{l}
%		\exsts{r'} l'_1 = s \composeL r' \;\land\; \extendsAMUpto{\amod{}'}{(n-1)}{s}{r'}{\amod{}}
%	\end{array} \label{LAE:Ass15}
%\end{align}\\
%%  (\ref{LAE:Ass})
%From (\ref{LAE:Ass10}) and (\ref{LAE:Ass15}) we have:
%%
%\begin{align}
%	\begin{array}{l}
%		\exsts{l_0, l'_0, f, f_0} \\
%	  \hspace*{0.5cm}
%	  \begin{array}{l}
%	  	\left(\for{l'} l' \leq l_0 \land l' \leq l'_0 \implies l' = \unitL\right)\;\land\\
%	  	l_1 = l_0 \composeL f_0 \composeL f \;\land\; l'_1 = l'_0 \composeL f_0 \composeL f   \;\land\\
%	  	(l_0 \composeL f,\ l'_0 \composeL f) \in \amod{}(\ca{}) 
%	  \end{array} \\
%%
%		\hspace*{2cm}\lor  \\
%		\exsts{r'} l'_1 = s \composeL r' \;\land\; \extendsAMUpto{\amod{}'}{(n-1)}{s}{r'}{\amod{}}
%	\end{array} \label{LAE:Ass16}
%\end{align}
%%(\ref{LAE:Ass})
%%
%%
%%
%%
%%
%\noindent\textbf{RTS. (\ref{LAE:Goal2})}\\
%Pick an arbitrary $\ca{}, d, e, l_4, l_3, l_1, l_2, f$ such that
%\begin{align}
%	& (l_1 \composeL f, l_2 \composeL f) \in \amod{}(\ca{}) \label{LAE:Ass17}\\
%	& \left(\for{l'} l' \leq l_1 \land l' \leq l_2 \implies l' = \unitL\right)\label{LAE:Ass18}\\
%	& l_1 \maxMeetL  s = l_3 \;\land\; l_1 = l_3 \composeL l_4 \label{LAE:Ass19}\\
%	& s = l_3 \composeL d \label{LAE:Ass20}\\
%	& s_e = l_4 \composeL e \label{LAE:Ass21}\\
%	& l_1 \composeL f \leq s \composeL s_e \label{LAE:Ass22}
%\end{align}
%%(\ref{LAE:Ass})
%From (\ref{LAE:Ass2}), (\ref{LAE:Ass18}) and (\ref{LAE:Ass22}) we have $l_1 \leq s$, that is:
%%
%\begin{align}
%	& \exsts{c} l_1 \composeL c = s \label{LAE:Ass23}
%\end{align}
%%
%From (\ref{LAE:Ass19}) and (\ref{LAE:Ass23}) we have:
%%
%\begin{align}
%	\begin{array}{l}
%		l_3 = l_1\\
%		l_4 = \unitL
%	\end{array} \label{LAE:Ass24}
%\end{align}
%%(\ref{LAE:Ass})
%From (\ref{LAE:Ass6}), (\ref{LAE:Ass17}), (\ref{LAE:Ass18}) and (\ref{LAE:Ass22}) we have:
%%
%\begin{align}
%	\begin{array}{l l}
%		\exsts{m, f_0, s_0} & m = f \maxMeetL s_e\\
%		& s_e = s_0 \composeL m\\
%		& f = f_0 \composeL m\\
%		& (l_1 \composeL f_0 \composeL m \composeL s_0, l_2 \composeL f_0 \composeL m \composeL s_0) \in \amod{}'(\ca{}) \;\lor\; l_2 \composeL f_0 \composeL m \composeL s_0 \text{ undefined}
%	\end{array} \label{LAE:Ass25}
%\end{align}
%
%\end{proof}
%\end{lemma}
%%
%%
\begin{lemma}\label{lem:localityPreservation}
%
%
%
\[
\begin{array}{l}
	\for{\amod{}, \amod{1} \in \AMods} \for{S, F \in \pset{\LState}}\\
	\hspace*{1cm}\local{\amod{}}{S} \;\land\; S \in \extend{F} \;\land\; \amod{} \weakenI{F} \amod{1} \implies
	\local{S}{\amod{1}}
\end{array}
\]
%
\begin{proof} 
Pick an arbitrary $\amod{}, \amod{1} \AMods$, $\ca{} \in \Caps$, $S, F \in \pset{\LState}$, $l_1, l_2, f \in \LState$, $s \in S$ and \\
\textbf{Assume:}
%
\begin{align}
	\local{\amod{}}{S} \label{LLP:Ass1}\\
	S \in \extend{F} \label{LLP:Ass2}\\
	\amod{} \weakenI{F} \amod{1} \label{LLP:Ass3}\\
	(l_1 \composeL f, l_2 \composeL f) \in \amod{1}(\ca{}) \label{LLP:Ass4}\\
	\for{l'} l' \leq l_1 \;\land\; l' \leq l_2 \implies l' = \unitL \label{LLP:Ass5}
\end{align}	
%
\textbf{Show:}
%
\begin{align}
	l_1 \leq s \;\lor\; l_1 \composeL f \meetL s = \emptyset \label{LLP:Goal}
\end{align}
%
Since $s \in S$, from (\ref{LLP:Ass2}) we know:
%
\begin{align}
	\exsts{l \in F} \exsts{r} s = l \composeL r \label{LLP:Ass6}
\end{align}
%
Assume
%
\begin{align}
	l_1 \not \leq l \composeL r \;\land\; l_1 \composeL f \meetL l \composeL r \not= \emptyset \label{LLP:Ass7}
\end{align}
%
From (\ref{LLP:Ass7}) and by definition of $\meetL$ we have:
%
\begin{align}
	\begin{array}{l l}
		\exsts{a, b, c} \hspace*{0.2cm} & l \composeL r = a \composeL b \\
		& l_1 \composeL f = b \composeL c \\
		& a \composeL b \composeL c \hspace*{0.2cm} \text {is defined}
	\end{array}
	\label{LLP:Ass8}
\end{align}
%
From (\ref{LLP:Ass8}) we can deduce:
%
\begin{align}
	l_1 \composeL f \leq l \composeL r \composeL c \label{LLP:Ass9}
\end{align}
%
From (\ref{LLP:Ass3})-(\ref{LLP:Ass5}), (\ref{LLP:Ass6}) and (\ref{LLP:Ass9}) we have:
%
\begin{align}
	\exsts{f'} (l_1 \composeL f', l_2 \composeL f') \in \amod{}(\ca{}) \;\land\; l_1 \composeL f' \leq l \composeL r \composeL c \label{LLP:Ass10}
\end{align}
%
From (\ref{LLP:Ass10}) and Lemma \ref{lem:nonEmptyOverlap} we have:
%
\begin{align}
	l_1 \composeL f' \meetL l \composeL r \composeL c \not= \emptyset \nonumber
\end{align}
%
and consequently from Lemma \ref{lem:noMeetByOrder} we have:
%
\begin{align}
	l_1 \composeL f' \meetL l \composeL r \not= \emptyset \label{LLP:Ass11}
\end{align}
%
From (\ref{LLP:Ass1}), (\ref{LLP:Ass6}) and (\ref{LLP:Ass11}) we have:
%
\begin{align}
	l_1 \leq l \composeL r \nonumber
\end{align}
%
However, this is a contradiction to our assumption in (\ref{LLP:Ass7}). Thus we can deduce that our assumption in (\ref{LLP:Ass7}) was wrong and consequently
%
\begin{align}
	l_1 \leq s \;\lor\; l_1 \composeL f \meetL s = \emptyset \nonumber
\end{align}
%
as required
\end{proof}
\end{lemma}
%
%
\begin{lemma}[]\label{lem:nonEmptyOverlap}
%
\[
	\for{a, b, c \in \LState} a \leq b \composeL c \implies a \meetL b \not= \emptyset
\]
%
\begin{proof} By contradiction.
Take arbitrary $a, b, c \in \LState$ such that 
%
\begin{equation}
	a \leq b \composeL c \label{L5:Ass1}
\end{equation}
%
and assume
%
\begin{equation}
	a \meetL b = \emptyset \label{L5:Ass2}
\end{equation}
%
From (\ref{L5:Ass2}) and by definition of $\meetL$, we have:
%
\begin{equation}
	\neg\exsts{d, e, f, g} a = d \composeL e \;\land\; b = e \composeL f \;\land\; d \composeL e \composeL f = g \label{L5:Ass3}
\end{equation}
%
From (\ref{L5:Ass1}) we have $\exsts{h} a \composeL h = b \composeL c$ and consequently by the cross-splittability property we have:
%
\begin{align}
	\exsts{ab, ac, hb, hc, t} \hspace*{2cm}&
	ab \composeL ac = a 	\label{L5:Ass4}\\
	& hb \composeL hc = h \label{L5:Ass5}\\
	& ab \composeL hb = b \label{L5:Ass6}\\
	& ac \composeL hc = c \label{L5:Ass7}\\
	& t = ab \composeL ac \composeL hb \composeL hc \label{L5:Ass8}
\end{align}
%
From (\ref{L5:Ass8}) we have:
%
\begin{equation}
	\exsts{s} ab \composeL ac \composeL hb = s \label{L5:Ass9}
\end{equation}
%
From (\ref{L5:Ass4}), (\ref{L5:Ass6}) and (\ref{L5:Ass9}) we have: 
%
\begin{equation}
	\exsts{d, e, f, g} a = d \composeL e \;\land\; b = e \composeL f \;\land\; d \composeL e \composeL f = g \label{L5:Ass10}
\end{equation}
%
From (\ref{L5:Ass3}) and (\ref{L5:Ass10}) we derive a contradiction and can hence deduce:
%
\begin{equation}
	a \meetL b \not= \emptyset \nonumber
\end{equation}
%
as required.
\end{proof}
\end{lemma}
%
%
\begin{lemma}[]\label{lem:divideUpper}
%
Given any separation algebra ($\mathcal{M}, \compose{\mathcal{M}}, \unit{\mathcal{M}}$) with the cross-split property:
\[
	\for{a, b, c \in \mathcal{M}} a \leq b \compose{\mathcal{M}} c \implies \exsts{m, n} a = m \compose{\mathcal{M}} n \;\land\; m \leq b \;\land\; n \leq c
\]
%
\begin{proof}
Pick an arbitrary $a, b, c \in \mathcal{M}$. Since $a \leq b \compose{\mathcal{M}} c$, we know $\exsts{d \in \mathcal{M}}$ such that:
%
\begin{equation}
	a \compose{\mathcal{M}} d = b \compose{\mathcal{M}} c \label{L6:Ass1}
\end{equation}
%
By the cross-split property of $\mathcal{M}$, We can deduce: $\exsts{ab, ac, db, dc \in \mathcal{M}}$ such that:
%
\begin{align}
	a = ab \compose{\mathcal{M}} ac \label{L6:Ass2}\\
	b = ab \compose{\mathcal{M}} db \label{L6:Ass3}\\
	c = ac \compose{\mathcal{M}} dc \label{L6:Ass4}\\
	d = db \compose{\mathcal{M}} dc \nonumber 
\end{align}
%
Since $ab \leq b$ (\ref{L6:Ass3}) and $ac \leq c$ (\ref{L6:Ass4}), from (\ref{L6:Ass2}) we can deduce:
%
\begin{equation}
	\exsts{m, n \in \mathcal{M}} a = m \compose{\mathcal{M}} n \;\land\; m \leq b \;\land\; n \leq c \nonumber
\end{equation}
%
as required.
\end{proof}
\end{lemma}
%
%
\begin{lemma}[]\label{lem:disjointByOrder}
Given any separation algebra ($\mathcal{M}, \compose{\mathcal{M}}, \unit{\mathcal{M}}$),
\[
	\for{a, b, c, d \in \mathcal{M}} a \compose{\mathcal{M}} b = d \;\land\; c \leq b \implies 
	\exsts{f \in \mathcal{M}} a \compose{\mathcal{M}} c = f
\]
%
\begin{proof}
Pick an arbitrary $a, b, c, d \in \mathcal{M}$ such that:
%
\begin{align}
	a \compose{\mathcal{M}} b = d \label{L7:Ass1}\\
	c \leq b \label{L7:Ass2}
\end{align}
%
From (\ref{L7:Ass2}), we have: 
%
\begin{equation}
	\exsts{e \in \mathcal{M}} c \compose{\mathcal{M}} e = b \label{L7:Ass3}
\end{equation}
%
and consequently from (\ref{L7:Ass1}) we have:
%
\begin{equation}
	a \compose{\mathcal{M}} c \compose{\mathcal{M}} e = d \label{L7:Ass4}
\end{equation}
%
Since $e \leq d$ (\ref{L7:Ass4}), we have: 
%
\begin{equation}
	\exsts{f \in \mathcal{M}} e \compose{\mathcal{M}} f = d \label{L7:Ass5}
\end{equation} 
%
From (\ref{L7:Ass4}), (\ref{L7:Ass5}) and cancellativity of separation algebras we have:
%
\begin{equation}
	a \compose{\mathcal{M}} c = f
\end{equation}
%
and thus
%
\begin{equation}
	\exsts{f \in \mathcal{M}} a \compose{\mathcal{M}} c = f
\end{equation}
%
as required.
\end{proof}
\end{lemma}
%%
%\begin{lemma}[]\label{lem:fencedPostAction}
%For $l, l_1, c, f, r \in \LState$, $F \in \pset{\LState}$, $\ca{} \in \Caps$ and $\amod{} \in \AMods$:
%%
%\[
%\begin{array}{l}
%	l = l_1 \composeL c \;\land\\
%	l \in F \;\land\\
%	F \fences \amod{} \;\land\\
%	(l_1 \composeL f, l_2 \composeL f) \in  \amod{}(\ca{}) \;\land\\
%	l_1 \composeL f \leq l \composeL r
%\end{array}
%\implies
%l_2 \composeL c \in F
%\]
%%
%\begin{proof}
%Pick arbitrary $l, l_1, c, f, r \in \LState$, $F \in \pset{\LState}$, $\ca{} \in \Caps$ and $\amod{} \in \AMods$ such that
%%
%\begin{align}
%	l = l_1 \composeL c \label{L8:Ass1}\\
%	l \in F \label{L8:Ass2}\\
%	F \fences \amod{} \label{L8:Ass3}\\
%	(l_1 \composeL f, l_2 \composeL f) \in  \amod{}(\ca{}) \label{L8:Ass4}\\
%	l_1 \composeL f \leq l \composeL r \label{L8:Ass5}
%\end{align}
%%
%From (\ref{L8:Ass1}) and (\ref{L8:Ass5}) we have: $l_1 \composeL f \leq l_1 \composeL c \composeL r$ and consequently, $f \leq c \composeL r$. From Lemma \ref{lem:divideUpper} we can then deduce:
%%
%\begin{align}
%	\exsts{a, b \in \LState} \hspace*{1cm}&f = a \composeL b \label{L8:Ass6}\\
%	& a \leq c \label{L8:Ass7}\\
%	& b \leq r \label{L8:Ass8}
%\end{align}
%%
%From (\ref{L8:Ass6}) and (\ref{L8:Ass4}) we have:
%%
%\begin{equation}
%	(l_1 \composeL a \composeL b, l_2 \composeL a \composeL b) \in \amod{}(\ca{}) \label{L8:Ass9}
%\end{equation}
%%
%From (\ref{L8:Ass7}) we have:
%%
%\begin{equation}
%	\exsts{d \in \LState} \hspace*{0.2cm}  a \composeL d = c \label{L8:Ass10}
%\end{equation}
%%
%and consequently from (\ref{L8:Ass1}):
%%
%\begin{align}
%	l = l_1 \composeL a \composeL d \label{L8:Ass11}
%\end{align}
%%
%From (\ref{L8:Ass11}), (\ref{L8:Ass8}), Lemma \ref{lem:disjointByOrder} and since $l \composeL r$ is defined (\ref{L8:Ass5}), we have:
%%
%\begin{equation}
%	l_1 \composeL a \composeL d \composeL b \hspace*{0.5cm}\text{is defined} \label{L8:Ass12}
%\end{equation}
%%
%From (\ref{L8:Ass9}), (\ref{L8:Ass11}), (\ref{L8:Ass12}) and (\ref{L8:Ass3}) we have:
%%
%\begin{equation}
%	l_2 \composeL a \composeL d \in F \nonumber
%\end{equation}
%%
%and consequently from (\ref{L8:Ass10}):
%%
%\begin{equation}
%	l_2 \composeL c \in F \nonumber
%\end{equation}
%%
%as required.
%%
%\end{proof}
%\end{lemma}
%%
%
%
%
%
%
\begin{lemma}[]\label{lem:amodClosureImplication}
For all $l, r \in \LState$, $\amod{}, \amod{}' \in \AMods$ and $n \in \NatsPlus$
%
\[
	\extendsAMUpto{\amod{}}{n}{l}{r}{\amod{}'} \implies \extendsAMUpto{\amod{}}{(n-1)}{l}{r}{\amod{}'}
\]
%
\begin{proof} By induction on number of steps $n$.\\
Pick an arbitrary $l, r \in \LState$, $\amod{}, \amod{}' \in \AMods$ and $n \in \NatsPlus$.\\
\textbf{Base case: n = 1}\\
\textbf{RTS.} 
%
\begin{equation}
	\extendsAMUpto{\amod{}}{0}{l}{r}{\amod{}'} \nonumber
\end{equation}
%
This holds trivially by definition of $\amod{}\!\downarrow_{0}$\\

\noindent\textbf{Inductive case}\\
\textbf{Assume:}
%
\begin{align}
	\extendsAMUpto{\amod{}}{(n+1)}{l}{r}{\amod{}'} \label{L9:Ass1}\\
	\extendsAMUpto{\amod{}}{n}{l}{r}{\amod{}'} \implies \extendsAMUpto{\amod{}}{(n-1)}{l}{r}{\amod{}'} \tag{I.H} \label{L9:I.H}
\end{align}
%
\textbf{Show:}
%
\begin{align}
	&\begin{array}{l}
			\for{\ca{}, c, d, l_3, l_4} \for{(l_1 \composeL f, l_2 \composeL f) \in \amod{}'(\ca{})} \\
%			
			\hspace*{0.2cm}\left(\for{l'} l' \leq l_1 \;\land\; l' \leq l_2 \implies l' = \unitL \right) \;\land\; l_1 \composeL f \leq l \composeL r \implies\\
%			
			\hspace*{0.5cm} l_1 = l_3 \composeL l_4 \;\land\; 	l_1 \maxMeetL l = l_3 \;\land\; l = l_3 \composeL c \;\land\; r = l_4 \composeL d \implies \\
%			
			\hspace*{0.8cm} \left((l_1 \composeL c \composeL d, l_2 \composeL c \composeL d) \in \amod{}(\ca{}) 
			\;\land\;
			\extendsAMUpto{\amod}{(n-1)}{l_2 \composeL c}{d}{\amod{}'}\right) \lor l_2 \composeL c \composeL d \text{ is undefined} 	
	\end{array}\label{L9:Goal1}\\
%
  &\for{\ca{}} \for{(l_1, l'_1) \in \amod{}(\ca{})} l_1 = l \composeL r \implies \nonumber\\
	&\exsts{l_0, l'_0, f, f_0} \nonumber\\
  &\hspace*{0.5cm}
  \begin{array}{l}
  	\left(\for{l'} l' \leq l_0 \land l' \leq l'_0 \implies l' = \unitL \right)\;\land\\
  	l_1 = l_0 \composeL f_0 \composeL f \;\land\; l'_1 = l'_0 \composeL f_0 \composeL f \;\land\\
  	(l_0 \composeL f,\ l'_0 \composeL f) \in \amod{}'(\ca{}) 
  \end{array} \nonumber\\
%
	&\hspace*{2cm}\lor  \nonumber\\
	&\exsts{r'} l'_1 = l \composeL r' \;\land\; \extendsAMUpto{\amod}{(n-1)}{l}{r'}{\amod{}'} \label{L9:Goal2}
\end{align}
%

\noindent\textbf{RTS. (\ref{L9:Goal1})}\\
Pick an arbitrary $\ca{}, c, d, l_1, l_2, l_3, l_4, f$ such that
%
\begin{equation}
\hspace*{-0.15cm}
\begin{array}{l}
	(l_1 \composeL f, l_2 \composeL f) \in \amod{}'(\ca{}) \;\land\; 
	\left(\for{l'} l' \leq l_1 \land l' \leq l_2 \implies l' = \unitL \right)
	\;\land\; l_1 \composeL f \leq l \composeL r\\
%
	\land\; l_1 = l_3 \composeL l_4 \;\land\; 	l_1 \maxMeetL l = l_3 \;\land\; l = l_3 \composeL c \;\land\; r = l_4 \composeL d
\end{array} \label{L9:Ass2}
\end{equation}
%
From (\ref{L9:Ass1}) and (\ref{L9:Ass2}) we have:
%
\begin{align}
	&\hspace*{1cm} \left((l_1 \composeL c \composeL d, l_2 \composeL c \composeL d) \in \amod{}(\ca{}) \land
	\extendsAMUpto{\amod}{n}{l_2 \composeL c}{d}{\amod{}'}\right) \lor l_2 \composeL c \composeL d \text{ is undefined} \nonumber
\end{align}
%
From (\ref{L9:I.H}) we can rewrite the above as 
%
\begin{align}
	&\hspace*{1cm} \left((l_1 \composeL c \composeL d, l_2 \composeL c \composeL d) \in \amod{}(\ca{}) \land
	\extendsAMUpto{\amod}{(n-1)}{l_2 \composeL c}{d}{\amod{}'}\right) \lor l_2 \composeL c \composeL d \text{ is undefined} \label{L9:Ass3}
\end{align}
%
as required.\\

\noindent\textbf{RTS. (\ref{L9:Goal2})}\\
Pick an arbitrary $\ca{}, l_1, l'_1$ such that
%
\begin{equation}
	(l_1, l'_1) \in \amod{}(\ca{}) \;\land\;  l_1 = l \composeL r   \label{L9:Ass4}
\end{equation}
%
From (\ref{L9:Ass1}) and (\ref{L9:Ass4}) we have:
%
\begin{align}
	&\for{\ca{}} \for{(l_1, l'_1) \in \amod{}(\ca{})} l_1 = l \composeL r \implies \nonumber\\
	&\exsts{l_0, l'_0, f, f_0} \nonumber\\
  &\hspace*{0.5cm}
  \begin{array}{l}
  	\left(\for{l'} l' \leq l_0 \land l' \leq l'_0 \implies l' = \unitL \right)\;\land\\
  	l_1 = l_0 \composeL f_0 \composeL f \;\land\; l'_1 = l'_0 \composeL f_0 \composeL f \;\land\\
  	(l_0 \composeL f,\ l'_0 \composeL f) \in \amod{}'(\ca{}) 
  \end{array} \nonumber\\
%
	&\hspace*{2cm}\lor  \nonumber\\
	&\exsts{r'} l'_1 = l \composeL r' \;\land\; \extendsAMUpto{\amod}{n}{l}{r'}{\amod{}'} \label{L9:Ass5}
\end{align}
%
From (\ref{L9:I.H}) we can rewrite (\ref{L9:Ass5}) as
%
\begin{align}
	&\for{\ca{}} \for{(l_1, l'_1) \in \amod{}(\ca{})} l_1 = l \composeL r \implies \nonumber\\
	&\exsts{l_0, l'_0, f, f_0} \nonumber\\
  &\hspace*{0.5cm}
  \begin{array}{l}
  	\left(\for{l'} l' \leq l_0 \land l' \leq l'_0 \implies l' = \unitL \right)\;\land\\
  	l_1 = l_0 \composeL f_0 \composeL f \;\land\; l'_1 = l'_0 \composeL f_0 \composeL f  \;\land\\
  	(l_0 \composeL f,\ l'_0 \composeL f) \in \amod{}'(\ca{}) 
  \end{array} \nonumber\\
%
	&\hspace*{2cm}\lor  \nonumber\\
	&\exsts{r'} l'_1 = l \composeL r' \;\land\; \extendsAMUpto{\amod}{(n-1)}{l}{r'}{\amod{}'} \label{L9:Ass6}
\end{align}
%
as required.
\end{proof}
\end{lemma}
%
%
%\begin{lemma}[]\label{lem:crossSplit}
%Given any separation algebra ($\mathcal{M}, \compose{\mathcal{M}}, \unit{\mathcal{M}}$) with the cross-split property,
%\[
%\begin{array}{l}
%	\for{l_1, l_2, l_3, l_4, f, g\in \mathcal{M}} l_1 \compose{\mathcal{M}} f = l_3 \compose{\mathcal{M}} g \;\land\; l_2 \compose{\mathcal{M}} f = l_4 \compose{\mathcal{M}} g  \implies \\
%	\hspace*{1cm}\exsts{a, b, d, e \in \mathcal{M}} l_1 = a \compose{\mathcal{M}} d \;\land\; l_2 = b  \compose{\mathcal{M}} d \;\land\; l_3 = q  \compose{\mathcal{M}} e \;\land\; l_4 = b  \compose{\mathcal{M}} e
%\end{array}
%\]
%%
%\begin{proof}
%
%\end{proof}
%\end{lemma}
\begin{lemma}[]\label{lem:noMeetByOrder}
Given any separation algebra ($\mathcal{M}, \compose{\mathcal{M}}, \unit{\mathcal{M}}$) with the cross-split property,
\[
\begin{array}{l}
	\for{a, b, c, d \in \mathcal{M}} a \meet{\mathcal{M}} b = \emptyset \;\land\; a \leq d \;\land\; b \leq c \implies \\
	\hspace*{1cm} c \meet{\mathcal{M}} d = \emptyset
\end{array}
\]
%
\begin{proof}
\todo
\end{proof}
\end{lemma}
%
%
\newpage	
\begin{lemma}[]\label{lem:amodWitness}
%
\[
\begin{array}{l}
	\for{\amod{}, \amod{0}, \amod{L} \in \AMods} \for{F_0, F \in \pset{\LStates}} \for{s, s_0}\\
	\hspace*{0.2cm} s_0 \in F_0 \;\land\; F_0 \strictfences \amod{0} \;\land\; s \in F \;\land\; F \strictfences \amod{L} 
%	\;\land\;  \dom{\amod{L}} \disjoint \dom{\amod{0}} 
	\implies \\
	\hspace*{0.5cm} \exsts{\amod{}', \fence{}'} s \composeL s_0 \in \fence{}' \;\land\; \fence{}' \strictfences \amod{L} \cup \amod{0} \;\land\; 
	 \extendsAM{\amod{}', \amod{L} \cup \amod{0}}{s_0}{s}{\amod{0}} \;\land\\
	 \hspace*{1.8cm}\for{l, r}\for{\amod{1} \subseteq \amod{L} } l \composeL r = s \;\land\; \extendsAM{\amod{}, \amod{L}}{l}{r}{\amod{1}} \implies\\
	 \hspace*{2cm}  \extendsAM{\amod{}', \amod{L} \cup \amod{0}}{l}{r \composeL s_0}{\amod{1}}

\end{array}
\]
%
\begin{proof}
Pick arbitrary $\amod{}$, $\amod{0}, \amod{L}, \amod{1} \subseteq \amod{L}, F_0, F, L, s_0, s$ such that 
%
\begin{align}
		&s_0 \in F_0 \;\land\; F_0 \strictfences \amod{0} \label{LAW:AssFence}\\
%		& \dom{\amod{L}} \disjoint \dom{\amod{0}} \;\land\; 
		&F \strictfences \amod{L}\label{LAW:dd}\\
		& s \in F \label{LAW:sfence}
\end{align}
%
Let
%
\[
	F' \eqdef
	\left\{
		f_0 \composeL f \;|\;
		f_0 \in F_0 \;\land\;  f \in F
	\right\}
\]
%
and pick  $\amod{}', \amod{0}'$ such that for each $\ca{} \in \Caps$:
%
\[
\amod{1}'(\ca{}) =
\left\{
	\begin{array}{l | l}
		(s_1 \composeL d \composeL f_0, s_2 \composeL d \composeL f_0)
		&
		\begin{array}{l}
			\hspace*{-0.2cm} \exsts{f \in F} \exsts{f_0 \in F_0} \exsts{f' \in F'}\\
			f' = f_0 \composeL  f\\
			(s_1 \composeL c, s_2 \composeL c) \in \amod{L}(\ca{})\\
			\for{s'} s' \leq s_1 \;\land\; s' \leq s_2 \implies s' = \unitL\\
			s_1 \composeL d = f \;\land\; s_1 \composeL c \leq f'\\
			s_2 \composeL d \composeL f_0 \hspace*{0.2cm} \text{is defined}
		\end{array}
	\end{array}
\right\}
\]
%
and
%
\[
\amod{0}'(\ca{}) =
\left\{
	\begin{array}{l | l}
		(s_1 \composeL d \composeL f, s_2 \composeL d \composeL f)
		&
		\begin{array}{l}
			\hspace*{-0.2cm} \exsts{f \in F} \exsts{f_0 \in F_0} \exsts{f' \in F'}\\
			f' = f_0 \composeL  f\\
			(s_1 \composeL c, s_2 \composeL c) \in \amod{0}(\ca{})\\
			\for{s'} s' \leq s_1 \;\land\; s' \leq s_2 \implies s' = \unitL\\
			s_1 \composeL d = f_0 \;\land\; s_1 \composeL c \leq f'\\
			s_2 \composeL d \composeL f \hspace*{0.2cm} \text{is defined}
		\end{array}
	\end{array}
\right\}
\]
%
Let 
%
\begin{align}
	\amod{}' = \amod{0}' \cup \amod{1}' \label{LAW:Ass1}
\end{align}
%(\ref{LAW:Ass})
\noindent\textbf{Show}
\begin{align}
	&F' \strictfences \amod{L} \cup \amod{0} \label{LAW:Goal1}\\
	&s \composeL s_0 \in F' \label{LAW:Goal2}\\
	&\for{n \in \Nats}\for{s \in F} \for{s_0 \in F_0}  s \composeL s_0 \text{ defined} \implies 
	\extendsAMUpto{\amod{}', \amod{L} \cup \amod{0}}{n}{s_0}{s}{\amod{0}} \label{LAW:Goal4}\\
	&\for{n \in \Nats}\for{l \composeL r \in F} \for{s_0 \in F_0} l \composeL r \composeL s_0 \text{ defined} \;\land\; \extendsAM{\amod{}, \amod{L}}{l}{r}{\amod{1}} \nonumber\\
	& \implies   \extendsAMUpto{\amod{}', \amod{L} \cup \amod{0}}{n}{l}{r \composeL s_0}{\amod{0}} \label{LAW:Goal5}
\end{align}
%
\textbf{RTS. (\ref{LAW:Goal4})}\\
\begin{proof}
By induction on the number of steps $m$.\\

\noindent\textbf{Base Case}\\
\textbf{RTS. } $\extendsAMUpto{\amod{}', \amod{L} \cup \amod{0}}{0}{s_0}{s}{\amod{0}}$\\
This holds trivially by definition of $-\downarrow_{0}$.\\

\noindent\textbf{Inductive Case}\\
\textbf{Assume}
\begin{align}
%	\begin{array}{l }
%		\exsts{s} l_1 \composeL  \cdots \composeL l_n = s \\
%	  \land\; \for{\amod{j}, \amod{k} \in \left(\amod{1} \cup \cdots \cup \amod{n}\right)} \dom{\amod{j}} \disjoint \dom{\amod{k}} \\
%	  \land\; \for{j \in 1, \cdots, n} l_j \in F_j  \;\land\; F_i \strictfences \amod{i} \\
%	  \hspace*{3cm} \implies\\
%	  \extendsAMUpto{\amod{}}{m-1}{l_i}{l_1 \composeL \cdots \composeL l_{i-1} \composeL l_{i+1} \composeL \cdots \composeL l_n}{\amod{i}}
%	\end{array}
	\for{s \in F}\for{s_0 \in F_0} s \composeL s_0 \text{ is defined} \implies \extendsAMUpto{\amod{}', \amod{L} \cup \amod{0}}{(m-1)}{s_0}{s}{\amod{0}}
	\tag{I.H} \label{LAW:I.H}
\end{align}
\textbf{Show } 
%
\begin{align}
	\extendsAMUpto{\amod{}', \amod{L} \cup \amod{0}}{m}{s_0}{s}{\amod{0}} \nonumber
\end{align}
%
That is:
%
\begin{align}
	&\begin{array}{l}
		\for{\ca{}, c, d, t, s_3, s_4}\for{(s_1 \composeL e, s_2 \composeL e) \in \amod{0}(\ca{})} \\
	\hspace*{0.2cm}\left(\for{l'} l' \leq s_1 \land l' \leq s_2 \implies l' = \unitL\right) \;\land\; s_1 \composeL e \leq  s \composeL s_0 \composeL t \;\land\\
	\hspace*{0.2cm} s_1 = s_3 \composeL s_4 \;\land\; s_1 \maxMeetL s_0 = s_3 \;\land\; s_0 = s_3 \composeL c \;\land\; s = s_4 \composeL d \implies\\
	\hspace*{0.4cm}\extendsAMUpto{\amod{}', \amod{L} \cup \amod{0}}{(m-1)}{s_2 \composeL c}{d}{\amod{0}} \;\land\\
	\hspace*{0.4cm} t = \unitL \implies (s_1 \composeL c \composeL d, s_2 \composeL c \composeL d) \in \amod{}'(\ca{}) \;\lor\; s_2 \composeL c \composeL d \text{ is undefined}
	\end{array} \label{LAW:Goal8} \\\nonumber\\
	&\begin{array}{l}
		\for{\ca{}} \for{(l_1 \composeL f, l_2 \composeL f) \in \amod{L} \cup \amod{0}(\ca{})} \for{c, d} \\
  \hspace*{0.2cm} l_1 \composeL f \composeL c = s \composeL s_0 \composeL d \;\land\; \left(\for{l'} l' \leq l_0 \land l' \leq l'_0 \implies l' = \unitL\right)  \;\implies\\
  \hspace*{0.4cm}\exsts{f', c'} (l_1 \composeL f', l_2 \composeL f') \in \amod{0}(\ca{}) \;\land\; l_1 \composeL f' \composeL c' =  s \composeL s_0 \composeL d\\
		\hspace*{0.4cm}\lor\;l_2 \composeL f \composeL c \text{ is undefined}\\
		\hspace*{0.4cm}\lor\; l_1 \maxMeetL s_0 = \unitL \;\land\; \exsts{r'} l_2 \composeL f \composeL c = s_0 \composeL r' \composeL d \;\land\\
		\hspace*{0.6cm} \extendsAMUpto{\amod{}', \amod{L} \cup \amod{0}}{(m-1)}{s_0}{r'}{\amod{0}}
%		
%  \hspace*{0.2cm} l_1 \composeL f \meetL l \composeL r \not= \emptyset \;\land\; \left(\for{l'} l' \leq l_0 \land l' \leq l'_0 \implies l' = \unitL\right)  \;\implies\\
%  \hspace*{0.4cm}\exsts{f'} (l_1 \composeL f', l_2 \composeL f') \in \amod{0} \;\land\; l_1 \composeL f' \meetL l \composeL r \not= \emptyset\\
%		\hspace*{0.4cm}\lor\;\for{c, d}  l_1 \composeL f \composeL c = l \composeL r \composeL d \implies\\
%		\hspace*{1cm}
%		\begin{array}{l l}
%			l_2 \composeL f \composeL c \text{ is undefined}\\
%			\lor\; \exsts{r'} l_2 \composeL f \composeL c = l \composeL r' \composeL d \;\land\; \left( d = \unitL \implies \extendsAMUpto{\amod{}, L \cup \{\amod{0}\}}{(m-1)}{l}{r'}{\amod{0}}\right)
%		\end{array}
	\end{array} \label{LAW:Goal9}
\end{align}
%

\noindent\textbf{RTS. (\ref{LAW:Goal8})} \\
Pick an arbitrary $\amod{L}, \ca{}, l_1, l_2, c, d, t, s_3, s_4, s_1, s_2, e$ such that:
%
\begin{align}
	& (s_1 \composeL e, s_2 \composeL e) \in \amod{0}(\ca{}) \label{LAW:Ass5} \\
	& \left(\for{l'} l' \leq s_1 \land l' \leq s_2 \implies l' = \unitL\right) \label{LAW:Ass6}\\
	& s_1 \composeL e \leq  s_0 \composeL s \composeL t\label{LAW:Ass7}\\
	& s_1 = s_3 \composeL s_4 \label{LAW:Ass8}\\
	& s_1 \maxMeetL s_0 = s_3 \label{LAW:Ass9}\\
	& s_0 = s_3 \composeL c \label{LAW:Ass10}\\
	& s= s_4 \composeL d \label{LAW:Ass11}
\end{align}
%(\ref{LAW:Ass})
From (\ref{LAW:AssFence}) we know that $s_1 \leq s_0$ and consequently from (\ref{LAW:Ass8})-(\ref{LAW:Ass11}) and definition of $\maxMeetL$ we have:
%
\begin{align}
\begin{array}{l}
	s_3 = s_1 \\
	s_4 = \unitL\\
	s = d
\end{array} \label{LAW:Ass12}
\end{align}
%(\ref{LAW:Ass})
Assume $t = \unitL$, then from (\ref{LAW:Ass1}), (\ref{LAW:AssFence}), (\ref{LAW:Ass5})-(\ref{LAW:Ass7}), (\ref{LAW:Ass10}) and (\ref{LAW:Ass12}) we have:
%
\begin{align}
	(s_1 \composeL c \composeL d, s_2 \composeL c \composeL d) \in \amod{}'(\ca{}) \;\lor\; s_2 \composeL c \composeL d \hspace*{0.2cm} \text{is undefined} \label{LAW:Ass13}
\end{align}
%
On the other hand, from (\ref{LAW:AssFence}), (\ref{LAW:Ass5}), (\ref{LAW:Ass8}), (\ref{LAW:Ass10}), (\ref{LAW:Ass12}) and definition of $\strictfences$ we have:
%
\begin{align}
	s_2 \composeL c \in F_0 \label{LAW:Ass14}
\end{align}
%(\ref{LAW:Ass})
%
From (\ref{LAW:AssFence}), (\ref{LAW:Ass12})-(\ref{LAW:Ass14}) and (\ref{LAW:I.H}) we have:
%
%
\begin{align}
\begin{array}{l}
	\extendsAMUpto{\amod{}', \amod{L} \cup \amod{0}}{(m-1)}{s_2 \composeL c}{d}{\amod{0}} \;\land\\
	t = \unitL \implies (s_1 \composeL c \composeL d, s_2 \composeL c \composeL d) \in \amod{}'(\ca{}) \;\lor\; s_2 \composeL c \composeL d \hspace*{0.2cm} \text{is undefined}
\end{array} \nonumber
\end{align}
%(\ref{LAW:Ass})
as required.\\

\noindent\textbf{RTS. (\ref{LAW:Goal9})}\\
Pick an arbitrary $\ca{}, l_1, l_2, f, c, d$ such that:
%
\begin{align}
	& (l_1 \composeL f, l_2 \composeL f) \in \amod{L} \cup \amod{0}(\ca{}) \label{LAW:Ass16}\\
	& l_1 \composeL f \composeL c = s \composeL s_0 \composeL d \label{LAW:Ass17}\\
	& \for{l'} l' \leq l_1 \;\land\; l' \leq l_2 \implies l' = \unitL
\end{align}
%
If $\ca{} \in \dom{\amod{0}}$ then the desired result holds trivially. On the other hand if $\ca{} \in \dom{\amod{L}}$, then from (\ref{LAW:AssFence}) and ((\ref{LAW:sfence}) we know $l_1 \leq s \;\land\; l_1 \maxMeetL s_0 = \unitL$, that is 
%
\begin{align}
	& l_1 \maxMeetL s_0 = \unitL \;\land\; \exsts{e} l_1 \composeL e = s \label{LAW:contained}
\end{align}
%
and thus
%
\begin{align}
	l_1 \composeL f \composeL c = s_0 \composeL l_1 \composeL e \composeL d \nonumber
\end{align}
%
and consequently by cancellativity of separation algebras:
%
\begin{align}
	f \composeL c = s_0 \composeL e \composeL d \label{LAW:Part1}
\end{align}
%
From (\ref{LAW:Part1}) and (\ref{LAW:contained}) we have:
%
\begin{align}
	l_1 \maxMeetL s_0 = \unitL \;\land\;  l_2 \composeL f \composeL c = l_2 \composeL s_0 \composeL e \composeL d \;\lor\; l_2 \composeL f \composeL c \text{ is undefined} \nonumber
\end{align}
%
Pick $r' = l_2 \composeL e$
%
\begin{align}
	l_2 \composeL f \composeL c \text{ is undefined} \;\lor\; l_1 \maxMeetL s_0 = \unitL \;\land\; \exsts{r'} l_2 \composeL f \composeL c = s_0 \composeL r' \composeL d \label{LAW:Part2}
\end{align}
%
Now assume $l_2 \composeL f \composeL c$ is defined, then from (\ref{LAW:contained}), (\ref{LAW:dd}) and by definition of $\strictfences$ we have:
%
\begin{align}
	l_2 \composeL e \in F \label{LAW:fenced}
\end{align}
%
and consequently by the (\ref{LAW:I.H})
%
\begin{align}
	\extendsAMUpto{\amod{}', \amod{L} \cup \amod{0}}{(m-1)}{s_0}{r'}{\amod{0}}
\end{align}
%(\ref{LAW:Ass})
as required.
\renewcommand{\qed}{}
\end{proof}

%
\noindent\textbf{RTS. (\ref{LAW:Goal5})}\\
\begin{proof}
By induction on the number of steps $m$. Pick an arbitrary $l, r, s_0, \amod{1} \subseteq \amod{L}$ such that:
%
\begin{align}
	&l \composeL r \in F \label{LAW2:Ass1}\\
	& s_0 \in F_0 \label{LAW2:Ass2}\\
	& l \composeL r \composeL s_0 \text{ is defined} \label{LAW2:Ass3} \\
	& \extendsAM{\amod{}, \amod{L}}{l}{r}{\amod{1}} \label{LAW2:Ass4}
\end{align}

\noindent\textbf{Base Case}\\
\textbf{RTS. } $\extendsAMUpto{\amod{}', \amod{L} \cup \amod{0}}{0}{s}{s_0}{\amod{L}}$\\
This holds trivially by definition of $-\downarrow_{0}$.\\

\noindent\textbf{Inductive Case}\\

\noindent\textbf{Assume}
\begin{align}
	\for{l \composeL r \in F}\for{s_0 \in F_0} l \composeL r \composeL s_0  \text{ is defined}  \;\land\; \extendsAM{\amod{}, \amod{L}}{l}{r}{\amod{1}} \nonumber \\
	\implies \extendsAMUpto{\amod{}', \amod{L} \cup \amod{0}}{(m-1)}{l}{r \composeL s_0}{\amod{1}}
	\tag{I.H} \label{LAW2:I.H}
\end{align}
\textbf{Show } 
%
\begin{align}
	\extendsAMUpto{\amod{}', \amod{L} \cup \amod{0}}{m}{l}{r \composeL s_0}{\amod{1}} \nonumber
\end{align}
%
That is:
%
\begin{align}
	&\begin{array}{l}
		\for{\ca{}, c, d, t, s_3, s_4}\for{(s_1 \composeL e, s_2 \composeL e) \in \amod{1}(\ca{})} \\
	\hspace*{0.2cm}\left(\for{l'} l' \leq s_1 \land l' \leq s_2 \implies l' = \unitL\right) \;\land\; s_1 \composeL e \leq  l \composeL r \composeL s_0 \composeL t\;\land\\
	\hspace*{0.2cm} s_1 = s_3 \composeL s_4 \;\land\; s_1 \maxMeetL l = s_3 \;\land\; l = s_3 \composeL c \;\land\; r \composeL s_0 = s_4 \composeL d \implies\\
	\hspace*{0.4cm} \extendsAMUpto{\amod{}', \amod{L} \cup \amod{0}}{(m-1)}{s_2 \composeL c}{d}{\amod{1}} \;\land\\
	\hspace*{0.4cm} t = \unitL \implies (s_1 \composeL c \composeL d, s_2 \composeL c \composeL d) \in \amod{}'(\ca{}) 
	\;\lor\; s_2 \composeL c \composeL d \text{ is undefined}
	\end{array} \label{LAW:Goal10} \\\nonumber\\
	&\begin{array}{l}
		\for{\ca{}} \for{(l_1 \composeL f, l_2 \composeL f) \in \amod{L} \cup \amod{0}(\ca{})} \for{c, d} \\
		\hspace*{0.2cm} l_1 \composeL f \composeL c = l \composeL r \composeL s_0 \composeL d \;\land\; \left(\for{l'} l' \leq l_0 \land l' \leq l'_0 \implies l' = \unitL\right)  \;\implies\\
  \hspace*{0.4cm}\exsts{f', c'} (l_1 \composeL f', l_2 \composeL f') \in \amod{1}(\ca{}) \;\land\; l_1 \composeL f' \composeL c' =  l \composeL r \composeL s_0 \composeL d\\
		\hspace*{0.4cm}\lor\;l_2 \composeL f \composeL c \text{ is undefined}\\
		\hspace*{0.4cm}\lor\; l_1 \maxMeetL l = \unitL \;\land\; \exsts{r'} l_2 \composeL f \composeL c = l \composeL r' \composeL d \;\land\\
		\hspace*{0.6cm} \extendsAMUpto{\amod{}', \amod{L} \cup \amod{0}}{(m-1)}{l}{r'}{\amod{1}}
	\end{array} \label{LAW:Goal11}
\end{align}
%

\noindent\textbf{RTS. (\ref{LAW:Goal10})} \\
Pick an arbitrary $\ca{}, c, d, t, s_3, s_4, s_1, s_2, e$ such that:
%
\begin{align}
	& (s_1 \composeL e, s_2 \composeL e) \in \amod{1}(\ca{}) \label{LAW2:Ass5} \\
	& \left(\for{l'} l' \leq s_1 \land l' \leq s_2 \implies l' = \unitL\right) \label{LAW:Ass6}\\
	& s_1 \composeL e \leq  s_0 \composeL l \composeL r \composeL t\label{LAW2:Ass7}\\
	& s_1 = s_3 \composeL s_4 \label{LAW2:Ass8}\\
	& s_1 \maxMeetL l = s_3 \label{LAW2:Ass9}\\
	& l = s_3 \composeL c \label{LAW2:Ass10}\\
	& r \composeL s_0 = s_4 \composeL d \label{LAW2:Ass11}
\end{align}
%(\ref{LAW:Ass})
From (\ref{LAW:dd}) we know that $s_1 \leq l \composeL r$ and consequently from (\ref{LAW:Ass8})-(\ref{LAW:Ass11}) and definition of $\maxMeetL$ we have:
%
\begin{align}
\begin{array}{l l}
	\exsts{g} & d = s_0 \composeL g \\
	& r = s_4 \composeL g
\end{array} \label{LAW2:Ass12}
\end{align}
%(\ref{LAW:Ass})
From (\ref{LAW2:Ass5})-(\ref{LAW2:Ass12}) and (\ref{LAW2:Ass4}) we have:
%
\begin{align}
	&\extendsAM{\amod{}, \amod{L}}{s_2 \composeL c}{g}{\amod{1}} \;\land \nonumber\\
	&t = \unitL \implies (s_1 \composeL c \composeL g, s_2 \composeL c \composeL g) \in \amod{}(\ca{}) \;\lor\; (s_2 \composeL c \composeL g) \text{ is undefined} \label{LAW2:Assmid}
\end{align}
%(\ref{LAW:Ass})
From the definition of $\strictfences$, (\ref{LAW:dd}), (\ref{LAW:Ass5}) and (\ref{LAW:Ass1}) we have:
%
\begin{align}
	s_2 \composeL c \composeL g \in F \label{LAW2:Assfencedpost}
\end{align}
%(\ref{LAW:Ass})
Assume $t = \unitL$, then from (\ref{LAW:Ass1}), (\ref{LAW:dd}), (\ref{LAW2:Ass5})-(\ref{LAW2:Ass7}), (\ref{LAW2:Ass10}) and (\ref{LAW2:Ass12}) we have:
%
\begin{align}
	(s_1 \composeL c \composeL d, s_2 \composeL c \composeL d) \in \amod{}'(\ca{}) \;\lor\; s_2 \composeL c \composeL d \hspace*{0.2cm} \text{is undefined} \label{LAW2:Ass13}
\end{align}
%
%(\ref{LAW:Ass})
%
From (\ref{LAW:dd}), (\ref{LAW2:Ass12}), (\ref{LAW2:Assmid}), (\ref{LAW2:Assfencedpost}), (\ref{LAW2:Ass13}) and (\ref{LAW2:I.H}) we have:
%
%
\begin{align}
\begin{array}{l}
	\extendsAMUpto{\amod{}', \amod{L} \cup \amod{0}}{(m-1)}{s_2 \composeL c}{d}{\amod{L}} \;\land \\
	t = \unitL \implies (s_1 \composeL c \composeL d, s_2 \composeL c \composeL d) \in \amod{}'(\ca{}) \;\lor\; s_2 \composeL c \composeL d \hspace*{0.2cm} \text{is undefined}
\end{array} \nonumber
\end{align}
%
as required.\\

\noindent\textbf{RTS. (\ref{LAW:Goal11})}\\
Pick an arbitrary $\ca{}, l_1, l_2, f, c, d$ such that:
%
\begin{align}
	& (l_1 \composeL f, l_2 \composeL f) \in \amod{L} \cup \amod{0}(\ca{}) \label{LAW2:Ass16}\\
	& l_1 \composeL f \composeL c = l \composeL r \composeL s_0 \composeL d \label{LAW2:Ass17}\\
	& \for{l'} l' \leq l_1 \;\land\; l' \leq l_2 \implies l' = \unitL \label{LAW2:Ass18}
\end{align}
%
There are two cases to consider:\\

\noindent\textbf{Case 1. }$\ca{} \in \dom{\amod{0}}$\\
From (\ref{LAW:AssFence}) and (\ref{LAW:sfence}) we know $l_1 \leq s_0 \;\land\; l_1 \maxMeetL l \composeL r = \unitL$, that is 
%
\begin{align}
	& l_1 \maxMeetL l = \unitL \;\land\; \exsts{e} l_1 \composeL e = s_0 \label{LAW2:contained}
\end{align}
%
and thus
%
\begin{align}
	l_1 \maxMeetL l = \unitL \;\land\;  l_1 \composeL f \composeL c = l \composeL  r \composeL l_1 \composeL e \composeL d \nonumber
\end{align}
%
and consequently by cancellativity of separation algebras:
%
\begin{align}
	f \composeL c = l \composeL r \composeL e \composeL d \label{LAW2:Part1}
\end{align}
%
From (\ref{LAW2:Part1}) and (\ref{LAW2:contained}) we have:
%
\begin{align}
	l_1 \maxMeetL l = \unitL \;\land\;  l_2 \composeL f \composeL c = l_2 \composeL l \composeL r \composeL e \composeL d \;\lor\; l_2 \composeL f \composeL c \text{ is undefined} \label{LAW2:Part2}
\end{align}
%
Now assume $l_2 \composeL f \composeL c$ is defined, then from (\ref{LAW2:contained}), (\ref{LAW:AssFence}), definition of $\strictfences$ and since $(l_1 \composeL f, l_2 \composeL f) \in \amod{0}(\ca{})$ we have:
%
\begin{align}
	l_2 \composeL e \in F_0 \label{LAW2:fenced}
\end{align}
%
and consequently from (\ref{LAW2:Ass1}), (\ref{LAW2:Ass4}) and (\ref{LAW2:I.H})
%
\begin{align}
	\extendsAMUpto{\amod{}', \amod{L} \cup \amod{0}}{(m-1)}{l}{r \composeL l_2 \composeL e}{\amod{1}}
\end{align}
%(\ref{LAW2:Ass})
Pick $r' = l_2 \composeL e \composeL r$. Then above, (\ref{LAW2:Part2})we have:
%
\begin{align}
\begin{array}{l l}
	l_2 \composeL f \composeL c \text{ is undefined} \;\lor & l_1 \maxMeetL l = \unitL \;\land\; \exsts{r'} l_2 \composeL f \composeL c = l \composeL r' \composeL d\\
	& \hspace*{0.1cm}\land\; \extendsAMUpto{\amod{}', \amod{L} \cup \amod{0}}{(m-1)}{l}{r'}{\amod{1}}
\end{array} \nonumber
\end{align}
%
as required.\\

\noindent\textbf{Case 2. }$\ca{} \in \dom{\amod{L}}$\\
From (\ref{LAW2:Ass16})-(\ref{LAW2:Ass18}), assumption of case 2 and (\ref{LAW2:Ass4}) we have:
%
\begin{align}
	\begin{array}{l}
  	\exsts{f', c'} (l_1 \composeL f', l_2 \composeL f') \in \amod{1}(\ca{}) \;\land\; l_1 \composeL f' \composeL c' =  l \composeL r \composeL s_0 \composeL d\\
		\lor\;l_2 \composeL f \composeL c \text{ is undefined}\\
		\lor\; l_1 \maxMeetL l = \unitL \;\land\; \exsts{r'} l_2 \composeL f \composeL c = l \composeL r' \composeL s_0 \composeL d \;\land\; \extendsAMUpto{\amod{}, \amod{L}}{(m-1)}{l}{r'}{\amod{1}}
	\end{array} \label{LAW2:Ass19}
\end{align}
%
From (\ref{LAW:dd}) we know $l_1 \leq l \composeL r$, in the case of the third disjunct above where $l_1 \maxMeetL l = \unitL$, we thus know $l_1 \leq r$, that is:
%(\ref{LAW2:Ass})
\begin{align}
	\exsts{g} l_1 \composeL g = r \label{LAW2:Ass20}
\end{align}
%(\ref{LAW2:Ass})
From (\ref{LAW:dd}), definition of $\strictfences$ and assumption of case 2 we have:
%
\begin{align}
	l \composeL l_2 \composeL g \in F \label{LAW2:Ass21}
\end{align}
%
On the other hand from (\ref{LAW2:Ass17}) and (\ref{LAW2:Ass20}) we know $l_1 \composeL f \composeL c = l \composeL l_1 \composeL g \composeL s_0 \composeL d$, and consequently by the cancellativity of separation algebras we know $f \composeL c = l \composeL g \composeL s_0 \composeL d$, thus we can rewrite (\ref{LAW2:Ass19}) as:
%
\begin{align}
	\begin{array}{l}
  	\exsts{f', c'} (l_1 \composeL f', l_2 \composeL f') \in \amod{1}(\ca{}) \;\land\; l_1 \composeL f' \composeL c' =  l \composeL r \composeL s_0 \composeL d\\
		\lor\;l_2 \composeL f \composeL c \text{ is undefined}\\
		\lor\; l_1 \maxMeetL l = \unitL \;\land\; l_2 \composeL f \composeL c = l \composeL l_2 \composeL g \composeL s_0 \composeL d \;\land\; \extendsAMUpto{\amod{}, \amod{L}}{(m-1)}{l}{l_2 \composeL g}{\amod{1}}
	\end{array} \label{LAW2:Ass22}
\end{align}
%(\ref{LAW2:Ass})
From (\ref{LAW2:Ass2}), (\ref{LAW2:Ass4}), (\ref{LAW2:Ass21}), (\ref{LAW2:I.H}) we can rewrite (\ref{LAW2:Ass22}) into
%
\begin{align}
	\begin{array}{l}
  	\exsts{f', c'} (l_1 \composeL f', l_2 \composeL f') \in \amod{1}(\ca{}) \;\land\; l_1 \composeL f' \composeL c' =  l \composeL r \composeL s_0 \composeL d\\
		\lor\;l_2 \composeL f \composeL c \text{ is undefined}\\
		\lor\; l_1 \maxMeetL l = \unitL \;\land\; l_2 \composeL f \composeL c = l \composeL l_2 \composeL g \composeL s_0 \composeL d \;\land\; \extendsAMUpto{\amod{}', \amod{L} \cup \amod{0}}{(m-1)}{l}{l_2 \composeL g \composeL s_0}{\amod{1}}
	\end{array} \label{LAW2:Ass22}
\end{align}
%(\ref{LAW2:Ass})
and consequently 
%
\begin{align}
	\begin{array}{l}
  	\exsts{f', c'} (l_1 \composeL f', l_2 \composeL f') \in \amod{1}(\ca{}) \;\land\; l_1 \composeL f' \composeL c' =  l \composeL r \composeL s_0 \composeL d\\
		\lor\;l_2 \composeL f \composeL c \text{ is undefined}\\
		\lor\; l_1 \maxMeetL l = \unitL \;\land\; \exsts{r'} l_2 \composeL f \composeL c = l \composeL r' \composeL d \;\land\; \extendsAMUpto{\amod{}', \amod{L} \cup \amod{0}}{(m-1)}{l}{r'}{\amod{1}}
	\end{array} \label{LAW2:Ass22}
\end{align}
%(\ref{LAW2:Ass})
as required.
\renewcommand{\qed}{}
\end{proof}



\end{proof}
\end{lemma}


