\begin{lemma}[\mergeRule-Closure]\label{lem:merge-closure}
For all $\lmod, \lmod_1, \lmod_2, \gmod \in \AMods$ and $s_p, s_c, s_q, r \in \LStates$,
%
\[
\begin{array}{l}
	\extendsAM{\lmod, \gmod}{s_p \composeL s_c}{s_q \composeL r}{\lmod_{1}} /| \extendsAM{\lmod, \gmod}{s_q \composeL s_c}{s_p \composeL r}{\lmod_{2}}
	\implies\\
	\hspace*{2cm} \extendsAM{\lmod, \gmod}{s_p \composeL s_c \composeL s_q}{r}{\lmod_{1} \cup \lmod_{2}}
\end{array}
\]
%
\begin{proof} Pick an arbitrary $\lmod, \lmod_1, \lmod_2, \gmod \in \AMods$ and $s_p, s_c, s_q, r \in \LStates$ such that 
%
\begin{align}
	\extendsAM{\lmod, \gmod}{s_p \composeL s_c}{s_q \composeL r}{\lmod_{1}} /| \extendsAM{\lmod, \gmod}{s_q \composeL s_c}{s_p \composeL r}{\lmod_{2}} \label{MC:Ass1}
\end{align} 
%
From the definition of $\downarrow$, it then suffices to show
%
\begin{align}
	& \lmod_1 \cup \lmod_2  \subseteq \lmod \label{MC:Goal1}\\
	& \for {n \in \Nats}  \extendsAMUpto{\lmod, \gmod}{n}{s_p \composeL s_c \composeL s_q}{r}{\lmod_{1} \cup \lmod_{2}} \label{MC:Goal2}
\end{align}
%

\noindent\textbf{RTS. (\ref{MC:Goal1})} \\
Since from (\ref{MC:Ass1}) and the definition of $\downarrow$ we have $\lmod_1 \subseteq \lmod /| \lmod_2 \subseteq \lmod$, we can thus conclude $\lmod_1 \cup \lmod_2 \subseteq \lmod$ as required. \\

\noindent\textbf{RTS. (\ref{MC:Goal2})} \\
Rather than proving (\ref{MC:Goal2}) directly, we first establish the following.
%
\begin{align}
	\for {n \in \Nats} \for{s_p, s_c, s_q, r \in \LStates} & \nonumber\\
	& \hspace{-4cm}\extendsAMUpto{\lmod, \gmod}{n}{s_p \composeL s_c}{s_q \composeL r}{\lmod_1} /| \extendsAMUpto{\lmod, \gmod}{n}{s_c \composeL s_q}{s_p \composeL r}{\lmod_2} \nonumber\\
	& \hspace{-3cm} \implies \extendsAMUpto{\lmod, \gmod}{n}{s_p \composeL s_c \composeL s_q}{r}{\lmod_1 \cup \lmod_2} \label{MC:Goal3}
\end{align}
%
We can then despatch (\ref{MC:Goal2}) from (\ref{MC:Ass1}) and (\ref{MC:Goal3}); since for an arbitrary $n \in \Nats$, from (\ref{MC:Ass1}) and the definition of $\downarrow$ we have $\extendsAMUpto{\lmod, \gmod}{n}{s_p \composeL s_c}{s_q \composeL r}{\lmod_1} /| \extendsAMUpto{\lmod, \gmod}{n}{s_c \composeL s_q}{s_p \composeL r}{\lmod_2}$ and consequently from (\ref{MC:Goal3}) we derive $\extendsAMUpto{\lmod, \gmod}{n}{s_p \composeL s_c \composeL s_q}{r}{\lmod_1 \cup \lmod_2}$ as required. \\

\noindent\textbf{RTS. (\ref{MC:Goal3})} \\
We proceed by induction on the number of steps $n$.\\

%\noindent Pick an arbitrary $s_1, s_2, r \in \LStates, \lmod, \lmod', \gmod \in \AMods$.\\
\noindent\textbf{Base case }$n=0$\\
Pick an arbitrary $s_1, s_2, r \in \LStates$. We are then required to show	$\extendsAMUpto{\lmod, \gmod}{0}{s_1}{s_2 \composeL r}{\lmod'} $ which follows trivially from the definition of $\downarrow_0$.\\
%\noindent\textbf{Base case} $n = 0$\\
%\textbf{RTS. }
%%
%\[
%\begin{array}{l}
%	\for{s_p, s_q, s_c, r \in \LStates} \for{\lmod, \lmod_{1}, \lmod_{2}, \gmod \in \AMods} \\
%	\hspace*{0.5cm} \extendsAMUpto{\lmod, \gmod}{0}{s_p \composeL s_c}{s_q \composeL r}{\lmod_{1}} \;\land\; \extendsAMUpto{\lmod, \gmod}{0}{s_q \composeL s_c}{s_p \composeL r}{\lmod_{2}}
%	\implies\\
%	\hspace*{2cm} \extendsAMUpto{\lmod, \gmod}{0}{s_p \composeL s_c \composeL s_q}{r}{\lmod_{1} \cup \lmod_{2}}
%\end{array}
%\]
%%
%Pick an arbitrary $s_p, s_q, s_c, r \in \LStates$, $\lmod, \lmod_{1}, \lmod_{2}, \gmod \in \AMods$. Since $ \extendsAMUpto{\lmod, \gmod}{0}{s_p \composeL s_c}{s_q \composeL r}{\lmod_{1}}$, we know $\lmod_{1} \subseteq \lmod$; similarly, from $\extendsAMUpto{\lmod, \gmod}{0}{s_q \composeL s_c}{s_p \composeL r}{\lmod_{2}}$ we have $\lmod_{2} \subseteq \lmod$. We can thus conclude $\lmod_{1} \cup \lmod_{2} \subseteq \lmod$ and consequently by definition of $\downarrow_0$ we have $\extendsAMUpto{\lmod, \gmod}{0}{s_p \composeL s_c \composeL s_q}{r}{\lmod_{1} \cup \lmod_{2}}$ as required.\\

\noindent\textbf{Inductive Step} Pick an arbitrary $s_p, s_q, s_c, r \in \LStates$ and $n \in \Nats$, such that
%
\begin{align}
	& \extendsAMUpto{\lmod, \gmod}{n}{s_p \composeL s_c}{s_q \composeL r}{\lmod_{1}} \label{LMC:Ass1}\\
	& \extendsAMUpto{\lmod, \gmod}{n}{s_q \composeL s_c}{s_p \composeL r}{\lmod_{2}} \label{LMC:Ass2}\\
	& \for{s_p, s_q, s_c, r \in \LStates} \nonumber \\
	&	\quad \extendsAMUpto{\lmod, \gmod}{(n-1)}{s_p \composeL s_c}{s_q \composeL r}{\lmod_{1}} \;\land\; \extendsAMUpto{\lmod, \gmod}{(n-1)}{s_q \composeL s_c}{s_p \composeL r}{\lmod_{2}} \nonumber \\
	&	\tag{I.H.} \qquad\quad\implies  \extendsAMUpto{\lmod, \gmod}{(n-1)}{s_p \composeL s_c \composeL s_q}{r}{\lmod_{1} \cup \lmod_{2}} \label{LMC:IH}
\end{align}
%
%
\noindent\textbf{RTS.}
%
\begin{align}
	& 
	\V{\ca{}}  \V{a \in \left(\lmod_{1} \cup \lmod_{2}\right)(\ca{})} \nonumber \\
  &\quad(\m{potential}(a,s_p \composeL s_c \composeL s_q \composeL r) /| \m{visible}(a,s_p \composeL s_c \composeL s_q)=> \nonumber \\
  &\qquad \for{(s', r') \in a[s_p \composeL s_c \composeL s_q , r]} \extendsAMUpto{\lmod, \gmod}{(n-1)}{s'}{r'}{\lmod_{1} \cup \lmod_{2}}) /| \label{LMC:Goal0}\\
%    
  &\quad(\m{potential}(a,s_p \composeL s_c \composeL s_q \composeL r) /| \neg \m{visible}(a,s_p \composeL s_c \composeL s_q) => \nonumber \\
  &\qquad \extendsAMUpto{\lmod, \gmod}{(n-1)}{s_p \composeL s_c \composeL s_q}{a[s_p \composeL s_c \composeL s_q \composeL r] - (s_p \composeL s_c \composeL s_q)}{\lmod_{1} \cup \lmod_{2}}) \label{LMC:Goal1}\\
%    
  &\quad\land(\m{enabled}(a,s_p \composeL s_c \composeL s_q \composeL r) => \nonumber\\
  &\qquad\qquad (s_p \composeL s_c \composeL s_q  \composeL r, a[s_p \composeL s_c \composeL s_q  \composeL r])\in \gmod(\ca{})) \land\label{LMC:Goal2}\\
%  
  &\V{\ca{}}\V{a\in \lmod(\ca{})}
  \m{potential}(a,s_p \composeL s_c \composeL s_q \composeL r) =>\null \nonumber \\
  &\quad \m{reflected}(a,s_p \composeL s_c \composeL s_q \composeL r,\left(\lmod_{1} \cup \lmod_{2}\right)(\ca{})) |/\null \nonumber \\
%  
  &\quad \neg\m{visible}(a,s_p \composeL s_c \composeL s_q) /| \exsts{r'}a[s_p \composeL s_c \composeL s_q \composeL r] = s_p \composeL s_c \composeL s_q \composeL r' /| \nonumber\\
  &\hspace{1cm}\extendsAMUptob{\lmod, \gmod}{(n-1)}{s_p \composeL s_c \composeL s_q, r'}{\lmod_{1} \cup \lmod_{2}} \label{LMC:Goal3}
\end{align}
%
%

\noindent\textbf{RTS. (\ref{LMC:Goal0})}\\
Pick an arbitrary $\ca{}$ and $a = (p, q) \in \left(\lmod_{1} \cup \lmod_{2} \right)(\ca{})$ such that
\begin{align}
	\m{potential}(a, s_p \composeL s_c \composeL s_q \composeL r) \label{LMC:Ass3}\\
	\m{visible}(a, s_p \composeL s_c \composeL s_q) \label{LMC:visible-pcq}
\end{align}
%
Pick an arbitrary $(s', r') \in a[s_p \composeL s_c \composeL s_q, r]$. Then from the definition of $a[s_p \composeL s_c \composeL s_q]$ and by the cross-split property we have:
%
\begin{align}
	\exsts{p_p, p_c, p_q, p_r, s_p', s_c', s_q'} & \fst{\updateFP{a}} = p_p \composeL p_c \composeL p_q \composeL p_r \label{LMC:Ass4}\\
	& s_p = p_p \composeL s_p' \label{LMC:Ass5}\\
	& s_c = p_c \composeL s_c' \label{LMC:Ass6}\\
	& s_q = p_q \composeL s_q' \label{LMC:Ass7}\\
	& r = p_r \composeL r' \label{LMC:Ass8}\\
	& s' = \snd{\updateFP{a}} \composeL s_p' \composeL s_c' \composeL s_q' \label{LMC:Ass9}
\end{align}
%
From the definition of $\lmod_{1} \cup \lmod_{2}$ we know:
%
\begin{align}
	a \in \lmod_{1}(\ca{}) \lor a \in \lmod_{2}(\ca{}) \nonumber
\end{align}
%(\ref{LMC:Ass})
There are two cases to consider:\\

\noindent\textbf{Case 1.} $a \in \lmod_{1}(\ca{})$\\
%
Since either $\m{visible}(a, s_p \composeL s_c)$ or $\neg\m{visible}(a, s_p \composeL s_c)$ there are two cases to consider:\\

%
\noindent\textbf{Case 1.1.} $\m{visible}(a, s_p \composeL s_c)$\\
From (\ref{LMC:Ass4})-(\ref{LMC:Ass8}) and by definition of $a[s_p \composeL s_c, s_q \composeL r]$ we have:
%
\begin{align}
	(\snd{\updateFP{a}} \composeL s_p' \composeL s_c', s_q' \composeL r') \in a[s_p \composeL s_c, s_q \composeL r] 
	\label{LMC:Ass10}
\end{align}
%
Thus from the assumptions of case 1, case 1.1, (\ref{LMC:Ass1}), (\ref{LMC:Ass3}) and (\ref{LMC:Ass10}) we have:
%
%
\begin{align}
	& \extendsAMUpto{\lmod, \gmod}{(n-1)}{\snd{\updateFP{a}} \composeL s_{p}' \composeL s_{c}'}{s_q' \composeL r'}{\lmod_1} 
	\label{LMC:Ass11}
\end{align}
%(\ref{LMC:Ass})
%
Since $\lmod_{1} \subseteq \lmod$, we know $a \in \lmod(\ca{})$. Consequently, from (\ref{LMC:Ass2}),(\ref{LMC:Ass3}), (\ref{LMC:Ass4})-(\ref{LMC:Ass8}) and by cancellativity of separation algebras we have:
%(\ref{LMC:Ass})
\begin{align*}
	&\m{reflected}(a, s_p \composeL s_c \composeL s_q \composeL r, \lmod_{2}(\ca{})) \lor \\
	&\fst{\updateFP{a}} \maxMeetL s_c \composeL s_q = \unitL \land  \exsts{r''} \snd{\updateFP{a}} \composeL s_p' \composeL s_c' \composeL s_q' \composeL r' = s_c \composeL s_q \composeL r'' /| \null \\
	& \qquad\extendsAMUpto{\lmod, \gmod}{(n-1)}{s_c \composeL s_q}{r''}{\lmod_2}
\end{align*}
%
There are yet two other cases to consider:\\

\noindent\textbf{Case 1.1.1.} 
\[
\begin{array}{l}
	\fst{\updateFP{a}} \maxMeetL s_c \composeL s_q  = \unitL \land  \exsts{r''} \snd{\updateFP{a}} \composeL s_p' \composeL s_c' \composeL s_q' \composeL r' = s_c \composeL s_q \composeL r'' /| \null \\
	\qquad\extendsAMUpto{\lmod, \gmod}{(n-1)}{s_c \composeL s_q}{r''}{\lmod_2}
\end{array}
\]
%(\ref{LMC:Ass})
By definition of $\maxMeetL$ and from the assumption of case 1.1 and (\ref{LMC:Ass4})-(\ref{LMC:Ass7}) we have:
%
\begin{align*}
	p_c = p_q = \unitL \land s_c = s_c' \land s_q = s_q'
	%\label{LMC:Ass17}
\end{align*}
%
Consequently from the assumption of case 1.1.1. and by cancellativity of separation algebras we have:
%(\ref{LMC:Ass})
\begin{align}
	\extendsAMUpto{\lmod, \gmod}{(n-1)}{s_c' \composeL s_q'}{\snd{\updateFP{a}} \composeL s_p' \composeL r'}{\lmod_2} \label{LMC:Ass12}
\end{align}
%
From (\ref{LMC:Ass11}), (\ref{LMC:Ass12}) and (\ref{LMC:IH}) we have:
%
\begin{align*}
	\extendsAMUpto{\lmod, \gmod}{(n-1)}{\snd{\updateFP{a}} \composeL s_p' \composeL s_c' \composeL s_q' }{r'}{\lmod_1 \cup \lmod_2}
\end{align*}
%(\ref{LMC:Ass})
Finally, from (\ref{LMC:Ass9}) we have
%
\begin{align*}
	\extendsAMUpto{\lmod, \gmod}{(n-1)}{s'}{r'}{\lmod_1 \cup \lmod_2}
\end{align*}
%
as required.\\
%
%
%
%
%

\noindent\textbf{Case 1.1.2.}
\[
\begin{array}{l l}
	\m{reflected}(a, s_p \composeL s_c \composeL s_q \composeL r, \lmod_{2}(\ca{})) 
\end{array}
\]
%(\ref{LMC:Ass})
From (\ref{LMC:Ass3}), the definition of $\m{potential}$ and by \lem~\ref{lem:nonEmptyOverlap} we have:
%
\begin{align*}
	& \exsts{l} \fst{a} < s_p \composeL s_c \composeL s_q \composeL r \composeL l /| \null\\
	& \exsts{l} \fst{\updateFP{a}} \composeL l = s_p \composeL s_c \composeL s_q \composeL r /| \snd{\updateFP{a}} \compatible l
\end{align*}
%
and thus from the assumption of case 1.1.2. we know there exists $a' \in \lmod_2(\ca{})$ such that: 
%
\begin{align*}
	& \updateFP{a'} = \updateFP{a} /|\\
	& \exsts{l} \fst{a'} < s_p \composeL s_c \composeL s_q \composeL r \composeL l /| \null\\
	& \exsts{l} \fst{\updateFP{a'}} \composeL l = s_p \composeL s_c \composeL s_q \composeL r /| \snd{\updateFP{a'}} \compatible l
\end{align*}
%
and consequently from the definition of $\m{potential}$ and \lem~\ref{lem:nonEmptyOverlap} we have: 
%
\begin{align}
	\updateFP{a'} = \updateFP{a} /| \m{potential}(a', s_p \composeL s_c \composeL s_q \composeL s_r) \label{LMC:Ass20}
\end{align}
%(\ref{LMC:Ass})
If $\neg\m{visible}(a', s_c \composeL s_q)$ and consequently $\neg\m{visible}(a, s_c \composeL s_q)$ then the proof is analogous to that of case 1.1.1. On the other hand, if $\m{visible}(a', s_c \composeL s_q)$, then from (\ref{LMC:Ass4})-(\ref{LMC:Ass8}), (\ref{LMC:Ass20}) and by definition of $a'[s_c \composeL s_q, s_p \composeL r]$ we have $(\snd{\updateFP{a}} \composeL s_c' \composeL s_q', s_p' \composeL r') \in a'[s_c \composeL s_q, s_p \composeL r]$ and thus from (\ref{LMC:Ass2}) and (\ref{LMC:Ass20}) we have:
%
\begin{align}
	\extendsAMUpto{\lmod, \gmod}{(n-1)}{\snd{\updateFP{a}} \composeL s_c' \composeL s_q'}{s_p' \composeL r'}{\lmod_2} \label{LMC:Ass21}
\end{align}
%(\ref{LMC:Ass})
From (\ref{LMC:Ass11}), (\ref{LMC:Ass21}) and (\ref{LMC:IH}) we have: 
%
\begin{align*}
	\extendsAMUpto{\lmod, \gmod}{(n-1)}{\snd{\updateFP{a}} \composeL s_p' \composeL s_c' \composeL s_q'}{r'}{\lmod_1 \cup \lmod_2} 
\end{align*}
%(\ref{LMC:Ass})
and finally from (\ref{LMC:Ass9}) we have: 
%
\begin{align*}
	\extendsAMUpto{\lmod, \gmod}{(n-1)}{s'}{r'}{\lmod_1 \cup \lmod_2} 
\end{align*}
%(\ref{LMC:Ass})
as required.\\
%
%
%
%
%








%
\noindent\textbf{Case 1.2.} $\neg\m{visible}(a, s_p \composeL s_c)$\\
By definition of $\m{visible}$ and from the assumption of case 1.2 and (\ref{LMC:Ass4})-(\ref{LMC:Ass7}) we have:
%
\begin{align}
	p_p = p_c = \unitL \land s_p = s_p' \land s_c = s_c' \label{LMC:noOverlapPC}
	%\label{LMC:Ass17}
\end{align}
%
and consequently since $a[s_p \composeL s_c \composeL s_q \composeL s_r] = s' \composeL r'$, from (\ref{LMC:Ass9}) we have:
%
\begin{align}
	a[s_p \composeL s_c \composeL s_q \composeL s_r] = \snd{\updateFP{a}} \composeL s_p \composeL s_c \composeL s_q' \composeL r' \label{LMC:Ass10-2}
\end{align} 
%
and thus from (\ref{LMC:Ass1}), (\ref{LMC:Ass3}) and assumptions of case 1. and case 1.2. we have: 
%
\begin{align*}
	& \extendsAMUpto{\lmod, \gmod}{(n-1)}{s_{p} \composeL s_{c}}{a[s_p \composeL s_c \composeL s_q \composeL s_r] - s_p \composeL s_c }{\lmod_1} 
\end{align*}
%(\ref{LMC:Ass})
and hence from (\ref{LMC:Ass10-2})
%
\begin{align}
	& \extendsAMUpto{\lmod, \gmod}{(n-1)}{s_{p} \composeL s_{c}}{\snd{\updateFP{a}} \composeL s_q' \composeL r'}{\lmod_1} 
	\label{LMC:Ass11-2}
\end{align}
%(\ref{LMC:Ass})
%
Since $\lmod_{1} \subseteq \lmod$, we know $a \in \lmod(\ca{})$. Consequently, from (\ref{LMC:Ass2}),(\ref{LMC:Ass3}), (\ref{LMC:Ass4})-(\ref{LMC:Ass8}) and by cancellativity of separation algebras we have:
%(\ref{LMC:Ass})
\begin{align*}
	&\m{reflected}(a, s_p \composeL s_c \composeL s_q \composeL r, \lmod_{2}(\ca{})) \lor \\
	&\neg\m{visible}(a, s_c \composeL s_q) \land  \exsts{r''} \snd{\updateFP{a}} \composeL s_p' \composeL s_c' \composeL s_q' \composeL r' = s_c \composeL s_q \composeL r'' /| \null \\
	& \qquad\extendsAMUpto{\lmod, \gmod}{(n-1)}{s_c \composeL s_q}{r''}{\lmod_2}
\end{align*}
%
There are yet two other cases to consider:\\

\noindent\textbf{Case 1.2.1.} 
\[
\begin{array}{l}
	\neg\m{visible}(a, s_c \composeL s_q) \land  \exsts{r''} \snd{\updateFP{a}} \composeL s_p' \composeL s_c' \composeL s_q' \composeL r' = s_c \composeL s_q \composeL r'' /| \null \\
	\qquad\extendsAMUpto{\lmod, \gmod}{(n-1)}{s_c \composeL s_q}{r''}{\lmod_2}
\end{array}
\]
%(\ref{LMC:Ass})
Since $\neg\m{visible}(a, s_c \composeL s_q)$ (assumption of case 1.2.1) and $\neg\m{visible}(a, s_p \composeL s_c)$ (assumption of case 1.2), from the definition of $\m{visible}$ we can deduce $\neg\m{visible}(a, s_p \composeL s_c \composeL s_q)$. However, this contradicts our assumption of (\ref{LMC:visible-pcq}) and hence this is not a valid case and we can deduce:
\begin{align}
	\m{visible}(a, s_c \composeL s_q) \label{LMC:visible-cq}
\end{align}
%
%
%
%
%

\noindent\textbf{Case 1.2.2.}
\[
\begin{array}{l l}
	\m{reflected}(a, s_p \composeL s_c \composeL s_q \composeL r, \lmod_{2}(\ca{})) 
\end{array}
\]
%(\ref{LMC:Ass})
From (\ref{LMC:Ass3}), the definition of $\m{potential}$ and by \lem~\ref{lem:nonEmptyOverlap} we have:
%
\begin{align*}
	& \exsts{l} \fst{a} < s_p \composeL s_c \composeL s_q \composeL r \composeL l /| \null\\
	& \exsts{l} \fst{\updateFP{a}} \composeL l = s_p \composeL s_c \composeL s_q \composeL r /| \snd{\updateFP{a}} \compatible l
\end{align*}
%
and thus from the assumption of case 1.2.2. we know there exists $a' \in \lmod_2(\ca{})$ such that: 
%
\begin{align*}
	& \updateFP{a'} = \updateFP{a} /|\\
	& \exsts{l} \fst{a'} < s_p \composeL s_c \composeL s_q \composeL r \composeL l /| \null\\
	& \exsts{l} \fst{\updateFP{a'}} \composeL l = s_p \composeL s_c \composeL s_q \composeL r /| \snd{\updateFP{a'}} \compatible l
\end{align*}
%
and consequently from the definition of $\m{potential}$ and \lem~\ref{lem:nonEmptyOverlap} we have: 
%
\begin{align}
	\updateFP{a'} = \updateFP{a} /| \m{potential}(a', s_p \composeL s_c \composeL s_q \composeL s_r) \label{LMC:Ass20-2}
\end{align}
%(\ref{LMC:Ass})
From (\ref{LMC:Ass4})-(\ref{LMC:Ass8}), (\ref{LMC:noOverlapPC}), (\ref{LMC:Ass20-2}) and by definition of $a'[s_c \composeL s_q, s_p \composeL r]$ we have $(\snd{\updateFP{a}} \composeL s_c \composeL s_q', s_p \composeL r') \in a'[s_c \composeL s_q, s_p \composeL r]$ and thus from (\ref{LMC:Ass2}), (\ref{LMC:Ass20-2}) and (\ref{LMC:visible-cq}) we have:
%
\begin{align}
	\extendsAMUpto{\lmod, \gmod}{(n-1)}{\snd{\updateFP{a}} \composeL s_c \composeL s_q'}{s_p \composeL r'}{\lmod_2} \label{LMC:Ass21-2}
\end{align}
%(\ref{LMC:Ass})
From (\ref{LMC:Ass11-2}), (\ref{LMC:Ass21-2}) and (\ref{LMC:IH}) we have: 
%
\begin{align*}
	\extendsAMUpto{\lmod, \gmod}{(n-1)}{\snd{\updateFP{a}} \composeL s_p \composeL s_c \composeL s_q'}{r'}{\lmod_1 \cup \lmod_2} 
\end{align*}
%(\ref{LMC:Ass})
and finally from (\ref{LMC:noOverlapPC}) and (\ref{LMC:Ass9}) we have: 
%
\begin{align*}
	\extendsAMUpto{\lmod, \gmod}{(n-1)}{s'}{r'}{\lmod_1 \cup \lmod_2} 
\end{align*}
%(\ref{LMC:Ass})
as required.\\
%
%
%
%
\noindent\textbf{Case 2.} $a \in \lmod_{2}(\ca{})$\\
The proof of this case is analogous to that of previous case and is omitted here.\\
%
%
%
%
%








\noindent\textbf{RTS. (\ref{LMC:Goal1})}\\
Pick an arbitrary $\ca{}$ and $a = (p, q) \in \left(\lmod_{1} \cup \lmod_{2} \right)(\ca{})$ such that
\begin{align}
	\m{potential}(a, s_p \composeL s_c \composeL s_q \composeL r) \label{LMC:ass3}\\
	\neg\m{visible}(a, s_p \composeL s_c \composeL s_q) \label{LMC:not-visible-pcq}
\end{align}
%
Then from the definition of $a[s_p \composeL s_c \composeL s_q \composeL r]$, (\ref{LMC:ass3}), (\ref{LMC:not-visible-pcq}) and by definition of $\m{visible}$ we know there exists $r'$ such that: 
%
\begin{align}
		&\fst{\updateFP{a}} \meetL s_p = \fst{\updateFP{a}} \meetL s_c = \fst{\updateFP{a}} \meetL s_q = \{\unitL \} \label{LMC:ass4}\\
		&r = \fst{\updateFP{a}} \composeL r' /| \snd{\updateFP{a}} \compatible s_p \composeL s_c \composeL s_q \composeL r' \label{LMC:ass5}\\
		&a[s_p \composeL s_c \composeL s_q \composeL r] = \snd{\updateFP{a}} \composeL s_p \composeL s_c \composeL s_q \composeL r' \label{LMC:ass6}
\end{align}
%
From the definition of $\lmod_{1} \cup \lmod_{2}$ we know:
%
\begin{align}
	a \in \lmod_{1}(\ca{}) \lor a \in \lmod_{2}(\ca{}) \nonumber
\end{align}
%(\ref{LMC:ass})
There are two cases to consider:\\

\noindent\textbf{Case 1.} $a \in \lmod_{1}(\ca{})$\\
%(\ref{LMC:ass})
Since from  (\ref{LMC:ass4}) we have $\neg\m{visible}(a, s_p \composeL s_c)$ , from (\ref{LMC:Ass1}),  (\ref{LMC:ass3}) and (\ref{LMC:ass6}) we have: 
%
\begin{align}
	& \extendsAMUpto{\lmod, \gmod}{(n-1)}{s_{p} \composeL s_{c}}{\snd{\updateFP{a}} \composeL  s_q \composeL r'}{\lmod_1} \label{LMC:ass7}
\end{align}
%(\ref{LMC:ass})
%(\ref{LMC:ass})
%
Since $\lmod_{1} \subseteq \lmod$, we know $a \in \lmod(\ca{})$. Consequently, from (\ref{LMC:Ass2}) and (\ref{LMC:ass3}) we have:
%(\ref{LMC:Ass})
\begin{align*}
	&\m{reflected}(a, s_p \composeL s_c \composeL s_q \composeL r, \lmod_{2}(\ca{})) \lor \\
	&\neg\m{visible}(a, s_c \composeL s_q) \land  \extendsAMUpto{\lmod, \gmod}{(n-1)}{s_c \composeL s_q}{a[s_p \composeL s_c \composeL s_q \composeL r]-(s_c \composeL s_q)}{\lmod_2}
\end{align*}
%
There are two cases to consider:\\

\noindent\textbf{Case 1.1.} 
\[
\begin{array}{l}
	\neg\m{visible}(a, s_c \composeL s_q) \land  \extendsAMUpto{\lmod, \gmod}{(n-1)}{s_c \composeL s_q}{a[s_p \composeL s_c \composeL s_q \composeL r]-(s_c \composeL s_q)}{\lmod_2}
\end{array}
\]
%(\ref{LMC:ass})
Consequently, from the assumption of case 1.1. and (\ref{LMC:ass6}) we have: 
\begin{align}
	\extendsAMUpto{\lmod, \gmod}{(n-1)}{s_c \composeL s_q}{\snd{\updateFP{a}} \composeL s_p \composeL r'}{\lmod_2} \label{LMC:ass8}
\end{align}
%
and thus from (\ref{LMC:ass7}), (\ref{LMC:ass8}) and (\ref{LMC:IH}) we have:
%
\begin{align*}
	\extendsAMUpto{\lmod, \gmod}{(n-1)}{s_p \composeL s_c \composeL s_q}{\snd{\updateFP{a}} \composeL r'}{\lmod_1 \cup \lmod_2}
\end{align*}
% (\ref{LMC:ass})
and hence from (\ref{LMC:ass6})
%
\begin{align*}
	\extendsAMUpto{\lmod, \gmod}{(n-1)}{s_p \composeL s_c \composeL s_q}{a[s_p \composeL s_c \composeL s_q \composeL r] - (s_p \composeL s_c \composeL s_q)}{\lmod_1 \cup \lmod_2}
\end{align*}
% (\ref{LMC:ass})
as required.\\
%
%
%
%
%

\noindent\textbf{Case 1.2.}
\[
\begin{array}{l l}
	\m{reflected}(a, s_p \composeL s_c \composeL s_q \composeL r, \lmod_{2}(\ca{})) 
\end{array}
\]
%(\ref{LMC:Ass})
From (\ref{LMC:Ass3}), the definition of $\m{potential}$ and by \lem~\ref{lem:nonEmptyOverlap} we have:
%
\begin{align*}
	& \exsts{l} \fst{a} < s_p \composeL s_c \composeL s_q \composeL r \composeL l /| \null\\
	& \exsts{l} \fst{\updateFP{a}} \composeL l = s_p \composeL s_c \composeL s_q \composeL r /| \snd{\updateFP{a}} \compatible l
\end{align*}
%
and thus from the assumption of case 1.2. we know there exists $a' \in \lmod_2(\ca{})$ such that: 
%
\begin{align*}
	& \updateFP{a'} = \updateFP{a} /|\\
	& \exsts{l} \fst{a'} < s_p \composeL s_c \composeL s_q \composeL r \composeL l /| \null\\
	& \exsts{l} \fst{\updateFP{a'}} \composeL l = s_p \composeL s_c \composeL s_q \composeL r /| \snd{\updateFP{a'}} \compatible l
\end{align*}
%
and consequently from the definition of $\m{potential}$ and \lem~\ref{lem:nonEmptyOverlap} we have: 
%
\begin{align}
	\updateFP{a'} = \updateFP{a} /| \m{potential}(a', s_p \composeL s_c \composeL s_q \composeL s_r) \label{LMC:ass20-2}
\end{align}
%(\ref{LMC:ass})
Since from (\ref{LMC:ass4}) and (\ref{LMC:ass20-2}) we have $\neg\m{visible}(a', s_c \composeL s_q)$, from (\ref{LMC:ass20-2}) and (\ref{LMC:Ass2}) we have:
%
\begin{align*}
	\extendsAMUpto{\lmod, \gmod}{(n-1)}{s_c \composeL s_q}{a[s_p \composeL s_c \composeL s_q \composeL r] - (s_c \composeL s_q}{\lmod_2}
\end{align*}
%(\ref{LMC:ass})
and thus from (\ref{LMC:ass6})
%
\begin{align}
	\extendsAMUpto{\lmod, \gmod}{(n-1)}{s_c \composeL s_q}{\snd{\updateFP{a}} \composeL s_p \composeL r'}{\lmod_2} \label{LMC:ass9}
\end{align}
%(\ref{LMC:ass})
From (\ref{LMC:ass7}), (\ref{LMC:ass9}) and (\ref{LMC:IH}) we have: 
%
\begin{align*}
	\extendsAMUpto{\lmod, \gmod}{(n-1)}{s_p \composeL s_c \composeL s_q}{\snd{\updateFP{a}} \composeL r'}{\lmod_1 \cup \lmod_2} 
\end{align*}
%(\ref{LMC:Ass})
and finally from (\ref{LMC:ass6}) we have: 
%
\begin{align*}
	\extendsAMUpto{\lmod, \gmod}{(n-1)}{s_p \composeL s_c \composeL s_q}{a[s_p \composeL s_c \composeL s_q \composeL r] - (s_p \composeL s_c \composeL s_q)}{\lmod_1 \cup \lmod_2} 
\end{align*}
%(\ref{LMC:Ass})
as required.\\
%
%
%
%

\noindent\textbf{Case 2.} $a \in \lmod_{2}(\ca{})$\\
The proof of this case is analogous to that of previous case and is omitted here.\\
%
%
%
%
%




\noindent\textbf{RTS. (\ref{LMC:Goal2})}\\
Pick an arbitrary $\ca{}$ and $a \in \left(\lmod_{1} \cup \lmod_{2} \right)(\ca{})$ such that
\begin{equation}
	\m{enabled}(a, s_p \composeL s_c \composeL s_q \composeL r) \label{LMC:Ass30}
\end{equation}
%
There are two cases to consider:\\

\noindent\textbf{Case 1.} $a \in \lmod_{1}(\ca{})$\\
From assumption of case 1, (\ref{LMC:Ass1}) and (\ref{LMC:Ass30}) we then have: 
%
\begin{align*}
	(s_p \composeL s_c \composeL s_q \composeL r, a[s_p \composeL s_c \composeL s_q \composeL r]) \in \gmod(\ca{})
\end{align*}
%
as required. \\

\noindent\textbf{Case 2.} $a \in \lmod_{2}(\ca{})$\\
The proof of this case is analogous to that of previous case and is omitted here.\\
%
%
%
%
%

\noindent\textbf{RTS. (\ref{LMC:Goal3})}\\
Pick an arbitrary $\ca{}$ and $a \in \left(\lmod_{1} \cup \lmod_{2} \right)(\ca{})$ such that
\begin{equation}
	\m{potential}(a, s_p \composeL s_c \composeL s_q \composeL r) \label{LMC:Ass40}
\end{equation}
%(\ref{LMC:Ass})
From (\ref{LMC:Ass1}) and (\ref{LMC:Ass40}) we have: 
%
\begin{align*}
	&\m{reflected}(a,s_p \composeL s_c \composeL s_q \composeL r,\lmod_{1}(\ca{})) |/\null \\
%  
  &\neg\m{visible}(a,s_p \composeL s_c) /| \exsts{r'_1}a[s_p \composeL s_c \composeL s_q \composeL r] = s_p \composeL s_c \composeL r'_1 /| \\
  &\hspace{1cm}\extendsAMUptob{\lmod, \gmod}{(n-1)}{s_p \composeL s_c, r'_1}{\lmod_{1}}
\end{align*}
%
and consequently from the definition of $\lmod_1 \cup \lmod_2$ we have: 
%
\begin{align}
	&\m{reflected}(a,s_p \composeL s_c \composeL s_q \composeL r, \left( \lmod_{1} \cup \lmod_2 \right) (\ca{})) |/\null \nonumber \\
%  
  &\neg\m{visible}(a,s_p \composeL s_c) /| \exsts{r'_1}a[s_p \composeL s_c \composeL s_q \composeL r] = s_p \composeL s_c \composeL r'_1 /| \nonumber\\
  &\hspace{1cm}\extendsAMUptob{\lmod, \gmod}{(n-1)}{s_p \composeL s_c, r'_1}{\lmod_{1}} \label{LMC:Ass41}
\end{align}
%(\ref{LMC:Ass})
Similarly, from (\ref{LMC:Ass2}) and (\ref{LMC:Ass40}) we have: 
%
\begin{align}
	&\m{reflected}(a,s_p \composeL s_c \composeL s_q \composeL r, \left(\lmod_1 \cup \lmod_{2}\right)(\ca{})) |/\null \nonumber \\
%  
  &\neg\m{visible}(a,s_c \composeL s_q) /| \exsts{r'_2}a[s_p \composeL s_c \composeL s_q \composeL r] = s_c \composeL s_q \composeL r'_2 /| \nonumber\\
  &\hspace{1cm}\extendsAMUptob{\lmod, \gmod}{(n-1)}{s_c \composeL s_q, r'_2}{\lmod_{2}} \label{LMC:Ass42}
\end{align}
%(\ref{LMC:Ass})
From (\ref{LMC:Ass41}) and (\ref{LMC:Ass42}) we have: 
%
\begin{align}
	&\m{reflected}(a,s_p \composeL s_c \composeL s_q \composeL r, \left(\lmod_1 \cup \lmod_{2}\right)(\ca{})) |/\null \nonumber \\
%  
  &\fst{\updateFP{a}} \maxMeetL s_p \composeL s_c = \unitL /| \exsts{r'_1}a[s_p \composeL s_c \composeL s_q \composeL r] = s_p \composeL s_c \composeL r'_1 /| \nonumber\\
  &\hspace{1cm}\extendsAMUptob{\lmod, \gmod}{(n-1)}{s_p \composeL s_c, r'_1}{\lmod_{1}} /| \nonumber\\
  &\fst{\updateFP{a}} \maxMeetL s_c \composeL s_q = \unitL /| \exsts{r'_2}a[s_p \composeL s_c \composeL s_q \composeL r] = s_c \composeL s_q \composeL r'_2 /| \nonumber\\
  &\hspace{1cm}\extendsAMUptob{\lmod, \gmod}{(n-1)}{s_c \composeL s_q, r'_2}{\lmod_{2}} \label{LMC:Ass43}
\end{align}
%(\ref{LMC:Ass})
Consequently from the definition of $a[s_p \composeL s_c \composeL s_q \composeL r]$ and by the cross-split property we have:
%
\begin{align*}
	\exsts{p_p, p_c, p_q, p_r, s_p', s_c', s_q', r'} & \fst{\updateFP{a}} = p_p \composeL p_c \composeL p_q \composeL p_r \\
%	\label{LMC:Ass44}\\
	& s_p = p_p \composeL s_p' \\
%	 \label{LMC:Ass45}\\
	& s_c = p_c \composeL s_c' \\
%	\label{LMC:Ass46}\\
	& s_q = p_q \composeL s_q' \\
%	\label{LMC:Ass47}\\
	& r = p_r \composeL r' \\
%	\label{LMC:Ass48}\\
	& a[s_p \composeL s_c \composeL s_q \composeL r] = \snd{\updateFP{a}} \composeL s_p' \composeL s_c' \composeL s_q' \composeL r'
\end{align*}
%
Thus from (\ref{LMC:Ass43}), by definition of $\maxMeetL$ and cancellativity of separation algebras we can deduce:
%
\begin{align*}
	&\begin{array}{l}
		p_p = p_c = p_q = \unitL \\
		s_p = s_p' /| s_c = s_c' /| s_q = s_q' \\
		r'_1 = s_q \composeL r' /| r'_2 = s_p \composeL r'
	\end{array}
%	\label{LMC:Ass49}
\end{align*}
%(\ref{LMC:Ass})
and thus we can rewrite (\ref{LMC:Ass43}) as:
%
\begin{align*}
	&\m{reflected}(a,s_p \composeL s_c \composeL s_q \composeL r, \left(\lmod_1 \cup \lmod_{2}\right)(\ca{})) |/\null  \\  
  &\fst{\updateFP{a}} \maxMeetL s_p \composeL s_c \composeL s_q = \unitL /| \exsts{r'}a[s_p \composeL s_c \composeL s_q \composeL r] = s_p \composeL s_c \composeL s_q \composeL r' /| \\
  &\hspace{0.5cm}\extendsAMUpto{\lmod, \gmod}{(n-1)}{s_p \composeL s_c}{s_q \composeL r'}{\lmod_{1}} /| 
  \extendsAMUpto{\lmod, \gmod}{(n-1)}{s_c \composeL s_q}{s_p \composeL r'}{\lmod_{2}} 
\end{align*}
%(\ref{LMC:Ass})
and finally from (\ref{LMC:IH}) we have: 
%
\begin{align*}
	&\m{reflected}(a,s_p \composeL s_c \composeL s_q \composeL r, \left(\lmod_1 \cup \lmod_{2}\right)(\ca{})) |/\null  \\  
  &\neg\m{visible}(a, s_p \composeL s_c \composeL s_q) /| \exsts{r'}a[s_p \composeL s_c \composeL s_q \composeL r] = s_p \composeL s_c \composeL s_q \composeL r' /| \\
  &\hspace{0.5cm}\extendsAMUpto{\lmod, \gmod}{(n-1)}{s_p \composeL s_c \composeL s_q}{r'}{\lmod_{1} \cup \lmod_{2}}
\end{align*}
%(\ref{LMC:Ass})
as required. 
%
%
%
%
%

\end{proof}
\end{lemma}
%
%