\begin{lemma}[\mergeRule-Closure]\label{lem:merge-closure}
For all $\lmod, \lmod_1, \lmod_2, \gmod \in \AMods$ and $s_p, s_c, s_q, r \in \LStates$,
%
\[
\begin{array}{l}
	\extendsAM{\lmod, \gmod}{s_p \composeL s_c}{s_q \composeL r}{\lmod_{1}} /| \extendsAM{\lmod, \gmod}{s_q \composeL s_c}{s_p \composeL r}{\lmod_{2}}
	\implies\\
	\hspace*{2cm} \extendsAM{\lmod, \gmod}{s_p \composeL s_c \composeL s_q}{r}{\lmod_{1} \cup \lmod_{2}}
\end{array}
\]
%
\begin{proof} Pick an arbitrary $\lmod, \lmod_1, \lmod_2, \gmod \in \AMods$ and $s_p, s_c, s_q, r \in \LStates$ such that 
%
\begin{align}
	\extendsAM{\lmod, \gmod}{s_p \composeL s_c}{s_q \composeL r}{\lmod_{1}} /| \extendsAM{\lmod, \gmod}{s_q \composeL s_c}{s_p \composeL r}{\lmod_{2}} \label{MC:Ass1}
\end{align} 
%
From the definition of $\downarrow$, it then suffices to show
%
\begin{align}
	& \lmod_1 \cup \lmod_2  \subseteq \lmod \label{MC:Goal1}\\
	& \for {n \in \Nats}  \extendsAMUpto{\lmod, \gmod}{n}{s_p \composeL s_c \composeL s_q}{r}{\lmod_{1} \cup \lmod_{2}} \label{MC:Goal2}
\end{align}
%

\noindent\textbf{RTS. (\ref{MC:Goal1})} \\
Since from (\ref{MC:Ass1}) and the definition of $\downarrow$ we have $\lmod_1 \subseteq \lmod /| \lmod_2 \subseteq \lmod$, we can thus conclude $\lmod_1 \cup \lmod_2 \subseteq \lmod$ as required. \\

\noindent\textbf{RTS. (\ref{MC:Goal2})} \\
Rather than proving (\ref{MC:Goal2}) directly, we first establish the following.
%
\begin{align}
	\for {n \in \Nats} \for{s_p, s_c, s_q, r \in \LStates} & \nonumber\\
	& \hspace{-4cm}\extendsAMUpto{\lmod, \gmod}{n}{s_p \composeL s_c}{s_q \composeL r}{\lmod_1} /| \extendsAMUpto{\lmod, \gmod}{n}{s_c \composeL s_q}{s_p \composeL r}{\lmod_2} \nonumber\\
	& \hspace{-3cm} \implies \extendsAMUpto{\lmod, \gmod}{n}{s_p \composeL s_c \composeL s_q}{r}{\lmod_1 \cup \lmod_2} \label{MC:Goal3}
\end{align}
%
We can then despatch (\ref{MC:Goal2}) from (\ref{MC:Ass1}) and (\ref{MC:Goal3}); since for an arbitrary $n \in \Nats$, from (\ref{MC:Ass1}) and the definition of $\downarrow$ we have $\extendsAMUpto{\lmod, \gmod}{n}{s_p \composeL s_c}{s_q \composeL r}{\lmod_1} /| \extendsAMUpto{\lmod, \gmod}{n}{s_c \composeL s_q}{s_p \composeL r}{\lmod_2}$ and consequently from (\ref{MC:Goal3}) we derive $\extendsAMUpto{\lmod, \gmod}{n}{s_p \composeL s_c \composeL s_q}{r}{\lmod_1 \cup \lmod_2}$ as required. \\

\noindent\textbf{RTS. (\ref{MC:Goal3})} \\
We proceed by induction on the number of steps $n$.\\

%\noindent Pick an arbitrary $s_1, s_2, r \in \LStates, \lmod, \lmod', \gmod \in \AMods$.\\
\noindent\textbf{Base case }$n=0$\\
Pick an arbitrary $s_1, s_2, r \in \LStates$. We are then required to show	$\extendsAMUpto{\lmod, \gmod}{0}{s_1}{s_2 \composeL r}{\lmod'} $ which follows trivially from the definition of $\downarrow_0$.\\

\noindent\textbf{Inductive Step} Pick an arbitrary $s_p, s_q, s_c, r \in \LStates$ and $n \in \Nats$, such that
%
\begin{align}
	& \extendsAMUpto{\lmod, \gmod}{n}{s_p \composeL s_c}{s_q \composeL r}{\lmod_{1}} \label{LMC:Ass1}\\
	& \extendsAMUpto{\lmod, \gmod}{n}{s_q \composeL s_c}{s_p \composeL r}{\lmod_{2}} \label{LMC:Ass2}\\
	& \for{s_p, s_q, s_c, r \in \LStates} \nonumber \\
	&	\quad \extendsAMUpto{\lmod, \gmod}{(n-1)}{s_p \composeL s_c}{s_q \composeL r}{\lmod_{1}} \;\land\; \extendsAMUpto{\lmod, \gmod}{(n-1)}{s_q \composeL s_c}{s_p \composeL r}{\lmod_{2}} \nonumber \\
	&	\tag{I.H.} \qquad\quad\implies  \extendsAMUpto{\lmod, \gmod}{(n-1)}{s_p \composeL s_c \composeL s_q}{r}{\lmod_{1} \cup \lmod_{2}} \label{LMC:IH}
\end{align}
%
%
\noindent\textbf{RTS.}
%
\begin{align}
	& 
	\V{\ca{}}  \V{a \in \left(\lmod_{1} \cup \lmod_{2}\right)(\ca{})} \nonumber \\
  &\quad(\m{potential}(a,s_p \composeL s_c \composeL s_q \composeL r) => \nonumber \\
  &\qquad \for{(s', r') \in a[s_p \composeL s_c \composeL s_q , r]} \extendsAMUpto{\lmod, \gmod}{(n-1)}{s'}{r'}{\lmod_{1} \cup \lmod_{2}}) /| \label{LMC:Goal1}\\
%    
  &\quad(\m{enabled}(a,s_p \composeL s_c \composeL s_q \composeL r) => \nonumber\\
  &\qquad\qquad (s_p \composeL s_c \composeL s_q  \composeL r, a[s_p \composeL s_c \composeL s_q  \composeL r])\in \gmod(\ca{})) \land\label{LMC:Goal2}\\
%  
  &\V{\ca{}}\V{a\in \lmod(\ca{})}
  \m{potential}(a,s_p \composeL s_c \composeL s_q \composeL r) =>\null \nonumber \\
  &\quad \m{reflected}(a,s_p \composeL s_c \composeL s_q \composeL r,\left(\lmod_{1} \cup \lmod_{2}\right)(\ca{})) |/\null \nonumber \\
%  
  &\quad \neg\m{visible}(a,s_p \composeL s_c \composeL s_q) /| \for{(s', r') \in a[s_p \composeL s_c \composeL s_q, r]}\extendsAMUpto{\lmod, \gmod}{(n-1)}{s'}{r'}{\lmod_{1} \cup \lmod_{2}} \label{LMC:Goal3}
\end{align}
%
%

\noindent\textbf{RTS. (\ref{LMC:Goal1})}\\
Pick an arbitrary $\ca{}$, $a = (p, q) \in \left(\lmod_{1} \cup \lmod_{2} \right)(\ca{})$ and $s', r'$ such that
\begin{align}
	& \m{potential}(a, s_p \composeL s_c \composeL s_q \composeL r) \label{LMC:Ass3}\\
	& (s', r') \in a[s_p \composeL s_c \composeL s_q, r] \label{LMC:visible-pcq}
\end{align}
%
Then from the definition of $a[s_p \composeL s_c \composeL s_q]$ and by the cross-split property we know there exists $p_p, p_c, p_q, s_p', s_c', s_q' \in \LStates$ such that :
\begin{align}
	\begin{array}{l}
		s' = s_p \composeL s_c \composeL s_q |/ \\
		\left(
		\begin{array}{l}
			(p_p > \unitL |/ p_c > \unitL |/ p_q > \unitL) /| s' = \snd{\updateFP{a}} \composeL s_p' \composeL s_c' \composeL s_q' 		\\
			/| \fst{\updateFP{a}} = p_p \composeL p_c \composeL p_q \composeL p_r \\
			/| s_p = p_p \composeL s_p' /| 
			s_c = p_c \composeL s_c' /|
			s_q = p_q \composeL s_q' /|
			r = p_r \composeL r' 
		\end{array}
		\right)
	\end{array}
	\label{LMC:ass4}
\end{align}
%
and consequently from the definitions of $a[s_p \composeL s_c, s_q \composeL r]$ and $a[s_c  \composeL s_q, s_p \composeL r]$ we have:
%
\begin{align}
\begin{array}{l}
	\left(
	\begin{array}{@{} l @{}}
		s' = s_p \composeL s_c \composeL s_q /| \\
		(s_p \composeL s_c, s_q \composeL r') \in a[s_p \composeL s_c, s_q \composeL r] /| (s_c \composeL s_q, s_p \composeL r') \in a[s_c  \composeL s_q, s_p \composeL r]
	\end{array}
	\right) \\
	|/ 
	\left(
	\begin{array}{@{} l @{}}
	 	s' = \snd{\updateFP{a}} \composeL s_p' \composeL s_c' \composeL s_q' /| \\
	 	\left(
	 	\begin{array}{@{} l @{}}
	 		\left(
	 		\begin{array}{@{} l @{}}
	 			(( p_c > \unitL |/ (p_c = \unitL /| p_p > \unitL /| p_q > \unitL))) /| \\
	 			(\snd{\updateFP{a}} \composeL s_p' \composeL s_c' 	, s_q' \composeL r') \in a[s_p \composeL s_c, s_q \composeL r] /|\\
	 			(\snd{\updateFP{a}} \composeL s_c' \composeL s_q', s_p' \composeL r') \in a[s_c  \composeL s_q, s_p \composeL r]
	 		\end{array}
	 		\right)\\
	 		|/
	 		\left(
	 		\begin{array}{@{} l@{}}
	 			p_p > \unitL /| p_c = p_q = \unitL /|\\
	 			(\snd{\updateFP{a}} \composeL s_p' \composeL s_c', s_q' \composeL r') \in a[s_p \composeL s_c, s_q \composeL r] /|\\
	 			(s_c' \composeL s_q', \snd{\updateFP{a}} \composeL s_p' \composeL r') \in a[s_c  \composeL s_q, s_p \composeL r] 
	 		\end{array}
	 		\right)\\
	 		|/
	 		\left(
	 		\begin{array}{@{} l@{}}
				p_q > \unitL /| p_c = p_p = \unitL /|\\
				(s_p' \composeL s_c', \snd{\updateFP{a}} \composeL s_q' \composeL r') \in a[s_p \composeL s_c, s_q \composeL r] /| \\
				(\snd{\updateFP{a}} \composeL s_c' \composeL s_q', s_p' \composeL r') \in a[s_c  \composeL s_q, s_p \composeL r]
	 		\end{array}
	 		\right)
	 	\end{array}
	 	\right)
	\end{array}
	\right)
\end{array}
\label{LMC:Ass5}
\end{align}
%
From the definition of $\lmod_{1} \cup \lmod_{2}$ we know:
%
\begin{align}
	a \in \lmod_{1}(\ca{}) \lor a \in \lmod_{2}(\ca{}) \nonumber
\end{align}
%(\ref{LMC:Ass})
There are two cases to consider:\\

\noindent\textbf{Case 1.} $a \in \lmod_{1}(\ca{})$\\
%
From (\ref{LMC:Ass5}), the assumption of case 1, (\ref{LMC:Ass1}) and (\ref{LMC:Ass3}) we have:
%
\begin{align}
\begin{array}{l}
	\left(
	\begin{array}{@{} l @{}}
		s' = s_p \composeL s_c \composeL s_q /| \\
		\extendsAMUpto{\lmod, \gmod}{(n-1)}{s_{p} \composeL s_{c}}{s_q \composeL r'}{\lmod_1} /| 
		(s_c \composeL s_q, s_p \composeL r') \in a[s_c  \composeL s_q, s_p \composeL r]
	\end{array}
	\right) \\
	|/ 
	\left(
	\begin{array}{@{} l @{}}
	 	s' = \snd{\updateFP{a}} \composeL s_p' \composeL s_c' \composeL s_q' /| \\
	 	\left(
	 	\begin{array}{@{} l @{}}
	 		\left(
	 		\begin{array}{@{} l @{}}
	 			(( p_c > \unitL |/ (p_c = \unitL /| p_p > \unitL /| p_q > \unitL))) /| \\
	 			\extendsAMUpto{\lmod, \gmod}{(n-1)}{\snd{\updateFP{a}} \composeL s_{p}' \composeL s_{c}'}{s_q' \composeL r'}{\lmod_1} /|\\
	 			(\snd{\updateFP{a}} \composeL s_c' \composeL s_q', s_p' \composeL r') \in a[s_c  \composeL s_q, s_p \composeL r]
	 		\end{array}
	 		\right)\\
	 		|/
	 		\left(
	 		\begin{array}{@{} l@{}}
	 			p_p > \unitL /| p_c = p_q = \unitL /|\\
	 			\extendsAMUpto{\lmod, \gmod}{(n-1)}{\snd{\updateFP{a}} \composeL s_{p}' \composeL s_{c}'}{s_q' \composeL r'}{\lmod_1} /| \\
	 			(s_c' \composeL s_q', \snd{\updateFP{a}} \composeL s_p' \composeL r') \in a[s_c  \composeL s_q, s_p \composeL r] 
	 		\end{array}
	 		\right)\\
	 		|/
	 		\left(
	 		\begin{array}{@{} l@{}}
				p_q > \unitL /| p_c = p_p = \unitL /|\\
				\extendsAMUpto{\lmod, \gmod}{(n-1)}{s_{p}' \composeL s_{c}'}{\snd{\updateFP{a}} \composeL s_q' \composeL r'}{\lmod_1}  /| \\
				(\snd{\updateFP{a}} \composeL s_c' \composeL s_q', s_p' \composeL r') \in a[s_c  \composeL s_q, s_p \composeL r]
	 		\end{array}
	 		\right)
	 	\end{array}
	 	\right)
	\end{array}
	\right)
\end{array}
\label{LMC:Ass6}
\end{align}
%(\ref{LMC:Ass})
%
Since $\lmod_{1} \subseteq \lmod$, we know $a \in \lmod(\ca{})$. Consequently, from (\ref{LMC:Ass2}) and (\ref{LMC:Ass3}) we have:
%(\ref{LMC:Ass})
\begin{align*}
	&\m{reflected}(a, s_p \composeL s_c \composeL s_q \composeL r, \lmod_{2}(\ca{})) \lor \\
	&\neg\m{visible}(a, s_c \composeL s_q) /| \extendsAMUpto{\lmod, \gmod}{(n-1)}{s_c \composeL s_q}{a[s_p \composeL s_c \composeL s_q \composeL e] - s_c \composeL s_q}{\lmod_2}
\end{align*}
%
There are two cases to consider:\\

\noindent\textbf{Case 1.1.} 
%
\[
\begin{array}{l}
	\neg\m{visible}(a, s_c \composeL s_q) /| \extendsAMUpto{\lmod, \gmod}{(n-1)}{s_c \composeL s_q}{a[s_p \composeL s_c \composeL s_q \composeL e] - s_c \composeL s_q}{\lmod_2}
\end{array}
\]
%(\ref{LMC:Ass})
By definition of $\m{visible}$ and from the assumption of case 1.1 and (\ref{LMC:Ass6}) we have:
%
\begin{align}
\begin{array}{l}
	\left(
	\begin{array}{@{} l @{}}
		s' = s_p \composeL s_c \composeL s_q /| \\
		\extendsAMUpto{\lmod, \gmod}{(n-1)}{s_{p} \composeL s_{c}}{s_q \composeL r'}{\lmod_1} /| 
		\extendsAMUpto{\lmod, \gmod}{(n-1)}{s_c \composeL s_q}{s_p \composeL r'}{\lmod_2}
	\end{array}
	\right) \\
	|/ 
	\left(
	\begin{array}{@{} l @{}}
	 	s' = \snd{\updateFP{a}} \composeL s_p' \composeL s_c' \composeL s_q' /| p_c = p_q = \unitL /| s_c' = s_c /| s_q' = s_q /| \\
	 			\extendsAMUpto{\lmod, \gmod}{(n-1)}{\snd{\updateFP{a}} \composeL s_{p}' \composeL s_{c}'}{s_q' \composeL r'}{\lmod_1} /| \\
	 			\extendsAMUpto{\lmod, \gmod}{(n-1)}{s_c' \composeL s_q'}{\snd{\updateFP{a}} \composeL s_p' \composeL r'}{\lmod_2} 
	\end{array}
	\right)
\end{array}
\label{LMC:Ass7}
\end{align}
%
From (\ref{LMC:Ass7}) and (\ref{LMC:IH}) we have:
\begin{align*}
\begin{array}{l}
	\extendsAMUpto{\lmod, \gmod}{(n-1)}{s'}{r'}{\lmod_1 \cup \lmod_2}
\end{array}
\end{align*}
%
as required.\\
%
%
%
%
%

\noindent\textbf{Case 1.2.}
\[
\begin{array}{l l}
	\m{reflected}(a, s_p \composeL s_c \composeL s_q \composeL r, \lmod_{2}(\ca{})) 
\end{array}
\]
%(\ref{LMC:Ass})
From (\ref{LMC:Ass3}), the definition of $\m{potential}$ and by \lem~\ref{lem:nonEmptyOverlap} we have:
%
\begin{align*}
	& \exsts{l} \fst{a} < s_p \composeL s_c \composeL s_q \composeL r \composeL l /| \null\\
	& \exsts{l} \fst{\updateFP{a}} \composeL l = s_p \composeL s_c \composeL s_q \composeL r /| \snd{\updateFP{a}} \compatible l
\end{align*}
%
and thus from the assumption of case 1.1.2. we know there exists $a' \in \lmod_2(\ca{})$ such that: 
%
\begin{align*}
	& \updateFP{a'} = \updateFP{a} /|\\
	& \exsts{l} \fst{a'} < s_p \composeL s_c \composeL s_q \composeL r \composeL l /| \null\\
	& \exsts{l} \fst{\updateFP{a'}} \composeL l = s_p \composeL s_c \composeL s_q \composeL r /| \snd{\updateFP{a'}} \compatible l
\end{align*}
%
and consequently from the definition of $\m{potential}$ and \lem~\ref{lem:nonEmptyOverlap} we have: 
%
\begin{align}
	\updateFP{a'} = \updateFP{a} /| \m{potential}(a', s_p \composeL s_c \composeL s_q \composeL s_r) \label{LMC:Ass20}
\end{align}
%(\ref{LMC:Ass})
On the other hand, from the definition of $a[s_c \composeL s_q, s_p \composeL r]$ and since $\updateFP{a'} = \updateFP{a}$(\ref{LMC:Ass20}), from (\ref{LMC:Ass6}) we have: 
%
\begin{align}
\begin{array}{l}
	\left(
	\begin{array}{@{} l @{}}
		s' = s_p \composeL s_c \composeL s_q /| \\
		\extendsAMUpto{\lmod, \gmod}{(n-1)}{s_{p} \composeL s_{c}}{s_q \composeL r'}{\lmod_1} /| 
		(s_c \composeL s_q, s_p \composeL r') \in a'[s_c  \composeL s_q, s_p \composeL r]
	\end{array}
	\right) \\
	|/ 
	\left(
	\begin{array}{@{} l @{}}
	 	s' = \snd{\updateFP{a}} \composeL s_p' \composeL s_c' \composeL s_q' /| \\
	 	\left(
	 	\begin{array}{@{} l @{}}
	 		\left(
	 		\begin{array}{@{} l @{}}
	 			(( p_c > \unitL |/ (p_c = \unitL /| p_p > \unitL /| p_q > \unitL))) /| \\
	 			\extendsAMUpto{\lmod, \gmod}{(n-1)}{\snd{\updateFP{a}} \composeL s_{p}' \composeL s_{c}'}{s_q' \composeL r'}{\lmod_1} /|\\
	 			(\snd{\updateFP{a}} \composeL s_c' \composeL s_q', s_p' \composeL r') \in a'[s_c  \composeL s_q, s_p \composeL r]
	 		\end{array}
	 		\right)\\
	 		|/
	 		\left(
	 		\begin{array}{@{} l@{}}
	 			p_p > \unitL /| p_c = p_q = \unitL /|\\
	 			\extendsAMUpto{\lmod, \gmod}{(n-1)}{\snd{\updateFP{a}} \composeL s_{p}' \composeL s_{c}'}{s_q' \composeL r'}{\lmod_1} /| \\
	 			(s_c' \composeL s_q', \snd{\updateFP{a}} \composeL s_p' \composeL r') \in a'[s_c  \composeL s_q, s_p \composeL r] 
	 		\end{array}
	 		\right)\\
	 		|/
	 		\left(
	 		\begin{array}{@{} l@{}}
				p_q > \unitL /| p_c = p_p = \unitL /|\\
				\extendsAMUpto{\lmod, \gmod}{(n-1)}{s_{p}' \composeL s_{c}'}{\snd{\updateFP{a}} \composeL s_q' \composeL r'}{\lmod_1}  /| \\
				(\snd{\updateFP{a}} \composeL s_c' \composeL s_q', s_p' \composeL r') \in a'[s_c  \composeL s_q, s_p \composeL r]
	 		\end{array}
	 		\right)
	 	\end{array}
	 	\right)
	\end{array}
	\right)
\end{array}
\label{LMC:Ass21}
\end{align}
%
and thus from (\ref{LMC:Ass2}) we have:
%
\begin{align}
\begin{array}{l}
	\left(
	\begin{array}{@{} l @{}}
		s' = s_p \composeL s_c \composeL s_q /| \\
		\extendsAMUpto{\lmod, \gmod}{(n-1)}{s_{p} \composeL s_{c}}{s_q \composeL r'}{\lmod_1} /| 
		\extendsAMUpto{\lmod, \gmod}{(n-1)}{s_{c} \composeL s_{q}}{s_p \composeL r'}{\lmod_2}
	\end{array}
	\right) \\
	|/ 
	\left(
	\begin{array}{@{} l @{}}
	 	s' = \snd{\updateFP{a}} \composeL s_p' \composeL s_c' \composeL s_q' /| \\
	 	\left(
	 	\begin{array}{@{} l @{}}
	 		\left(
	 		\begin{array}{@{} l @{}}
	 			(( p_c > \unitL |/ (p_c = \unitL /| p_p > \unitL /| p_q > \unitL))) /| \\
	 			\extendsAMUpto{\lmod, \gmod}{(n-1)}{\snd{\updateFP{a}} \composeL s_{p}' \composeL s_{c}'}{s_q' \composeL r'}{\lmod_1} /|\\
	 			\extendsAMUpto{\lmod, \gmod}{(n-1)}{\snd{\updateFP{a}} \composeL s_{c}' \composeL s_{q}'}{s_p' \composeL r'}{\lmod_2} 
	 		\end{array}
	 		\right)\\
	 		|/
	 		\left(
	 		\begin{array}{@{} l@{}}
	 			p_p > \unitL /| p_c = p_q = \unitL /|\\
	 			\extendsAMUpto{\lmod, \gmod}{(n-1)}{\snd{\updateFP{a}} \composeL s_{p}' \composeL s_{c}'}{s_q' \composeL r'}{\lmod_1} /| \\
	 			\extendsAMUpto{\lmod, \gmod}{(n-1)}{s_{c}' \composeL s_{q}'}{\snd{\updateFP{a}} \composeL s_p' \composeL r'}{\lmod_2} 
	 		\end{array}
	 		\right)\\
	 		|/
	 		\left(
	 		\begin{array}{@{} l@{}}
				p_q > \unitL /| p_c = p_p = \unitL /|\\
				\extendsAMUpto{\lmod, \gmod}{(n-1)}{s_{p}' \composeL s_{c}'}{\snd{\updateFP{a}} \composeL s_q' \composeL r'}{\lmod_1}  /| \\
				\extendsAMUpto{\lmod, \gmod}{(n-1)}{\snd{\updateFP{a}} \composeL s_{c}' \composeL s_{q}'}{s_p' \composeL r'}{\lmod_2}
	 		\end{array}
	 		\right)
	 	\end{array}
	 	\right)
	\end{array}
	\right)
\end{array}
\label{LMC:Ass22}
\end{align}
%
%(\ref{LMC:Ass})
From (\ref{LMC:Ass22}) and (\ref{LMC:IH}) we have: 
%
\begin{align*}
	\extendsAMUpto{\lmod, \gmod}{(n-1)}{s'}{r'}{\lmod_1 \cup \lmod_2} 
\end{align*}
%(\ref{LMC:Ass})
as required.\\
%
%
%
%
%
%

\noindent\textbf{Case 2.} $a \in \lmod_{2}(\ca{})$\\
The proof of this case is analogous to that of previous case and is omitted here.\\
%
%
%
%
%




\noindent\textbf{RTS. (\ref{LMC:Goal2})}\\
Pick an arbitrary $\ca{}$ and $a \in \left(\lmod_{1} \cup \lmod_{2} \right)(\ca{})$ such that
\begin{equation}
	\m{enabled}(a, s_p \composeL s_c \composeL s_q \composeL r) \label{LMC:Ass30}
\end{equation}
%
There are two cases to consider:\\

\noindent\textbf{Case 1.} $a \in \lmod_{1}(\ca{})$\\
From assumption of case 1, (\ref{LMC:Ass1}) and (\ref{LMC:Ass30}) we then have: 
%
\begin{align*}
	(s_p \composeL s_c \composeL s_q \composeL r, a[s_p \composeL s_c \composeL s_q \composeL r]) \in \gmod(\ca{})
\end{align*}
%
as required. \\

\noindent\textbf{Case 2.} $a \in \lmod_{2}(\ca{})$\\
The proof of this case is analogous to that of previous case and is omitted here.\\
%
%
%
%
%

\noindent\textbf{RTS. (\ref{LMC:Goal3})}\\
Pick an arbitrary $\ca{}$ and $a \in\lmod(\ca{})$ such that
\begin{equation}
	\m{potential}(a, s_p \composeL s_c \composeL s_q \composeL r) \label{LMC:Ass40}
\end{equation}
%(\ref{LMC:Ass})
From (\ref{LMC:Ass1}) and (\ref{LMC:Ass40}) we have: 
%
\begin{align*}
	&\m{reflected}(a,s_p \composeL s_c \composeL s_q \composeL r,\lmod_{1}(\ca{})) |/\null \\
%  
  &\neg\m{visible}(a,s_p \composeL s_c) /| \for{(s', r') \in a[s_p \composeL s_c, s_q \composeL r]} \extendsAMUpto{\lmod, \gmod}{(n-1)}{s'}{r'}{\lmod_{1}}
\end{align*}
%
and consequently from the definition of $\lmod_1 \cup \lmod_2$ we have: 
%
\begin{align}
	&\m{reflected}(a,s_p \composeL s_c \composeL s_q \composeL r, \left( \lmod_{1} \cup \lmod_2 \right) (\ca{})) |/\null \nonumber \\
%  
  &\neg\m{visible}(a,s_p \composeL s_c) /| \for{(s', r') \in a[s_p \composeL s_c, s_q \composeL r]} \extendsAMUpto{\lmod, \gmod}{(n-1)}{s'}{r'}{\lmod_{1}} \label{LMC:Ass41}
\end{align}
%(\ref{LMC:Ass})
Similarly, from (\ref{LMC:Ass2}) and (\ref{LMC:Ass40}) we have: 
%
\begin{align}
	&\m{reflected}(a,s_p \composeL s_c \composeL s_q \composeL r, \left(\lmod_1 \cup \lmod_{2}\right)(\ca{})) |/\null \nonumber \\
%  
  &\neg\m{visible}(a, s_c \composeL s_q) /| \for{(s', r') \in a[s_c \composeL s_q, s_p \composeL r]} \extendsAMUpto{\lmod, \gmod}{(n-1)}{s'}{r'}{\lmod_{2}} \label{LMC:Ass42}
\end{align}
%(\ref{LMC:Ass})
From (\ref{LMC:Ass41}) and (\ref{LMC:Ass42}) we have: 
%
\begin{align}
	&\m{reflected}(a,s_p \composeL s_c \composeL s_q \composeL r, \left(\lmod_1 \cup \lmod_{2}\right)(\ca{})) |/\null \nonumber \\
%  
  &\neg\m{visible}(a,s_p \composeL s_c) /| \for{(s', r') \in a[s_p \composeL s_c, s_q \composeL r]} \extendsAMUpto{\lmod, \gmod}{(n-1)}{s'}{r'}{\lmod_{1}} \nonumber\\
  &\neg\m{visible}(a, s_c \composeL s_q) /| \for{(s', r') \in a[s_c \composeL s_q, s_p \composeL r]} \extendsAMUpto{\lmod, \gmod}{(n-1)}{s'}{r'}{\lmod_{2}} \label{LMC:Ass43}
\end{align}
%(\ref{LMC:Ass})
If the first disjunct is the case then the desired result holds trivially. On the other hand, in case of the second disjunct we have:
%
\begin{align}
	& 
	\begin{array}{@{} l @{}}
		\neg\m{visible}(a,s_p \composeL s_c) /| \neg\m{visible}(a, s_c \composeL s_q) \\
		/| \for{(s', r') \in a[s_p \composeL s_c, s_q \composeL r]} \extendsAMUpto{\lmod, \gmod}{(n-1)}{s'}{r'}{\lmod_{1}} \\
  	/| \for{(s', r') \in a[s_c \composeL s_q, s_p \composeL r]} \extendsAMUpto{\lmod, \gmod}{(n-1)}{s'}{r'}{\lmod_{2}} 
	\end{array}
	\label{LMC:Ass44}
\end{align}
%
From (\ref{LMC:Ass44}) and by definition of $\m{visible}$ we have:
%
\begin{align}
	\neg\m{visible}(a, s_p \composeL s_c \composeL s_q)
	\label{LMC:Ass45}
\end{align}
%
Pick an arbitrary $s', r' \in \LStates$ such that
%
\begin{align}
	(s', r') \in a[s_p \composeL s_c \composeL s_q, r]
	\label{LMC:Ass46}
\end{align}
%(\ref{LMC:Ass})
Then from (\ref{LMC:Ass45}), (\ref{LMC:Ass46}) and the definitions of $a[s_p \composeL s_c, s_q \composeL r]$ and $a[s_c \composeL s_q, s_p \composeL r]$ we know 
%
\begin{align}
	& s' = s_p \composeL s_c \composeL s_q \label{LMC:Ass47}\\
	&(s_p \composeL s_c, s_q \composeL r') \in a[s_p \composeL s_c, s_q \composeL r] /| 
	(s_c \composeL s_q, s_p \composeL r') \in a[s_c \composeL s_q, s_p \composeL r]
	\label{LMC:Ass48}
\end{align}
%
Consequently from (\ref{LMC:Ass44}) and (\ref{LMC:Ass48}) we have
%
\begin{align*}
	\extendsAMUpto{\lmod, \gmod}{(n-1)}{s_p \composeL s_c}{s_q \composeL r'}{\lmod_1} /| \extendsAMUpto{\lmod, \gmod}{(n-1)}{s_c \composeL s_q}{s_p \composeL r'}{\lmod_2}
\end{align*}
% 
and thus from (\ref{LMC:IH}) and (\ref{LMC:Ass47})
%
\begin{align}
	\extendsAMUpto{\lmod, \gmod}{(n-1)}{s'}{r'}{\lmod_1 \cup \lmod_2}
	\label{LMC:Ass49}
\end{align}
%(\ref{LMC:Ass})
Finally, from (\ref{LMC:Ass45}), (\ref{LMC:Ass46}) and (\ref{LMC:Ass49}) we have:
%
\begin{align*}
	& \neg\m{visible}(a, s_p \composeL s_c \composeL s_q)  /| \\
	& \for{(s', r') \in a[s_p \composeL s_c \composeL s_q, r]} \extendsAMUpto{\lmod, \gmod}{(n-1)}{s'}{r'}{\lmod_1 \cup \lmod_2}
\end{align*}
%
as required.
%
%
%
%
%

\end{proof}
\end{lemma}
%
%