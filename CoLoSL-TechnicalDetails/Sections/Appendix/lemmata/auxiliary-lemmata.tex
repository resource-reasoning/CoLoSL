%\begin{lemma}[Sequential Command Soundness]\label{lem:seqSoundness}
%For all $\seq{1} \in \Seqs$, $\left(\Hp{1}, \seq{1}, \Hp{2}\right) \in \AxiomsSeq$ and $\h{} \in \Heaps$:
%%
%\[
%	\opSemSeq{\seq{1}}{\reifyH{\Hp{1} \composeH \{\h{}\}}} \subseteq \reifyH{\Hp{2} \composeH \{\h{}\}}
%\]
%%
%\begin{proof}
%By induction over the structure of $\seq{}$. Pick an arbitrary $\h{} \in \Heaps$.\\
%
%\noindent\textbf{Case \hspace*{0.3cm}}\bc{}\\
%This follows from parameter \ref{par:basicSoundness}.\\
%
%
%\noindent\textbf{Case \hspace*{0.3cm}\skipC}\\
%\textbf{RTS.}
%%
%\[
%	\opSemSeq{\seq{1}}{\reifyH{\Hp{} \composeH \{\h{}\}}} 
%	\subseteq \reifyH{\Hp{} \composeH \{\h{}\}}
%\]
%%
%\begin{proof}
%%
%\[
%\begin{array}{r l}
%	\opSemSeq{\seq{1}}{\reifyH{\Hp{} \composeH \{\h{}\}}} 
%	= &
%	\reifyH{\Hp{} \composeH \{\h{}\}}\\
%
%	\subseteq & \reifyH{\Hp{} \composeH \{\h{}\}}
%\end{array}
%\]
%%
%as required.
%\renewcommand{\qed}{}
%\end{proof}
%%
%%
%
%\noindent\textbf{Case \hspace*{0.3cm}}$\seq{1} ; \seq{2}$\\
%\textbf{RTS.}
%%
%\[
%	\opSemSeq{\seq{1}; \seq{2}}{\reifyH{\Hp{} \composeH \{\h{}\}}} 
%	\subseteq \reifyH{\Hp{}' \composeH \{\h{}\}}
%\]
%%
%where $\left(\Hp{}, \seq{1}, \Hp{}'' \right), \left(\Hp{}'', \seq{2}, \Hp{}' \right)  \in \AxiomsSeq$.
%\begin{proof}
%%
%\[
%\begin{array}{r l}
%	
%	\opSemSeq{\seq{1}; \seq{2}}{\reifyH{\Hp{} \composeH \{\h{}\}}} 
%	= &  
%	\opSemSeq{\seq{2}}{ \opSemSeq{\seq{1}}{\reifyH{\Hp{} \composeH \{\h{}\}}}}\\
%
%	\text{(I.H.) \hspace*{0.5cm}}
%	\subseteq &
%	\opSemSeq{\seq{2}}{\reifyH{\Hp{}'' \composeH \{\h{}\}}}\\
%	
%	\text{(I.H.) \hspace*{0.5cm}}
%	\subseteq &
%	\reifyH{\Hp{}' \composeH \{\h{}\}}
%	
%\end{array}
%\]
%%
%as required.
%\renewcommand{\qed}{}
%\end{proof}
%%
%%
%
%\noindent\textbf{Case \hspace*{0.3cm}}$\seq{1} + \seq{2}$\\
%\textbf{RTS.}
%%
%\[
%	\opSemSeq{\seq{1} + \seq{2}}{\reifyH{\Hp{} \composeH \{\h{}\}}} 
%	\subseteq \reifyH{\Hp{}' \composeH \{\h{}\}}
%\]
%%
%where $\left(\Hp{}, \seq{1}, \Hp{}' \right), \left(\Hp{}, \seq{2}, \Hp{}' \right)  \in \AxiomsSeq$.
%\begin{proof}
%%
%\[
%\hspace*{-0.2cm}
%\begin{array}{r l}
%	
%	\opSemSeq{\seq{1} + \seq{2}}{\reifyH{\Hp{} \composeH \{\h{}\}}} 
%	= &  
%	\opSemSeq{\seq{1}}{\reifyH{\Hp{} \composeH \{\h{}\}}} \;\cup\; \opSemSeq{\seq{2}}{\reifyH{\Hp{} \composeH \{\h{}\}}}\\
%
%	\text{(I.H.) \hspace*{0.5cm}}
%	\subseteq &
%	\reifyH{\Hp{}' \composeH \{\h{}\}} \;\cup\; \reifyH{\Hp{}' \composeH \{\h{}\}}\\
%	
%	\subseteq &
%	\reifyH{\Hp{}' \composeH \{\h{}\}}
%	
%\end{array}
%\]
%%
%as required.
%\renewcommand{\qed}{}
%\end{proof}
%%
%%
%
%\noindent\textbf{Case \hspace*{0.3cm}}$\seq{}^{*}$\\
%\textbf{RTS.}
%%
%\[
%	\opSemSeq{\seq{}^{*}}{\reifyH{\Hp{} \composeH \{\h{}\}}} 
%	\subseteq \reifyH{\Hp{} \composeH \{\h{}\}}
%\]
%%
%where $\left(\Hp{}, \seq{}, \Hp{} \right)  \in \AxiomsSeq$.
%\begin{proof}
%%
%\[
%\begin{array}{r l}
%	
%	\opSemSeq{\seq{}^{*}}{\reifyH{\Hp{} \composeH \{\h{}\}}} 
%	= &  
%	\opSemSeq{\skipC + \seq{}; \seq{}^{*}}{\reifyH{\Hp{} \composeH \{\h{}\}}} \\
%	
%	= & \opSemSeq{\skipC}{\reifyH{\Hp{} \composeH \{\h{}\}}} 
%		\cup \opSemSeq{\seq{}; \seq{}^{*}}{\reifyH{\Hp{} \composeH \{\h{}\}}} \\
%
%	\text{(I.H.) \hspace*{0.5cm}}
%	\subseteq &
%	\reifyH{\Hp{} \composeH \{\h{}\}} \cup \reifyH{\Hp{} \composeH \{\h{}\}}\\
%	
%	\subseteq &
%	\reifyH{\Hp{} \composeH \{\h{}\}}
%	
%\end{array}
%\]
%%
%as required.
%\renewcommand{\qed}{}
%\end{proof}
%%
%%
%\end{proof}
%\end{lemma}
%%
%%
%\begin{lemma}[] \label{lem:updateGContainment}
%
\[
\begin{array}{l}
	\for{w_1, w_2, w \in \Worlds} w_1 \composeW w_2 = w \implies\\
	\hspace*{1cm}(l'_1, s', \amod{}') \in \updateG(w_1) \implies (\localPart{(w_2)}, s', \amod{}') \in \updateR(w_2)
\end{array}
\]
%
where given any relation $S \in (A \times A)$ we write
%
\[
\begin{array}{l}
	S(w) \eqdef \left\{w' \;|\; (w, w') \in S \right\}\\
\end{array}
\]
%
\begin{proof} Pick an arbitrary $(l_1, s_1, \amod{1}), (l_2, s_2, \amod{2}), w$ and $(l'_1, s', \amod{}')$ such that:
%
\begin{align}
	(l_1, s_1, \amod{1}) \composeW (l_2, s_2, \amod{2}) = w \label{L11:Ass1}\\
	(l'_1, s', \amod{}') \in \updateG(l_1, s_1, \amod{1}) \label{L11:Ass2}
\end{align}
%
\textbf{RTS.}
%
\begin{align}
	(\localPart{(w_2)}, s', \amod{}') \in \updateR(l_2, s_2, \amod{2}) \label{L11:Goal}
\end{align}
From (\ref{L11:Ass1}) we know:
%
\begin{align}
	s_1 = s_2 \label{L11:Ass3}\\
	\amod{1} = \amod{2} \label{L11:Ass4}
\end{align}
%
By definition of \updateG and from (\ref{L11:Ass2}) and (\ref{L11:Ass4}) we know:
%
\begin{align}
	& \amod{1} = \amod{2} = \amod{}' \label{L11:Ass5} \\
	& \capSize{\capPart{l_1 \composeL s_1)}} \subseteq \capSize{\heapPart{(l'_1 \composeL s')}} \label{L11:Perms} \\
	& s' = s_1 \lor 
	\left(\begin{array}{l}
		\exsts{\ca{} \leq \capPart{(l_1)}}  (s_1, s') \in \amod{1}(\ca{}) \;\land \\
		\heapSize{\heapPart{l_1 \composeL s_1)}} = \heapSize{\heapPart{(l'_1 \composeL s')}}
	\end{array} \right) \nonumber
\end{align}
%
There are two cases to consider:\\

\noindent\textbf{Case 1.} $s_1 = s'$\\
From (\ref{L11:Ass3}) and the assumption of the case we know $s' = s_2$. Consequently, from (\ref{L11:Ass5}) we have:

\begin{equation}
	(\localPart{(w_2)}, s', \amod{}') = (l_2, s_2, \amod{2}) \label{L11:Ass6}
\end{equation}
%
By definition of \updateR\ and from (\ref{L11:Ass6}) we can conclude:
%
\begin{equation}
	(\localPart{(w_2)}, s', \amod{}') \in \updateR(l_2, s_2, \amod{2}) \label{L11:Ass7}
\end{equation}
%
\textbf{Case 2.} 
%
\begin{align}
	\exsts{\ca{} \leq \capPart{(l_1)}} & (s_1, s') \in \amod{1}(\ca{}) \label{L11:Ass8} \;\land\\
	&\heapSize{\heapPart{l_1 \composeL s_1)}} = \heapSize{\heapPart{(l'_1 \composeL s')}} \label{L11:Ass9}
\end{align}
%
From (\ref{L11:Ass1}), (\ref{L11:Ass3}) and (\ref{L11:Ass4}) we know that 
%
\begin{equation}
	w = (l_1 \composeL l_2, s_2, \amod{2}) \label{L11:Ass10}
\end{equation}
%
Since $\wf{w}$ (by definition of \Worlds) and from (\ref{L11:Ass3}) we know:
\begin{align}
	&\capPart{(l_1 \composeL l_2 \composeL s_2)} = \capPart{(l_1)} \composeCap \capPart{(l_2)} \composeCap {(s_2)} = \capPart{(l_1 \composeL s_1)} \composeCap \capPart{(l_2)}\hspace*{0.2cm} \text{is defined} \label{L11:Ass11}\\
%	
	&\heapPart{(l_1 \composeL l_2 \composeL s_1)} = \heapPart{(l_1 \composeH s_1)} \composeH \heapPart{(l_2)}  \hspace*{0.2cm} \text{is defined} \label{L11:Ass12}
\end{align}
%
Since $\ca{1} \leq \capPart{(l_1)}$ (\ref{L11:Ass8}), from (\ref{L11:Ass11}) and Lemma \ref{lem:disjointByOrder}, we know:
%
\begin{equation}
	\ca{} \disjoint\ \ \capPart{(l_2)} \composeCap \capPart{(s_2)} \label{L11:Ass13}
\end{equation}
%
From (\ref{L11:Ass3}), (\ref{L11:Ass9}) and (\ref{L11:Ass12}) we know
%
\begin{equation}
	\heapPart{(l'_1 \composeL s')} \composeH \heapPart{(l_2)} = \heapPart{(l'_1 \composeL l_2 \composeL s')}\hspace*{0.2cm}\text{ is defined} \label{L11:Ass14}
\end{equation}
%
From (\ref{L11:Perms}) and (\ref{L11:Ass11}) we know
%
\begin{equation}
	\capPart{(l'_1 \composeL s')} \composeCap \capPart{(l_2)} = \capPart{(l'_1 \composeL l_2 \composeL s')}\hspace*{0.2cm}\text{ is defined} \label{L11:Ass15}
\end{equation}
%
From (\ref{L11:Ass14}) and (\ref{L11:Ass15}) we know $l'_1 \composeL l_2 \composeL s'$ is defined and consequently:
%
\begin{equation}
	l_2 \composeL s' \hspace*{0.2cm} \text{ is defined} \label{L11:Ass16}
\end{equation}
%
From (\ref{L11:Ass4}), (\ref{L11:Ass8}), (\ref{L11:Ass13}), (\ref{L11:Ass16}) and by definition of \updateR, we have:
%
\begin{equation}
	(\localPart{(w_2)}, s', \amod{}') = (l_2, s', \amod{}) \in \updateR(l_2, s_2, \amod{})  \label{L11:Ass17}
\end{equation}
%
From (\ref{L11:Ass7}) and (\ref{L11:Ass17}) we can dismiss (\ref{L11:Goal}).
\end{proof}
%
\end{lemma}
%
%
\begin{lemma}[]\label{lem:extendGContainment}
%
\[
\begin{array}{l}
	\for{w_1, w_2, w \in \Worlds} w_1 \composeW w_2 = w \implies\\
	\hspace*{1cm}(l'_1, s', \amod{}') \in \extendG(w_1) \implies (\localPart{(w_2)}, s', \amod{}') \in \extendR(w_2)
\end{array}
\]
%
\begin{proof} Pick an arbitrary $(l_1, s_1, \amod{1}), (l_2, s_2, \amod{2}), w$ and $(l'_1, s', \amod{}')$ such that:
%
\begin{align}
	(l_1, s_1, \amod{1}) \composeW (l_2, s_2, \amod{2}) = w \label{L12:Ass1}\\
	(l'_1, s', \amod{}') \in \extendG(l_1, s_1, \amod{1}) \label{L12:Ass2}
\end{align}
%
\textbf{RTS.}
%
\begin{align}
	(\localPart{(w_2)}, s', \amod{}') \in \extendR(l_2, s_2, \amod{2}) \label{L12:Goal}
\end{align}
From (\ref{L12:Ass1}) we know:
%
\begin{align}
	s_1 = s_2 \label{L12:Ass3}\\
	\amod{1} = \amod{2} \label{L12:Ass4}
\end{align}
%
By definition of \extendG and from (\ref{L12:Ass2}), (\ref{L12:Ass3}) and (\ref{L12:Ass4}) we know:
%
\begin{align}
	\exsts{l_3, l_4, \ca{1}, \ca{2}, s''} \hspace*{0.2cm}& l_1 = l_3 \composeL l_4 \;\land\; l'_1 = l_3 \composeL (\unitH, \ca{1}) \;\land \label{L12:Ass5}\\
	& s'' = l_4 \composeL (\unitH, \ca{2}) \;\land\; s' = s_2 \composeL s'' \label{L12:Ass6}\\
%
	\exsts{F, \amod{0}} \hspace*{0.2cm} & \for{\ca{} \in \left(\{\ca{1}, \ca{2}\} \cup \dom{\amod{0}}\right)} \for{\ca{}' \in \dom{\amod{}}}\;\; \ca{} \disjoint \ca{} \label{L12:Ass7}\\
%
	& s'' \in \fence{} \;\land\; \fence{} \strictfences \amod{0} \label{L12:fence}\\
%	
	&\for{l_7, r_7, \amod{7}} l_7 \composeL r_7 = s_2 \land \extendsAM{\amod{}}{l_7}{r_7}{\amod{7}} \implies \nonumber\\ 
	&\extendsAM{\amod{}'}{l_7}{r_7 \composeL s''}{\amod{7}}  \label{L12:Ass9}\\
%	& \extendsAM{\amod{}'}{s_2}{s''}{\amod{2}} \label{L12:Ass9}\\
%
	& \extendsAM{\amod{}'}{s''}{s_2}{\amod{0}}  \label{L12:Ass10}
\end{align}
%
%Since $\wf{l_1, s_1, \amod{1}}$, we know 
%\begin{equation}
%	\contains{\dom{\amod{1}}}{\capPart{(l_1 \composeL s_1)}} \nonumber
%\end{equation}
%%
%and consequently from (\ref{L12:Ass4}) and (\ref{L12:Ass5})
%\begin{equation}
%	\contains{\dom{\amod{2}}}{\capPart{(l_4)}} \label{L12:Ass11}
%\end{equation}
%%
%From (\ref{L12:Ass5}), (\ref{L12:Ass6}) and (\ref{L12:Ass11}), we have
%%
%\begin{equation}
%	\contains{K \cup \dom{\amod{2}}}{\capPart{s''}} \label{L12:Ass12}
%\end{equation}
%%
%From (\ref{L12:Ass5})-(\ref{L12:Ass10}) and (\ref{L12:Ass12}) we have:
From (\ref{L12:Ass6}) and (\ref{L12:Ass9}) we have:
%
\begin{equation}
	(\localPart{(w_2)}, s', \amod{}') = (l_2, s', \amod{}') \in \extendR(l_2, s_2, \amod{2}) \nonumber
\end{equation}
% 
%and consequently by definition of \rely, 
%%
%\begin{equation}
%	(\localPart{(w_2)}, s', \amod{}') \in \rely(l_2, s_2, \amod{2}) \nonumber
%\end{equation}
%% 
as required.
\end{proof}
%
%
\end{lemma}
%
%
\begin{lemma}[] \label{lem:guaranteeContainment}
%
\[
\begin{array}{l}
	\for{w_1, w_2, w \in \Worlds} w_1 \composeW w_2 = w \implies\\
	\hspace*{1cm}(l', s', \amod{}') \in \guarantee(w_1) \implies (\localPart{(w_2)}, s', \amod{}') \in \rely(w_2)
\end{array}
\]
%
\begin{proof} Pick an arbitrary $w_1, w_2, w$ and $(l'_1, s', \amod{}')$ such that:
%
\begin{align}
	w_1 \composeW w_2 = w \label{L13:Ass1}\\
	(l'_1, s', \amod{}') \in \guarantee(w_1) \label{L13:Ass2}
\end{align}
%
\textbf{RTS.}
%
\begin{align}
	(\localPart{(w_2)}, s', \amod{}') \in \rely(w_2) \label{L13:Goal}
\end{align}
From (\ref{L13:Ass2}) and by definition of \guarantee\ we know:
%
\begin{align}
	(l'_1, s', \amod{}') \in \left(\updateG \cup \extendG \right)^{*}(w_2) \label{L13:Ass3}
\end{align}
%
From (\ref{L13:Ass1}), (\ref{L13:Ass3}) and by Lemmata \ref{lem:updateGContainment} and \ref{lem:extendGContainment} we have:
%
\begin{align}
	(\localPart{(w_2)}, s', \amod{}') \in \left(\updateR \cup \extendR \right)^{*}(w_2) \nonumber
\end{align}
%
and consequently 
%
\begin{align}
	(\localPart{(w_2)}, s', \amod{}') \in \rely(w_2) \nonumber
\end{align}
%
as required.
\end{proof}
\end{lemma}


%\begin{lemma}[]
%%
%\[
%		\fence{} \fences I \;\land\; I \weakenI{\fence{}} I_1 \implies \fence{} \fences I_1
%\]
%%
%\begin{proof}
%\todo
%\end{proof}
%\end{lemma}
%%
%%
%
%
%
\begin{lemma}[]\label{lem:nonEmptyOverlap}Given any separation algebra $(\mathbb{A}, \compose{\mathbb{A}}, \unit{\mathbb{A}})$ with the cross-split property, for any $a, b \in \mathbb{A}$:
%
\[
	\exsts{c \in \mathbb{A}} a \leq b \compose{\mathbb{A}} c \iff a \meet{\mathbb{A}} b \not= \emptyset
\]
%
$\m{Proof} \implies$. We proceed with proof by contradiction.
Take arbitrary $a, b, c \in \mathbb{A}$ such that 
%
\begin{equation}
	a \leq b \composeL c \label{L5:Ass1}
\end{equation}
%
and assume
%
\begin{equation}
	a \meet{\mathbb{A}} b = \emptyset \label{L5:Ass2}
\end{equation}
%
From (\ref{L5:Ass2}) and by definition of $\meet{\mathbb{A}}$, we have:
%
\begin{equation}
	\neg\exsts{d, e, f, g} a = d \compose{\mathbb{A}} e /| b = e \compose{\mathbb{A}} f /| d \compose{\mathbb{A}} e \compose{\mathbb{A}} f = g \label{L5:Ass3}
\end{equation}
%
From (\ref{L5:Ass1}) we have $\exsts{h} a \compose{\mathbb{A}} h = b \compose{\mathbb{A}} c$ and consequently by the cross-split property we have:
%
\begin{align}
	\exsts{ab, ac, hb, hc, t} &
	ab \compose{\mathbb{A}} ac = a 	\label{L5:Ass4}\\
	& hb \compose{\mathbb{A}} hc = h \label{L5:Ass5}\\
	& ab \compose{\mathbb{A}} hb = b \label{L5:Ass6}\\
	& ac \compose{\mathbb{A}} hc = c \label{L5:Ass7}\\
	& t = ab \compose{\mathbb{A}} ac \compose{\mathbb{A}} hb \compose{\mathbb{A}} hc \label{L5:Ass8}
\end{align}
%
From (\ref{L5:Ass8}) we have:
%
\begin{equation}
	\exsts{s} ab \compose{\mathbb{A}} ac \compose{\mathbb{A}} hb = s \label{L5:Ass9}
\end{equation}
%
From (\ref{L5:Ass4}), (\ref{L5:Ass6}) and (\ref{L5:Ass9}) we have: 
%
\begin{equation}
	\exsts{d, e, f, g} a = d \compose{\mathbb{A}} e /| b = e \compose{\mathbb{A}} f /| d \compose{\mathbb{A}} e \compose{\mathbb{A}} f = g \label{L5:Ass10}
\end{equation}
%
From (\ref{L5:Ass3}) and (\ref{L5:Ass10}) we derive a contradiction and can hence deduce:
%
\begin{equation}
	a \meet{\mathbb{A}} b \not= \emptyset \nonumber
\end{equation}
%
as required.\\


\noindent$\m{Proof} \Leftarrow$. Take arbitrary $a, b \in \mathbb{A}$ such that 
%
\begin{align*}
	a \meet{\mathbb{A}}	b \not= \emptyset
\end{align*}
%
Then by definition of $\meet{\mathbb{A}}$ we have: 
%
\begin{align*}
	\exsts{d, e, f \in \mathbb{A}} & a = d \compose{\mathbb{A}} e\\
	& b = e \compose{\mathbb{A}} f \\
	& d \compose{\mathbb{A}} e \compose{\mathbb{A}} f \text{ is defined}
\end{align*}
% 
and thus by definition of $\leq$ we have $a \leq b \compose{\mathbb{A}} d$ and consequently
%
\begin{align*}
	\exsts{c} a \leq b \compose{\mathbb{A}} c
\end{align*}
%
as required.
\qed
\end{lemma}
%
%
\begin{lemma}[]\label{lem:divideUpper}
%
Given any separation algebra ($\mathcal{M}, \compose{\mathcal{M}}, \unit{\mathcal{M}}$) with the cross-split property:
\[
	\for{a, b, c \in \mathcal{M}} a \leq b \compose{\mathcal{M}} c \implies \exsts{m, n} a = m \compose{\mathcal{M}} n \;\land\; m \leq b \;\land\; n \leq c
\]
%
\begin{proof}
Pick an arbitrary $a, b, c \in \mathcal{M}$. Since $a \leq b \compose{\mathcal{M}} c$, we know $\exsts{d \in \mathcal{M}}$ such that:
%
\begin{equation}
	a \compose{\mathcal{M}} d = b \compose{\mathcal{M}} c \label{L6:Ass1}
\end{equation}
%
By the cross-split property of $\mathcal{M}$, We can deduce: $\exsts{ab, ac, db, dc \in \mathcal{M}}$ such that:
%
\begin{align}
	a = ab \compose{\mathcal{M}} ac \label{L6:Ass2}\\
	b = ab \compose{\mathcal{M}} db \label{L6:Ass3}\\
	c = ac \compose{\mathcal{M}} dc \label{L6:Ass4}\\
	d = db \compose{\mathcal{M}} dc \nonumber 
\end{align}
%
Since $ab \leq b$ (\ref{L6:Ass3}) and $ac \leq c$ (\ref{L6:Ass4}), from (\ref{L6:Ass2}) we can deduce:
%
\begin{equation}
	\exsts{m, n \in \mathcal{M}} a = m \compose{\mathcal{M}} n \;\land\; m \leq b \;\land\; n \leq c \nonumber
\end{equation}
%
as required.
\end{proof}
\end{lemma}
%
%
\begin{lemma}[]\label{lem:disjointByOrder}
Given any separation algebra ($\mathcal{M}, \compose{\mathcal{M}}, \unit{\mathcal{M}}$),
\[
	\for{a, b, c, d \in \mathcal{M}} a \compose{\mathcal{M}} b = d \;\land\; c \leq b \implies 
	\exsts{f \in \mathcal{M}} a \compose{\mathcal{M}} c = f
\]
%
\begin{proof}
Pick an arbitrary $a, b, c, d \in \mathcal{M}$ such that:
%
\begin{align}
	a \compose{\mathcal{M}} b = d \label{L7:Ass1}\\
	c \leq b \label{L7:Ass2}
\end{align}
%
From (\ref{L7:Ass2}), we have: 
%
\begin{equation}
	\exsts{e \in \mathcal{M}} c \compose{\mathcal{M}} e = b \label{L7:Ass3}
\end{equation}
%
and consequently from (\ref{L7:Ass1}) we have:
%
\begin{equation}
	a \compose{\mathcal{M}} c \compose{\mathcal{M}} e = d \label{L7:Ass4}
\end{equation}
%
Since $e \leq d$ (\ref{L7:Ass4}), we have: 
%
\begin{equation}
	\exsts{f \in \mathcal{M}} e \compose{\mathcal{M}} f = d \label{L7:Ass5}
\end{equation} 
%
From (\ref{L7:Ass4}), (\ref{L7:Ass5}) and cancellativity of separation algebras we have:
%
\begin{equation}
	a \compose{\mathcal{M}} c = f
\end{equation}
%
and thus
%
\begin{equation}
	\exsts{f \in \mathcal{M}} a \compose{\mathcal{M}} c = f
\end{equation}
%
as required.
\end{proof}
\end{lemma}
%
%


