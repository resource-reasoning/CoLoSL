\begin{lemma}[\forgetRule-Closure]\label{lem:forget-closure}
For all $\lmod, \lmod', \gmod \in \AMods$ and $s_1, s_2, r \in \LStates$
%
\[
\begin{array}{l}
	\extendsAM{\lmod, \gmod}{s_1 \composeL s_2}{r}{\lmod'} \implies 
	\extendsAM{\lmod, \gmod}{s_1}{s_2 \composeL r}{\lmod'}
\end{array}
\]
%
\begin{proof} Pick an arbitrary $\lmod, \lmod', \gmod \in \AMods$ and $s_1, s_2, r \in \LStates$ such that 
%
\begin{align}
	\extendsAM{\lmod, \gmod}{s_1 \composeL s_2}{r}{\lmod'} \label{FC:Ass1}
\end{align} 
%
From the definition of $\downarrow$, it then suffices to show
%
\begin{align}
	& \lmod' \subseteq \lmod \label{FC:Goal1}\\
	& \for {n \in \Nats}  \extendsAMUpto{\lmod, \gmod}{n}{s_1}{s_2 \composeL r}{\lmod'} \label{FC:Goal2}
%	& \for {n \in \Nats} \for{s_1, s_2, r \in \LStates} \extendsAMUpto{\lmod, \gmod}{n}{s_1 \composeL s_2}{r}{\lmod'} \implies \extendsAMUpto{\lmod, \gmod}{n}{s_1}{s_2 \composeL r}{\lmod'} \label{LFC:Goal2}
\end{align}
%

\noindent\textbf{RTS. (\ref{FC:Goal1})} \\
This follows trivially from (\ref{FC:Ass1}) and the definition of $\downarrow$.\\

\noindent\textbf{RTS. (\ref{FC:Goal2})} \\
Rather than proving (\ref{FC:Goal2}) directly, we first establish the following.
%
\begin{align}
	\for {n \in \Nats} \for{s_1, s_2, r \in \LStates} & \nonumber\\
	& \hspace{-2cm}\extendsAMUpto{\lmod, \gmod}{n}{s_1 \composeL s_2}{r}{\lmod'} \implies \extendsAMUpto{\lmod, \gmod}{n}{s_1}{s_2 \composeL r}{\lmod'} \label{FC:Goal3}
\end{align}
%
We can then despatch (\ref{FC:Goal2}) from (\ref{FC:Ass1}) and (\ref{FC:Goal3}); since for an arbitrary $n \in \Nats$, from (\ref{FC:Ass1}) and the definition of $\downarrow$ we have $\extendsAMUpto{\lmod, \gmod}{n}{s_1 \composeL s_2}{r}{\lmod'}$ and consequently from (\ref{FC:Goal3}) we derive $\extendsAMUpto{\lmod, \gmod}{n}{s_1}{s_2 \composeL r}{\lmod'} $ as required. \\

\noindent\textbf{RTS. (\ref{FC:Goal3})} \\
We proceed by induction on the number of steps $n$.\\

%\noindent Pick an arbitrary $s_1, s_2, r \in \LStates, \lmod, \lmod', \gmod \in \AMods$.\\
\noindent\textbf{Base case }$n=0$\\
Pick an arbitrary $s_1, s_2, r \in \LStates$. We are then required to show	$\extendsAMUpto{\lmod, \gmod}{0}{s_1}{s_2 \composeL r}{\lmod'} $ which follows trivially from the definition of $\downarrow_0$.\\

\noindent\textbf{Inductive Step} Pick an arbitrary $n \in \Nats$ and $s_1, s_2, r \in \LStates$ such that
%
\begin{align}
	& \extendsAMUpto{\lmod, \gmod}{n}{s_1 \composeL s_2}{r}{\lmod'} \label{LCS:Ass1}\\
%
	&	\for{s_1, s_2, r \in \LStates} \nonumber \\	
	& \tag{I.H.} \quad\extendsAMUpto{\lmod, \gmod}{(n-1)}{s_1 \composeL s_2}{r}{\lmod'} \implies 
										\extendsAMUpto{\lmod, \gmod}{(n-1)}{s_1}{s_2 \composeL r}{\lmod'} \label{LCS:IH}
\end{align}
%
\textbf{RTS. }
%
\begin{align}
	& 
	\V{\ca{}}  \V{a\in \lmod'(\ca{})} \nonumber \\
  &\quad (\m{potential}(a, s_1 \composeL s_2\composeL r) /| \m{visible}(a, s_1)=> \nonumber\\
  & \qquad \qquad\for{(s', r') \in a[s_1, s_2 \composeL r]} \extendsAMUpto{\lmod, \gmod}{(n-1)}{s'}{r'}{\lmod'}) \label{LCS:Goal0}\\
%  
  &\quad (\m{potential}(a, s_1 \composeL s_2\composeL r) /| \neg\m{visible}(a, s_1)=> \nonumber\\
  & \qquad \qquad \extendsAMUpto{\lmod, \gmod}{(n-1)}{s_1}{a[s_1 \composeL s_2 \composeL r] - s_1}{\lmod'}) \label{LCS:Goal1}\\
%    
  &\quad\land \m{enabled}(a,s_1 \composeL s_2 \composeL r)
  => (s_1 \composeL s_2 \composeL r,
  a[s_1 \composeL s_2 \composeL r])\in \gmod(\ca{}))
  /|\null \label{LCS:Goal2}\\
%  
  &\V{\ca{}}\V{a\in \lmod(\ca{})}
  \m{potential}(a,s_1 \composeL s_2 \composeL r) =>\null \nonumber \\
  &\ \m{reflected}(a,s_1 \composeL s_2 \composeL r,\lmod'(\ca{})) |/\null \nonumber \\
%  
  &\ \neg\m{visible}(a,s_1) /| \exsts{r'} a[s_1 \composeL s_2 \composeL r] = s_1 \composeL r' /| \extendsAMUpto{\lmod, \gmod}{(n-1)}{s_1}{r'}{\lmod'}  \label{LCS:Goal3}
%
\end{align}
%
\textbf{RTS. (\ref{LCS:Goal0})}\\
Pick an arbitrary $\ca{}$, $a \in \lmod'(\ca{})$ and $(s', r')$ such that 
\begin{align}
	\m{potential}(a, s_1 \composeL s_2 \composeL r) /| \m{visible}(a, s_1) \label{LCS:Ass2}\\
	(s', r') \in a[s_1, s_2 \composeL r] \label{LCS:Ass3}
\end{align}
%
Then from the definition of $a[s_1, s_2 \composeL r]$, the cross-split property and since $\m{visible}(a, s_1)$, we have:
%
\begin{align}
	\exsts{ps_1 > \unitL, ps_2, pr, s'_1, s'_2, r'}\ & s_1 = ps_1 \composeL  s'_1 \label{LCS:Ass4-1}\\
	& s_2 = ps_2 \composeL s'_2 \label{LCS:Ass4-2}\\
	& r = pr \composeL r' \label{LCS:Ass4-3}\\
	& \m{fst}(\updateFP{a}) = ps_1 \composeL ps_2 \composeL pr \label{LCS:Ass4-4}\\
	& \m{snd}(\updateFP{a}) \compatible s'_1 \composeL s'_2 \composeL r'  \label{LCS:Ass4-5}\\
	& s' = \m{snd}(\updateFP{a}) \composeL s'_1 \label{LCS:Ass4-6}\\
	& r' = s'_2 \composeL r' \label{LCS:Ass4-7}
\end{align}
% 
Then from (\ref{LCS:Ass4-1})-(\ref{LCS:Ass4-5}) and by definition of $a[s_1 \composeL s_2, r]$ we know $\m{visible}(a, s_1 \composeL s_2) /| (\m{snd}(\updateFP{a}) \composeL s'_1 \composeL s'_2, r') \in a[s_1 \composeL s_2, r]$. Consequently from (\ref{LCS:Ass1}) and (\ref{LCS:Ass2}) we have:
%
\begin{equation}
	\extendsAMUpto{\lmod, \gmod}{(n-1)}{\m{snd}(\updateFP{a} \composeL s'_1 \composeL s'_2}{r'}{\lmod'} \nonumber
\end{equation}
%
and thus from (\ref{LCS:IH})
%
\begin{align}
	\extendsAMUpto{\lmod, \gmod}{(n-1)}{\m{snd}(\updateFP{a} \composeL s'_1}{s'_2 \composeL r'}{\lmod'} \label{LCS:Ass5}
\end{align}
% 
Finally, from (\ref{LCS:Ass4-6}), (\ref{LCS:Ass4-7}) and (\ref{LCS:Ass5}) we have: 
%
\begin{align}
	\extendsAMUpto{\lmod, \gmod}{(n-1)}{s'}{r'}{\lmod'} \nonumber
\end{align}
% 
as required.\\

%
%
%
\noindent\textbf{RTS. (\ref{LCS:Goal1})}\\
Pick an arbitrary $\ca{}$, $a \in \lmod'(\ca{})$ and $s''$ such that 
\begin{align}
	\m{potential}(a, s_1 \composeL s_2 \composeL r) /| \neg\m{visible}(a, s_1) \label{LCS:ass2}\\
	s'' = a[s_1 \composeL s_2 \composeL r] \label{LCS:ass3}
\end{align}
%
There are two cases to consider:\\
\textbf{Case 1. }$\m{visible}(a, s_1 \composeL s_2, r)$\\
Since $a[s_1 \composeL s_2 \composeL r]$ is defined (\ref{LCS:ass2}), we know $\exsts{(s', r')} \in a[s_1 \composeL s_2, r]$. Consequently, from the definition of $a[s_1 \composeL s_2, r]$, the cross-split property and since $\neg\m{visible}(a, s_1)$ (\ref{LCS:ass2}) and $\m{visible}(a, s_2)$ (Assumption of case 1.) and from (\ref{LCS:ass3}) we have:
%
\begin{align}
	\exsts{ps_2, pr, s'_2, r'}\ & s_2 = ps_2 \composeL s'_2 \label{LCS:ass4-1}\\
	& r = pr \composeL r' \label{LCS:ass4-2}\\
	& \m{fst}(\updateFP{a}) = ps_2 \composeL pr \label{LCS:ass4-3}\\
	& \m{snd}(\updateFP{a}) \compatible s_1 \composeL s'_2 \composeL r'  \label{LCS:ass4-4}\\
	& s' = \m{snd}(\updateFP{a}) \composeL s_1 \composeL  s'_2 \label{LCS:ass4-5}\\
	& s'' = \m{snd}(\updateFP{a}) \composeL s_1 \composeL  s'_2 \composeL r' \label{LCS:ass4-6}
\end{align}
% 
Consequently, from (\ref{LCS:ass2}), (\ref{LCS:Ass1}) and assumption of case 1. we have:
%
\begin{align*}
	\extendsAMUpto{\lmod, \gmod}{(n-1)}{s'}{r'}{\lmod'}
\end{align*}
% 
and consequently, from (\ref{LCS:ass4-5}) and (\ref{LCS:IH})
%
\begin{align*}
	\extendsAMUpto{\lmod, \gmod}{(n-1)}{s_1}{\m{snd}(\updateFP{a}) \composeL  s'_2 \composeL r'}{\lmod'}
\end{align*}
%
and finally, from (\ref{LCS:ass4-6}), (\ref{LCS:ass3}) and the definition of $a[s_1 \composeL s_2 \composeL r] - s_1$, we have:
%
\begin{align*}
	\extendsAMUpto{\lmod, \gmod}{(n-1)}{s_1}{a[s_1 \composeL s_2 \composeL r] - s_1}{\lmod'}
\end{align*}
% 
as required.\\
%
\noindent\textbf{Case 2. }$\neg\m{visible}(a, s_1 \composeL s_2, r)$\\
From the assumption of case 2, (\ref{LCS:ass2}) and (\ref{LCS:Ass1}) we have:
%
\begin{align*}
	\extendsAMUpto{\lmod, \gmod}{(n-1)}{s_1 \composeL s_2}{a[s_1 \composeL s_2 \composeL r] - (s_1 \composeL s_2)}{\lmod'}
\end{align*}
% 
and consequently from (\ref{LCS:IH})
%
\begin{align*}
	\extendsAMUpto{\lmod, \gmod}{(n-1)}{s_1}{a[s_1 \composeL s_2 \composeL r] - (s_1 \composeL s_2) \composeL s_2}{\lmod'}
\end{align*}
% 
that is, 
%
\begin{align*}
	\extendsAMUpto{\lmod, \gmod}{(n-1)}{s_1}{a[s_1 \composeL s_2 \composeL r] - s_1}{\lmod'}
\end{align*}
% 
as required.\\
%%
%%
%%



\noindent\textbf{RTS. (\ref{LCS:Goal2})}\\
Pick an arbitrary $\ca{}$ and $a \in \lmod'(\ca{})$ such that 
\begin{align}
	\m{enabled}(a, s_1 \composeL s_2 \composeL r) \label{LCS:Ass9}
\end{align}
%
Then from (\ref{LCS:Ass1}) and (\ref{LCS:Ass9}) we have:
%
\begin{align}
	(s_1 \composeL s_2 \composeL r, a[s_1 \composeL s_2 \composeL r]) \in \gmod(\ca{}) \nonumber
\end{align}
% 
as required.

\noindent\textbf{RTS. (\ref{LCS:Goal3})}\\
Pick an arbitrary $\ca{}$ and $a \in \lmod(\ca{})$ such that 
\begin{align}
	\m{potential}(a, s_1 \composeL s_2 \composeL r) \label{LCS:Ass12}
\end{align}
%(\ref{LCS:Ass})
Then from (\ref{LCS:Ass1}) we have: 
%
\begin{align*}
  &\m{reflected}(a,s_1 \composeL s_2 \composeL r,\lmod'(\ca{})) |/\null \nonumber \\
%  
  &\neg\m{visible}(a,s_1 \composeL s_2) /| \exsts{r'} a[s_1 \composeL s_2 \composeL r] = s_1 \composeL s_2 \composeL r' /| \extendsAMUpto{\lmod, \gmod}{(n-1)}{s_1 \composeL s_2}{r'}{\lmod'}  \nonumber
\end{align*}
%
and consequently from (\ref{LCS:IH})
%
\begin{align*}
  &\m{reflected}(a,s_1 \composeL s_2 \composeL r,\lmod'(\ca{})) |/\null \nonumber \\
%  
  &\neg\m{visible}(a,s_1 \composeL s_2) /| \exsts{r'} a[s_1 \composeL s_2 \composeL r] = s_1 \composeL s_2 \composeL r' /| \extendsAMUpto{\lmod, \gmod}{(n-1)}{s_1}{s_2 \composeL r'}{\lmod'}  \nonumber
\end{align*}
%
%Then from (\ref{LCS:Ass1}), the definition of $a[s_1 \composeL s_2, r]$, the cross-split property and the cancellativity of logical states separation algebras we have:
%%
%\begin{align}
%&\begin{array}{l}
%	\m{reflected}(a, s_1 \composeL s_2 \composeL r, \lmod'(\ca{}) \lor\\
%	(\neg\m{visible}(a, s_1 \composeL s_2) \land \\
%	\begin{array}{@{} l l @{}}
%		\exsts{ps_1, ps_2, s'_1, s'_2, pr, r', s''} & s_1 = ps_1 \composeL s'_1\\
%		& s_2 = ps_2 \composeL s'_2\\
%		& r = pr \composeL r'\\
%		& \m{fst}(\updateFP{a}) = ps_1 \composeL ps_2 \composeL pr\\
%		& \m{snd}(\updateFP{a}) = ps_1 \composeL ps_2 \composeL s'' \land\\
%		& \extendsAMUpto{\lmod, \gmod}{(n-1)}{s_1 \composeL s_2}{s'' \composeL r'}{\lmod'}
%	\end{array}\\
%	)
%\end{array}
%\label{LCS:Ass13}
%\end{align}
%% 
%Consequently from (\ref{LCS:IH}) we have: 
%\begin{align}
%&\begin{array}{l}
%	\m{reflected}(a, s_1 \composeL s_2 \composeL r, \lmod'(\ca{}) \lor\\
%	(\neg\m{visible}(a, s_1 \composeL s_2) \land \\
%	\begin{array}{@{} l l @{}}
%		\exsts{ps_1, ps_2, s'_1, s'_2, pr, r', s''} & s_1 = ps_1 \composeL s'_1\\
%		& s_2 = ps_2 \composeL s'_2\\
%		& r = pr \composeL r'\\
%		& \m{fst}(\updateFP{a}) = ps_1 \composeL ps_2 \composeL pr\\
%		& \m{snd}(\updateFP{a}) \composeL s'_1= s_1 \composeL ps_2 \composeL s'' \land\\
%		& \extendsAMUpto{\lmod, \gmod}{(n-1)}{s_1}{ps_2 \composeL s'' \composeL s'_2 \composeL r'}{\lmod'}
%	\end{array}\\
%	)
%\end{array}
%\nonumber
%\end{align}
%% 
%%
%and by definition of $a[s_1, s_2 \composeL r]$ we have:
%\begin{align}
%&\begin{array}{l}
%	\m{reflected}(a, s_1 \composeL s_2 \composeL r, \lmod'(\ca{}) \lor\\
%	(\neg\m{visible}(a, s_1 \composeL s_2) \land \\
%	\begin{array}{@{} l l @{}}
%		\exsts{s'''} & \m{fst}(a[s_1, s_2 \composeL r]) = s_1 \composeL s''' \land\\
%		& \extendsAMUpto{\lmod, \gmod}{(n-1)}{s_1}{s''' \composeL \m{snd}(a[s_1, s_2 \composeL r])}{\lmod'}
%	\end{array}\\
%	)
%\end{array}
%\label{LCS:Ass14}
%\end{align}
Finally from the definition of $\m{visible}$ and by \lem~\ref{lem:disjointByOrder} we can rewrite the above as: 
%
%
\begin{align*}
  &\m{reflected}(a,s_1 \composeL s_2 \composeL r,\lmod'(\ca{})) |/\null \nonumber \\
%  
  &\neg\m{visible}(a,s_1) /| \exsts{r'} a[s_1 \composeL s_2 \composeL r] = s_1 \composeL r' /| \extendsAMUpto{\lmod, \gmod}{(n-1)}{s_1}{r'}{\lmod'}  \nonumber
\end{align*}
%
%
%\begin{align}
%&\begin{array}{l}
%	\m{reflected}(a, s_1 \composeL s_2 \composeL r, \lmod'(\ca{}) \lor\\
%	(\neg\m{visible}(a, s_1) \land \\
%	\begin{array}{@{} l l @{}}
%		\exsts{s'''} & \m{fst}(a[s_1, s_2 \composeL r]) = s_1 \composeL s''' \land\\
%		& \extendsAMUpto{\lmod, \gmod}{(n-1)}{s_1}{s''' \composeL \m{snd}(a[s_1, s_2 \composeL r])}{\lmod'}
%	\end{array}\\
%	)
%\end{array}
%\nonumber
%\end{align}
%%
as required.
%
\end{proof}
\end{lemma}
%
%