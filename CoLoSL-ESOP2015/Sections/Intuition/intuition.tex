\section{Informal Development}
\label{sec:intuition}

We illustrate our \colosl reasoning principles by sketching a proof of
a variation of Dijkstra's token ring mutual exclusion
algorithm~\cite{dijkstra74}.  These principles are laid out in
\fig\ref{fig:principles}, together with the usual concurrency rule of
separation logic~\cite{csl-tcs}, and the rule of consequence from the 
Views framework~\cite{views}. Note that $(P => Q) => (P ===> Q)$.  We will
introduce them informally as needed during the proof of our example,
and present them in more details in \S\ref{sec:colosl}.

\begin{figure}
\centering
\noindent\hrule
\begin{align*}
  \shared{P * Q}{I} &=> \shared{P}{I}  &\mathllap{(\forgetRule)}&
  &
  \shared{P}{I_1}\!\! * \shared{Q}{I_2} &=> \shared{P \sepish Q}{I_1
    \cup I_2}
  \tag{\mergeRule}
  \\
  (P => Q)
  &=>
  \shared{P}{I} => \shared{Q}{I}
  &(\weakenRule)&
  &
  I \weakenI{P} I'
  \text{ implies }&
  \shared{P}{I} ===> \shared{P}{I'}
  &\tag{\shiftRule}
  \\
  \shared{P}{I} &
  => \shared{P}{I} * \shared P I
  &(\copyRule)&
  &
  P \containI I 
%  \text{ and }
%  \fresh{\vec{x}, \capAss{1} * \capAss{2}}
  \text{ implies }&
  P ===>
  \exsts{\token A, \token A'} [\token A] * \shared{P * [\token A']}{I}
  \tag{\extendRule}
\end{align*}

\vspace{-15pt}
\begin{mathpar}
\infer[\parRule]
	{
		\{P_1 * P_2\} \;\mathbb{C}_1 || \mathbb{C}_2\; \{Q_1 * Q_2\}		
	}
	{
		\{P_1\} \;\mathbb{C}_1\; \{Q_1\}
		\\
		\{P_2\} \;\mathbb{C}_2\; \{Q_2\}
	}

\infer[\conseqRule]
	{
	  \hoare{P}{\mathbb{C}}{Q}
	}
	{
          P ===> P'&
	  \hoare{P'}{\mathbb{C}}{Q'}&
          Q' ===> Q
	}
\end{mathpar}


\hrule
\caption{\colosl main reasoning principles.}
\label{fig:principles}
\end{figure}

\colosl introduces a new assertion $\shared{P} I$ called a
\emph{subjective view}, which comprises an assertion $P$ describing
\emph{parts} of the global shared state and an interference assertion
$I$ which characterises how this partial shared state may be changed
by the thread or the environment. Similar to interference assertions
of CAP, $I$ declares transitions of the form $[\token a] : Q \swap R$,
where a thread in possession of the $[\token a]$ capability (in its
private state) may carry out its transition and update parts of the
shared state that satisfy $Q$ to those described by $R$. Assertions in
Hoare triples must be {\em stable}; that is, robust with respect to
interferences from the environment (as prescribed in the interference
relations attached to that assertion's subjective views). The $*$
(resp.\ $**$) connective is that of separation logic and means that
the current state is a disjoint (resp.\ potentially overlapping)
composition of two states satisfying each sub-formula. The overlapping
conjunction $**$ has been used in the past to reason about sharing in
data structures~\cite{rey-slnotes,js-popl12,ramification}.

\begin{figure}
\centering
\noindent\hrule\vspace{10pt}
\begin{tabular}{@{} l @{\hspace{25pt}} l@{\ }}
$\mathbb{INC}$:&
$\mathbb{P}_{\li x}$:
\vspace{-5pt}\\
{\hspace{20pt}\begin{lstlisting}[numbers=left,numbersep=10pt]
//$\color{blue}\{\tx x|-> - * \tx{y}|-> - * \tx{z}|-> - \}$
x = 0; y = 0; z = 0;
//$\color{blue} \{\tx x|-> 0 * \tx{y}|-> 0 * \tx{z}|-> 0 \}$
//$\color{blue} \left\{\begin{array}{@{}l<{\null}@{}l<{\null}@{}}\shared{\tx x|-> 0 * \tx{y}|-> 0 * \tx{z}|-> 0} I\\ \null*[\token a_{\tx x}] * [\token a_{\tx y}] * [\token a_{\tx z}]\end{array}\right\}$
($\mathbb{P}_{\tx x} \ \mid\ \mid\ \mathbb{P}_{\tx y} \ \mid\ \mid\  \mathbb{P}_{\tx z}$)
//$\color{blue}\left\{\begin{array}{@{}l<{\null}@{}l<{\null}@{}}\shared{\tx x|-> \!\!10 * \tx{y}|-> \!\!10 * \tx{z}|-> \!\!10} I\\ \null*[\token a_{\tx x}] * [\token a_{\tx y}] * [\token a_{\tx z}]\end{array}\right\}$
\end{lstlisting}}
&
\begin{lstlisting}
//$\color{blue} \{\shared{\cell{\tx z}{0} * \cell{\tx x}{0}}{I_{\tx x}'} * [\token a_{\tx x}]\}$
while(x != 10)$\{$
//$\color{blue} \left\{\shared{\begin{array}{@{}l<{\null}@{}l<{\null}@{}}\exsts{v}\cell{\tx z}{v} * \cell{\tx x}{v} \lor\\ \cell{\tx z}{v} * \cell{\tx x}{v+1}\end{array}}{I_{\tx x}'}\!\!\!\!\!\! * [\token a_{\tx x}]\right\}$
  $\langle$if (x == z) x++;$\rangle$ 
$\}$
//$\color{blue} \left\{\shared{\begin{array}{@{}l<{\null}@{}l<{\null}@{}}\cell{\tx z}{10} * \cell{\tx x}{10} \lor\\ \cell{\tx z}{9} * \cell{\tx x}{10}\end{array}}{I_{\tx x}'} * [\token a_{\tx x}]\right\}$
\end{lstlisting}\\
{$\mathbb{P}_{\li y}$:}&
{$\mathbb{P}_{\li z}$:}
\vspace{-5pt}\\
\begin{lstlisting}
//$\color{blue}\left\{\shared{\begin{array}{@{}l<{\null}@{}l<{\null}@{}}\cell{\tx x}{0} * \cell{\tx y}{0} \lor\\ \cell{\tx x}{1} * \cell{\tx y}{0}\end{array}}{I_{\tx y}'}\!\!\!\!\!\! * [\token a_{\tx y}]\right\}$
while(y != 10)$\{$
//$\color{blue}\left\{\shared{\begin{array}{@{}l<{\null}@{}l<{\null}@{}}\exsts{v}\cell{\tx x}{v} * \cell{\tx y}{v} \lor\\ \cell{\tx x}{v+1} * \cell{\tx y}{v}\end{array}}{I_{\tx y}'}\!\!\!\!\!\! * [\token a_{\tx y}]\right\}$
  $\langle$if (y $<$ x) y++;$\rangle$ 
$\}$
//$\color{blue}\left\{\shared{\begin{array}{@{}l<{\null}@{}l<{\null}@{}}\cell{\tx x}{10} * \cell{\tx y}{10} \lor\\ \cell{\tx x}{11} * \cell{\tx y}{10}\end{array}}{I_{\tx y}'}\!\!\!\!\!\! * [\token a_{\tx y}]\right\}$
\end{lstlisting}
&
\begin{lstlisting}
//$\color{blue}\left\{\shared{\begin{array}{@{}l<{\null}@{}l<{\null}@{}}\cell{\tx y}{0} * \cell{\tx z}{0} \lor\\ \cell{\tx y}{1} * \cell{\tx z}{0}\end{array}}{I_{\tx z}'}\!\!\!\!\!\! * [\token a_{\tx z}]\right\}$
while(z != 10)$\{$
//$\color{blue}\left\{\shared{\begin{array}{@{}l<{\null}@{}l<{\null}@{}}\exsts{v}\cell{\tx y}{v} * \cell{\tx z}{v} \lor\\ \cell{\tx y}{v+1} * \cell{\tx z}{v}\end{array}}{I_{\tx z}'}\!\!\!\!\!\! * [\token a_{\tx z}]\right\}$
  $\langle$if (z $<$ y) z++;$\rangle$
$\}$
//$\color{blue}\left\{\shared{\begin{array}{@{}l<{\null}@{}l<{\null}@{}}\cell{\tx y}{10} * \cell{\tx z}{10} \lor\\ \cell{\tx y}{11} * \cell{\tx z}{10}\end{array}}{I_{\tx z}'}\!\!\!\!\!\! * [\token a_{\tx z}]\right\}$
\end{lstlisting}
\end{tabular}
\[
\begin{array}{@{}r@{}l@{}l@{}}
  a_{\li x} &\null ==
  [\token a_{\li x}]: \exsts{v} \cell{\li{z}}{v} * \cell{\li{x}}{v} &\null\swap
  \cell{\li{z}}{v} * \cell{\li{x}}{v+1}\\
  a_{\li y} &\null ==
  [\token a_{\li y}]: \exsts{v} \cell{\li{x}}{v+1} * \cell{\li{y}}{v} &\null\swap
  \cell{\li{x}}{v+1} * \cell{\li{y}}{v+1}\\
  a_{\li z} &\null ==
  [\token a_{\li z}]: \exsts{v} \cell{\li{y}}{v+1} * \cell{\li{z}}{v} &\null\swap
  \cell{\li{y}}{v+1} * \cell{\li{z}}{v+1}\\
  a_{\li x}' &\null ==
  [\token a_{\li x}]: \exsts{v} \cell{\li{z}}{v} * \cell{\li{x}}{v} * \cell{\li{y}}{v} &\null\swap
  \cell{\li{z}}{v} * \cell{\li{x}}{v+1} * \cell{\li{y}}{v}\\
  a_{\li y}' &\null ==
  [\token a_{\li y}]: \exsts{v} \cell{\li{x}}{v+1} * \cell{\li{y}}{v} * \cell{\li{z}}{v} &\null\swap
  \cell{\li{x}}{v+1} * \cell{\li{y}}{v+1} * \cell{\li{z}}{v}\\
  a_{\li z}' &\null ==
  [\token a_{\li z}]: \exsts{v} \cell{\li{y}}{v+1} * \cell{\li{z}}{v} * \cell{\li{x}}{v+1}&\null\swap
  \cell{\li{y}}{v+1} * \cell{\li{z}}{v+1} * \cell{\li{x}}{v+1}
\end{array}
\]
\vspace{-5pt}
\begin{mathpar}
  I \eqdef \{a_{\li x}, a_{\li y}, a_{\li z}\}

  I_{\li x}' \eqdef \{a_{\li x}, a_{\li z}'\}

  I_{\li y}' \eqdef \{a_{\li x}', a_{\li y}\}

  I_{\li z}' \eqdef \{a_{\li y}', a_{\li z}\}
\end{mathpar}
\vspace{-5pt}\hrule
\caption{The concurrent increment program together with a \colosl proof sketch. Lines starting with $\color{blue}{//}$ contain formulas that describe  the local state and the subjective shared state at the relevant program point.}
\label{fig:concurrentInc}
\end{figure}

Consider the program $\mathbb{INC}$ defined in \fig\ref{fig:concurrentInc}, ignoring the
assertions. It is written in pseudo-code resembling C with additional
constructs for concurrency: atomic sections $\atomic \_$ which
declare that code behaves atomically; and
parallel composition $\_ ||\_ $  which spawns threads then waits until
they complete. In our example, after
initialisation of the variables to $0$, three threads are spawned to
increment each variable in a lock-step fashion: $\mathbb{P}_{\li x}$
is the first allowed to run its increment operation; then
$\mathbb{P}_{\li y}$; and finally $\mathbb{P}_{\li z}$. This process
repeats until $\li{x} = \li{y} = \li{z} = 10$.  This example code is
interesting because the threads are intricately intertwined. In the case of the $\mathbb{P}_{\li y}$ thread, the programmer knows that the code depends on the values of variables
\li{x} and \li{y}  and that it can increment  \li{y} as long as its value is less than that of \li{x}.
However, she also knows a much more complex behaviour of the thread: given the initial setting where all variables have value $0$, then the thread can only increase the value of \li{y} by $1$ if \li{x} is one more than \li{y},  and the environment can only increase \li{x} by one if \li{x} and \li{z} (and in fact \li{y}) have the same value. Finally, the programmer knows that at the end all the variables will have value $10$. 

\colosl can simply specify this complex behaviour of the resources associated with thread $\mathbb{P}_{\li y}$.
Consider the \colosl assertions accompanying  $\mathbb{INC}$.
After initialisation, line~3 of $\mathbb{INC}$ provides a standard
assertion from separation logic~\cite{seplog} with variables as
resource~\cite{variablesAsResource}. The assertion  declares  that the
variable cells addressed by \li{x}, \li{y} and \li{z}  have value $0$. This variable resource in the thread-local state is fully
owned by  the thread. Using the \extendRule\ principle, the thread is
able to give up  this local resource and transfer it  to the global
shared state.

In its general form, the \extendRule adds locally-held resources
described by $P$ to the shared state, subject to the new interference
relation $I$, and creates fresh sets of tokens $\token A$, $\token A'$, some
kept locally and some shared. The side-condition $P\containI I$
ensures that $I$ cannot modify resource outside of the new extension
delimited by $P$, in particular preserving any pre-existing subjective
view of the shared state which otherwise would need to suddenly
account for new interference in $I$. As will be crucial for the set
example in \S\ref{sec:examples} (but unimportant here), a novel
feature of this rule is that actions in $I$ may refer to existing
shared resources, unlike CAP for example, where all possible futures of the
region must be accounted for upon its creation.  The \conseqRule rule is
then applied to start using the resulting assertion in Hoare triples.

Line~4 demonstrates the creation of a subjective view
$\shared{\li{x}|-> 0 * \li{y}|-> 0 * \li{z}|-> 0}I$ under three
similar actions. For instance, $a_{\li y}$ can increment \li{y} under
the condition that the value of \li{x} is one more than \li{y}.  This
update can only be carried out by the thread that holds $[\token a_{\li y}]$ \emph{capability} in its local state. For this particular example, the
assertion in line~4 simply has all the capabilities $[\token a_{\li
    x}] * [\token a_{\li y}] * [\token a_{\li z}]$ of the actions of
$I$ in the local state; in general, capabilities can be buried inside
boxes, only to emerge as a consequence of an action (see
\S\ref{sec:examples}).

This subjective view may then be freely duplicated using the \copyRule
principle and each thread is given a copy together with the
appropriate capability using the \parRule rule. For instance, thread
$\mathbb{P}_{\li y}$ gets
$
\shared{\li{x}|-> 0 * \li{y}|->0 * \li{z}|-> 0}{I} *[\token a_{\li y}]
$.

However, this precondition is more complicated than we
need. Intuitively, the specification of each thread should only use
the variable resource relevant to that thread, and need only consider
actions that affect that resource. In this example, the extraneous
piece of state is the variable cell \li{z}. This additional resource
might seem an acceptable price to pay, but straightforward
generalisations to $n$ participants yields extra state of $n\!-\!2$
variable cells with their associated interferences which are of no
interest to the particular thread.  Fundamentally, for large systems,
the burden of carrying the whole shared state around to analyse all
threads can lead to intractable proofs. In our example this would be
acceptable since the footprint of each thread is small, but in general needlessly
verbose states require extraneous stability checks at every program
point.

Because subjective views only describe \emph{parts} of the shared
state, we can use the \forgetRule principle to obtain a weaker view of the shared
state that ignores cell \li{z}. The \shiftRule principle allows us to
replace $I$ with any interference relation $I'$ that has the same
observable effect on the subjective assertion $P$ (written $I
\weakenI{P} I'$). Without the \li{z} cell, action $a_{\li z}$ does not
have observable effects since it leaves other cells unchanged. We thus
get:
\begin{align*}
 	\shared{\li{x}|-> 0 * \li{y}|->0 * \li{z}|-> 0}{I} *[\token a_{\li y}]&
	\stackrel{(\forgetRule)}{=>} \shared{\li{x}|-> 0 * \li{y}|->0}{I} *[\token a_{\li y} ]\\
 	&\stackrel{(\shiftRule)}{\semimplies}
        \shared{\li{x}|-> 0 * \li{y}|->0 }{I\setminus a_{\small\li z}} *[\token a_{\li y}]\\
	&\stackrel{(\weakenRule)}{=>}
        \shared{\exsts{v, v'}  (\li{x}|-> v \!*\! \li{y}|->v' ) \!\land\! v\geq v'}{I\setminus a_{\small\li z}} \!\!*[\token a_{\li y}]
\end{align*}

%Finally, we stabilise the resulting subjective view such that whatever the environment does to the state, the assertion inside the box still holds (the assertion is \emph{stable}).
%it is invariant under all possible actions by the environment.. 
Finally, we stabilise (weaken) the resulting subjective view such that it is invariant under all possible actions by the environment.
%Since the thread owns $[\token{a}_{\li{y}}]$ locally, the environment can't perform its action and thus the subjective view is stable against it. On the other hand, as the environment may hold $[\token{a}_{\li{x}}]$, we should stabilise with respect to its action. 
Since we no longer know the value of \li{z}, after stabilising against $a_{\small\li{x}}$, the resulting assertion is too weak. Intuitively, we know that \li{x} can only be incremented when its value is equal to \li{z} and \li{y}. However, this is not reflected in $a_{\small\li{x}}$. Since we have forgotten \li{z}, there is nothing to constrain the increment on \li{x}. Hence, we can only stabilise in a general way as given, losing information about how the values of \li{x} and \li{y} are connected together through \li{z}.
%
It is however possible to give a stronger specification, as follows: \vspace*{-8pt} 
%
\begin{align*}
  \shared{\li{x}|-> 0 * \li{y}|->0 * \li{z}|-> 0}{I} *[\token a_{\small\li y}]
  &\stackrel{(\shiftRule)}{\semimplies}
  \shared{\li{x}|-> 0 * \li{y}|->0 * \li{z}|-> 0}{\{a_{\small\li x}', a_{\small\li y}, a_{\small\li z}\}} *[\token a_{\small\li y} ] \vspace*{-5pt}\\
  &\stackrel{(\forgetRule)}{=>}
  \shared{\li{x}|-> 0 * \li{y}|->0 }{\{a_{\small\li x}', a_{\small\li y}, a_{\small\li z}\}} *[\token a_{\small\li y}]\\
  &\stackrel{(\shiftRule)}{\semimplies}
  \shared{\li{x}|-> 0 * \li{y}|->0 }{I_{\small\li y}'} *[\token a_{\small\li y}] \\
  &\stackrel{(\weakenRule)}{=>}
  \shared{
    \begin{array}{@{}l<{\null}@{}l<{\null}@{}}
      \cell{\li{x}}{0} * \cell{\li{y}}{0} |/ 
      \cell{\li{x}}{1} * \cell{\li{y}}{0}
    \end{array}
  }{I_{\small\li y}'}
  * [\token a_{\small\li y}]\vspace*{-5pt}
\end{align*}
%
This implication involves subtle interaction between the assertion and
the interference relation of the subjective view.  Consider the $a_{\small\li x}$ action and the initial state with value $0$ in all the
cells. This action can be replaced by $a_{\small\li x}'$ because, as the
programmer knows, whenever \li{x} and \li{z} have the same value then
\li{y} also has the same value which, under these conditions, is not
changed by the actions in $I$.  This amended action reflects stronger
knowledge about when \li{x} can be incremented and how its value is
related to \li{y} and \li{z}.
%This amended action gives full information about how the resource of the subjective view changes as \li{x} is being updated.
As we justify in \S\ref{sec:colosl}, by using the judgement
$I\weakenI{\cell{\li{x}}{0} * \cell{\li{y}}{0} * \cell{\li{z}}{0}}
I_{\small\li y}'$, we can use the \shiftRule principle to replace
$I$ by $\{a'_{\small\li x}, a_{\small\li y}, a_{\small\li z}\}$. 
%%
%\[
%I\weakenIb{\cell{\li{x}}{0} * \cell{\li{y}}{0} * \cell{\li{z}}{0}} I_{\small\li y}
%\]
%%
Only then do we use \forgetRule to lose the \li{z} cell and obtain the subjective view  $\shared{\li{x}|-> 0 * \li{y}|->0 }{I_{\small\li y}'}$ as the new action $a_{\small\li x}'$ in $I'_{\small\li y}$ retains enough information about how \li{x}, \li{y} and \li{z} are related.
The interference relation is now as simple as it can get, whilst retaining enough information about the connection between \li{x}, \li{y} and \li{z}. Finally, we stabilise the subjective view with respect to $I_{\small\li y}'$ and obtain our final precondition of $\mathbb{P}_{\small\li y}$.
%, which retains much more information about the values of \li{x} and \li{y}. %

%% There is one more subtlety to mention about this precondition. The thread has the capability $[\token a_{\small\li y}]$ to perform its action which only requires the resources described by the subjective view. However, the action $a_{\small\li x}$ depends on \li{z}, which is no longer a part of the subjective view of the thread. Since another thread in the environment may own the $[\token{a_{\small\li x}}]$ capability, it may perform its action whenever its subjective view is {\em compatible} with the pre-condition of the action.
%% When that is the case, the piece of the state corresponding to the overlap between the state and the
%% precondition of the action is removed, and the entire postcondition of the action is added in its place.
%% %Diagrammatically, a subjective state (represented by the circle) is affected as follows by an action $P\swap Q$:\vspace*{-8pt}
%% %%
%% %\[
%% % \begin{tikzpicture}[baseline]
%% % \draw[thick] (0,0) circle (.4cm);
%% % \draw[densely dotted] (0,0) rectangle (.4cm,.6cm);
%% % \end{tikzpicture}
%% % \quad\swap\quad
%% % \begin{tikzpicture}[baseline]
%% % \draw[thick] (0,0) circle (.4cm);
%% % \draw[thick,fill=white] (0,0) rectangle (.6cm,.4cm);
%% % \end{tikzpicture}
%% % \qquad
%% % \left(\text{where }
%% % \begin{tikzpicture}[baseline,yshift=-.25cm]
%% % \draw[thick] (0,0) rectangle (.4cm,.6cm);
%% % \draw[densely dotted] (.4cm,0) arc (0:90:.4cm);
%% % \end{tikzpicture}|= P
%% % \text{ and }
%% % \begin{tikzpicture}[baseline]
%% % \draw[thick,yshift=-.1cm] (0,0) rectangle (.6cm,.4cm);
%% % \end{tikzpicture}|= Q\right)\vspace*{-5pt}
%% %\]
%% %%
%% This is due to the fact that the subjective view describes the thread's partial knowledge about the shared state, while the environment may have some additional knowledge to what the thread knows. In this case, while the thread does not have the capability to do the action of $[\token a_{\small\li x}]$, the environment might.

The proof of the specification of the thread $\mathbb{P}_{\small\li y}$ is
now relatively straightforward. By inspection (or using the rules of
\S\ref{sec:prules}), the invariant of the while loop is stable with
respect to $I_{\small\li y}'$. The atomic section allows safe manipulation
of the contents of the subjective view.  The final postcondition of
$\mathbb{P}_{\small\li y}$ follows from the invariant and the boolean
expression of the while. We join up the postconditions of the threads
using \mergeRule, which embodies a crucial feature of \colosl:
different subjective views \emph{overlap}. Since $|/$ distributes over
$**$, the subjective view simplifies to $\shared{\li{x}|->10 *
  \li{y}|->10 * \li{z}|->10}{I_{\small\li x}'\cup I_{\small\li y}'\cup I_{\small\li z}'}
$.  Finally, since $ I_{\small\li x}'\cup I_{\small\li y}'\cup I_{\small\li z}'$
$\weakenI{\li{x}|->10 * \li{y}|->10 * \li{z}|->10} I $, by the
\shiftRule principle, we get the postcondition of $\mathbb{INC}$.


%\paragraph{Comparison with other formalisms}
This concludes our \colosl proof of $\mathbb{INC}$. Our expansion and
contraction of subjective views, in particular with shifting of
interference assertions in key places, enables us to confine the
specification and verification of each thread to just the resources
they need. This is not possible in existing frameworks.


We complete our semi-formal overview by noting that unlike
CAP~\cite{cap-ecoop10} and as in iCAP~\cite{icap}, we do not provide
an explicit \emph{unsharing} mechanism to claim shared
resources and render them thread-local. Instead, this behaviour can
be simply encoded as an additional action with the second behaviour
described above. For instance, in o
%
%We also note that, as in the CAP family~\cite{cap,icap,tada}, \colosl
%cannot ensure that proved programs do not leak memory. Indeed, one may
%at any point of a proof stash leaked resources into a new subjective
%view, then weaken the resulting assertion to remove that view from the
%current state, rendering the leak invisible as far as the proof system
%is concerned.

%% This finishes the justification of all uses of \eqref{eq:shift} in the
%% derivations of \S\ref{subsec:threads}, and thus the derivations
%% themselves.  In this section, we have informally justified
%% manipulations of the interference relation by semantic
%% arguments. However, \S\ref{sec:shiftax} will show how the reasoning
%% involved can be reduced to simple logical implications between
%% separation logic formulas that do not mention the shared state.



%% In the case of the less local predicate $G$, whenever the
%% pre-condition of the action associated with $\token a_{\small\li z}$ is satisfied
%% (third disjunct), it is also the case that $\cell{\li{x}}{v+1}$. In other
%% words, in all the cases where the shared state satisfies
%% $\cell{\li{y}}{v+1} * \cell{\li{z}}{v}$, it also satisfies $\cell{\li{x}}{v+1} *
%% \cell{\li{y}}{v+1} * \cell{\li{z}}{v}$. However, this information is not
%% reflected in $I$ and as a result when weakening $G$, we need to
%% stabilise the resultant assertion which proves to be very weak. To
%% remedy this, in \colosl we introduce the notion of action
%% \emph{shifting} (rewriting) with respect to the \emph{invariant} of
%% the shared state. Given $\shared{P}{I'}$, we write $\fence{} \fences
%% (P, I')$ - read ``$\fence{}$ fences $P$ with respect to $I'$'' - to
%% indicate that i) $\fence{}$ contains all states associated with $P$
%% and ii) it is closed under $I'$; that is, given any action in $I'$
%% whose pre-condition is satisfied by a state in $\fence{}$, the state
%% resulting from the action is also in $\fence{}$. For instance, given
%% the $G$ predicate of \fig\ref{fig:concurrentIncCoLoSLSpec}, we have
%% $\fence{G} \fences (P_G, I)$ where $P_G$ denotes the assertion inside
%% the box and $\fence{G}$ is as specified below.
%% \[
%% 	\begin{array}{l l}
%% 		\fence{G} = \hspace*{-5pt}& \left\{\cell{\li{x}}{v} * \cell{\li{y}}{v} * \cell{\li{z}}{v} ||| v \in \{0, \cdots 10\} \right\} \\
%% 		& \cup \left\{\cell{\li{x}}{v+1} * \cell{\li{y}}{v} * \cell{\li{z}}{v} ||| v \in \{0, \cdots 9 \} \right\} \\
%% 		& \cup \left\{\cell{\li{x}}{v+1} * \cell{\li{y}}{v+1} * \cell{\li{z}}{v} ||| v \in \{0, \cdots 9\} \right\}\\
%% 	\end{array}
%% \]
%% %as well as the action associated with its update
%% Given the above invariant, we can now \emph{shift} the action associated with $\token a_{\small\li z}$ in $I$ and arrive at $I'$ where
%% \[
%% 	I' \eqdef \left\{
%% 		\begin{array}{@{}l@{}}
%% 			\token a_{\small\li x}:\, \exsts{v} \cell{\li{x}}{v} * \cell{\li{z}}{v}  \swap  \cell{\li{x}}{v+1} * \cell{\li{z}}{v}\\
%% 			\token a_{\small\li y}:\, \exsts{v} \cell{\li{x}}{v+1} * \cell{\li{y}}{v}  \swap  \cell{\li{x}}{v+1} * \cell{\li{y}}{v+1}\\
%% 			\token a_{\small\li z}:\, \exsts{v} \cell{\li{x}}{v+1} *
%%                         \cell{\li{y}}{v+1} * \cell{\li{z}}{v} \swap\null\\
%% 			\hspace*{2cm} \cell{\li{x}}{v+1} * \cell{\li{y}}{v+1} * \cell{\li{z}}{v+1}\\
%% 		\end{array}			
%% 	\right.
%% \]
%% Note that in doing so we have neither restricted nor relaxed the action of $\token a_{\small\li z}$ in that it can be carried out in exactly the same states given the invariant $\fence{G}$. This is formalised by the \shiftRule\ rule where $I \weakenI{\fence{}} I'$ denotes the shifting of $I$ with respect to $\fence{}$ and we defer its formalisation to \S\ref{sec:logic}.
%% \[
%% 	\text{if}\hspace*{0.25cm} \fence{} \fences (P, I) 
%% 	\hspace*{0.25cm}\text{and}\hspace*{0.25cm} I \weakenI{\fence{}} I'
%% 	\hspace*{0.25cm}\text{then}\hspace*{0.25cm}
%% 	\shared{P}{I} ===> \shared{P}{I'} \hspace*{0.5cm} \shiftRule
%% \]
%% Given predicate $G$ of \fig\ref{fig:concurrentIncCoLoSLSpec}, we can first shift $I$ into $I'$ (specified above) using the \textsc{Shift} rule and then apply the \forgetRule\ rule to forget variable \li{y} and obtain $\shared{a_{\small\li x}}{I'}$.
%% %%
%% %\[
%% %	X' \eqdef \shared{\exsts{v} \cell{\li{x}}{v} * \cell{\li{z}}{v} \lor \cell{\li{x}}{v+1} * \cell{\li{z}}{v}}{I'} 
%% %\]
%% %%
%% We have almost arrived at a local specification for $\mathbb{P}_{\small\li x}$. However, the action of $\token a_{\small\li y}$ is still visible in $I'$ even though it does not affect the values of \li{x} or \li{z}. Through interference shifting, we can not only rewrite actions with respect to the invariant, we can also \emph{forget} actions that affect \emph{none} of the states contained in the invariant. For instance, let $\fence{\li{x}} = \left\{\cell{\li{x}}{v} * \cell{\li{z}}{v} ||| v \in \{0, \cdots 10\} \right\} \cup \left\{\cell{\li{x}}{v+1} * \cell{\li{z}}{v} ||| v \in \{0, \cdots 9 \} \right\}$, then we have $\fence{X} \fences (X, I')$. In the case of the action of $\token a_{\small\li y}$, given any state $p$ in the action pre-condition (e.g. $p = \cell{\li{x}}{1} * \cell{\li{y}}{0}$), for an arbitrary state $s \in \fence{X}$ (e.g. $s = \cell{\li{x}}{1} * \cell{\li{z}}{0}$), \emph{all overlaps} of $p$ and $s$ ($p \meetL s = \{\cell{\li{x}}{1}\}$) are preserved by the action. We give the formal definition of overlap operator $\meetL$ in \S\ref{sec:logic}. 

%% Given $I'$ and $\fence{X}$ we can again apply the \textsc{Shift} rule in order to forget about the action of $\token a_{\small\li y}$ and obtain $I_{\small\li x}'$ as specified in \fig\ref{fig:concurrentIncSubjectiveSpec}. We can take analogous steps in order to obtain subjective views $S_{\small\li y}$ and $S_{\small\li z}$ for threads $\tau_{\small\li y}$ and $\tau_{\small\li z}$. We then pass the $\token a_{\small\li x}$, $\token a_{\small\li y}$ and $\token a_{\small\li z}$ capabilities to $\tau_{\small\li x}$, $\tau_{\small\li y}$ and $\tau_{\small\li z}$ and verify $\mathbb{INC}$ as follows:
%% \[
%% \hspace*{-0.2cm}
%% \begin{array}{c}
%% 	\color{blue}{\left\{\token a_{\small\li x} * \token a_{\small\li y} *  \token a_{\small\li z} *  S_{\small\li x} * S_{\small\li y} * S_{\small\li z} \right\}}\vspace*{2pt}\\
	
%% 	\begin{array}{c || c || c}
%% 		\color{blue}{\left\{\token a_{\small\li x} * S_{\small\li x} \right\}} & \color{blue}{\left\{\token a_{\small\li y} * S_{\small\li y} \right\}} & \color{blue}{\left\{\token a_{\small\li z} * S_{\small\li z} \right\}}\\
%% 		&&\vspace*{-7pt}\\
%% 		\mathbb{P}_{\small\li x} & \mathbb{P}_{\small\li y} & \mathbb{P}_{\small\li z}\\
%% 		&&\vspace*{-5pt}\\

%% 		\color{blue}{
%% 			\left\{
%% 					\token a_{\small\li x} * S'_{\small\li x}
%% 			\right\}
%% 		} 
%% 		& 
%% 		\color{blue}{
%% 			\left\{
%% 				\token a_{\small\li y} * S'_{\small\li y}
%% 			\right\}
%% 		} 

%% 		&
		
%% 		\color{blue}{
%% 			\left\{
%% 				\token a_{\small\li z} * S'_{\small\li z}
%% 			\right\}
%% 		} 		
%% 		\vspace*{3pt}
%% 	\end{array}\\
%% 	\color{blue}{\left\{\token a_{\small\li x} * \token a_{\small\li y} *  \token a_{\small\li z} *  S'_{\small\li x} * S'_{\small\li y} * S'_{\small\li z} \right\}}\\
%% \end{array}
%% \]
%% with
%% \[
%% \begin{array}{l l}
%% 	S'_{\small\li x} \eqdef & \shared{\cell{\li{x}}{10} * \cell{\li{z}}{10} \lor \cell{\li{x}}{10} * \cell{\li{z}}{9} }{I_{\small\li x}'}\\
%% 	S'_{\small\li y} \eqdef & \shared{\cell{\li{x}}{11} * \cell{\li{y}}{10} \lor \cell{\li{x}}{10} * \cell{\li{y}}{10} }{I_{\small\li y}'}\\
%% 	S'_{\small\li z} \eqdef & \shared{\cell{\li{y}}{11} * \cell{\li{z}}{10} \lor \cell{\li{y}}{10} * \cell{\li{z}}{10} }{I_{\small\li z}'}
%% \end{array}
%% \]







%% \pgcomment{KEy point of intro}

%% A central property of
%% \colosl proofs is that multiple threads can view different,t
%% potentially overlapping parts of the shared state. As such, the
%% separating conjunction $*$ behaves as \emph{overlapping conjunction}
%% $\sepish$~\cite{rey-slnotes,ramification} (sometimes called
%% \emph{sepish}~\cite{gareth-js12}) over subjective states (in contrast
%% with most existing formalisms, which would use \emph{classical
%%   conjunction} instead).
%% \begin{align*}
%%   \shared{P}{I} &=> \shared{P}{I} * \shared{P}{I}
%%   \tag{\eqref{eq:split}}\\
%%   \shared{P}{I_1} * \shared{Q}{I_2} &=> \shared{P \sepish Q}{I_1\cup I_2} \tag{\eqref{eq:merge}}
%% \end{align*}
%% On local states, $*$ behaves as \emph{disjoint union}, as is standard.

%% \pgcomment{ ISn't it
%%   double-arrow on Merge, perhaps called spilt/merge} 

%% \julescomment{No it's not, only this direction is valid
%%   (\eqref{eq:split} is the valid corresponding other direction).}

%% This directly justifies the \eqref{eq:split} and \eqref{eq:merge}
%% steps of the two derivations above.
