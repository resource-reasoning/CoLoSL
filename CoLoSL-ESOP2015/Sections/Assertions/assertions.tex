\section{\colosl Assertions}\label{sec:assertions}
We describe the \emph{assertions} of \colosl and provide their semantics by relating them to sets of worlds. We establish the validity of \copyRule, \forgetRule and \mergeRule principles introduced in the preceding sections by establishing their truth for all possible worlds and interpretations.

Our assertions extend standard assertions from separation logic with \emph{subjective views} and \emph{capability assertions}. 
We assume an infinite set, $\set{PVar}$, of program variables; an infinite set, $\set{LVar}$, of \emph{logical variables}; and a set of \emph{logical environments} $\lenv \in \LEnv: \pset{\set{LVar} --> \set{Val}}$ that associate logical variables with their values.

\colosl is parametric in the assertions of machine states and capabilities and can be instantiated with any assertion language over machine states $\Heaps$ and capabilities $\Caps$. In this paper, we use standard heap and stack assertions as machine state assertions; and instantiate capability assertions with tokens.
%We thus assume an infinite set, $\set{Token}$, of tokens.
%
%
\begin{definition}[Assertion syntax]\label{def:assertions}
The assertions of \colosl are elements of $\Assertions$ described by the grammar below, where $x \in \set{LVar}$, $\li{x} \in \set{PVar}$, and $\token{a} \in \set{Token}$. We write $\overline{[\token{a}]}^{\token{A}}$ as a shorthand for $\oast_{\token{a} \in \token{A}}[\token{a}]$.
%
\begin{align*}	 
  A &::=\m{false} \mid \emp \mid \li{x}|-> E \mid E_1 |-> E_2 \mid [\token{a}] \mid [x]
  \quad
  E ::= x \!\mid\! E_1 + E_2 \!\mid\! \cdots \\
%
  \LAssertions \ni \lass{p}, \lass{q} & ::= A \mid \lass{P} \land \lass{Q} \mid \lass{P} \lor \lass{Q} \mid \exsts{x} \lass{P} \mid \lass{P} * \lass{Q} \mid \lass{P} ** \lass{Q} \mid \lass{p} \septraction \lass{q}\\
%
	\Assertions \ni P, Q  &::=  \lass{p} \mid P \land Q \mid P \lor Q \mid \exsts{x} P \mid P * Q \mid P \sepish Q \mid \shared{P}{I}  \\
%
	\IAssertions \ni I &::= \emptyset \mid \{\interAss{\overline{[\token{a}]}^{\token{A}}}{\bar{y}}{P}{Q}\} \cup I
\end{align*}
\end{definition}
%
This syntax follows from standard separation logic with variables as resource~\cite{variablesAsResource} (where expressions $E$ do not allow program variables), with the exception of the subjective views $\shared P I$. 
Machine state and capability assertions are interpreted over a world's local state: $\li{x}|-> E$ (resp.\ $E_1|->E_2$) is true of the singleton stack (resp.\ heap) where only \li{x} (resp.\ address $E_1$) is allocated and has value $E$ (resp.\ points to $E_2$); the capability assertion $[\token{a}]$ is true of the capability $\{\token{a}\}$. 
$\emp$ is true of states where the the units of $\composeW$. 
$P * Q$ is true of worlds that can be split into two according to $\composeW$ such that one state satisfies $P$ and the other satisfies $Q$; $P**Q$ is the \emph{overlapping conjunction}, true of worlds can be split three-way according to $\composeW$, such that the $\composeW$-composition of the first two worlds satisfies $P$ and the $\composeW$-composition of the last two satisfy $Q$~\cite{rey-slnotes}; classical predicates and connectives have their standard classical meaning. Interference assertions $I$ describe actions enabled by a given capability, in the form of a pre- and post-condition.

A subjective view $\shared P I$ is true of $(l,g,\lmod)$ when $l = \unitL$ and a \emph{subjective view} $s$ can be found in the global shared state $g$ (i.e. $g = s\composeL r$ for some \emph{context} $r$), such that $s$ satisfies $P$ in the standard separation logic sense, and $I$ and $\lmod$ \emph{agree} given the
decomposition $s$, $r$, in the following sense:\vspace{-10pt}
%
\begin{enumerate}
	\item every action in $I$ is reflected in $\lmod{}$;
	
	\item every action in $\lmod$ that has a visible effect on $s$ is reflected in $I$;
	
	\item the above holds after any number of actions in $\lmod$ takes place.
%	\item the above is true for any state resulting from any number of actions of $\lmod$ applied to $g$.
%	\item the above is true after any number of action applications in $I$ that affect $g$ and any number of action applications in $\gmod$ that affect $r$ but not $s$.
\end{enumerate}\vspace{-10pt}
%
These conditions will be captured by the \emph{action model closure} relation $\extendsAM{\lmod}{s}{r}{\semI{I}}$ given by the upcoming \defin~\ref{def:actclos} (where $\semI{I}$ is the interpretation of $I$ given a logical environment $\lenv$).

The semantics of \colosl assertions is given by a forcing relation $w,\lenv|= P$ between a world $w$, a logical environment $\lenv\in\LEnv$, and a formula $P$. We use two auxiliary forcing relations. The first one $l,\lenv\slsat P$ interprets formulas $P$ in the usual separation
logic sense over a logical state $l$ (and ignores shared state assertions). The second one $s,\lenv|=_{g,\lmod} P$ interprets assertions over a \emph{subjective view} $s$ that is part of the global shared state $g$, subject to action model $\lmod$. This third form of satisfaction is needed to deal with nesting of subjective views\footnote{This presentation with several forcing relations differs from the usual CAP presentation~\cite{cap-ecoop10}, where formulas are first interpreted over worlds that are not necessarily well-formed, and then cut down to well-formed ones. The CAP presentation strays from separation logic models in some respects; e.g. in CAP, $*$ is not the adjoint of $--*$, the ``magic wand'' connective of separation logic. Although we have omitted this connective from our presentation, its definition in \colosl would be standard and satisfy the adjunction with $*$.}.
%
Moreover, since logical connectives are interpreted uniformly in all cases, we write $|=_\dagger$ for either of the three satisfaction relations, and then write $u$ for elements of either $\Worlds$ or $\LStates$, and $\grey$ for either $\composeW$ or $\composeL$, depending on whether the satisfaction relation is $|=$, or either $\slsat$ or $|=_{l,\lmod}$, respectively.
%
%
\begin{definition}[Assertion semantics]\label{def:assertion-semantics}
Given a logical environment $\lenv \in \LEnv$, the semantics of \colosl assertions is as follows, where $\semI[(.)]{.}: \LEnv \rightarrow \AMods$ denotes the semantics of interference assertions and $\extendsAM{\lmod}{s}{r}{\semI{I}}$ is formalised in \defin~\ref{def:actclos}.
\vspace{-1ex}
%
\[
\begin{array}{R>{\null}l@{\ \,}c@{\ \,}l}
  (l,g,\lmod), \lenv &|= \lass{p} &\text{iff}& l,\lenv \slsat \lass{p}\\
  
  
  (l,g,\lmod), \lenv &|= \shared P I &\text{iff}&
  l = \unitL \text{ and }
  \exsts{s,r}
  g = s\composeL r
  \text{ and}\\
  &&&s, \lenv |=_{g,\lmod} P\text{ and }
  \extendsAM{\lmod}{s}{r}{\semI[\lenv]{I}} \vspace{5pt}\\
  
  
  s, \lenv &|=_{g,\lmod} \lass{p} &\text{iff}& s, \lenv \slsat \lass{p}\\
  
  
  s, \lenv &|=_{g,\lmod} \shared P I &\text{iff}&
  (s,g,\lmod), \lenv |= \shared P I\\
  

  u,\lenv &|=_\dagger P \land Q
  &\text{iff}& u,\lenv |=_\dagger P\text{ and }u,\lenv |=_\dagger Q\\
  
  u,\lenv &|=_\dagger P \lor Q
  &\text{iff}& u,\lenv |=_\dagger P\text{ or }u,\lenv |=_\dagger Q\\
  
  u,\lenv &|=_\dagger \exsts x P
  &\text{iff}& \exsts v u, [\lenv|||x:v] |=_\dagger P\\
  
  
  u, \lenv &|=_\dagger P_1 * P_2 &\text{iff}&
  \exsts{u_1,u_2} u = u_1 \grey u_2\text{ and}\\
  &&& u_1, \lenv |=_\dagger P_1 \text{ and }u_2, \lenv |=_\dagger P_2\\
  
  
  u, \lenv &|=_\dagger P_1 ** P_2 &\text{iff}&
  \exsts{u',u_1,u_2} u = u'\grey u_1\grey u_2\text{ and}\\
  &&&
  u' \grey u_1, \lenv |=_\dagger P_1 \text{ and }
  u' \grey u_2, \lenv |=_\dagger P_2\\
  

% 	u, \lenv &|=_\dagger \lass{P} \septraction \lass{Q} &\text{iff}&
%	\exsts{u'} u', \lenv |=_\dagger \lass{P} \text{ and }
%	u \sharp u'\\
%	&&&\text{ implies }u \grey u', \lenv |=_\dagger \lass{Q} \\
	
  
%  l, \lenv &\slsat [E]
%  &\text{iff}&
%  l = (\emptyset,\unitH, \{[|E|]_{\lenv}\})\\
%  &&\cdots\\


		
  l,\lenv &\slsat \m{false}
  && \text{never}\\
  
  
  l, \lenv &\slsat \emp &\text{iff}&l = \unitL\\
  
  
  %  l,\lenv &\slsat E_1 = E_2
%  &\text{iff}& [|E_1|]_{\lenv} = [|E_2|]_{\lenv}\\


  l, \lenv &\slsat \li{x}|->E
  &\text{iff}&
  l = ([\li x: [|E|]_{\lenv}], \emptyset, \emptyset)\\
  
  
  l, \lenv &\slsat E_1|->E_2 
  &\text{iff}&
  l =
  (\emptyset, [[|E_1|]_{\lenv}: [|E_2|]_{\lenv}], \emptyset)\\
  
  
  l, \lenv & \slsat [\token{a}]
  & \text{iff} &
  l = (\emptyset, \emptyset, \{\token{a}\})\\
  
  
  l, \lenv &\slsat \lass{P} \septraction \lass{Q} &\text{iff}&
	\exsts{l'} l', \lenv \slsat \lass{P} \text{ and }
	l \sharp l'\\
	&&&\text{ implies }l \composeL l', \lenv \slsat \lass{Q} 
	
%  l, \lenv &\slsat \shared P I&
%  \text{ iff }& l = \unitL
\end{array}
\]
\vspace{-1em}
%
\begin{align*}
\semI[\lenv]{I}(\token{A}) &==
  \left\{
  \begin{array}{@{}l@{\ }|@{\ }r@{}}
    (p \composeL r, q \composeL r)&
    \begin{array}{@{}l@{}}
      \interAss{\overline{[\token{a}]}^{\token{A}}}{\bar{y}}{P}{Q} \in I /|  \exsts{\bar{v}, \lmod} \\
%      \lenv' = [\lenv|||\bar y:\bar v] \\
%      /| (p \composeL r), \lenv'\slsat \sumA{P} 
%      /| (q \composeL r), \lenv' \slsat \sumA{Q} \\
      \quad p, [\lenv|||\bar y:\bar v]  |=_{p \composeL r, \lmod} P 
      /| q, [\lenv|||\bar y:\bar v]  |=_{q \composeL r, \lmod} Q 
    \end{array}
  \end{array}
  \right\}
\end{align*}
%
\end{definition}
%
%
Although \colosl allows for nested subjective views (e.g. $\shared{P * \shared{Q}{I'}}{I}$), it is often useful to \emph{flatten} them into equivalent assertions with no nested boxes. 
We introduce 
%\emph{local assertions}, $\LAssertions \subset \Assertions$, a subset of assertions that do not contain any subjective views; as well as \emph{flat assertions}, 
\emph{flat assertions}, $\FAssertions \subset \Assertions$, to capture a subset of assertions with no nested subjected views and provide a \emph{flattening mechanism} to rewrite \colosl assertions as equivalent flat assertions. 
%
%
\begin{definition}[Flattening]
%The set of \emph{local assertions} $\LAssertions$, and 
The set of \emph{flat assertions} $\FAssertions$ is defined by the following grammar. \vspace*{-5pt}
%
\begin{align*}
%	\LAssertions \ni \lass{P}, \lass{Q} & ::= A \mid \lass{P} \land \lass{Q} \mid \lass{P} \lor \lass{Q} \mid \exsts{x} \lass{P} \mid \lass{P} * \lass{Q} \mid \lass{P} \sepish \lass{Q}\\
%
	\FAssertions \ni \fass{P}, \fass{Q} & ::= \lass{P} \mid \shared{\lass{P}}{I} \mid \fass{P} \land \fass{Q} \mid \fass{P} \lor \fass{Q} \mid \exsts{x} \fass{P} \mid \fass{P} * \fass{Q} \mid \fass{P} \sepish \fass{Q}\vspace*{-5pt}
\end{align*}
%
The \emph{flattening} function $\un{.} : \Assertions \rightarrow \FAssertions$ is defined inductively as follows where $x$ does not appear free in $I$ and $\odot \in \{\land, \lor, *, **\}$. 
%
\begin{align*}
	& \un{\lass{p}} \eqdef  \lass{p} \quad &&  
	\un{P \odot Q} \eqdef  \un{P} \odot \un{Q}  \\%\quad \text{for } \odot \in \{\land, \lor, *, \sepish\}\\
%
%
	& \un{\exsts{x} P} \eqdef  \exsts{x} \un{P} &&
	\un{\shared{\shared{P}{I} * Q}{I'}} \eqdef  \un{\shared{P}{I}} * \un{\shared{Q}{I'}}\\
%
%
	& \un{\shared{\lass{P}}{I}} \eqdef  \shared{\lass{P}}{I} \quad &&
	\un{\shared{\shared{P}{I} \sepish Q}{I'}} \eqdef  \un{\shared{P}{I}} * \un{\shared{Q}{I'}}\\
%
%
	& \un{\shared{\exsts{x} P}{I}} \eqdef  \exsts{x} \un{\shared{P}{I}} &&
	\un{\shared{P \lor Q}{I}} \eqdef \un{\shared{P}{I}} \lor \un{\shared{Q}{I}}\\
%
%
	& \un{\shared{\shared{P}{I} \land L}{I'}} \eqdef \m{false} &&
	\un{\shared{\shared{P}{I} \land ((L \lor Q) \odot R)}{I'}} \eqdef 
	\un{\shared{\shared{P}{I} \land (Q \odot R)}{I'}}\\
%
%
	&&&
	\un{\shared{\shared{P}{I} \land B}{I'}} \eqdef
	\un{\shared{P}{I}} * \un{\shared{B}{I'}} \vspace*{-30pt}
\end{align*}\vspace{-10pt}
%
%
%
\begin{mathpar}
	B ::= \emp|\, \shared{P}{I}|\, B \odot B
	
	L ::= \m{false} \mid \cell{x}{v} \mid [\token{a}] \mid L \ominus Q \quad \text{ for }\ominus \in \{\land, *, \sepish\}
\end{mathpar}
%
%\begin{mathpar}
%	\un{A} \eqdef  A
%
%	\un{P \odot Q} \eqdef  \un{P} \odot \un{Q}  \quad \text{for } \odot \in \{\land, \lor, *, \sepish\}
%	
%	\un{\exsts{x} P} \eqdef  \exsts{x} \un{P} 
%	
%	\un{\shared{\lass{P}}{I}} \eqdef \shared{\lass{P}}{I}
%	
%	\un{\shared{\exsts{x} P}{I}} \eqdef  \exsts{x} \un{\shared{P}{I}}
%	
%	\un{\shared{\shared{P}{I} * Q}{I'}} \eqdef  \un{\shared{P}{I}} * \un{\shared{Q}{I'}}
%	
%	\un{\shared{\shared{P}{I} \sepish Q}{I'}} \eqdef  \un{\shared{P}{I}} * \un{\shared{Q}{I'}}
%	
%	\un{\shared{P \lor Q}{I}} \eqdef  \un{\shared{P}{I}} \lor \un{\shared{Q}{I}}
%	
%%%	\un{\shared{\shared{P}{I} \land \emp}{I'}} \eqdef & \un{\shared{P}{I}} * \shared{\emp}{I'}\\
%%
%	\un{\shared{\shared{P}{I} \land B}{I'}} \eqdef \un{\shared{P}{I}} * \un{\shared{B}{I'}}
%	
%	\un{\shared{\shared{P}{I} \land (L \odot Q)}{I'}} \eqdef 
%	\m{false} \quad \text{ where } \odot \in \{\land, *, \sepish\}
%	
%	\un{\shared{\shared{P}{I} \land (L \lor Q)}{I'}} \eqdef 
%	\un{\shared{\shared{P}{I} \land Q}{I'}}
%%	\begin{cases}
%%		\un{\shared{P}{I}} * \un{\shared{Q}{I'}} & \text{ if } Q \implies \emp\\
%%		\m{false} & \text{ otherwise }
%%	\end{cases}
%\end{mathpar}
%where
%%
%\begin{align*}
%	B ::= & \emp \mid \shared{P}{I} \mid B * B \mid B \sepish B \mid B \land B \mid B \lor B\\
%	L ::= & \m{false} \mid \cell{x}{v} \mid [\token{a} ]
%\end{align*}
\end{definition}
%
%
\begin{definition}[Erasure]
The \emph{erasure} of an assertion $\erase{(.)}: \Assertions \rightarrow \LAssertions$ is defined inductively over the structure of assertions as follows with $\odot\in\{\land, \lor, *, \sepish \}$\vspace*{-5pt}
%
\begin{mathpar}
	\erase{A} \eqdef A  

	\erase{\left(\shared{P}{I}\right)} \eqdef \emp 

	\erase{(\exsts{x} P)} \eqdef \exsts{x}\erase{P} 

	\erase{(P \odot Q)} \eqdef  \erase{P} \odot \erase{Q}\vspace*{-5pt}
\end{mathpar}
%
%Observe that $P \implies \erase{P}$ holds for any assertion $P$.
%
\end{definition}
%
%
%%
%\begin{definition}[Interference semantics]\label{def:interference-semantics}
%The \emph{semantics of interference assertions}, $\semI[(.)]{.}: \LEnv --> \AMods$, is defined as follows for $\set{A} \in \pset{\set{Token}}$.
%\begin{align*}
%  \semI[\lenv]{I}(\token{A}) &==
%  \left\{
%  \begin{array}{@{}l@{\ }|@{\ }r@{}}
%    (p \composeL r, q \composeL r)&
%    \begin{array}{@{}l@{}}
%      \interAss{\overline{[\token{a}]}^{\token{A}}}{\bar{y}}{P}{Q} \in I /|  \exsts{\bar{v}, \lmod} \lenv' = [\lenv|||\bar y:\bar v] \\
%      /| (p \composeL r), \lenv'\slsat \sumA{P} 
%      /| (q \composeL r), \lenv' \slsat \sumA{Q} \\
%      /|\ p, \lenv' |=_{p \composeL r, \lmod} P 
%      /| q, \lenv' |=_{q \composeL r, \lmod} Q 
%    \end{array}
%  \end{array}
%  \right\}
%  \end{align*}
%\end{definition}
%%
%
\begin{lemma}\label{lem:assertion-facts}
The following entailments are valid.
%
%
%\begin{figure}
%\hrule\vspace{5pt}
\begin{mathpar}
	\infer{
		P <=> \un{P}
	}
	{}
	
	\infer{
		P \entails \erase{P}
	}
	{}
	
	\infer{
		\shared{P}{I} \entails \shared{Q}{I}
	}
	{
		P \entails Q
	}
	
	\infer{
		\shared{P}{I} * \shared{Q}{I'} \entails \m{false}
	}
	{
		P \sepish Q \entails \m{false}
	}
	
	\infer{
		\shared{P}{I} * \shared{Q}{I'} \entails \shared{R}{I} * \shared{Q}{I'}
	}
	{
		R \entails P
		&
		P \sepish Q \entails R \sepish Q
	}
\end{mathpar}
%\hrule
%\caption{Useful entailments where we write $P <=> Q$ for $P \entails Q \land Q \entails P$.}
%\label{fig:assertion-facts}
%\end{figure}
%
%\begin{mathpar}
%	P => \erase{P} %\shared{P}{I} \implies \emp
%	
%	P <=> \un{P}
%\end{mathpar}
%%
\begin{proof}
The proof follows immediately from the semantics of \colosl assertions and is provided in~\cite{colosl-tr14}.
%  The proof of the first formula follows immediately from the semantics of \colosl. The proof of second formula is provided in~\cite{colosl-tr14}.
\end{proof}
%
\end{lemma}
%
%
%%%%%%%%%%%%%%%%%%%%%%%%%%%%%%%%%%%
\paragraph{\textbf{Action Model Closure}}
Let us now turn to the definition of action model closure, as informally introduced at the beginning of this section. First, we need to revisit the effect of actions to take into account the splitting of the global shared state into a subjective state $s$ and a context $r$.
%
%
\begin{definition}[Action application (cont.)]\label{def:actionApplicationPair}
The \emph{application} of action $a$ on the subjective state $s$ together with the context $r$, written $a[s,r]$, is defined provided that $a[s \composeL r]$ is defined.
%there exist $l$ such that
%%
%\[
%\begin{array}{L}
%	\m{fst}(a)\meetL (s\composeL r) /= \emptyset /| 
%	\m{fst}(\updateFP{a}) \composeL l = s \composeL r /|
%	\m{snd}(\updateFP{a}) \compatible l 
%%  \m{fst}(a)\meetL (s\composeL r) /= \emptyset /|
%%  ps \in s \meetL \m{fst}(\updateFP{a}) /| \null \\
%%  
%%  \begin{array}{@{} l @{\hspace{2pt}} l @{}}
%%  	\E{p_r, r'} & s = p_s \composeL s' /|  \m{fst}(\updateFP{a}) = p_s\composeL p_r /| \null\\
%%  	& r = p_r \composeL r' /| \m{snd}(\updateFP{a})\composeL s'\composeL r'\text{ is defined}
%%  \end{array}
%\end{array}
%\]
%
When that is the case, we write $a[s,r]$ for
\[
a[s, r] ==
\begin{cases}
	(\m{snd}(\updateFP{a})\composeL s', r')
	&\text{if } \exsts{p_s > \unitL} \exsts{p_r}\\	
	& \quad \m{fst}(\updateFP{a}) = p_s\composeL p_r /| 
  s = p_s \composeL s' /|   r = p_r \composeL r' \\
  
  \left(s, \m{snd}(\updateFP{a})\composeL r'\right)
  & \text{if }\neg\m{visible}(a, s) /| \exsts{r'} r = \m{fst}(\updateFP{a}) \composeL r' 
\end{cases}
\]
\end{definition}
%
%
\noindent We observe that this new definition and \defin~\ref{def:actionApplication} are linked in the following way: 
%
\[
	\fst{a[s, r]} \composeL \snd{a[s, r]} = a[s\composeL r]
\]
%
In our informal description of action model closure in \defin~\ref{def:assertions} we stated that some actions must be \emph{reflected} in some action models. 
Intuitively, an action is reflected in an action model if for every state in which the action can take place, the action model includes an action with a similar effect that can also occur in that state. In other words, $a$ is reflected in $\lmod$ from a state $l$ if whenever $l$ contains the pre-state of $a$ ($\fst{a} \leq l$), then there exists an action $a' \in \lmod$ with the same active part ($\updateFP{a} = \updateFP{a'}$) whose pre-state is also contained in $l$ ($\fst{a'} \leq l$).
We proceed with the definition of action reflection.
%
%
\begin{definition}
An action $a$ is \emph{reflected} in a set of actions $A$ from a state $l$, written $\m{reflected}(a,l,A)$, if
%
\[
  \V r \m{fst}(a)\leq l\composeL r =>
  \E{a'\in A} \updateFP{a'} = \updateFP{a} /| \m{fst}(a')\leq l\composeL r
\]
\end{definition}
%
%
%Let us now give the formal definition of action model closure. For each condition mentioned in \defin\ref{def:assertions}, we annotate which part of the definition implements them. We write
%%
%\[
%\m{enabled}(a,g) == \m{potential}(a,g) /| \m{fst}(a)\leq g
%\]
%%
%That is, $a$ can actually happen in $g$ since $g$ holds all the resources in the pre-state of $a$.
%
%
\begin{definition}[Action model closure]\label{def:actclos}
An action model $\lmod$ is \emph{closed} under a subjective state $s$, context $r$, and action model $\lmod'$, written $\extendsAM{\lmod}{s}{r}{\lmod'}$, if $\lmod' \subseteq \lmod$ and $\extendsAMUpto{\lmod}{n}{s}{r}{\lmod '}$ for all $n\in\Nats$, where $\downarrow_n$ is defined recursively as follows:\vspace*{-5pt}
%
%
\[
\begin{array}{@{} L @{\hspace*{2pt}} L @{}}
  \extendsAMUpto{\lmod}{\mathrlap{0}\phantom{n}}{s}{r}{\lmod'}
  \iffdef
  &\m{true}
  \\
  
  \extendsAMUpto{\lmod}{n+1}{s}{r}{\lmod'} \iffdef &
	(\V{\ca{}}\V{a\in \lmod(\ca{})}
	\m{potential}(a,s\composeL r) =>\null\\
	  
		&\quad 
		  \left(
		  	\m{reflected}(a,s\composeL r,\lmod'(\ca{})) |/ \neg\m{visible}(a,s) 
		  \right)\qquad \hfill \text{(2)} \\
		& \quad  /| \extendsAMUpto{\lmod}{n}{\fst{a[s, r]}}{\snd{a[s, r]}}{\lmod'} \qquad \hfill \text{(3)}\vspace*{-5pt}
\end{array}
\]
%
\end{definition}
%
Recall from the semantics of the assertions (\defin~\ref{def:assertion-semantics}) that $\lmod'$ corresponds to the interpretation of an interference assertion $I$ in a subjective view. The first condition($\lmod' \subseteq\lmod$) asserts that the subjective action model $\lmod'$ is contained in $\lmod$ ({\it cf.} property (1) in \defin~\ref{def:assertions}); consequently (from the semantics of subjective assertions), $\lmod$ represents the superset of all interferences known to subjective views.
The above states that the $\lmod$ is closed under $(s, r, \lmod')$ if the closure relation holds for any number of steps $n \in \Nats$ where $s$ denotes the subjective view of the shared state, $r$ denotes the context, $s \composeL r$ captures the entire shared state, and a step corresponds to the occurrence of an action as prescribed in $\lmod$.
% \vspace*{-5pt}
%\begin{itemize}
%	\item $s$ denotes the subjective view of the shared state; \vspace{0pt}
%	\item $r$ denotes the context; \vspace{0pt}
%	\item $s \composeL r$ captures the entire shared state; \vspace{0pt}
%	\item a step corresponds to the occurrence of an action as prescribed in $\lmod$.\vspace*{-5pt}
%%	which may or may not be found in $\lmod'$ (since $\lmod \subseteq \lmod$). 
%\end{itemize}   
%
The relation is satisfied trivially for no steps ($n = 0$); for an arbitrary $n\in \Nats$ the relation holds iff for any action $a$ in $\lmod$, where $a$ is potentially enabled in $s \composeL r$, then:	\vspace*{-5pt}
\begin{enumerate}\renewcommand{\theenumi}{\alph{enumi}}
	\item \textit{either} $a$ is reflected in $\lmod'$ ($a$ is known to the subjective view); 
	\textit{or} $a$ does not affect the subjective state $s$ ($a$ is not visible in $s$); and 
	\item $\lmod$ is also closed under the subjective state $s'$ and context $r'$ resulting from the application of $a$ ($\,a[s, r]= (s', r')$).\vspace*{-5pt}
\end{enumerate}
%
%
We make further observations about this definition. First, property~(3) makes our assertions robust with respect to future extensions of the shared state, where potentially enabled actions may be become enabled using additional catalyst that is not immediately present. Second, $\lmod'$ (and thus interference assertions) need not reflect actions that have no visible effect on the subjective state.\\

This completes the definition of assertion semantics. We can now show that the logical principles of \colosl are sound. The proof of the \shiftRule\ and \extendRule\ principles will be delayed until the following section, where $===>$ is defined.

\begin{lemma}\label{lem:semprinciples}
The logical implications \copyRule, \forgetRule, and \mergeRule\ are \emph{valid}; \textit{i.e.} true of all worlds and logical interpretations.
%
%
\begin{proof}
The case of \copyRule\ is straightforward from the semantics of assertions.
For \forgetRule, we remark that whenever the action model closure relation holds for a subjective state $s\leq g$, then it holds of any $s'\leq s$. For \mergeRule, we remark that the two worlds corresponding to each subjective view
need to have the same action model to be compatible, which ensures closure under the combined subjective views. The full proof is provided in~\cite{colosl-tr14}.
\end{proof}
%
%
\end{lemma}
%
%
Note that, the version of \forgetRule\ where predicates are conjoined using $**$ is also valid for all $P$, $Q$, and $I$, as shown by the following derivation, where the first implication follows from the properties of $**$ and the rule of consequence.
%
\[
  \vspace{-1ex}
\shared{P ** Q}{I} \stackrel{\proofRule{Cons }}{=> }
\shared{P * \m{true}}{I} \stackrel{\forgetRule }{=> }
\shared{P}{I}
\]
%
