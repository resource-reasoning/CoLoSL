\section{Introduction}
\label{sec:introduction}

%\pg{This is what we should answer by the introduction: 
%What's the problem with the current state of the art?
%What is our solution?
%Why is the problem we're solving hard?
%How do we solve the challenges?
%Why do we do better than existing work?
%What are the lessons learnt from the paper?}

A key difficulty in verifying properties of shared-memory concurrent
programs is to be able to reason compositionally about each thread in
isolation, even though in reality the correctness of the whole system
is the collaborative result of intricately intertwined actions of the
threads. Such compositional reasoning is essential for 1) verifying
large concurrent systems since it allows them to be verified
component-wise; 2) verifying library code and incomplete programs
since it avoids the need to know about the context of execution and
the precise implementation of functions called in the code; and 3)
replicating a programmer's intuition about why their implementations
are correct since informal arguments are usually kept local and do not
involve details about the whole system.

In the sequential case, separation logic~\cite{rey02,seplog} famously
achieves compositional reasoning for heap-manipulating
programs. There, compositionality rests on two powerful mechanisms: a
function can be specified using only those resources that it actually
accesses, and this specification can be reused in any context that
merely contains these resources. This is embodied by the following
\emph{frame rule}%
% between Hoare triples, from separation logic
:
\[
\infer[\frameRule]{\hoare{P * R}{\mathbb{C}}{Q * R}}
{\hoare{P}{\mathbb{C}}{Q}}
\]
The $*$ connective between assertions means that they describe
\emph{disjoint} sets of resources. By making the specification
$\hoare{P}{\mathbb{C}}{Q}$ as tight as possible, we can ensure that it
can be reused in a large set of possible contexts via the frame
rule. Specification reuse is paramount to achieve compositional
reasoning: if the specification of a piece of code can be reused in
any valid context, one only has to prove it once and for all. In
non-compositional approaches, one very often needs to reprove a piece
of code under different specifications.

For shared-memory concurrency, rely-guarantee~\cite{rg} and proof
techniques that combine rely-guarantee and separation logic
principles~\cite{viktor-marriage,dg,lrg,cap-ecoop10,icap,tada} have
been proposed to gain compositionality in the face of the added
difficulty of inter-thread interaction. Unfortunately, none of these
approaches are fully compositional. In particular, despite successive
significant strides in the kind of programs they are able to verify,
the following three interconnected problems remain open.

First and foremost, the region of memory shared by the threads is
spatially coarse-grained, usually comprising the entirety of the
shared state regardless of how much of it is actually accessed by
individual threads or individual functions within these threads. Thus,
a crucial ingredient of compositionality is missing: one cannot give
small specifications to programs manipulating shared data, nor can one
use the frame rule over shared state. Some formalisms can mitigate
this issue, most notably Concurrent Abstract Predicates
(CAP)~\cite{cap-ecoop10} and Local Rely-Guarantee reasoning
(LRG)~\cite{lrg}, but only when it is possible to decompose the shared
state into entirely disjoint regions without cross-regions
interference. For instance, given a linked list of $n+1$ nodes
accessed concurrently by $n$ threads (numbered $1, \cdots, n$) where
the $i$th thread requires access to two successive nodes at indices
$i-1$ and $n$, no such splitting is possible as each node of the
linked list will be shared between two neighbouring threads, hence the
shared state must contain the entire linked list. Similarly, when data
structures with intrinsic sharing such as graphs are accessed
concurrently, one cannot reason about their sub-parts in isolation due
to arbitrary sharing between them.

Second, the interference associated with each region is defined when it is created and cannot change. As such, upon its creation, provision has to be made for
all future extensions of it and it is generally not possible to
arbitrarily grow it with extra state under new interference rules.
As we discuss in \S\ref{sec:examples}, this results in unnatural specifications such as the concurrent set module example in~\cite{cap-ecoop10}. 

Finally, the description of the interference on the shared state is
fixed throughout its entire lifetime, even if a transition on the shared state 
becomes irrelevant (either because the resources it acts on have been
deallocated or extra knowledge gained about the shared state
falsifies its requirements). This incurs
a burden in the proof effort, as stability against these irrelevant
actions still has to be checked at every program point.

These issues, in particular the first one, not only affect the
locality of the reasoning, but also impede \emph{modularity} and proof
reuse. When the implementation of a module $M_1$ is verified against
its specification, its proof can only be reused from within another
module $M_2$ if either the parts of the shared state accessed by $M_2$
are disjoint from those of $M_1$, or the shared state is partitioned
identically for both modules. The probability that several modules
agree on the exact shape of the part of the state they consider as
shared is slim.

This paper introduces the program logic \colosl, for Concurrent Local
Subjective Logic, which addresses all three issues. \colosl models
shared resources as a single state, with a twist: the assertion
language allows threads to hold their own \emph{subjective views} of
this state. Each subjective (personalised) view is an assertion which
describes of \emph{parts} of the shared state, and how the thread and
the environment may affect these shared resources. Subjective views
expand and contract to the natural resources and the interference
required by the current thread. Moreover, the subjective views of
different threads represent potentially \emph{overlapping} parts of
the shared state, in contrast with previous approaches that either
mandate that all threads view the same entire shared state, or
strictly disjoint parts of it. This allows us to provide a framing
mechanism both on shared resources as well as their interferences even
in the presence of overlapping footprints and entangled
resources. This flexibility provides truly compositional reasoning for
shared-memory concurrency.

\paragraph{Contributions and outline} 
We give the main ideas behind the semantic model and the soundness of
\colosl in \S\ref{sec:logic}. In this section, we also provide proof
rules to discharge \colosl-specific proof obligations and turn them
into regular separation logic validity checks. This enables one to use
\colosl without having to reason directly in the model. We demonstrate
how \colosl allows local reasoning about the shared state on a range
of examples.  The first example \S\ref{sec:intuition}, admittedly
serves to exhibit the main principles of \colosl through a variant of
Dijkstra's token-ring mutual exclusion
algorithm~\cite{dijkstra74}. This example illustrates distributed,
global knowledge between threads. As well as the local knowledge about
how each thread behaves, the programmer also has global knowledge
about how threads constrain one another. Pleasantly, our reasoning
captures the spirit of local reasoning about individual threads while
retaining global knowledge about the overall interactions between
threads. In \S\ref{sec:examples}, we study further, more challenging
examples. We verify a concurrent spanning tree algorithm for
graphs. This algorithm uses threads spawned on arbitrary overlapping
graphs. We demonstrate that the flexible, overlapping subjective views
of \colosl are just what we need.  We also verify a concurrent set
module implemented using a hand-over-hand list-locking algorithm that
significantly improves over the existing proof~\cite{cap-ecoop10}.
%
%Finally, we demonstrate how the mutual exclusion ring of \S\ref{sec:intuition} can be generalised to synchronise access to a concurrent database.

Most of the technical details have been left out due to space
constraints, and are provided in appendices as well as in an
accompanying technical report~\cite{colosl-tr14}.

%This paper presents the program logic \colosl, which achieves
%compositional reasoning for concurrent programs by enabling
%\emph{subjective} views of the shared state. Subjective views may be
%composed arbitrarily and manipulated so as to retain only the portions
%of local and shared state actually accessed by each thread in the
%program, and to consider only the interferences relevant to that piece
%of shared state. \colosl achieves a greater degree of compositionality
%than existing work by enabling \emph{finer-grained} sharing: the
%program logic can consider that threads share only what is relevant to
%their function, a key ingredient of compositionality.
%
%
%
%Driven by the ever-increasing need for concurrency in software,
%%  spurred by recent hardware developments, 
%program logics for shared-memory concurrency have progressed towards
%the twin ideals of fine-grain reasoning and compositionality. The
%former enables elegant proof techniques about increasingly subtle
%concurrency idioms~\cite{vv06popl,vv07msc,todo}, while the latter
%allows programs to be proved component-wise and their proofs to be
%reusable as-is in any client
%program~\cite{csl-tcs,cap-ecoop10,icap}.
%%  Central to these compositional verification frameworks is the
%% formalism used to describe both the state shared between program
%% threads and the possible interference on that state.
%%
%% dating back from Owicki~\cite{owicki}, and later Jones who
%% integrated the notion of interference in the logic
%% itself~\cite{rg}.
%% 
%
%% Consider for instance
%% a program where threads operate on subgraphs of a global graph.
%
%
%%  dubbed their \emph{subjective states}.  One may then reuse these
%%   specifications in the context of any larger local state (as is
%%   standard in separation logic~\cite{rey02}), and, crucially for
%%   compositional reasoning about concurrent programs, any larger
%%   shared state. The subjective states of different threads in a
%%   program are allowed to overlap arbitrarily, ensuring maximum
%%   reusability of proofs.
%
%
%
%
% Subjective views can expand or contract depending on the resources required by the thread, which has subtle consequences for the logical reasoning. In particular, it is possible to strengthen actions in $I$ to record  some global information known to the subjective view.
%Then, the view can be weakened to just the resources and possibly strengthened actions appropriate to  the thread. The strengthened actions retain  some global information despite the weakened view. 


\paragraph{Related work}
\colosl stems from a long lineage of program logics for shared-memory
concurrency, dating back to the Owicki-Gries logic~\cite{owicki}, each of
which made substantial advances in fine-grained compositional
reasoning about more and more complex concurrent programs. As noted by
Jones and motivating his definition of rely-guarantee~\cite{rg},
Owicki-Gries's system is fundamentally non-compositional as assertions
at each program point in each thread has to be checked against every
instruction of every other threads. While rely-guarantee addresses
this concern by summarising potential interferences in terms of rely
and guarantee conditions, its model of the shared state is global,
even encompassing the threads' private state. Later techniques
combined rely-guarantee with separation logic, starting with
RGSep~\cite{viktor-marriage}, which separates private (or local) and
shared state, allowing further compositionality. However, the view of
the shared state and interference on it is still globally shared
across threads. The next breakthrough as far as compositionality is
concerned came from LRG~\cite{lrg}, wherein the shared state may be
split into completely independent pieces, thus partially solving the
first problem outlined above. This provides a way to frame off parts
of the shared state, but, as noted by its author, the strong
disjointness restrictions make this applicable only at the level of
disjoint modules in the code. Independently from LRG, the reasoning
framework of CAP~\cite{cap-ecoop10}
and its extended variants~\cite{icap,tada} introduced the notion
of \emph{regions} of shared state. Much like LRG, the division into
regions is only possible for entirely disjoint pieces of state, with
fixed and disjoint interference relations.

Finally, Subjective Concurrent Separation Logic (SCSL)~\cite{SCSL}
uses \emph{subjective auxiliary states}, similar in spirit to our
subjective views for the shared state, to reason about coarse-grained
concurrency. As noted by its authors, CAP (and by design \colosl) can
be employed to reason about their studied examples, SCSL cannot be
used to reason about scenarios involving fine-grained concurrency.
