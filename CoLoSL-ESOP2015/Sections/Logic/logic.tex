\section{\colosl Logic}\label{sec:logic}
In this section we revisit the \colosl principles introduced in the earlier sections and provide judgements for establishing interference confinement (see \extendRule), action shifting (see \shiftRule) and stability. While these judgements are not complete, we found them sufficient in dismissing the proof obligations of our examples in this exposition.

%
\todo segue why we need these next definitions. 
%

As we demonstrate in Appendix~\ref{sec:exist-promotion}, given any assertion $P$, it is always possible to rewrite it into an equivalent assertion where all the existential quantifications are promoted to the front of the assertion; that is, $P <=> \exsts{\bar{x}} P'$ where there are no bound variables in $P'$.
%
\begin{definition}[Gather/merge]
Given an assertion $P \eqdef \exsts{\bar{x}} P' \in \Assertions$ (where there are no bound variables in $P'$), the \emph{gather}, $\prodA{(.)}: \Assertions \rightarrow \LAssertions$, and \emph{merge}, $\sumA{(.)}: \Assertions \rightarrow \LAssertions$, operations are defined as follows:
%
\begin{mathpar}
	\prodA{P} \eqdef \exists{\bar{x}}.\; \lass{P} * \lass{Q}
	
	\sumA{P} \eqdef  \exists{\bar{x}}.\; \lass{P} \sepish \lass{Q}
	
	\text{where } (\lass{P}, \lass{Q}) = \ub{\un{P'}} 
\end{mathpar}
%
%
with the auxiliary function $\ub{.}: \FAssertions \rightarrow (\LAssertions \times \LAssertions)$ defined inductively over the structure of local assertions as follows where $\odot \in \{\land, *, \sepish\}$.
% We assume that the bound logical variables of the left-hand assertion are pairwise distinct (and are otherwise renamed to be distinct in a capture-avoiding manner). 
 In what follows, $\ub{\fass{P}} = (\lass{P}, \lass{P}')$ and $\ub{\fass{Q}} = (\lass{Q}, \lass{Q}')$ where applicable.
%
\begin{mathpar}
	\ub{\lass{P}} \!\!\eqdef\! (\lass{P}, \emp) 
	
	\ub{\shared{\lass{P}}{I}} \!\!\eqdef\!  (\emp, \lass{P})
	
	\ub{\fass{P} \odot \fass{Q}} \!\!\eqdef\! \left(\lass{P} \odot \lass{Q}, \lass{P}' \sepish \lass{Q}' \right)

	\ub{\fass{P} \lor \fass{Q}} \!\!\eqdef\! \left(\lass{P} \lor \lass{Q}, \lass{P}' \lor \lass{Q}' \right)
\end{mathpar}
%
%
\end{definition}
%
\todo interference entailment
%
\begin{figure}
\hrule\vspace{5pt}
\begin{mathpar}
%	\infer{
%		P \confines I	
%	}
%	{
%		\erase{P} \strictfences I	
%	}
	\infer{
		I \cup I_1 \entailsI I \cup I_2
	}
	{
		I_1 \entailsI I_2
	}	
	
	\infer{
		\left\{ \interAss{\capAss{}}{\bar{y}}{P}{Q} \right\}	
		\entailsI
		\left\{ \interAss{\capAss{}}{\bar{y}}{\sumA{P}}{\sumA{Q}} \right\}	
	}{}
		
	\infer{
		\left\{ \interAss{\capAss{}}{\bar{y}}{P}{Q} \right\}	
		\entailsI
		\left\{ \interAss{\capAss{}}{\bar{y}}{P'}{Q'} \right\}	
	}
	{
		P \entails\! P'
		&
		Q \entails Q'	
	}
%	
\end{mathpar}
\hrule
\caption{Interference entailment judgements.}
\label{fig:interference-entailment-rules}
\end{figure}
%

Recall from \S\ref{sec:introduction} that when extending the shared state with locally held resources described by $P$, the interference governing the newly-shared resources $I$ must be confined to $P$. This is captured by the confinement condition $P \confines I$ (see \extendRule). Confinement of interference assertions is simply the lifting of action model confinement (\defin\ref{def:amod-confinement}) to assertions as defined below.
%
%
\begin{definition}[Confinement/Local fences]
An interference assertion $I$ is \emph{confined} to the set of states described by assertion $P$, written $P \confines I$, if for all $\lenv \in \LEnv$:
%
\qquad
$
	\left\{l \mid l, \lenv \slsat P \right\} \confines \semI[\lenv]{I}
$
%

\noindent Similarly, a local assertion $\lass{f}$ \emph{locally fences} an interference assertion $I$, written $\lass{f} \strictfences I$, if for all $\lenv \in \LEnv$,
%
\qquad
$
	\left\{l \mid l, \lenv \slsat \lass{F} \right\} \strictfences \semI[\lenv]{I}
$
%
\end{definition}
%

\paragraph{\textbf{Confinement/local fencing judgements}}\fig~\ref{fig:local-fencing-rules} presents a number of judgements that reduce interference confinement and local fences to logical entailments. 
%While these judgements are not complete, we found them sufficient for establishing confinement in our examples of this exposition. 

Note that although local fencing judgements are given for a local assertion $\lass{f} \in \LAssertions$, since $\erase{P} \in \LAssertions$ and $P \entails \erase{P}$ given any assertion $P$ (\lem\ref{lem:assertion-facts}), in the top left confinement judgement one can simply substitute $\erase{P}$ for $P$. \textit{Mutatis mutandis} for $P'$ and $Q'$ in the left-most judgements of the second row.
%in the top left confinement judgement one can simply check whether $\erase{P} \strictfences I$. 

Pleasantly, since our local assertions do not contain subjective (boxed) assertions, entailments of the form $\lass{f} \slentails \lass{f'}$ are the familiar entailments of standard separation logic except for the overlapping conjunction ($**$) and septraction ($\septraction$) connectives. We thus provide elimination rules for $**$ in Appendix~\ref{sec:sepish-judgements}; $\septraction$ can be eliminated as described by Calcagno et al. in~\cite{vv07msc}.
%
\begin{figure}
\hrule\vspace{5pt}
\begin{mathpar}
	\infer{
		P \confines I	
	}
	{
		P \entails P'
		\;\;
		P' \strictfences I	
	}	
%	
	\hspace{10pt}
%	
	\infer{
		\fenceAss{} \strictfences \emptyset	
	}{}
%	
	\hspace{10pt}
%	
	\infer={
		\fenceAss{} \strictfences I_1 \cup I_2	
	}
	{
		\fenceAss{} \strictfences I_1
		&
		\fenceAss{} \strictfences I_2	
	}		
%	
%	\infer{
%		\fenceAss{} \strictfences I
%	}
%	{
%		\fenceAss{} \strictfences I'
%		&
%		I' \weakenI{\fenceAss{}} I	
%	}	
%	
%	\infer{
%		\fenceAss{} \strictfences \left\{\capAss{}: P \swap Q \right\}	
%	}
%	{
%		\fenceAss{} \strictfences \left\{\capAss{}: \sumA{P} \swap \sumA{Q} \right\}	
%	}	
%		
%	\infer{
%		\fenceAss{} \strictfences \left\{\capAss{}: P \swap Q \right\}	
%	}
%	{
%		P \entails\! P'
%		&
%		Q \entails Q'
%		&
%		\fenceAss{} \strictfences \left\{\capAss{}\!\!:\! P'\! \swap\! Q' \right\}	
%	}		
%
	\hspace{10pt}
%	
	\infer{
		\fenceAss{} \strictfences I	
	}
	{
		I \entailsI I' 
		&
		\fenceAss{} \strictfences I'
	}	
%	
	\hspace{10pt}	
%	
	\infer{
		\fenceAss{} \strictfences \left\{\capAss{}: \lass{P} * \lass{R} \swap \lass{Q} * \lass{R} \right\}	
	}
	{
		\exact{\lass{R}}
		&
		\fenceAss{} \strictfences \left\{\capAss{}\!\!:\! \lass{P} \swap \lass{Q} \right\}	
	}	
	
	\infer{
		\fenceAss{} \!\strictfences\! \left\{\capAss{}\!\!:\! \lass{P} \swap \lass{Q} \right\}		
	}
	{
		\fenceAss{} \sepish \lass{P} \slentails \m{false}
	}	
	
%	\infer{
%		\fenceAss{} \strictfences \left\{\capAss{}: \lass{P} \swap \lass{Q} \right\}		
%	}
%	{
%		\fenceAss{}' \slentails \fenceAss{}
%		&
%		\fenceAss{} \sepish \lass{P} \slentails \fenceAss{}' \sepish \lass{P}
%		&
%		\fenceAss{}' \strictfences \left\{\capAss{}: \lass{P} \swap \lass{Q} \right\}		
%	}
%		
	\infer{
		\fenceAss{} \strictfences \left\{\capAss{}: \lass{P} \swap \lass{Q} \right\}		
	}
	{	
%		\separate{\lass{P}}{\lass{Q}}
%		&
		\left(\lass{P} \septraction \fenceAss{} \right) * \lass{Q} \slentails \fenceAss{}	
		&
		\fenceAss{} \!<=>\! \bigvee\limits_{i \in I}\fenceAss{i} 		
		&
		(\precise{\fenceAss{i}}
		\land
		\fenceAss{i} \sepish \lass{P} \slentails \fenceAss{i}
		\;\;\text{for } i \in I)
	}	
%	{
%		\begin{array}{@{} l @{}}			
%			\separate{\lass{P}}{\lass{Q}}
%			\quad
%			\left(\lass{P} \septraction \fenceAss{} \right) * \lass{Q} \slentails \fenceAss{}	
%			\quad
%			\fenceAss{} <=> \bigvee\limits_{i \in I}\fenceAss{i} 		\\
%			%
%			\precise{\fenceAss{i}}
%			\quad 
%			\fenceAss{i} \sepish \lass{P} \slentails \fenceAss{i}
%			\quad
%			\text{for } i \in I
%		\end{array}
%	}	
%	
\end{mathpar}
\hrule
\caption{Confinement/local fencing judgements with $P, Q \in \Assertions$; $\lass{P}, \lass{Q}, \lass{f} \in \LAssertions$.}
\label{fig:local-fencing-rules}
\end{figure}
%
%
%
\begin{definition}[Shifting/fences]
An interference assertion $I'$ is a \emph{shifting} of $I$ with respect to assertion $P$, written $I \weakenI{P} I'$, if for all $\lenv \in \LEnv$:
%
\[
	\semI[\lenv]{I} \weakenI{\left\{l \mid l, \lenv \slsat P \right\}} \semI[\lenv]{I'}
\]
%
Similarly, a local assertion $\lass{f}$ \emph{fences} an interference assertion $I$, written $\lass{f} \fences I$, if for all $\lenv \in \LEnv$,
%
\qquad
$
	\left\{l \mid l, \lenv \slsat \lass{F} \right\} \fences \semI[\lenv]{I}
$
%
\end{definition}
%
\paragraph{\textbf{Shifting/fencing judgements}}We present a number of judgements that reduce fencing conditions and action shifting to logical entailments in \fig~\ref{fig:fencing-rules} and~\ref{fig:shifting-rules}. As with the local fencing judgements, the entailments in the premises of these rules are those of standard separation logic ($\slentails$) where the ($**$) and ($\septraction$) connectives can be eliminated as described above. 

The premises of the form $\separate{\lass{P}}{\lass{Q}}$ assert that the states described by $\lass{P}$ and $\lass{Q}$ \emph{do not intersect}; that is, for all $\lenv \in \LEnv$, and for all $l_1, l_2$ such that $l_1, \lenv \slsat \lass{p}$ and $l_2, \lenv \slsat \lass{Q}$, then: 
%
\qquad
$
	\for{l \in \LStates} l \leq l_1 \land l \leq l_2 \implies l = \unitL
$\\
%
The $\separate{\lass{P}}{\lass{Q}}$ entailment can be rewritten equivalently as follows:
%
\[
	\separate{\lass{P}}{\lass{Q}} <=> \lass{P} \slentails \neg\left( \m{true} * (\neg\emp \land (\m{true} \septraction \lass{Q}) ) \right)
\]
%
However, in Appendix~\ref{sec:intersection-judgements} we provide a set of judgements for establishing non-intersection directly rather than performing semantic checks.
%
\begin{figure}
\hrule\vspace{5pt}
\begin{mathpar}
	\infer{
		\m{true} \fences I	
	}{}	
	
%	\infer{
%		\fenceAss{} \fences \emptyset
%	}{}
%	
	\infer{
		\fenceAss{} \fences I	
	}{
		\fenceAss{} \strictfences I	
	}
		
	\infer={
		\fenceAss{} \fences I_1 \cup I_2	
	}
	{
		\fenceAss{} \fences I_1
		&
		\fenceAss{} \fences I_2	
	}		
	
	\infer{
		\fenceAss{} \fences I
	}
	{
		I \entailsI I' 
		&
		\fenceAss{} \fences I'
	}
%	
%	\infer{
%		\fenceAss{} \fences \left\{\capAss{}: P \swap Q \right\}	
%	}
%	{
%		\fenceAss{} \fences \left\{\capAss{}: \sumA{P} \swap \sumA{Q} \right\}	
%	}
%	
%	\infer{
%		\fenceAss{} \fences \left\{\capAss{}: P \swap Q \right\}	
%	}
%	{
%		\fenceAss{} \fences \left\{\capAss{}: \erase{P} \swap \erase{Q} \right\}	
%	}
%	

	\infer{
		\fenceAss{} \fences \left\{\capAss{}: \lass{P} * \lass{R} \swap \lass{Q} * \lass{R} \right\}	
	}
	{
		\exact{\lass{R}}
		&
		\fenceAss{} \!\fences\! \left\{\capAss{}\!\!:\! \lass{P} \swap \lass{Q} \right\}	
	}
		
	\infer{
		\fenceAss{} \fences I
	}
	{
		\fenceAss{} \fences I'
		&
		I' \weakenI{\fenceAss{}} I	
	}
	
	\infer{
		\fenceAss{} \fences \left\{\capAss{}\!\!:\! \lass{P} \swap \lass{Q} \right\}	
	}
	{
		\separate{\fenceAss{}}{\lass{P}}
	}	
	
%	\infer{
%		\fenceAss{} \fences \left\{\capAss{}: \lass{P} \swap \lass{Q} \right\}		
%	}
%	{
%		\fenceAss{} \sepish \lass{P} \slentails \m{false}
%	}
%	
	\infer{
		\fenceAss{} \fences \left\{\capAss{}: \lass{P} \swap \lass{Q} \right\}		
	}
	{
		\separate{\lass{P}}{\lass{Q}}
		&
		\left(\lass{P} \septraction (\fenceAss{} \sepish \lass{P}) \right) * \lass{Q} \slentails \fenceAss{}
	}	
	
%	\infer[?]{
%	\fenceAss{} \fences \left\{\capAss{}: \lass{P} \swap \lass{Q} \right\}		
%	}
%	{
%		\fenceAss{}' \slentails \fenceAss{}
%		&
%		\lass{P} \sepish \fenceAss{} \slentails \lass{P} \sepish \fenceAss{}'
%		&
%		\fenceAss{}' \fences \left\{\capAss{}: \lass{P} \swap \lass{Q} \right\}		
%	}	
%	
\end{mathpar}
\hrule
\caption{Fencing judgements.}
\label{fig:fencing-rules}
\end{figure}
%
%
\begin{figure*}
\hrule\vspace*{5pt}
\begin{mathpar}
%	\infer{
%		I \weakenI{P} I'
%	}
%	{
%		I \weakenI{\erase{P}} I'	
%	}
%	
%	
	\infer{
		I \weakenI{P} I'	
	}
	{
		P \entails Q
		&
		I \weakenI{Q} I'
	}	
	
	\infer{
		I \cup I_1 \weakenI{\fenceAss{}} I \cup I_2
	}
	{
		\fenceAss{} \fences I \cup I_1	
		&
		I_1 \weakenI{\fenceAss{}} I_2
	}	
	
	\infer{
		I  \weakenI{\fenceAss{}} \emptyset
	}
	{
		I  \entailsI I'
		&
		I' \weakenI{\fenceAss{}} \emptyset
	}
%	
%	\infer{
%		I \cup I' \weakenI{\fenceAss{}} I
%	}
%	{
%		I \cup I' \entailsI I''
%		&
%		I'' \weakenI{\fenceAss{}} I
%	}
%		
%	\infer{
%		\left\{ \capAss{}: P \swap Q \right\} \weakenI{\fenceAss{}} I'
%	}
%	{
%		\left\{ \capAss{}: \erase{P} \swap \erase{Q} \right\} \weakenI{\fenceAss{}} I'
%	}
		
	\infer
%	[\proofRule{Disj-L}]
	{
		\bigcup\limits_{i \in I}\!\! \left\{\capAss{}\!\!:\!\exists\bar{y}. P_i \!\swap\! Q \right\}
		\!\approx^{\m{true}}\!\!
		\left\{\!\capAss{}\!\!:\!\!\exists\bar{y}.\! \bigvee\limits_{i \in I}\!\! P_i \!\swap\! Q \right\} 
	}
	{
	}
		
	\infer
%	[\proofRule{Disj-R}]
	{
		\bigcup\limits_{i \in I}\!\! \left\{\capAss{}\!\!:\!\exists\bar{y}. P \!\swap\! Q_i \right\}
		\!\approx^{\m{true}}\!\!
		\left\{\!\capAss{}\!\!:\!\!\exists\bar{y}.\! P \!\swap\! \bigvee\limits_{i \in I}\!\! Q_i \right\} 
	}
	{
	}
		
%	\infer
%%	[\proofRule{Disj-R}]
%	{
%		\bigcup\limits_{i \in I}  \left\{ \capAss{}\!\!:\! \lass{P} \swap \lass{Q}_i \right\}
%		\approx^{\m{true}}
%		\left\{\capAss{}\!\!:\! \lass{P} \swap \left( \bigvee\limits_{i \in I} \lass{Q}_i \right) \right\} 
%	}
%	{
%	}
%	
	\infer
%	[\proofRule{Exist}]
	{
		\left\{ \capAss{}: \exists\overline{v_i \in S_i}^{i \in I}.\, P \swap Q \right\} 
		\approx^{\m{true}}
		\bigcup\limits_{\overline{w_i \in S_i}^{i \in I}} \left\{\capAss{}: P \overline{[w_i /v_i]}^{i \in I} \swap  Q \overline{[w_i /v_i]}^{i \in I} \right\} 
	}
	{
	}	
	
	\infer
%	[\proofRule{Hide}]
	{
		\left\{ \capAss{}: \lass{P} * \lass{R}   \swap \lass{Q} * \lass{R} \right\} \weakenI{\fenceAss{}} 
		\emptyset 	
	}{	
		\exact{\lass{R}}
		&
		\separate{\fenceAss{}}{\lass{P}}
	}

	\infer
%	[\proofRule{False-L}]
	{	
		\left\{ \capAss{}: \lass{P} \swap \lass{Q} \right\} \weakenI{\fenceAss{}} 
		\emptyset 	
	}{
%		& \fenceAss{} \fences  \left\{ \capAss{}: \left\{ P \swap Q \right\}\right\}
		\fenceAss{} \sepish \lass{P} \slentails \m{false}
	}	
	
	\infer
%	[\proofRule{False-R}]
	{	
		\left\{ \capAss{}: \lass{P} \swap \lass{Q} \right\} \weakenI{\fenceAss{}} 
		\emptyset 	
	}
	{
%		& \fenceAss{} \fences  \left\{ \capAss{}: \left\{ P \swap Q \right\}\right\}
		\left(\lass{P} \septraction (\lass{P} \sepish \fenceAss{})\right) * \lass{Q} \slentails \m{false}
	}	
	
	\infer
%	[\proofRule{Expand/Contract}]
	{
		\left\{ \capAss{}: \lass{P} \swap \lass{Q} \right\} \;\approx^{\fenceAss{}}\;  \bigcup_{i \in I} \left\{ \capAss{}: \lass{P} * \lass{R}_i \swap \lass{Q} * \lass{R}_i \right\}	
	}
	{
		\fenceAss{} \sepish \lass{P} \slentails \bigvee_{i \in I} \fenceAss{} \sepish \left(\lass{P} * \lass{R}_i \right)
		&
		\exact{\lass{R}_i} \text{ for } i \in I
	}
\end{mathpar}
\hrule
%\vspace*{5pt}
\caption{Action shifting judgements; we write $I \approx^{\fenceAss{}} I'$ for $I \weakenI{\fenceAss{}} I' /| I' \weakenI{\fenceAss{}} I$.}
\label{fig:shifting-rules}
\end{figure*}
%
%

Recall from \S\ref{sec:intuition} that when reasoning we often need to \emph{stabilise} our assertions describing the shared state. That is, we need to weaken our knowledge of the shared state such that tit is robust to the possible changes it may be subjected to by the environment. In \S\ref{sec:semantics} we formulate the \emph{rely} relation that describes the ways in which the environment may manipulate the shared state. We then formalise what it means for an assertion to be \emph{stable} with respect to the rely relation. In the meantime, we provide syntactic judgements that allows us to ascertain the stability of an assertion without performing the necessary semantic checks. 

\todo Why do we need this next definition?
%
\begin{definition}[Combination]
The \emph{combination} of an assertion $P \eqdef \exsts{\bar{x}}P'$ describing the current state, and an assertion $Q \eqdef \exsts{\bar{y}}Q'$ describing an action precondition, $\combine{.}{.}: \Assertions \times \Assertions \rightarrow \LAssertions$, is defined as follows. We assume that the bound variables of  $P$ and $Q$ have been promoted to the top as described above such that $P'$ and $Q'$ have no bound variables.
%
\[
\begin{array}{l l}
	\combine{P}{Q} \eqdef & \exsts{\bar{x}, \bar{y}} \lass{P} * (\lass{P}' \sepish \lass{Q} \sepish \lass{Q}')\\
	& \quad \text{where } (\lass{P}, \lass{P}') = \ub{\un{P'}} \text{ and }  (\lass{Q}, \lass{Q}') = \ub{\un{Q'}} 
\end{array}
\]
%
\end{definition}
%
%
\paragraph{\textbf{Stability judgements}}
\fig~\ref{fig:stability-rules} presents a number of judgements that reduce stability checks to logical entailments. \todo complete this.
%
\begin{figure*}
\hrule\vspace*{5pt}
\begin{mathpar}
	\infer{
		\stable{\lass{p}}	
	}{}
	
%	\infer{
%		\stable{P}	
%	}
%	{
%		\stable{\un{P}}	
%	}
%
	\infer{
		\stable{P \odot Q}	
	}
	{
		\stable{P}
		&
		\stable{Q}	
	}

	\infer{
		\stable{\exsts{x} P}	
	}
	{
		\stable{P}
	}

%	\infer{
%		\stable{\shared{\lass{P}}{I}}	
%	}
%	{
%		\lass{P} \fences I	
%	}	
%	
%	\infer[?]{
%		\stable{\shared{\lass{P}}{I}}	
%	}
%	{
%		\stable{\shared{\lass{P}}{I'}} 
%		&
%		I' \weakenI{\lass{P}} I
%	}
%	
%	
	\infer{
		\stable{P * Q}	
	}
	{
		\stableTo{P}{Q}
		&
		\stableTo{Q}{P}	
	}
\end{mathpar}\vspace{5pt}\\
%
%
\begin{mathpar}
	\infer{
		\stableTo{P}{R}	
	}
	{
		\stable{P}
	}	
	
	\infer{
		\stableTo{P * Q}{R}
	}
	{
		\stableTo{P}{Q * R}
		&
		\stableTo{Q}{P * R}
	}	
	
	\infer{
		\stableTo{P}{R * R'}
	}
	{
		\stableTo{P}{R}
	}	
		
	\infer{
		\stableTo{P \ominus Q}{R}
	}
	{
		\stableTo{P}{R}
		&
		\stableTo{Q}{R}
	}	
	
	\infer{
		\stableTo{\exsts{x} P}{R}
	}
	{
		\stableTo{P}{R}
	}
	
	\infer{
		\stableTo{\shared{\lass{P}}{I}}{R}
	}
	{
		\stableTo{\shared{\lass{P}}{I'}}{R}
		&
		I' \weakenI{\lass{P}} I
	}
	
	
	\infer{
		\stableTo{\shared{\lass{P}}{I}}{R}
	}
	{
		\stableIn{\lass{P}}{I}{R * \shared{\lass{P}}{I}}
	}
\end{mathpar}\vspace{5pt}\\
%
%
\begin{mathpar}
	\infer{
		\stableIn{\lass{P}}{I}{R * Q}
	}
	{
		\stableIn{\lass{P}}{I}{R}
	}
%	
%	
%	\infer{
%		\stableIn{\lass{P}}{I}{\fass{R}}	
%	}
%	{
%		\stableIn{\lass{P}}{I'}{\fass{R}}
%		&
%		I' \weakenI{\lass{P}} I
%	}	
	
	\infer{
		\stableIn{\lass{P}}{I_1 \cup I_2}{R}	
	}
	{
		\stableIn{\lass{P}}{I_1}{R}
		&
		\stableIn{\lass{P}}{I_2}{R}
	}	
	
	\infer{
		\stableIn{\lass{P}}{\left\{\interAss{\capAss{}}{\bar{y}}{Q_1}{Q_2}\right\}}{R}	
	}
	{
		\stableIn{\lass{P}}{\left\{\interAss{\capAss{}}{\bar{y}}{\sumA{Q_1}}{\sumA{Q_2}}\right\}}{R}	
	}	
	
	\infer{
		\stableIn{\lass{P}}{\left\{\interAss{\capAss{}}{\bar{y}}{Q_1}{Q_2}\right\}}{R}	
	}
	{
		Q_1 \entails Q'_1
		&
		Q_2 \entails Q'_2
		&
		\stableIn{\lass{P}}{\left\{\interAss{\capAss{}}{\bar{y}}{Q'_1}{Q'_2}\right\}}{R}	
	}
	
	
	\infer{
		\stableIn{\lass{P}}{\left\{\interAss{\capAss{}}{\bar{y}}{\lass{Q}_1}{\lass{Q}_2}\right\}}{R}	
	}
	{
		\capAss{} * \prodA{R} \slentails \m{false}
	}
	
	
	\infer{
		\stableIn{\lass{P}}{\left\{\capAss{}: \lass{Q}_1 \swap \lass{Q}_2 \right\}} {R}	
	}
	{
		\combine{R}{\lass{Q}_1} \slentails \m{false}
	}
	
	
	\infer{
		\stableIn{\lass{P}}{\left\{\capAss{}: \lass{Q}_1 \swap \lass{Q}_2 \right\}}{R}	
	}
	{
		\left(\lass{Q}_1 \septraction \combine{R}{\lass{Q}_1} \,\right) * \lass{Q}_2 \slentails \lass{P} * \m{true}
	}
%	
%	
%	
\end{mathpar}
\hrule
%\vspace*{5pt}
\caption{Stability judgements where $P, Q, Q_1, Q_2, R \in \FAssertions$; $\lass{P}, \lass{q}_1, \lass{q}_2 \in \LAssertions$; $\dot \in \{\land, \lor, *, **\}$ and $\ominus \in \{\land, \lor, **\}$.}
\label{fig:stability-rules}
\end{figure*}
%
