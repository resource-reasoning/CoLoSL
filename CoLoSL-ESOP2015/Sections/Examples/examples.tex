\newpage
\section{Examples}\label{sec:examples}



\subsection{Concurrent Spanning Tree}\label{subsec:CST-example}
Programs manipulating arbitrary graphs present the following significant
challenge for compositional verification: because of deep sharing
between different components of a graph, changes to one subgraph may
affect other subgraphs which may point into it. This makes it hard to
reason about updates to each subgraph in isolation. In a concurrent
setting, this difficulty compounds with the fact that threads working
on different parts of the graph may affect each other in ways that are
difficult to reason about locally to each subgraph. Central to those
issues is the fact that different subgraphs present \emph{unspecified
  sharing} and may overlap in arbitrary ways. We now demonstrate on a
concurrent spanning tree example that \colosl may be just the tool you
need in this case, as it naturally deals with arbitrary overlapping
views of the shared state, and allows one to tailor interferences to a
given subjective view.

Our example, presented in \fig\ref{fig:conSpanningTree}, operates on a
\emph{directed binary graph} (henceforth simply \emph{graph}),
\textit{i.e.} a directed graph where each node has at most two
successors, called its left and right children. The program computes a
in-place spanning tree of the graph, \textit{i.e.} a tree that covers
all nodes of the graphs from a given root, concurrently as follows:
each time a new node is encountered, two new threads are created that
each prune the edges of the left and right children recursively. A
mark bit is associated to each node to keep track of which ones have
already been visited. Each thread returns whether the root
of the subgraph it was operating on had already been visited; if so,
the parent thread removes the link from its root node to the
corresponding child. Intuitively, it is allowed to do so because,
since the child was marked by another thread, the child has to be
already reachable via some other path in the graph.

%% marking the vertices to keep track of those already visited; the
%% return value b records the outcome of marking with \li{b}$=1$ when the
%% top node $x$ is unmarked (not visited yet) and \li{b}$=0$
%% otherwise. Assuming that the root vertex $x$ is unmarked initially,
%% the algorithm continues by first marking $x$ and subsequently spanning
%% the left and right subgraphs concurrently. When the top node of the
%% left subgraph is already marked (\li{!b1}), the edge from $x$ into it
%% is replaced by a null pointer. This corresponds to the case where the
%% node has already been visited by another thread and is thus reachable
%% from the root; \emph{mutatis mutandis} for the right subgraph.

We will prove that, given a shared graph as input, the program always
returns a tree, \textit{i.e.} all sharing and cycles have been
appropriately removed. Pleasingly, the \colosl specification achieves
maximal locality and allows us to reason only on the current subgraph
manipulated by a thread, instead of the whole graph. Because of
arbitrary sharing between the two, a global specification would be
unpleasant indeed!  However, we do not establish that the final tree
indeed spans the original graph. The reason it does is subtle indeed,
as are the invariants required, but in ways unrelated to our main
issue, which is to provide tight specifications for each thread.

To reason about this program, following Hobor and
Villard~\cite{ramification}, we use two representations of graphs. The
first is a mathematical representation $\gamma = (V, E)$ where $V$ is
a finite set of vertices and $E: V \rightarrow (V \uplus
\{\li{null}\}) \times (V \uplus \{\li{null}\})$ is a function
associating each vertex with at most two successors, where \li{null}
denotes the absence of an edge from the node.  We write $n \in \gamma$
for $n \in V$, $\gamma(n)$ for $E(n)$ and $|\gamma|$ for $|V|$.


%% Jules: definition of spanning tree is wrong (what's minimal?) since
%% graphs need not be connected now. Not needed anyway, as we don't
%% show that we have a spanning tree.
%% 
%% Similarly, let $\theta = (V, E)$ denote an \emph{acyclic} DCB graph
%% with no sharing between subgraphs, \textit{i.e.}, a \emph{tree}. For
%% brevity, in this section we refer to a DCB graph and an acyclic DCB
%% graph simply as a graph and a tree, respectively. Given a graph
%% $\gamma$, a tree $\theta$ \emph{spans} $\gamma$ if it includes all
%% vertices of $\gamma$ with a minimal set of edges where every edge in
%% $\theta$ is also an edge in $\gamma$.  For instance,
%% \fig\ref{fig:graphAndTree} depicts a DCB graph and a possible spanning
%% tree where the dashed lines indicate those edges that have been
%% removed from the graph after spanning it.



\begin{wrapfigure}[8]{r}{0.3\columnwidth}
	\centering
	\begin{tabular}{|c |}
		\hline
			\includegraphics[scale=0.27]{Sections/Examples/Images/graph.pdf} \\
		\hline
	\end{tabular}
\caption{A graph.}
\label{fig:graphAndTree}
\end{wrapfigure}
%%
%\begin{figure}
%\hrule
%\begin{tabular}{c c c}
%	\begin{subfigure}[b]{0.3\columnwidth}
%      \centering	
%      \includegraphics[scale=0.3]{Sections/FurtherExamples/Images/graph.pdf}
%    \caption{}
%    \label{subfig:graph}
%    \end{subfigure}
%    &
%    \begin{subfigure}[b]	{0.2\columnwidth}		
%      \centering	
%      \includegraphics[scale=0.3]{Sections/FurtherExamples/Images/tree.pdf}
%    \caption{}
%    \label{subfig:tree}
%    \end{subfigure}
%    &
%    \begin{subfigure}[b]{0.3\columnwidth}
%      \centering	
%      \includegraphics[scale=0.3]{Sections/FurtherExamples/Images/graphWithRootEdge.pdf}
%    \caption{}
%    \label{subfig:graphWithRootEdge}
%    \end{subfigure}
%    \end{tabular}
%\hrule
%\caption{A directed connected binary graph (\subref{subfig:graph}), a possible spanning tree (\subref{subfig:tree}), and the same graph with the logical edge $\rootEdge$ (\subref{subfig:graphWithRootEdge}).}
%\label{fig:graphAndTree}
%\end{figure}
%%

Mathematical graphs are connected to a second, in-memory
representation by an inductive predicate $\graph{x}{\gamma}$, denoting
a spatial (in-heap) graph rooted at address $x$ corresponding to the
mathematical graph $\gamma$. The predicate definition uses the
overlapping conjunction to account for the sharing between the left
and right children and for potential cycles in the graph, as shown in
\fig\ref{fig:globalCST}.  The basic action we allow on spatial graphs
is to \emph{mark} a node, changing its mark field from $0$ to $1$ and
claiming ownership of its left and right pointers in the process. Such
an action is allowed by a \emph{marking} capability of the form
$\markT{n}{e}$ where $n$ denotes the vertex (address) and $e$ the edge
via which vertex $n$ is visited. For instance, the capabilities
associated with marking of vertex $z$ in \fig\ref{fig:graphAndTree}
are $\markT{z}{y.r}$ and $\markT{z}{w.l}$. Note that the
parameterisation of our actions are merely a notational convenience
and can be substituted for their full definitions. Given a graph at
root address $x$, in order to account for the ability to mark the root
vertex $x$, we introduce a logical (virtual) root edge $\rootEdge$
into $x$ as depicted in \fig\ref{fig:graphAndTree} together with its
associated marking capability $\markT{x}{\rootEdge}$. The shared state
contains node $x$ which can be either unmarked ($\unmarked{x}{l}{r}$)
or marked ($\marked{x}$); as well as the left and right subgraphs
captured recursively by $\G{l}{\gamma}$ and $\G{r}{\gamma}$.
%% Note that the two subgraphs and vertex $x$ are combined by
%% the overlapping conjunction $\sepish$ since the graph can be cyclic
%% and each node may be reachable via more than one path.

Each vertex is represented as three consecutive cells in the heap
tracking the mark bit and the addresses of the left ($l$) and right
($r$) subgraphs. For brevity, we write $\cell{x}{m, l, r}$ for
$\cell{x}{m} * \cell{x+1}{l} * \cell{x+2}{r}$, and $x.m$, $x.l$, and
$x.r$ for $x$, $x+1$, and $x+2$, respectively. When vertex $x$ is in
the unmarked state, the whole cell $\cell{x}{0,l,r}$ and the
capabilities to mark the children reside in the shared state. In the
marked state, the shared state only contains $\cell{x.m}{1}$: the left
and right subgraphs (pointers and capabilities) have been claimed by
the thread who marked the node, while other threads need not access
the children of $x$ once they see that $x$ is already marked. The
atomic \li{CAS} instruction prevents several threads to concurrently
mark the same node and claim ownership of the same resource.

The interference associated with the graph is described as the union
of interferences pertaining to the vertices of the graph ($n \in
\gamma$). For each vertex $n \in \gamma$, the only permitted action is
that of marking $n$ which can be carried out by any of the marking
capabilities associated with node $n$ ($\markT{n}{-}$). Note that the
anonymous quantification $-$ is yet another notational shorthand and
can be substituted for the following more verbose definition.
%
\[
\vspace{-1ex}
I(n) \eqdef \bigcup\limits_{p \in \gamma} \left(\bigcup_{e \in \{p.l,
  p.r, \rootEdge\}}\!\!\! \markT{n}{e} : \exsts{l, r} \unmarked{n}{l}{r} \swap \marked{n} \right)
\]
%
%
\begin{figure}
%
\hrule
\[
\begin{array}{r @{\hspace*{2pt}} l}
	\graph{x}{\gamma} \eqdef & \left[\markT{x}{\rootEdge}\right] * \shared{\G{x}{\gamma}}{I_\gamma} \hspace*{0.5cm} I_\gamma \eqdef \bigcup\limits_{n \in \gamma}I(n)\\
%	
	\G{x}{\gamma} \eqdef & (x = \li{null} \land \emp) \lor x \in \gamma \land \exsts{l, r} \gamma(x) = (l, r) \\
	& \land \left( \unmarked{x}{l}{r} \lor \marked{x}\right) \sepish \G{l}{\gamma} \sepish \G{r}{\gamma}\\
%
	\unmarked{x}{l}{r} \eqdef & \cell{x}{0, l, r} * \left[\markT{l}{x.l}\right] * \left[\markT{r}{x.r} \right]\\
%	
	\marked{x} \eqdef & \cell{x}{1}\\
%
	I(n) \eqdef & \left\{ \markT{n}{-}: \exsts{l, r} \unmarked{n}{l}{r} \swap \marked{n}\right\}\\
%
%	\tree{x}{\gamma} \eqdef & \markT{x}{\rootEdge} * \shared{\G{x}{\gamma}}{\bigcup\limits_{n \in \gamma}I(n)}\\
\end{array}
\]
%
\hrule
\caption{Global specification of the graph predicate.}
\label{fig:globalCST}
\end{figure}
%
%
\fig\ref{fig:conSpanningTree} shows an in place concurrent algorithm for calculating a spanning tree of a graph. 
%
\begin{figure}
%
\hrule
\[
\begin{array}{r @{\hspace*{2pt}} l}
	\g{x}{\gamma} \eqdef & (x = \li{null} \land \emp) \lor x \in \gamma \land \exsts{l, r} \gamma(x) = (l, r) \land\\
	& \shared{\unmarked{x}{l}{r} \lor \marked{x}}{I(x)} * \g{l}{\gamma} * \g{r}{\gamma}\\
	
	\tr{x}{\gamma} \eqdef & (x = \li{null} \land \emp) \lor (x \in \gamma \land \exsts{l, r} \gamma(x) = (l, r) \land \exsts{l' \in \{l, \li{null}\}} \exsts{r' \in \{r, \li{null}\}}\\
	& \shared{\marked{x}}{I(x)} 
	\left[\markT{l}{x.l}\right] * \cell{x.l}{l'} * \tr{l'}{\gamma} * 
	\left[\markT{r}{x.r}\right] * \cell{x.r}{r'} * \tr{r'}{\gamma})
\end{array}
\]
\hrule
\caption{Local specification of the graph predicate.}
\label{fig:localCST}
\end{figure}
%
The $\graph{x}{\gamma}$ predicate defined in \fig\ref{fig:globalCST} is a \emph{global} account of the graph in that it captures all vertices and the interference associated with them. However, our spanning tree algorithm operates \emph{locally} as it is called upon recursively for each node. That is, for each \li{span(n)} call (where $\cell{\li{n}}{n}$ and $n \in \gamma$), the footprint of the call is limited to node $n$. Moreover, in order to reason about the concurrent recursive calls $\li{span(x.l)} || \li{span(x.r)}$, we need to \emph{split} the state into two $*$-composed states prior to the calls, pass each constituent state onto the relevant thread and combine the resulting states by $*$ composition through an application of the \parRule\ rule. We thus provide a \emph{local} specification of the graph, $\g{x}{\gamma}$ as defined in \fig\ref{fig:localCST} such that for all $n, p \in \gamma$ and $e \in \{p.l, p.r, \rootEdge \}$
%
\[
\begin{array}{@{} c @{} }
	\color{blue}{
	\Big\{
		\cell{\li{n}}{n} * \cell{\li{b}}{-} * 
		\left[\markT{n}{e}\right] * 
		\g{n}{\gamma}
	\Big\} 
	} \\
%	
	\command{b:= span(n)} \\ 
%
	\color{blue}{
	\Big\{
		\cell{\li{n}}{n} *  
		\left[\markT{n}{e}\right] * 
		\left(
%		\begin{array}{@{} l @{}}
			\cell{\li{b}}{1} * \tr{n}{\gamma} \lor
			\cell{\li{b}}{0} *  \tr{\li{null}}{\gamma}
%		\end{array}
		\right)
	\Big\}
	}
\end{array}
\]
%
The definition of the $\g{x}{\gamma}$ predicate is similar to that of $\shared{\G{x}{\gamma}}{I_{\gamma}}$ except that the global view $\shared{\G{x}{\gamma}}{I_{\gamma}}$ that describes the resources associated with all $|\gamma|$ vertices has been replaced by $|\gamma|$ $*$-composed more local views, each describing the resources of a vertex $n \in \gamma$. Moreover, the interference of each local view concerning a vertex $n \in \gamma$ has been shifted from $I_{\gamma}$ to $I(n)$ as to reflect only those actions that affect $n$.  

Similarly, the $\tr{x}{\gamma}$ predicate represents a \emph{tree}
rooted at $x$, as is standard in separation logic~\cite{rey02}, and
consists of $|\gamma|$ subjective views one for each vertex in
$\gamma$. The assertion of each subjective view reflects that the
corresponding vertex ($x$) has been marked
$\shared{\marked{x}}{I(x)}$. The resources associated with each node
$x$, namely the left and right pointers and the corresponding marking
capabilities have been claimed by the marking thread and moved into
the local state. The vertex addressed by the left pointer of $x$
(\textit{i.e.} $l'$) corresponds to either the initial value prior to
marking ($l$ where $\gamma(x) = (l, r)$) or \li{null}\ when $l$ has
more than one predecessors and has been marked by another thread,
making the whole predicate stable against actions of the program and
the environment.

We now demonstrate how to obtain the local specification $\g{x}{\gamma}$ from the global specification of \fig\ref{fig:globalCST}. 
When expanding the definition of $\G{x}{\gamma}$, there are two cases to consider depending on whether or not $x = \li{null}$. In what follows we only consider the case where $x \not= \li{null}$ since the derivation in the case of $x = \li{null}$ is trivial.
Let $P$ and $Q$ predicates be defined as below.
%
\[
\begin{array}{l l}
	P \eqdef & \iterStar_{n \in \gamma} \left( \gamma(n) = (l, r) \land (\unmarked{n}{l}{r} \lor \marked{n}) \; \right)\\
	
	Q \eqdef & \iterStar_{n \in \gamma} \left( \gamma(n) = (l, r) \land \shared{\unmarked{n}{l}{r} \lor \marked{n}}{I(n)} \right)\\
\end{array}	
\]
%
From the definitions of $\G{x}{\gamma}$ and $\g{x}{\gamma}$ we then have:
%
\begin{mathpar}
	\G{x}{\gamma} \iff  P
	
	\g{x}{\gamma} \iff Q
\end{mathpar}
%
%% %
%% \[
%% \begin{array}{l @{\hspace*{1cm}} c @{\hspace*{1cm}} l}
%% 	\G{x}{\gamma} \iff  P & \text{and} & \g{x}{\gamma} \iff Q
%% \end{array}
%% \]
%% %
In order to derive the local specification $\g{x}{\gamma}$ from the global specification $\G{x}{\gamma}$, it thus suffices to show $\shared{P}{I_{\gamma}} \semimplies Q$ as demonstrated below.
%
%\begin{align*}
%	&\shared{P}{I_{\gamma}}\!\!\!\!
%	\color{blue}\stackrel{(\textsc{Copy})}{\implies}\color{black}
%	\underbrace{\shared{P}{I_{\gamma}} * \cdots * \shared{P}{I_{\gamma}}}_{|\gamma| \text{ times}}\!\!\!
%	\stackrel{(\textsc{Forget})}{\implies}
%	\iterStar_{n \in \gamma} \left( \gamma(n) = (l, r) \land \shared{\unmarked{n}{l}{r} \lor \marked{n}}{I_{\gamma}}  \right)\\
%	&\stackrel{(\textsc{Shift})}{\semimplies}
%	\iterStar_{n \in \gamma} \left( \gamma(n) = (l, r) \land \shared{\unmarked{n}{l}{r} \lor \marked{n}}{I(n)}  \right)
%	\iffdef Q
%\end{align*}
%
\begin{align*}
	\shared{P}{I_{\gamma}} &
	\color{blue}\stackrel{(\textsc{Copy})}{\implies}\color{black}
	\underbrace{\shared{P}{I_{\gamma}} * \cdots * \shared{P}{I_{\gamma}}}_{|\gamma| \text{ times}}\\
	&\color{blue}\stackrel{(\textsc{Forget})}{\implies}\color{black}
	\iterStar_{n \in \gamma} \left( \gamma(n) = (l, r) \land \shared{\unmarked{n}{l}{r} \lor \marked{n}}{I_{\gamma}}  \right)\\
	& \color{blue}\stackrel{(\textsc{Shift})}{\semimplies}\color{black}
	\iterStar_{n \in \gamma} \left( \gamma(n) = (l, r) \land \shared{\unmarked{n}{l}{r} \lor \marked{n}}{I(n)}  \right)
	\color{blue}\iffdef \color{black}Q
\end{align*}
%
%%
%\[
%\begin{array}{@{} c @{} l @{}}
%	&\shared{P}{\bigcup\limits_{n \in S} I(n)}  \\
%	
%	\stackrel{(\textsf{G}\ \defin)}{\implies} & \shared{\exsts{l, r} (\unmarked{x}{l}{r} \lor \marked{x}) \sepish \G{l}{S} \sepish \G{r}{S}}{\bigcup\limits_{n \in S} I(n)} \\
%	
%	\implies &   \exsts{l, r}  \shared{(\unmarked{x}{l}{r} \lor \marked{x}) \sepish \G{l}{S} \sepish \G{r}{S}}{\bigcup\limits_{n \in S} I(n)} \\
%	
%	\stackrel{(\textsc{Copy})}{\implies} &
%	\exsts{l, r}  
%	\shared{(\unmarked{x}{l}{r} \lor \marked{x})  \sepish \G{l}{S} \sepish \G{r}{S}}{\bigcup\limits_{n \in S} I(n)} \\
%	& * \shared{(\unmarked{x}{l}{r} \lor \marked{x}) \sepish \G{l}{S} \sepish \G{r}{S}}{\bigcup\limits_{n \in S} I(n)} \\
%	& * \shared{(\unmarked{x}{l}{r} \lor \marked{x})  \sepish \G{l}{S} \sepish \G{r}{S}}{\bigcup\limits_{n \in S} I(n)} \\
%	
%	
%	\stackrel{(\textsc{Forget})}{\implies} &
%	\exsts{l, r}  
%	\shared{\unmarked{x}{l}{r} \lor \marked{x}  }{\bigcup\limits_{n \in S} I(n)} \\
%	& * \shared{\G{l}{S}}{\bigcup\limits_{n \in S} I(n)} * \shared{\G{r}{S}}{\bigcup\limits_{n \in S} I(n)} \\
%	
%	
%	
%	\stackrel{(?)}{\semimplies} &
%	\exsts{l, r}  
%	\shared{\unmarked{x}{l}{r} \lor \marked{x}}{\bigcup\limits_{n \in S} I(n)} * \g{l}{S} * \g{r}{S}\\
%	
%	
%	\stackrel{(\textsc{Shift})}{\semimplies} &
%	\exsts{l, r}  
%	\shared{\unmarked{x}{l}{r} \lor \marked{x} }{I(x)} * \g{l}{S} * \g{r}{S}\\
%	
%	
%	\iffdef & \g{x}{S}
%	
%\end{array}
%\]
%%
%
\begin{figure}
\hrule     
\begin{lstlisting}
  //<@\codecomment{$\cell{\tx{x}}{x} * \cell{\tx{b}}{-} * \graph{x}{\gamma}$}@>
  //<@\codecomment{$\cell{\tx{x}}{x} *  \cell{\tx{b}}{-} * \markT{x}{\rootEdge} * \shared{\G{x}{\gamma}}{I_{\gamma}}\}$}@>
  //<@\codecomment{$\{\cell{\tx{x}}{x} * \cell{\tx{b}}{-} * \left[\markT{x}{\rootEdge}\right] * \g{x}{\gamma}\}$}@>
  b:= span(x) $\{$
  //<@\codecomment{$\left\{\begin{array}{@{}l@{}} \cell{\tx{x}}{x} * \cell{\tx{b}}{-} * \left[\markT{x}{\rootEdge}\right] * \exsts{l, r} \shared{\unmarked{x}{l}{r} \lor \marked{x}}{I(x)} * \g{l}{\gamma} * \g{r}{\gamma} \end{array} \right\}$}@>
    res:= $\langle$ CAS(x.m, 0, 1) $\rangle$;
    //<@\codecomment{$\left\{\begin{array}{@{} l @{}} \cell{\tx{x}}{x} * \cell{\tx{b}}{-} * \left[\markT{x}{\rootEdge}\right] * \shared{\marked{x}}{I(x)} * \exsts{l, r} \g{l}{\gamma} * \g{r}{\gamma} \\ * \left(\cell{\tx{res}}{0} \lor  \left(\begin{array}{l} \cell{\tx{res}}{1} * \cell{x.l}{l} * \cell{x.r}{r} * \left[\markT{l}{x.l}\right] * \left[\markT{r}{x.r}\right] \end{array}\right)\right)\end{array}\right\}$}@>
    if (res) then $\{$ 
      //<@\codecomment{$\left\{\begin{array}{@{} l @{}} \cell{\tx{x}}{x} * \cell{\tx{b}}{-} * \left[\markT{x}{\rootEdge} \right] * \shared{\marked{x}}{I(x)} * \cell{\tx{res}}{1}\\ * \exsts{l, r} \cell{x.l}{l} * \cell{x.r}{r} * \left[\markT{l}{x.l} \right] * \g{l}{\gamma} * \left[\markT{r}{x.r}\right] * \g{r}{\gamma}  \end{array} \right\}$}@>
      //<@\codecomment{$\left\{ \left[\markT{l}{x.l} \right] * \g{l}{\gamma} * \left[\markT{r}{x.r} \right] * \g{r}{\gamma}  \right\}$}@>
      b1:= span(x.l) || b2:= span(x.r)
      //<@\codecomment{$\left\{\begin{array}{@{} l @{}}  \left[\markT{l}{x.l} \right] * \left( (\cell{\tx{b1}}{1} *  \tr{l}{\gamma}) \lor \cell{\tx{b1}}{0}\right) *\\ \left[ \markT{r}{x.r} \right] * \left( (\cell{\tx{b2}}{1} * \tr{r}{\gamma}) \lor \cell{\tx{b2}}{0} \right)  \end{array}\right\}$}@>
      //<@\codecomment{$\left\{\begin{array}{@{} l @{} }  \cell{\tx{x}}{x} * \cell{\tx{b}}{-} * \left[ \markT{x}{\rootEdge} \right] * \shared{\marked{x}}{I(x)} * \cell{\tx{res}}{1} \\ \begin{array}{@{}l @{\hspace{2pt}} l @{} } * \exsts{l, r} &\cell{x.l}{l} * \left[ \markT{l}{x.l} \right] * \left( (\cell{\tx{b1}}{1} *  \tr{l}{\gamma}) \lor \cell{\tx{b1}}{0} \right) *\\  &\cell{x.r}{r} *\left[ \markT{r}{x.r} \right] * \left( (\cell{\tx{b2}}{1} * \tr{r}{\gamma}) \lor \cell{\tx{b2}}{0} \right) \end{array}   \end{array}\right\}$}@>
      if (!b1) then 
        $\text{[}$x.l$\text{]}$:= null
      if (!b2) then 
        $\text{[}$x.r$\text{]}$:= null
      //<@\codecomment{$\left\{\begin{array}{@{} l @{}}  \cell{\tx{x}}{x} * \cell{\tx{b}}{-} * \left[ \markT{x}{\rootEdge} \right] * \shared{\marked{x}}{I(x)} * \cell{\tx{res}}{1} * \cell{\tx{b1}}{-} * \cell{\tx{b2}}{-}\\  \begin{array}{@{} l @{\hspace{2pt}} l @{}} * \exsts{l, r}  & \exsts{l' \in \{l, \tx{null}\}} \left[ \markT{l}{x.l} \right] * \cell{x.l}{l'} * \tr{l'}{\gamma} *\\ & \exsts{r' \in \{r, \tx{null}\}} \left[ \markT{r}{x.r} \right] * \cell{x.r}{r'} * \tr{r'}{\gamma} \end{array} \end{array}\right\}$}@>
      //<@\codecomment{$\left\{\begin{array}{@{} l @{}}  \cell{\tx{x}}{x} * \cell{\tx{b}}{-} * \cell{\tx{res}}{1} *\cell{\tx{b1}}{-} * \cell{\tx{b2}}{-} * \markT{x}{\rootEdge} *  \tr{x}{\gamma}    \end{array}\right\}$}@>
    $\}$    
    //<@\codecomment{$\left\{ \begin{array}{@{} l @{}} \cell{\tx{x}}{x} * \cell{\tx{b}}{-} * \left[ \markT{x}{\rootEdge} \right]*  \\  (\cell{\tx{res}}{1}  * \tr{x}{\gamma}) \lor \left(\begin{array}{@{} l @{}}\cell{\tx{res}}{0} * \shared{\marked{x}}{I(x)} * \g{l}{\gamma} * \g{r}{\gamma} \end{array}\right) \end{array} \right\}$}@>
    //<@\codecomment{$\left\{ \begin{array}{@{} l @{}} \cell{\tx{x}}{x} * \cell{\tx{b}}{-} * \left[ \markT{x}{\rootEdge} \right] *   (\cell{\tx{res}}{1}  * \tr{x}{\gamma}) \lor (\cell{\tx{res}}{0} ) \end{array} \right\}$}@>
    return res
  $\}$
  //<@\codecomment{$\left\{ \begin{array}{@{} l @{}} \cell{\tx{x}}{x}  * \left[ \markT{x}{\rootEdge} \right] *  ((\cell{\tx{b}}{1}  * \tr{x}{\gamma}) \lor \cell{\tx{b}}{0}) \end{array} \right\}$}@>
\end{lstlisting}
\hrule\vspace*{-6pt}
\caption{Concurrent Spanning Tree Implementation}
\label{fig:conSpanningTree}
\end{figure}
%
%
\vspace{-2ex}
%\clearpage
