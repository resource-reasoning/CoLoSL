\section{\colosl Model}\label{sec:model}
We formally describe the underlying model of \colosl and present the various ingredients necessary for defining \emph{worlds}: which are \colosl's notion of states.
% that track the resources held by each thread, the shared resources accessible to all threads, as well the ways in which the shared resources may be manipulated by each thread.
We formalise the notions of interference confinement and action shifting as stipulated by the \extendRule and \shiftRule principles.
%
%
%
\paragraph{\textbf{Worlds overview}}
A \emph{world} is a triple $(l, g, \lmod)$ that is \emph{well-formed} where $l$ and $g$ are \emph{logical states} and $\lmod$ is an \emph{action model}; let us explain the role of each component informally. The \emph{local logical state}, or simply local state, $l$ represents the locally owned resources of a thread. The \emph{shared logical state}, or shared state, $g$ represents the \emph{entire} (global) shared state, accessible to all threads, subject to interferences as described by the action models.

An action model is a partial function from \emph{capabilities} to sets of \emph{actions}. An action is a pair $(p,q)$ of logical states where $p$ is the \emph{pre-state} of the action and $q$ its \emph{post-state}.  The \emph{action model} $\lmod$ corresponds directly to the (semantic interpretation of) an interference assertion $I$. Although worlds do not put further constraints on the
relationship between $\lmod$ and $g$, they are linked more tightly in the semantics of assertions (\S\ref{sec:assertions}). 
%As $g$ represents the global shared state, as part of the well-formedness condition we require that actions in $\lmod$ and $\gmod$ are \emph{confined} to $g$.  We write $g \containI \lmod$ to denote that those parts of the shared state mutated by actions in $\lmod$ are fully contained in $g$. As threads may unilaterally decide to introduce part of their local states into the shared state at any point (by \extendRule), this ensures that existing actions cannot affect future extensions of shared state. Similarly, we require that the new actions associated with newly shared state are confined to that extension in the same vein, hence extending the shared state cannot retroactively invalidate the views of other threads. As we will demonstrate, confinement does not however prohibit \emph{referring} to existing parts of the shared state in the new actions; rather, it only safeguards against \emph{mutation} of the already shared resources through new actions.
Finally, the composition of two worlds will be defined whenever their local states are disjoint and they agree on all other two components, hence have identical knowledge of the shared state and possible interferences.\\

%\paragraph{Worlds and Actions}
We proceed by defining logical states, which are \colosl's notion of \emph{resource}, in the standard separation logic sense. Logical states have two components: one describes machine states (\textit{e.g.} stacks and heaps); the other
represents \emph{capabilities}. The latter are inspired by the capabilities in deny-guarantee reasoning~\cite{dg}: a thread in possession of a given capability is allowed to perform the associated actions (as prescribed by the \emph{action model} components of each world, defined below), while any capability \emph{not} owned by a thread means that the environment can perform the action.
%
%
\begin{definition}[Logical states]
A \emph{logical state} is a tuple $((@s, h), \ca{})$, also written $(@s,h,\ca{})$, of a finite partial \emph{stack} $@s\in\set{Stack}$ associating program variables with values, a heap $h \in \set{Heap}$ associating heap locations with values, and a capability $\ca{}\in\Caps$. We assume infinite disjoint sets $\set{PVar}$, $\set{LVar}$, and $\set{Token}$ of program variables, logical variables, and tokens,
respectively.
%
\begin{mathpar}
	\set{Stack} == \set{PVar} --`_{\m{fin}} \set{Val}

	\set{Heap} == \set{Loc} --`_{\m{fin}} \set{Val}

	\Heaps == \set{Stack}\times \set{Heap}

	\Caps == \powerset(\set{Token})

	\LStates == \Heaps\times \Caps
\end{mathpar}
%
We write $l, l_1, \cdots, l_n$ to range over either arbitrary logical states or those representing the local logical state. Similarly, we write $g, g_1, \cdots, g_n$ to range over logical states when representing the shared (global) state. We write $\unitL$ for the logical state $(\emptyset, \unitH, \unitCap)$. The \emph{composition of logical states} $ \composeL : \LStates \times \LStates \rightharpoonup \LStates $ is defined as the disjoint union $\uplus$ of their individual components:
%
\[
  (@s, h,\ca{}) \composeL (@s', h', \ca{}') \eqdef
  (@s\uplus @s', h\composeH h', \ca{}\composeCap \ca{}')
\]
%
The \emph{separation algebra of logical states} is given by $(\LStates, \composeL, \unitL)$.  
\end{definition}
%
%
Although, for this presentation, we build logical states from a fixed $\Heaps$ and $\Caps$, the general theory of \colosl~\cite{colosl-tr14} allows these two objects to be instantiated with \emph{any} separation algebras (i.e. a cancellative, partial commutative monoid~\cite{asl}) that satisfies the \emph{cross-split} property\footnote{
A separation algebra $(\mathbb{B}, +, \mathbf{1})$ satisfies the cross-split property iff for all $a, b, c, d \in \mathbb{B}$ where $a + b = c + d$, there exists $ac, ad, bc, bd \in \mathbb{B}$ s.t.
%
\vspace{-3pt}
\[
	a = ac + ad /| b = bc + bd /|
	c = ac + bc /| d = ad + bd \vspace{-10pt}
\]
}.
As we demonstrate in the examples of \S\ref{sec:examples}, our programs often call for a more complex model of machine states and capabilities. For instance, we may need our capabilities to be fractionally owned, where ownership of a \emph{fraction} of a capability grants the right to perform the action to both the thread and the environment, while a fully-owned capability by the thread \emph{denies} the right to the environment to perform the associated action. 

Oftentimes, we need to compare two logical states $l_1 \leq l_2$ (or their constituent components) defined when there exists $l$ such that $l\composeL l_1 = l_2$. We also write $l_2 - l_1$ to denote the unique (by cancellativity) such $l$. When $l_1$ and $l_2$ are compatible according to $\composeL$, we write $l_1\compatL l_2$.
%
%

We now proceed with the next ingredient of a \colosl world, namely, action models. Recall from above that an action is simply a pair of logical states describing the pre- and post-states of the action, while an action model describes the set of actions associated with each capability.
%
%
\begin{definition}[Actions, action models]
The set of \emph{actions}, $\set{Action}$, ranged over by $a, a_1, \cdots, a_n$; and the set of \emph{action models}, $\AMods$, ranged over by $\lmod, \lmod_1, \cdots, \lmod_n$ are defined as follows.
%
\begin{mathpar}
  \set{Action} == \LStates \times \LStates

	\AMods == \Caps \rightharpoonup \pset{\set{Action}}
\end{mathpar}
%  
We write $\unitAM$ for an action model with empty domain.
\end{definition}
%
%
\paragraph{\textbf{Action confinement}} 
Given a world $(l, g, \lmod)$, since $g$ represents the \emph{entire} shared state, as part of the well-formedness condition of worlds we require that the actions in $\lmod$ are \emph{confined} to $g$.  
%We write $g \containI \lmod$ to denote that those parts of the shared state mutated by actions in $\lmod$ are fully contained in $g$. 
Let us elaborate on the necessity of the confinement condition.

As threads may unilaterally decide to introduce part of their local states into the shared state at any point (by \extendRule), confinement ensures that existing actions cannot affect future extensions of the shared state. Similarly, we require that the new actions associated with newly shared state are confined to that extension in the same vein, hence extending the shared state cannot retroactively invalidate the views of other threads. However, as we will demonstrate, confinement does not prohibit \emph{referring} to existing parts of the shared state in the new actions; rather, it only safeguards against \emph{mutation} of the already shared resources through new actions.

Through confinement we ensure that the \emph{effect} of actions in the action model are contained to the shared state. That is, given an action $a = (p,q)$ and a shared state $g$, whenever the precondition $p$ \emph{agrees} with $g$ then the part of the state mutated by the action is contained in $g$. 
Agreement of $p$ and $g$ merely means that $p$ and $g$ agree on the resources they have in common. 
In particular, $g$ need not contain $p$ entirely for $a$ to take effect. This relaxation is due to the fact that other threads may extend the shared state; in particular, the extension may provide the missing resources for $p$ to be contained in the shared state, thus allowing the extending thread to perform action $a$. Crucially, however, it is not the case that any part of $p$ is allowed to be missing from $g$. Rather, only parts of $p$ that are not mutated by $a$ (and thus can also be found in $q$), can escape $g$'s grasp. Mutated parts must always be contained in $g$, so that extensions need not worry about existing actions interfering with new resources that were never shared beforehand. In order to distinguish between parts that must be contained in $g$ and those that may be missing, we characterise the \emph{active} and \emph{passive} parts of actions, and then continue to define the effect of actions. This will enable us to define our confinement condition.
%, and then the set of well-formed worlds.

An action $a = (p, q)$ is typically of the form $(p' \composeL c, q' \composeL c)$, where part of the state $c$ required for the action is \emph{passive} and acts as a mere \emph{catalyst} for the action: it has to be present for the action to take effect, but is left unchanged by the action. The \emph{active} part of the action is then the pair $(p',q')$, which should be maximal in the sense that no further, non-empty catalyst can be found in $(p',q')$.
%
%
\begin{definition}[Active part of an action]
Given an action $a = (p, q)$, its \emph{active part}, $\updateFP{(p,q)}$, is defined as the pair $(p', q')$ such that:
%
\[
	\E c p = p' \composeL c /| q = q' \composeL c /| \V{c'} c'\leq p' /|
  c'\leq q' => c' = \unitL
\]
%
\end{definition}
%
%
%For instance, the interpretations of the actions associated with token $\token a_{\li{x}}$ in the previous section all have the same active parts $([x:v],\emptyset,\emptyset),([x:v+1],\emptyset,\emptyset)$. Their catalysts are of the form $([z:v],\emptyset,\emptyset)$ in the interpretations of $I$, $I_{\li{x}}$, and $I_{\li{z}}$, and $([y:v,z:v],\emptyset,\emptyset)$ in $I_{\li{y}}$.
%
%
The agreement of an action pre-state $p$ and the shared state $g$ will be defined using the following notion of \emph{intersection} of logical states.
%
%
\begin{definition}[Intersection]
The \emph{intersection} function over logical states,
$
\meetL : \left(\LStates \times \LStates \right) \rightarrow \pset{\LStates}
$, is defined as follows.
%
\[
	l_1 \meetL l_2 \eqdef 
	\left\{ 
		l  \mid
		\exsts{l', l_1', l_2'}\ l_1 = l \composeL l_1' \land l_2 = l \composeL l_2' \land l \composeL l_1' \composeL l_2' = l'
	\right\}
\]
%
\end{definition}
%
Given our choice of logical states in this paper, $l_1\meetL l_2$ yields at most one element for any given $l_1$ and $l_2$; we thus usually omit the set notation\footnote{In general, 
%separation algebras may lack the \emph{disjointness} property, 
intersection between two states may yield more than one state. Thus, in our general definitions we use the set notation.}.
%% One can check that $l_1\maxMeetL l_2$ indeed yields at most one
%% element for any given $l_1$ and $l_2$~\cite{colosl-tr14}.
%

An action pre-state $p$ and a shared state $g$ then agree if their intersection is non-empty, i.e. $p \meetL g \not= \emptyset$. We can now define action confinement.
%
%
\begin{definition}[Action confinement]\label{def:actconf}
An action $a$ is \emph{confined} to a logical state $g$, written $g\containI a$, if:
%
\[
	\m{fst}(a)\meetL g \neq\emptyset => \m{fst}(\updateFP{a})\leq g 
\]
\end{definition}
%
As discussed, only the \emph{active} precondition of $a$, i.e. the part actually mutated by the action, has to be contained in $g$. This way, future extensions of $g$ need not account for existing actions interfering with new resources.

Given a shared state $g$ and an action model $\lmod$, we require that all actions of $\lmod$ are confined in all possible \emph{futures} of $g$, i.e. all shared states resulting from $g$ after any number of applications of actions in $\lmod$. For that we define \emph{action application} that describes the effect of an action on a logical state. Moreover, for some of the actions in $\lmod$, the active pre-state may not affect $g$, i.e. its intersection with $g$ may be the empty state $\unitL$. In that case, we find that even though that action is potentially enabled, we do not need to account for it since it leaves $g$ unchanged. We thus introduce the notion of \emph{visible actions} to quantify over those actions that affect (mutate) $g$. 
%
\begin{definition}[Action application and visibility]\label{def:actionApplication}
The \emph{application} of an action $a$ on a logical state $g$, written $a[g]$, is defined provided there exists $l$ such that
%
\vspace{-7pt}
\[
	\m{fst}(a) \meetL g /=\emptyset /|
	g = \m{fst}(\updateFP{a}) \composeL l /|
	\m{snd}(\updateFP{a})\compatL l
\]
%
When that is the case, we write $a[g]$ for the (uniquely defined) logical state $\m{snd}(\updateFP{a})\composeL l$. We write $\m{potential}(a,g)$ to denote that $a[g]$ is defined.
An action $a$ is called \emph{visible in $g$}, written $\m{visible}(a,g)$, when:
%
\quad
$
	\m{fst}(\updateFP{a})\meetL g /= \unitL
$
%
%
\end{definition}
%
Observe that $\m{potential}(a,g)$ implies $g\containI a$.  
%
%
We are now ready to define our confinement condition on action models. Inspired by Local RG~\cite{lrg}, we introduce the concept of locally fenced action models to capture all possible states reachable from the current state via some number of action applications. A set of states $\fence{}$ \emph{fences} an action model if it is invariant under interferences perpetrated by the corresponding actions. An action model is then confined to a logical state $l$ if it can be fenced by a set of states that includes $l$. In the following we write $\m{rg}(f)$ to denote the \emph{range} (or co-domain) of a function $f$.
%
%
\begin{definition}[Locally-fenced action model]\label{def:localFence}
An action model $\lmod \in \AMods$ is \emph{locally fenced} by $\fence{} \in \pset{\LStates}$, written $\fence{} \strictfences \lmod$, iff for all $g \in \fence{}$ and all $a\in\m{rg}(\lmod)$,
%
\vspace{-7pt}
\[
\begin{array}{L}
  g\containI a /|
%  (\m{potential}(a,g) /| (\m{visible}(a,g) |/ \fst{\updateFP{a}} = \unitL ) => a[g]\in\fence{})
	(\m{potential}(a,g) => a[g]\in\fence{})
\end{array}
\]
\end{definition}
%
%
\begin{definition}[Action model confinement]\label{amod-confinement}
An action model $\lmod$ is \emph{confined} to a set of logical state $\mathcal{P}$, written $\mathcal{P} \confines \lmod$, if there exists a local fence $\fence{} \in \pset{\LStates}$ such that:
%
\qquad
$
\mathcal{P} \subseteq \fence{} \land \fence{}\strictfences{}\lmod
$
\end{definition}
%
%
Later in \S\ref{sec:logic} we lift this definition to assertions and write $P \confines I$ to denote that the interpretation of $I$ is confined to the states described by $P$. We present a number of judgements that provide a syntactic mechanism for despatching the confinement obligation rather than performing semantic checks. As such we reduce the stipulated side condition of the \extendRule principle to logical entailments.

We can now formalise the notion of well-formedness. A world $(l ,g, \lmod)$ is well-formed if $\lmod$ is confined to $g$, $l$ and $g$ are compatible, and the capabilities found in $l$ and $g$ are contained in the domain of $\lmod$. Since capabilities enable the manipulation of the shared state through their associated actions in the action model, the last constraints ensures that \emph{all} capabilities found in $l \composeL g$ are accounted for in $\lmod$. 
%We proceed by defining what it means for capabilities to be contained in an action model and subsequently formalise the notion of well-formed states.
%
%
\begin{definition}[Well-formedness]
A triple $(l, g, \lmod)$ is \emph{well-formed}, written $\wf{l, g, \lmod}$, iff
%
\vspace{-1ex}
\[
\begin{array}{L}
	g \containI \lmod
	/|
	(\exsts{@s, h, \ca{}}
	l \composeL g = (@s, h, \ca{}) \land \ca{} \containedIn \lmod)
\end{array}
\vspace{-1ex}
\]
%
\end{definition}
%

\begin{definition}[Worlds]\label{def:worlds}
The set of \emph{worlds} is defined as
%
\[
	\Worlds \eqdef 
	\{w\in \LStates\times\LStates\times\AMods ||| \wf{w}\}
\]
%
The \emph{composition} $w\composeW w'$ of two worlds $\composeW: \Worlds \rightarrow \Worlds \rightharpoonup \Worlds$, is defined as:.
% w = (l,g,\lmod,\gmod)$ and $w' = (l',g',\lmod',\gmod')$ is defined as $(l\composeL l', g, \lmod, \gmod)$ when $g = g'$, $\lmod = \lmod'$, $\gmod = \gmod'$, and $\p{wf}((l\composeL l', g, \lmod, \gmod))$, and is undefined otherwise.
%
\[
	(l,g,\lmod) \composeW (l',g',\lmod') \eqdef
	\begin{cases}
		(l\composeL l', g, \lmod) &
		\begin{array}[t]{L}
			\text{if }
			g = g' \text{, and }
			\lmod = \lmod' \text{, and }\p{wf}((l\composeL l', g, \lmod))
		\end{array}\\
		\textit{undefined}&\text{otherwise}
	\end{cases}
\]
%
The set of worlds with composition $\composeW$ forms a separation algebra with multiple units: all well-formed states of the form $(\unitL,g,\lmod)$. 
%Given a world $w$, we write $\localPart{w}$ for the first projection.
%
\end{definition}
%
%
%\paragraph{\textbf{Shared State Extension}}
%When extending the shared state using currently owned local resources, one specifies a new interference assertion over these newly shared resources. While in \colosl the new interferences may mention parts of the shared state beyond the newly added resources (in particular the existing shared state), they must not allow visible updates to those parts, so as not to invalidate other threads' views of existing resources. We thus impose a locality condition on the newly added behaviour to ensure sound extension of the shared state, similarly to the confinement constraint of local fences of \defin\ref{def:actconf}. We first motivate this constraint with an example.
%
%\begin{example}\label{ex:badExtension}
%Let $P \eqdef \cell{x}{1} * \shared{\cell{y}{1} \lor \cell{y}{2}}{I}$ denote the view of the current thread with $I \eqdef \left(\token{b}: \left\{\cell{y}{1} \swap \cell{y}{2}\right\} \right)$. Since the current thread owns the location addressed by $x$, it can extend the shared state as $Q \eqdef [\token{a}] * \shared{\left(\cell{y}{1} |/  \cell{y}{2} \right) * \cell{x}{1}}{I \cup I'}$ where $ I' \eqdef \left( \token{a}: \cell{x}{1} \swap \cell{x}{2} \right) $.
%In extending the shared state, the current thread also extended the interference allowed on the shared state by adding a new action associated with the newly generated capability resource $[\token{a}]$, as given in $I'$, which updates the value of location $x$. Since location $x$ was previously owned privately by the current thread and was hence not visible to other threads, this new action will not invalidate their view of the shared state, hence this extension is a valid one.
%
%If, on the other hand, $I'$ is replaced with $I'' \eqdef \left( \token{a}: \left\{\cell{y}{1} \swap \cell{y}{3}
%\right\}\right)$, where location $y$ can be mutated, the situation above is not allowed.  Indeed, other threads may rely on the fact that the only updates allowed on location $y$ are done through the $[\token{b}]$ capability as specified in $I$, and would be spooked by this new possible behaviour they were not aware of (as it is not
%in $I$).
%\end{example}
%%
%%
%In order to ensure sound extension of the shared state, we require in \extendRule\ that the newly introduced interferences are confined to the locally owned resources.
%
%\begin{definition}\label{def:confinement}
%A set of states $\mathcal P$ \emph{confines} an action model $\lmod$, written $\mathcal P \containI \lmod$, if
%%
%\[
%  \E{\fence{}} \mathcal P\subseteq \fence{} /| \fence{} \strictfences \lmod
%\]
%\end{definition}
%%
%In \S\ref{sec:logic} we lift confinement to assertions and provide judgements that provide a syntactic mechanism for despatching the confinement obligation rather than performing semantic checks. As such we reduce the stipulated side condition of the \extendRule principle ($P \confines I$) to logical entailments. 
%%The $P\containI I$ notation used in \extendRule is then a straightforward lift of this definition to assertions $P \in \Assertions$ and interference assertions $I \in \IAssertions$:
%%%
%%\[
%%	P \containI I \iffdef  \for{\lenv} \left\{l \mid l, \lenv \slsat P \right\} \containI \semI[\lenv]{I}
%%\]
%%%
%%%
\paragraph{\textbf{Action Shifting}}
Recall from \S\ref{sec:intuition} that given a subjective view $\shared{P}{I}$, a thread may rewrite the actions in $I$, or forget about those actions that are not relevant to the current subjective view ($P$) as well as any evolution of it according to potential actions, both from the thread and the environment. To capture this set of possible futures, we refine our notion of local fences (\defin~\ref{def:localFence}), which was defined in the context of the global shared state, to the case where we consider only a subjective state within the global shared state.
For this, we need to also refine a second time our notion of action application to ignore the context of a subjective state, which as far as a subjective view is concerned could be anything.
%
%
\begin{definition}[Subjective action application]
The \emph{subjective application} of an action $a$ on a logical state $s$, written $a(s)$ is defined provided that there exists a context $r$ for which $a[s \composeL r]$ is defined. When that is the case, we write $a(s)$ for
%%
%\[
%  \left\{ \snd{\updateFP{a}} \composeL s' \ \middle|\ 
%  \begin{array}{l}
%%  	\exsts{p_s > \unitL} \exsts{s', r}\\
%		\exsts{p_s, r}\
%  	\fst{\updateFP{a}}= p_s \composeL r /| s = p_s \composeL s' /| s \compatible r 
%  \end{array}\; \right\}
%\]
%%
%
\[
  \left\{ s' \ \middle|\ 
  \begin{array}{l}
%  	\exsts{p_s > \unitL} \exsts{s', r}\\
		\exsts{r}\
  	s \compatible r /| (s', -) \in a[s, r]
  \end{array}\; \right\}
\]
%
\end{definition}
Note that, in contrast with $a[l]$, only \emph{parts} of the active precondition has to intersect with the subjective view $s$ for $a(s)$ to apply. Thus, we fabricate a context $r$ that is compatible with the subjective view and satisfies the rest of the precondition. 
%Note that, in contrast with $a[l]$, only \emph{parts} of the active precondition $p' = \m{fst}(\updateFP{a})$ (namely, $p_s$) has to intersect with the subjective view $s$ for $a(s)$ to apply. Thus, we fabricate a context $r$ that is compatible with the subjective view and satisfies the rest of $p'$.
% to be able to reuse our $a[s,r]$
%definition (\defin\ref{def:actionApplicationPair}). We observe that $\m{snd} \left( a[s, \m{fst}(\updateFP{a}) - (s\meetL \m{fst}(\updateFP{a}))] \right) = \unitL$ always, since it is fabricated as the minimum context needed. 

\begin{definition}[Fenced action model]
An action model $\lmod \in \AMods$ is \emph{fenced} by $\fence{} \in \pset{\LStates}$, written $\fence{} \fences \lmod$, if, for all $l \in \fence{}$ and all $a \in\m{rg}(\lmod)$,
\[
\begin{array}{L}
%  \m{visible}(a,l) =>  \for{s' \in a(l)} s' \in\fence{}
	a(l) \text{ is defined} =>  a(l) \subseteq \fence{}
\end{array}
\]
\end{definition}

In contrast with local fences, fences do not require that actions be confined inside the subjective state.  %We lift the notion of fences to assertions as follows, given $\fenceAss{} \in \Assertions$ and $I \in \IAssertions$:
%%
%\begin{align*}
%  \fenceAss{} \fences I & \iffdef \for{\lenv}
%  \{ l \mid l, \lenv \slsat F \} \fences \semI[\lenv]{I}
%\end{align*}
%%
%
%For instance, a possible fence for the interference assertion $I_{\li{y}}$ of \fig\ref{fig:concurrentInc2} is denoted by the following assertion.
%%
%\[
%	F_{\li{y}} \eqdef \bigvee_{v = 0}^{10} (x|-> v * y|->v) |/ (x|->v+1 * y|->v)
%\]
%%
We proceed with the definition of action shifting.
%
%
\begin{definition}[Action shifting]
Given $\lmod, \lmod' \in \AMods$ and $\mathcal{P} \in \pset{\LStates}$, $\lmod'$ is a \emph{shifting} of $\lmod$ with respect to $\mathcal{P}$, written $\lmod \weakenI{\mathcal{P}} \lmod'$, if there exists a fence $\fence{}$ s.t.
%
\[
\begin{array}{L}
	\mathcal{P} \subseteq \fence{} \land \fence{} \fences \lmod
	\land
	\for{l\in \fence{}}\for{\ca{}}\\
	\ (\for{a\in\lmod'(\ca{})}
	\m{reflected}(a,l,\lmod(\ca{}))) /|\null\\
	\ \for{a\in\lmod(\ca{})}
	a(l)\text{ is defined} {/|}\m{visible}(a,l) {=>}
	\m{reflected}(a,l,\lmod'(\ca{}))
\end{array}
\]
%
\end{definition}

The conjunct on the second line expresses the fact that the new action model $\lmod'$ cannot introduce new actions not present in $\lmod$, and the last conjunct that $\lmod'$ has to include all the visible potential actions of $\lmod$. In \S\ref{sec:logic} we lift $\weakenI{}$ to assertions and provide a a number of judgements that reduce the stipulated side condition of the \shiftRule principle ($I \weakenI{P} I'$) to logical entailments.  
%%
%\[
%	I \weakenI{P} I' == \V{\lenv} \semI[\lenv]{I} \weakenI{\{s|||  s,\lenv\slsat P\}} \semI[\lenv]{I'}
%\]
%%

%% Intuitively, there exists an invariant $\fence{}$ that contains the states in $\mathcal{P}$ and encompasses the behaviour of actions in $\gmod$ and 
%% \begin{enumerate}
%% 	\item if an action in $\gmod'$ is possible given a state $l \in \fence{}$ and an arbitrary context $r$ (its precondition is satisfiable by $l \composeL r$), then there exists a similar action with the same update footprint in $\gmod$ whose precondition is also satisfiable by $l \composeL r$. 
%% 	\item if an action in $\gmod$ is possible given a state $l \in \fence{}$ and an arbitrary context $r$ (its precondition is satisfiable by $l \composeL r$), then \emph{either}  there exists a similar action with the same update footprint in $\gmod'$ whose precondition is also satisfiable by $l \composeL r$; \emph{or} the action does not affect $l$, that is  $l \meetL p' = \{\unitL\}$ where $(p', q')$ denotes the update footprint of the action; \emph{or}  the resultant state ($q' \composeL l'$) is undefined.
%% \end{enumerate} 


%We provide a set of rules in \fig\ref{fig:shiftRules} that reduce action shifting to logical entailments. While these rules are not complete, we found them sufficient to reason about our examples. The $\approx^R$ notation in the conclusion of \proofRule{Expand/Contract}, denotes that the shifting is valid both ways (\textit{i.e.} $I_1 \approx^R I_2$ iff $I_1 \weakenI{R} I_2$ and $I_2 \weakenI{R} I_1$). We write $\exact{P}$ to denote that the assertion $P$ is \emph{exact}. That is, there exists $l$ such that for all $\lenv$ and $l'$ where $l', \lenv \slsat P$ then $l = l'$. 
%%
