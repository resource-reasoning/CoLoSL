\subsection*{Conclusion and future work}

We have introduced \colosl, a new program logic that allows a flexible
treatment of the shared state to enable truly compositional
verification of fine-grained concurrent programs. By rethinking the
semantic model underlying the reasoning about such programs, new
paradigms have emerged that improve over the state of the art as far
as compositionality is concerned. However, \colosl is still young, and
lacks many features of its various cousins.  There are many
interesting ideas present in the literature: e.g.\ abstract states
governed by state transition systems~\cite{caresl}; higher-order
reasoning~\cite{icap}; and abstract atomicity~\cite{tada}. All these
ideas require further investigation. Here, our aim was to simply
introduce subjective views as a fundamental new way of underpinning of
such reasoning.

In another direction, \colosl makes heavy use of a richer fragment
of separation logic than is typically considered (in particular by
automatic tools), in particular due to the inclusion of the $**$ and
$--o$ connectives. Improved ways of reasoning about these connectives
would greatly facilitate more complex proofs, be it in \colosl or
other program logics based on separation logic.
