\section{N-ary Mutual Exclusion Ring}
In this section we demonstrate how we can generalise the concepts introduced in \S\ref{chapter:intuition} to generate an n-ary mutual exclusion device to manage concurrent accesses to a shared resource such as a database. 
%
%
\begin{figure}
%
\hrule
\[
\begin{array}{@{} l @{}}
\begin{array}{@{} r l @{}}
	\dring{x}{i}{n} \eqdef & \shared{\exsts{v} \oast_{k = 0}^{i-1} \cell{x + k}{v} * \oast_{k = i}^{n-1} \cell{x + k}{v-1}}{\bigcup\limits_{j=0}^{n-1}I'(j)}\\
	
	\dnode{x}{i}{n} \eqdef& [\token{a}_{i}] 
	* \shared{
		\begin{array}{@{} l @{}}
			\exsts{v} \left(\cell{x+a}{v} * \cell{x + b}{v}\right)\\
			 \lor \left(\cell{x+a}{v+1} * \cell{x + b}{v} \right)
		\end{array}			
	}{I(i)}\\
	& \text{where } a = min((i\!-\!1)\% n, i) \text{ and }\ b = max((i\!-\!1)\% n, i)\\\\
	
%	\dturn{x}{i}{n} \eqdef& 
%	\begin{cases}
%%		[\token{a}_{i}] *
%		\shared{
%			\begin{array}{@{} l @{}}
%				\exsts{v} \cell{x}{v} * \cell{x + (n-1)}{v}
%			\end{array}		
%		}{I(i)}
%		&\text{if } i=0\\
%		
%%		[\token{a}_{i}] *
%		\shared{
%			\begin{array}{@{} l @{}}
%				\exsts{v} \cell{x+(i-1)}{v+1} * \cell{x + i}{v}
%			\end{array}		
%		}{I(i)}	
%		&\text{otherwise}	
%	\end{cases}

	\dturn{x}{i}{n} \eqdef& 
	\begin{cases}
		\exsts{v} \cell{x}{v} * \cell{x + (n-1)}{v}
		&\text{if } i=0\\
		
		\exsts{v} \cell{x+(i-1)}{v+1} * \cell{x + i}{v}
		&\text{otherwise}	
	\end{cases}
\end{array}\vspace{10pt}\\
%
%
\begin{array}{@{} l @{}}	
	I(i) \eqdef   I'(i) \cup I''(i) \vspace{10pt}\\
	
	I'(i) \!\eqdef\! 
	\begin{cases}
		\left\{
		\begin{array}{@{} l l @{}}
			[\token{a}_{i}]: 
				&\exsts{v}\cell{x + (i\!-\!1)}{v} * \cell{x + i}{v-1} \swap\\
				&\qquad\cell{x + (i-1)}{v} * \cell{x + i}{v}
		\end{array}
		\right\}
		& \text{if } i = 0\\
		
		\left\{
		\begin{array}{@{} l l @{}}
			[\token{a}_{0}]: 
				&\exsts{v}\cell{x}{v} * \cell{x + (n-1)}{v} \swap\\
				&\qquad\cell{x}{v+1} * \cell{x + (n-1)}{v}
		\end{array}
		\right\}
		& \text{otherwise}
	\end{cases}\\\\
	
	I''(i) \!\eqdef\! \\\;
	\begin{cases}
		\!\left\{
			\begin{array}{@{} l @{} l @{}}
				[\token{a}_{n-1}]\!: & \exsts{v}\! \cell{x}{v} * \cell{x + (n-2)}{v} * \cell{x + (n-1)}{v-1} \\
				& \qquad \swap\cell{x}{v} * \cell{x + (n-2)}{v} * \cell{x + (n-1)}{v}
			\end{array}
		\right\}
		& \text{if } i = 0  \vspace*{5pt}\\
		
		\!\left\{
			\begin{array}{@{} l @{}}
				[\token{a}_{0}]\!: \\
				\;\exsts{v}\! \cell{x}{v} * \cell{x + 1}{v} * \cell{x + (n-1)}{v-1} \\
				\; \qquad \swap\cell{x}{v+1} * \cell{x + 1}{v} * \cell{x + (n-1)}{v}
			\end{array}
		\right\}
		& \text{if } i = 1 \vspace*{5pt}\\
		
		\!\left\{
			\begin{array}{@{} l @{}}
				[\token{a}_{i-1}]\!: \\
				\;\exsts{v}\! \cell{x + (i-2)}{v+1} * \cell{x + (i-1)}{v} * \cell{x + i}{v} \\
				\;\swap\cell{x + (i-2)}{v+1} * \cell{x + (i-1)}{v+1} * \cell{x + i}{v}
			\end{array}
		\right\}
		& \text{otherwise}
	\end{cases}\\
\end{array}
\end{array}
\]
\hrule
%
\caption{The \colosl specification of an n-ary mutual exclusion ring.}
\label{fig:n-place-ring-spec}
\end{figure}
%
%

We generalise the ternary specification of Dijkstra's token ring described in \S\ref{chapter:intuition} to n places as presented in \fig~\ref{fig:n-place-ring-spec}. The $\dring{x}{i}{n}$ predicate describes the members of the ring as $n$ consecutive heap cells between address $x$ and address $x + (n-1)$, inclusively where the element at index $i$ is the next to be incremented (hereafter, the leading element or component). That is, all components upto (but not including) index $i$ have value $v$ while those between $i$ and $n-1$ (end of the ring) have value $v-1$ for some value $v$. The interference on the ring is the combined interference associated with each member of the ring ($I'(0) \cup \cdots \cup I'(n-1)$). 

While the $\dring{x}{i}{n}$ predicate provides a \emph{global} account of all members of the ring, we also provide a \emph{local} description of each of the components in the ring. The $i$th component of the $n$-ary ring at address $x$ is captured by the $\dnode{x}{i}{n}$ predicate. As before, given a node at index $i$ and its predecessor at index $(i-1)\%n$ (where $\%$ denotes the remainder operation), at any given time either their values are equal, or the node at the lower index (\,$min((i-1)\%n, i)$\,) is greater by one. The interference associated with the $i$th node consists of the updates allowed on the node itself ($I'(i)$) as well as the updates associated with its predecessor ($I''(i)$). The $\dturn{x}{i}{n}$ predicate states that the leading component of the $n$-ary ring at address $x$ resides at index $i$. That is, until the value of the member at index $i$ is incremented no other component may be updated and as such, the thread associated with this element bears the sole right to manipulate the ring.

As we will shortly demonstrate, each of the $n$ threads in the mutual exclusion ring maintains a local (subjective) view of the ring when interacting with it. That is, when a thread inspects the current state of the ring (in order to infer whether it is associated with the leading component) or when incrementing the value of its ring member, it does so locally by enquiring only the values of its own component and its immediate left neighbour. As such, the specification of these operations are also local and are concerned with the values of the node and its neighbour alone (see \fig~\ref{fig:DME-methods}). On the other hand, the $n$-threaded concurrent database is equipped with the global specification of the mutual exclusion ring as to reflect the overall state of the ring and convey the identity of the leading thread (the thread next in line to access the database).
%
\begin{figure}
\hrule
\[
\begin{array}{@{} l @{}}
	\color{blue}{
		\left\{
			\cell{\li x}{-}
		\right\}
	}	\\
	\li{x:= DME($n$)}\text{\{}\\
	\begin{array}{l}
	
		\color{blue}{
			\left\{
				\cell{\li x}{-}
			\right\}
		}	\\
		
		\li{x:= alloc(n)}\\
		
		\color{blue}{
			\left\{
				\exsts{x}
				\cell{\li x}{x} *
				\oast_{i =0}^{n-1} \left(\cell{x+i}{0}\right)
			\right\}
		}	\\
		
		\color{darkgreen}{\text{//Apply the \extendRule principle}}\\
		
		\color{blue}{
			\left\{
				\exsts{x}
				\cell{\li x}{x} *
				\oast_{i =0}^{n-1} \left([\token{a}_{i}]\right) * 
				\shared{\oast_{i =0}^{n-1} \left(\cell{x+i}{0}\right)}{\bigcup\limits_{i = 0}^{n-1} I'(i)}
			\right\}
		}	\\
		
		
		\color{darkgreen}{\text{//Apply the \copyRule principle } n \text{ times.}}\\
		
		\color{blue}{
			\left\{
			\begin{array}{@{} l @{}}
				\!\exsts{x}\! \cell{\li x}{x} \!*\!
				\oast_{i =0}^{n-1} \left([\token{a}_{i}]\right)  \\
				* \underbrace{
					\shared{\oast_{i =0}^{n-1} \left(\cell{x+i}{0}\right)}{\bigcup\limits_{i = 0}^{n-1} I'(i)}
					\!\!\!*
					\!\cdots
					*
					\shared{\oast_{i =0}^{n-1} \left(\cell{x+i}{0}\right)}{\bigcup\limits_{i = 0}^{n-1} I'(i)}
				}_{n+1 \text{ copies}}
			\end{array}
			\right\}
		}	\\
		
		\color{blue}{
			\left\{
			\begin{array}{@{} l @{}}
				\!\exsts{x}\! \cell{\li x}{x} \!*\!
				\oast_{i =0}^{n-1} \left([\token{a}_{i}]\right) 
				* \dring{x}{0}{n}\\
				
				* \underbrace{
					\shared{\oast_{i =0}^{n-1} \left(\cell{x+i}{0}\right)}{\bigcup\limits_{i = 0}^{n-1} I'(i)}
					\!\!\!*
					\!\cdots
					*
					\shared{\oast_{i =0}^{n-1} \left(\cell{x+i}{0}\right)}{\bigcup\limits_{i = 0}^{n-1} I'(i)}
				}_{n \text{ copies}}\\
				
				
			\end{array}
			\right\}
		}	\\
		
		\color{darkgreen}{\text{//Apply the \shiftRule principle } n \text{ times.}}\\
		
		\color{blue}{
			\left\{
			\begin{array}{@{} l @{}}
				\!\exsts{x}\! \cell{\li x}{x} \!*\!
				\oast_{i =0}^{n-1} \left([\token{a}_{i}]\right) 
				* \dring{x}{0}{n}\\
				
				* \underbrace{
					\shared{\oast_{i =0}^{n-1} \left(\cell{x+i}{0}\right)}{\bigcup\limits_{i = 0}^{n-1} I(i)}
					\!\!\!*
					\!\cdots
					*
					\shared{\oast_{i =0}^{n-1} \left(\cell{x+i}{0}\right)}{\bigcup\limits_{i = 0}^{n-1} I(i)}
				}_{n \text{ copies}}\\
				
				
			\end{array}
			\right\}
		}	\\
		
		\color{darkgreen}{\text{//Apply the \forgetRule principle } n \text{ times.}}\\
		
		\color{blue}{
			\left\{
			\begin{array}{@{} l @{}}
				\exsts{x} \cell{\li x}{x} *
				\oast_{i =0}^{n-1} \bigg([\token{a}_{i}] * 
				\shared{\cell{x+i}{0} * \cell{x+((i-1) \% n)}{0}}{\bigcup\limits_{i = 0}^{n-1} I(i)}\bigg)\\
				* \dring{x}{0}{n}
			\end{array}
			\right\}
		}	\\
		
		\color{darkgreen}{\text{//Apply the \shiftRule principle } n \text{ times.}}\\
		
		\color{blue}{
			\left\{
			\begin{array}{@{} l @{}}
				\exsts{x} \cell{\li x}{x} *
				\oast_{i =0}^{n-1} \left([\token{a}_{i}] * 
				\shared{\cell{x+i}{0} * \cell{x+((i-1) \% n)}{0}}{I(i)}\right)\\
				* \dring{x}{0}{n}
			\end{array}
			\right\}
		}	\\
		
		\color{blue}{
			\left\{
				\exsts{x} \cell{\li x}{x} *
				\oast_{i =0}^{n-1} \;\dnode{x}{i}{n}
				* \dring{x}{0}{n}
			\right\}
		}	\\
	\end{array}\\
	
	\text{\}}\\
	
	\color{blue}{
			\left\{
				\exsts{x} \cell{\li x}{x} *
				\oast_{i =0}^{n-1} \;\dnode{x}{i}{n}
				* \dring{x}{0}{n}
			\right\}
		}		
\end{array}
\]
\hrule
\caption{Instantiation of the $n$-ary mutual exclusion ring.}
\label{fig:DME}
\end{figure}
%
%

\fig~\ref{fig:DME} presents a possible implementation for instantiating an $n$-place mutual exclusion ring along with an outline of the reasoning steps taken. Upon allocation of $n$ consecutive cells at heap address $x$ as the members of the ring, the shared state is extended by the ring components and their associated interference producing a \emph{global} specification of the ring, similar to the initial ternary specification of \S\ref{chapter:intuition}. Using the \copyRule, \shiftRule and \forgetRule principles, we then obtain a \emph{local} specification for each of the ring components as captured by the final $*$-composed $\textsf{node}$ predicates. 

\fig~\ref{fig:DME-methods} presents two auxiliary methods for maintaining the $n$-ary ring at address $x$, namely \li{isTurn(x,i,n)} and \li{next(x,i)}, along with their \colosl proof sketches. The \li{isTurn(x,i,n)} method can be used to ascertain whether the element at index \li{i} is the leading component (i.e. $\dturn{x}{i}{n}$ holds); while \li{next(x,i)} is used to increment the value of the component at index \li{i}. 
%
%
\begin{figure}
%
\hrule\vspace{5pt}
\[
\begin{array}{@{} r l @{}}
	\db{d}{x}{n} \eqdef& 
	\shared{
		\begin{array}{@{} l @{}}
			\exsts{i} \dring{x}{i}{n} * \\
			\left((\database{d} * \dbtoken{d}{0}) \lor (\dbtoken{d}{1} * [\token{a}_{i}])	\right)
		\end{array}
	}{I(d)}\\\\
	
	\database{d} \eqdef & \cell{d}{-}\\
	I(d) \eqdef &
	\bigcup\limits_{i = 1}^{n}
	\left\{
	\begin{array}{r l}
		[\token{a}_i]: &
		\left\{
		\begin{array}{@{} l @{}}
			\dring{x}{i}{n} * \database{d} * \dbtoken{d}{0} \\
			\qquad \swap \dring{x}{i}{n} * \dbtoken{d}{1} * [\token{a}_{i}]
		\end{array}
		\right\}
%		&\cup
%		\left\{
%		\begin{array}{@{} l @{}}
%	 		\dring{x}{i}{n} * \dbtoken{d}{1} * [\token{a}_{i}]\swap\\
%			\qquad \dring{x}{i}{n} * \database{d} * \dbtoken{d}{0}
%		\end{array}
%		\right\}
	\end{array}
	\right\}\\
%
	&
	\;\bigcup
	\left\{
	\begin{array}{r l}
		true: &
		\left\{
		\begin{array}{@{} l @{}}
	 		\dring{x}{i}{n} * \dbtoken{d}{1} * [\token{a}_{i}]\swap\\
			\qquad \dring{x}{i}{n} * \database{d} * \dbtoken{d}{0}
		\end{array}
		\right\}
	\end{array}
	\right\}
\end{array}
\]
\hrule
%
\caption{The \colosl specification of a shared database accessible by $n$ threads synchronised through an $n$-ary mutual exclusion ring.}
\label{fig:database}
\end{figure}
%
%

We proceed with the specification of a shared database concurrently accessed by $n$ threads as presented in \fig~\ref{fig:database}. The $\db{d}{x}{n}$ predicate describes a database whose contents is captured by the $\database{d}$ predicate and concurrent access to it is synchronised by means of an $n$-ary mutual exclusion ring at address $x$ as described above. An auxiliary capability $\dbtoken{d}{-}$ tracks the status of the database: $\dbtoken{d}{0}$ denotes that the database is \emph{idle} (not being accessed by any threads) and thus its contents ($\database{d}$) are in the shared state, visible to all threads. Conversely, $\dbtoken{d}{1}$ indicates that the database is \emph{busy} (under manipulation by a thread) and thus its contents have been claimed by the leading thread and are missing from the shared state. Moreover, the thread manipulating the database has temporarily turned its update capability ($[\token{a}_i]$) to the shared state; this is used to track the identity of the leading thread.

The specification of the database itself ($\database{d}$) is orthogonal to the objective of this example and we thus represent it as a single heap cell at address $d$. In general, $\database{d}$ can be defined arbitrarily to capture the contents of a database at location $d$  so long as it is disjoint from the mutual exclusion ring. 

The interference associated with the database ($I(d)$), asserts that $i$th thread (in the $n$-ary mutual exclusion ring at address $x$) associated with the leading component ($\dring{x}{i}{n}$) may change the status of the database from idle to busy and claim the contents of the database in return for its update capability $[\token{a}_i]$. That is, the leading thread at index $i$ who possesses the sole right to manipulate the mutual exclusion ring, may remove the contents of the database from the shared state and update it locally. Conversely, once the leading thread has completed its operations on the database, it may return its contents to the shared state and reset its status from busy to idle while claiming its update capability. 

We finalise this section by providing an example implementation of an $n$-threaded concurrent database and its proof sketch as shown in \fig~\ref{fig:database-implementation}. At the beginning of the database program in \fig~\ref{fig:database-implementation} (\li{concurrent-database(d,n)}), the contents of the database is unshared and is held locally as captured by the $\database{d}$ predicate in the precondition. The program proceeds by first generating an $n$-ary mutual exclusion ring by calling the instantiation method (\li{DME(n)}). It then extends the shared state by the contents of the database and in doing so it makes the database a shared resource accessible by all $n$ threads in the mutual exclusion ring at $x$. The program then continues by executing in parallel the programs associated with each of the $n$ threads for concurrently manipulating the database (\li{process([d],[x],$i$,[n])} for $i \in \{0,\cdots,n-1\}$). 

The \li{i}th thread executing the (\li{process(d,x,i,n)}) program for manipulating the database, continuously \emph{spins} until it becomes the leading thread. It then claims the database locally and proceeds with its intended modifications. Once complete, it returns the database to the shared state and advances the mutual exclusion ring by calling the \li{next(x,i)} method. It then resumes spinning and the same process repeats indefinitely. 
%
%
\begin{figure}
\hrule
\[
\begin{array}{@{} l @{}}
	\color{blue}{
		\left\{
			\cell{\li x}{-} * \cell{\li d}{d} * \cell{\li n}{n} * \database{d}
		\right\}
	}	\\
	\li{concurrent-database(d, n)}\text{\{}\\
	\begin{array}{l}
	
		\color{blue}{
			\left\{
				\cell{\li x}{-} * \cell{\li d}{d} * \cell{\li n}{n} * \database{d}
			\right\}
		}	\\
			
		\li{x := DME(n)} \\
		
		\color{blue}{
			\left\{
			\begin{array}{@{} l @{}}
				\exsts{x} \cell{\li x}{x} * \cell{\li d}{d} * \cell{\li n}{n} * \database{d} *\\
				\oast_{i = 0}^{n-1}\left(\dnode{x}{i}{n}\right) * \dring{x}{0}{n}
			\end{array}
			\right\}
		}	\\
		
		\color{darkgreen}{\text{//Apply the \extendRule principle}}\\
		
		\color{blue}{
			\left\{
				\exsts{x} \cell{\li x}{x} * \cell{\li d}{d} * \cell{\li n}{n} * \db{d}{x}{n} * \oast_{i = 0}^{n-1}\left(\dnode{x}{i}{n}\right)
			\right\}
		}	\\
		
		\color{darkgreen}{\text{//Apply the \copyRule principle $n$ times}}\\
		
		\color{blue}{
			\left\{
				\exsts{x} \cell{\li x}{x} * \cell{\li d}{d} * \cell{\li n}{n} * \oast_{i = 0}^{n-1}\left(\dnode{x}{i}{n} * \db{d}{x}{n} \right)
			\right\}
		}	\\
			
		\begin{array}{@{\hspace{4pt}} l }
			\color{blue}{
				\left\{
					\left(\dnode{x}{0}{n} * \db{d}{x}{n} \right) * \cdots * \left(\dnode{x}{n-1}{n} * \db{d}{x}{n} \right)
				\right\}
			}	\\
		
			

			\li{process([d],[x],0,[n])} \;\mid\mid\; \cdots  \;\mid\mid\; \li{process([d],[x],[n]-1,[n])}\\
			
			\color{blue}{
				\left\{
					\left(\dnode{x}{0}{n} * \db{d}{x}{n} \right) * \cdots * \left(\dnode{x}{n-1}{n} * \db{d}{x}{n} \right)
				\right\}
			}	
			
			
		\end{array}\\
		
		\color{blue}{
			\left\{
				\exsts{x} \cell{\li x}{x} * \cell{\li d}{d} * \cell{\li n}{n} * \oast_{i = 0}^{n-1}\left(\dnode{x}{i}{n} * \db{d}{x}{n} \right)
			\right\}
		}	

	\end{array}\\
	
	\text{\}}\\
		
	\color{blue}{
		\left\{
			\exsts{x} \cell{\li x}{x} * \cell{\li d}{d} * \cell{\li n}{n} * \oast_{i = 0}^{n-1}\left(\dnode{x}{i}{n} * \db{d}{x}{n} \right)
		\right\}
	}	
\end{array}
\]
%
%
\hrule	
\caption{An $n$-threaded concurrent database where the implementation of the \li{process} method is given \fig~\ref{fig:database-process}.}
\label{fig:database-implementation}
\end{figure}
%
%
\begin{figure}
\[
\begin{array}{@{} l @{}}
	\color{blue}{
		\left\{
			\cell{\li x}{x} * \cell{\li d}{d} * \cell{\li i}{i} * \cell{\li n}{n} * \dnode{x}{i}{n} * \db{d}{x}{n} 
		\right\}
	}\\
	
	\li{process(d,x,i,n)}\text{\{}\\
	\begin{array}{@{\hspace{5pt}}l}
		
		\li{while(true)}\text{\{}\\
		
		\begin{array}{@{\hspace{5pt}}l}
			\color{blue}{
				\left\{
					\cell{\li x}{x} * \cell{\li d}{d} * \cell{\li i}{i} * \cell{\li n}{n} * \dnode{x}{i}{n} * \db{d}{x}{n} 
				\right\}
			}\\
			
			\li{do}\text\{\\
				\quad \li{b:= isTurn(x,i,n)}\\
			\text{\}} \li{while(!b)}\\
			
			\color{blue}{
				\left\{
					\cell{\li x}{x} * \cell{\li d}{d} * \cell{\li i}{i} * \cell{\li n}{n} * [\token{a}_i] * \shared{\dturn{x}{i}{n}}{I(i)} * \db{d}{x}{n} 
				\right\}
			}\\
		
			\color{blue}{
				\left\{
				\begin{array}{@{} l @{}}
					\cell{\li x}{x} * \cell{\li d}{d} * \cell{\li i}{i} * \cell{\li n}{n} * [\token{a}_i] * \shared{\dturn{x}{i}{n}}{I(i)} \\
					* \shared{\exsts{j} \dring{x}{j}{n} * 
							\left((\database{d} * \dbtoken{d}{0}) \lor (\dbtoken{d}{1} * [\token{a}_{j}])	\right)
						}{I(d)}
				\end{array}
				\right\}
			}\\
			
			\color{blue}{
				\left\{
				\begin{array}{@{} l @{}}
					\cell{\li x}{x} * \cell{\li d}{d} * \cell{\li i}{i} * \cell{\li n}{n} * [\token{a}_i] * \shared{\dturn{x}{i}{n}}{I(i)} \\
					* \shared{
							\dring{x}{i}{n} * \database{d} * \dbtoken{d}{0}
						}{I(d)}
				\end{array}
				\right\}
			}\\
			
			\color{darkgreen}{\text{//Thread has exclusive access to the critical section}}\\
			\color{darkgreen}{\text{//Claim the database locally by applying the action of }[\token{a}_{i}].}\\
			
			\color{blue}{
				\left\{
				\begin{array}{@{} l @{}}
					\cell{\li x}{x} * \cell{\li d}{d} * \cell{\li i}{i} * \cell{\li n}{n} * \shared{\dturn{x}{i}{n}}{I(i)} \\
					* \database{d} 				
					* \shared{
							\dring{x}{i}{n} * \dbtoken{d}{1} * [\token{a}_i] 
						}{I(d)}
				\end{array}
				\right\}
			}\\
			
	%		\quad
	%		\color{blue}{ \database{d} 		}\\
			
			\color{darkgreen}{\text{//Mutate the database as needed.}}\\
			
	%		\quad
	%		\color{blue}{ \database{d} 		}\\
			
			\color{blue}{
				\left\{
				\begin{array}{@{} l @{}}
					\cell{\li x}{x} * \cell{\li d}{d} * \cell{\li i}{i} * \cell{\li n}{n} * \shared{\dturn{x}{i}{n}}{I(i)} \\
					* \database{d} 				
					* \shared{
							\dring{x}{i}{n} * \dbtoken{d}{1} * [\token{a}_i] 
						}{I(d)}
				\end{array}
				\right\}
			}\\
			
			\color{darkgreen}{\text{//Return the database to the shared region.}}\\
			
			\color{blue}{
				\left\{
				\begin{array}{@{} l @{}}
					\cell{\li x}{x} * \cell{\li d}{d} * \cell{\li i}{i} * \cell{\li n}{n} * [\token{a}_i] * \shared{\dturn{x}{i}{n}}{I(i)}
					* \db{d}{x}{n}				
				\end{array}
				\right\}
			}\\
			
			\li{next(x,i)}\\
			
			\color{blue}{
				\left\{
				\begin{array}{@{} l @{}}
					\cell{\li x}{x} * \cell{\li d}{d} * \cell{\li i}{i} * \cell{\li n}{n} *  \dnode{x}{i}{n}
					* \db{d}{x}{n}				
				\end{array}
				\right\}
			}\\
			
		\end{array}\\

	\text{\}}

	\end{array}	\\

	\text{\}}\\
	
	\color{blue}{
		\left\{
		\begin{array}{@{} l @{}}
			\cell{\li x}{x} * \cell{\li d}{d} * \cell{\li i}{i} * \cell{\li n}{n} *  \dnode{x}{i}{n}
			* \db{d}{x}{n}				
		\end{array}
		\right\}
	}
\end{array}
\]
\hrule
\caption{The code for each of the $n $ threads of the concurrent database.}
\label{fig:database-process}
\end{figure}
%
%
%
%
\begin{figure}
\hrule
\[
\begin{array}{@{} l @{}}
	\color{blue}{
		\left\{
			\cell{\li x}{x} * \cell{\li i}{i} * \cell{\li n}{n} * \cell{\li b}{-} * \dnode{x}{i}{n}
		\right\}
	}	\\
	\li{b:= isTurn(x, i, n)}\text{\{}\\
	\begin{array}{l}
	
		\color{blue}{
			\left\{
			\begin{array}{@{} l @{}}
				\cell{\li x}{x} * \cell{\li i}{i} * \cell{\li n}{n} * \cell{\li b}{-} *  \dnode{x}{i}{n}
			\end{array}
			\right\}
		}	\\
		
		\color{blue}{
			\left\{
			\begin{array}{@{} l @{}}
				\cell{\li x}{x} * \cell{\li i}{i} * \cell{\li n}{n} * \cell{\li b}{-} \\
				*  (\dnode{x}{i}{n} \lor [\token{a}_i] * \shared{\dturn{x}{i}{n}}{I(i)})
			\end{array}
			\right\}
		}	\\
		
		\li{if (i == 0)}\text{\{}\\
			
		\begin{array}{@{\hspace{4pt}} l }
			\color{blue}{
				\left\{
				\begin{array}{@{} l @{}}
					\cell{\li x}{x} * \cell{\li i}{0} * \cell{\li n}{n} * \cell{\li b}{-} *\\
					(\dnode{x}{0}{n} \lor [\token{a}_0] * \shared{\dturn{x}{0}{n}}{I(i)})
				\end{array}
				\right\}
			}	\\
			
			\li{b := ([x] == [x+(n-1)])}\\
			
			\color{blue}{
				\left\{
				\begin{array}{@{} l @{}}
					\cell{\li x}{x} * \cell{\li i}{0} * \cell{\li n}{n} \\
					* ((\cell{\li b}{0} * \dnode{x}{0}{n}) \lor (\cell{\li b}{1} * [\token{a}_0] * \shared{\dturn{x}{0}{n}}{I(i)}))
				\end{array}
				\right\}
			}	\\
			
			\color{blue}{
				\left\{
				\begin{array}{@{} l @{}}
					\cell{\li x}{x} * \cell{\li i}{i} * \cell{\li n}{n} \\
					*  ((\cell{\li b}{0} * \dnode{x}{i}{n}) \lor (\cell{\li b}{1} *[\token{a}_i] * \shared{\dturn{x}{i}{n}}{I(i)}))
				\end{array}
				\right\}
			}	
		\end{array}\\
		
		\text{\}}\\
		
		\li{else} \text{\{}\\
		
		\begin{array}{@{\hspace{4pt}} l }
			\color{blue}{
				\left\{
					\cell{\li x}{x} * \cell{\li i}{i} * \cell{\li n}{n} *  (\dnode{x}{i}{n} \lor [\token{a}_i] * \shared{\dturn{x}{i}{n}}{I(i)})
				\right\}
			}	\\
			
			\li{b := ([x+i]} < \li{[x+(i-1)])}\\
			
			\color{blue}{
				\left\{
				\begin{array}{@{} l @{}}
					\cell{\li x}{x} * \cell{\li i}{i} * \cell{\li n}{n} \\
					* ((\cell{\li b}{0} * \dnode{x}{i}{n}) \lor (\cell{\li b}{1} * [\token{a}_i] * \shared{\dturn{x}{i}{n}}{I(i)}))
				\end{array}
				\right\}
			}	
		\end{array}\\
		
		\text{\}}\\
		
		\color{blue}{
			\left\{
			\begin{array}{@{} l @{}}
				\cell{\li x}{x} * \cell{\li i}{i}8 \cell{\li n}{n} \\
				*  ((\cell{\li b}{0} * \dnode{x}{i}{n}) \lor (\cell{\li b}{1} * [\token{a}_i] * \shared{\dturn{x}{i}{n}}{I(i)}))
			\end{array}
			\right\}
		}	
	\end{array}\\
	
	\text{\}}\\
	
	\color{blue}{
		\left\{
		\begin{array}{@{} l @{}}
			\cell{\li x}{x} * \cell{\li i}{i} * \cell{\li n}{n} \\
			*  ((\cell{\li b}{0} * \dnode{x}{i}{n}) \lor (\cell{\li b}{1} * [\token{a}_i] * \shared{\dturn{x}{i}{n}}{I(i)}))
		\end{array}
		\right\}
	}	\vspace{5pt}\\

\hline\vspace{-5pt}\\

	\color{blue}{
		\left\{
			\cell{\li x}{x} * \cell{\li i}{i} * [\token{a}_i] * \shared{\dturn{x}{i}{n}}{I(i)}
		\right\}
	}	\\
	\li{next(x, i)}\text{\{}\\
	\begin{array}{l}
	
		\color{blue}{
			\left\{
				\cell{\li x}{x} * \cell{\li i}{i} *  [\token{a}_i] * \shared{\dturn{x}{i}{n}}{I(i)}
			\right\}
		}	\\
		
		\atomic{\li{[x+i]+= 1}}\\
		
		\color{blue}{
			\left\{
				\cell{\li x}{x} * \cell{\li i}{i} *  \dnode{x}{i}{n}
			\right\}
		}	\\
		
	\end{array}\\
	
	\text{\}}\\
	
	\color{blue}{
		\left\{
			\cell{\li x}{x} * \cell{\li i}{i} *  \dnode{x}{i}{n}
		\right\}
	}	
\end{array}
\]
\hrule
\caption{Auxiliary operations of the $n$-ary mutual exclusion ring.}
\label{fig:DME-methods}
\end{figure}
%