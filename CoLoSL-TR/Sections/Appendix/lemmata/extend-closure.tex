\newpage	
%
\begin{lemma}[\extendRule-Containment]\label{lem:extend-containment}
Given any two action models $\lmod, \lmod' \in \AMods$, 
\[
\begin{array}{l l}
	\for{n \in \Nats} \for{l \in \LStates} \isContainedAM{\lmod}{n}{l}{\lmod \cup \lmod'}
\end{array}
\]
%
\begin{proof}
Pick an arbitrary $\lmod, \lmod' \in \AMods$. We then proceed by induction on $n$.\\

\noindent\textbf{Base case $n = 0$}\\
Pick an arbitrary $l \in \LStates$. From the definition of $\leq_0$ we then trivially have $\isContainedAM{\lmod}{0}{l}{\lmod \cup \lmod'}$ as required.\\

\noindent\textbf{Inductive case}\\
Pick an arbitrary $n \in \Nats^{+}$ such that
%
\begin{align}
	\for{l' \in \LStates} \isContainedAM{\lmod}{n-1}{l'}{\lmod \cup \lmod'} \label{ECon:IH} \tag{I.H.}
\end{align}
%
Pick an arbitrary $l \in \LStates$. We are then required to show 
%
\begin{align*}
	\for{\ca{}} \for{a \in \lmod(\ca{})} \m{reflected}(a, l, (\lmod \cup \lmod')(\ca{})) \land \isContainedAM{\lmod}{n-1}{a[l]}{\lmod \cup \lmod'} 
\end{align*}
%
Pick an arbitrary $\ca{}$ and $a \in \lmod(\ca{})$. From the definition of $\m{reflected}$ we trivially have:
%
\begin{align*}
	\m{reflected}(a, l, (\lmod \cup \lmod')(\ca{}))
\end{align*}
%
If $a[l]$ is undefined then we vacuously have $\isContainedAM{\lmod}{n-1}{a[l]}{\lmod \cup \lmod'}$. On the other hand, if $a[l]$ is defined then from (\ref{ECon:IH}) we have $\isContainedAM{\lmod}{n-1}{a[l]}{\lmod \cup \lmod'}$ as required.
\end{proof}
%

\end{lemma}
%
%%
%\begin{lemma}[\extendRule-Containment]\label{lem:extend-containment2}
%Given any two action models $\lmod, \lmod' \in \AMods$, 
%\[
%\begin{array}{l l}
%	\for{n \in \Nats} \for{l, r \in \LStates} \isContainedAM{\lmod}{n}{l}{\lmod'} \land l \compatL r \implies \isContainedAM{\lmod}{n}{l \composeL r}{\lmod'}
%\end{array}
%\]
%%
%\begin{proof}
%Pick an arbitrary $\lmod, \lmod' \in \AMods$. We then proceed by induction on $n$.\\
%
%\noindent\textbf{Base case $n = 0$}\\
%Pick an arbitrary $l, r \in \LStates$ such that $ \isContainedAM{\lmod}{0}{l}{\lmod'}$ and $l \compatL r$. From the definition of $\leq_0$ we then trivially have $\isContainedAM{\lmod}{0}{l \composeL r}{\lmod'}$ as required.\\
%
%\noindent\textbf{Inductive case}\\
%Pick an arbitrary$n \in \Nats^{+}$ and  $l, r \in \LStates$ such that
%%
%\begin{align}
%	l \compatL r \land \isContainedAM{\lmod}{n}{l}{\lmod'} \label{ECon2:Ass1}
%	\for{l', r' \in \LStates} \isContainedAM{\lmod}{n-1}{l'}{\lmod'} \land l' \compatL r' \implies \isContainedAM{\lmod}{n-1}{l' \composeL r'}{\lmod'} \label{ECon2:IH} \tag{I.H.}
%\end{align}
%%
%\textbf{RTS.}
%%
%\begin{align*}
%	\for{\ca{}} \for{a \in \lmod(\ca{})} \m{reflected}(a, l \composeL r, \lmod'(\ca{})) \land \isContainedAM{\lmod}{n-1}{a[l \composeL r]}{\lmod'} 
%\end{align*}
%%
%Pick an arbitrary $\ca{}$ and $a = (p, q, c) \in \lmod(\ca{})$. From the definition of $\m{reflected}$ we trivially have:
%%
%\begin{align*}
%	\m{reflected}(a, l, (\lmod \cup \lmod')(\ca{}))
%\end{align*}
%%
%If $a[l]$ is undefined then we vacuously have $\isContainedAM{\lmod}{n-1}{a[l]}{\lmod \cup \lmod'}$. On the other hand, if $a[l]$ is defined then from (\ref{ECon:IH}) we have $\isContainedAM{\lmod}{n-1}{a[l]}{\lmod \cup \lmod'}$ as required.
%\end{proof}
%%
%
%\end{lemma}
%%
%
\begin{lemma}[\extendRule-Closure-1]\label{lem:extend-closure}
%
For all $\lmod, \lmod_{e}, \lmod_{0} \in \AMods$ such that $\for{\ca{} \in \dom{\lmod_0}} \lmod_0(\ca{}) = \emptyset$; and for all $g, s_e \in \LStates$ and $\fence{}, \fence{e} \in \pset{\LStates}$,
\[
\begin{array}{l l}
	g \in \fence{} /| \fence{} \strictfences \lmod /| s_e \in \fence{e} /| \fence{e} \strictfences \lmod_e \cup \lmod_0
	\implies  \extendsAM{\lmod \cup \lmod_{e} \cup \lmod_0}{s_{e}}{g}{\lmod_{e}}
\end{array}
\]
%
%where
%%
%\[
%\begin{array}{@{} l  l @{}}
%	\left( (\lmod_1, \fence{1}) + (\lmod_{2}, \fence{2}) \right)(\ca{}) \eqdef  & 
%	\left\{
%		(f', a[f']) \;\;\middle|
%		\begin{array}{ l @{}}
%			\left(a \in \lmod_{1}(\ca{}) |/ a \in \lmod_{2}(\ca{})  \right) /| \\
%			f' \in \fence{1} \composeL \fence{2} /| \m{enabled}(a, f')	
%		\end{array}		  
%	\right\}\\
%\end{array}
%\]
%%
%and
%%
%\[
%\begin{array}{@{} l l @{}}
%	\fence{1} \composeL \fence{2} \eqdef & 
%	\left\{
%		f_{1} \composeL f_{2} \;\;\middle|\;\; 
%		f_{1} \in \fence{1} /| f_{2} \in \fence{2}
%%		\begin{array}{l @{\hspace{5pt}} l @{}}
%%			\exsts{\fence{}, \fence{e}} & g \in \fence{} /| f \in \fence{} /| \fence{} \strictfences \lmod /| \\
%%			& s_{e} \in \fence{e} /| f_{e} \in \fence{e} /| \fence{e} \strictfences \lmod_{e} 
%%		\end{array}
%	\right\}
%\end{array}
%\]
%%
\begin{proof} Pick an arbitrary $\lmod, \lmod_e, \lmod_0 \in \AMods$, $g, s_e \in \LStates$ and $\fence{}, \fence{e} \in \pset{\LStates}$ such that 
%
\begin{align}
	& \for{\ca{} \in \dom{\lmod_0}} \lmod_0(\ca{}) = \emptyset \label{EC:Ass0}\\
	& g \in \fence{} /|  s_e \in \fence{e}   \label{EC:Ass1}\\
	& \fence{} \strictfences \lmod /| \fence{e} \strictfences \lmod_e \cup \lmod_0 \label{EC:Ass2}
\end{align} 
%
From the definition of $\downarrow$, it then suffices to show
%
\begin{align}
	& \for {n \in \Nats} \isContainedAM{\lmod_e}{n}{s_e \composeL g}{\lmod \cup \lmod_e \cup \lmod_0} \label{EC:Goal1}\\
	& \for {n \in \Nats}  \extendsAMUpto{\lmod \cup \lmod_e}{n}{s_e}{g}{\lmod_e} \label{EC:Goal2}
\end{align}
%
\noindent\textbf{RTS. (\ref{EC:Goal1})} \\
This follows immediately from \lem~\ref{lem:extend-containment}.\\

\noindent\textbf{RTS. (\ref{EC:Goal2})} \\
Rather than proving (\ref{EC:Goal2}) directly, we first establish the following.
%
\begin{align}
	& \for {n \in \Nats} \for{g, s_e \in \LStates} \nonumber\\
	& \;\; g \in \fence{} /| s_e \in \fence{e} \implies \extendsAMUpto{\lmod \cup \lmod_e \cup \lmod_0}{n}{s_e}{g}{\lmod_e} \label{EC:Goal3}
\end{align}
%
We can then despatch (\ref{EC:Goal2}) from (\ref{EC:Ass1}) and (\ref{EC:Goal3}); since for an arbitrary $n \in \Nats$, from (\ref{EC:Ass1}) and (\ref{EC:Goal3}) we have $\extendsAMUpto{\lmod \cup \lmod_e \cup \lmod_0}{n}{s_e}{g}{\lmod_e}$ as required. \\

\noindent\textbf{RTS. (\ref{EC:Goal3})} \\
We proceed by induction on the number of steps $n$.\\

%\noindent Pick an arbitrary $s_1, s_2, r \in \LStates, \lmod, \lmod', \gmod \in \AMods$.\\
\noindent\textbf{Base case }$n=0$\\
Pick an arbitrary $g, s_e \in \LStates$. We are then required to show	$\extendsAMUpto{\lmod \cup \lmod_e \cup \lmod_0}{0}{s_e}{g}{\lmod_e}$ which follows trivially from the definition of $\downarrow_0$.\\


\noindent\textbf{Inductive case }\\
Pick an arbitrary $n \in \Nats$ and $g, s_e \in \LStates$ such that
%
\begin{align}
	& g \in  \fence{} \label{LEC:Ass1}\\
	& s_e \in \fence{e} \label{LEC:Ass2}\\
%	& \gmod = (\lmod, g) + (\lmod_{e}, s_e) \label{LEC:Ass3}\\
	& \tag{I.H} \for{g', s_e'}  g' \in \fence{} /| s'_e \in \fence{e} \implies \extendsAMUpto{\lmod \cup \lmod_{e} \cup \lmod_0}{(n-1)}{s_e'}{g'}{\lmod_{e}} \label{LEC:IH}
\end{align}
%
\textbf{RTS.}
%
\begin{align}
%	& 
%	\V{\ca{}}  \V{a\in \lmod_e(\ca{})} \nonumber \\
%  &\quad (\m{potential}(a, s_e \composeL g)  => \nonumber\\
%  & \quad\qquad\for{(s', r') \in a[s_e, g]} \extendsAMUpto{\lmod \cup \lmod_e \cup \lmod_0, \gmod}{(n-1)}{s'}{r'}{\lmod_e}) \label{LEC:Goal1}\\
%%  
%  &\quad \m{enabled}(a,s_e \composeL g)
%  => (s_e \composeL g, a[s_e \composeL g])\in \gmod(\ca{})) \label{LEC:Goal2}\\
%  
  &\V{\ca{}}\V{a\in \left(\lmod \cup \lmod_e \cup \lmod_0 \right) (\ca{})}
  \m{potential}(a,s_e \composeL g) =>\null \nonumber \\
%  &\ \m{reflected}(a, s_e \composeL g,\lmod_e(\ca{})) |/\null \nonumber \\
%%  
%  &\ \neg\m{visible}(a, s_e) /| \for{(s', r') \in a[s_e, g]} \extendsAMUpto{\lmod \cup \lmod_e \cup \lmod_0, \gmod}{(n-1)}{s'}{r'}{\lmod_e}  
	& \quad
	\begin{array}{@{} l @{}}
		\left(\m{reflected}(a,s_e \composeL g, \lmod_e(\ca{})) |/ \neg\m{visible}(a, s_e) \ \right) /| \\
		\for{(s', r') \in a[s_e, g]} \extendsAMUpto{\lmod \cup \lmod_e \cup \lmod_0}{(n-1)}{s'}{r'}{\lmod_e}
 	\end{array} 
  \label{LEC:Goal3}
\end{align}
%
%

%\noindent\textbf{RTS. \ref{LEC:Goal1}}\\

%Since from (\ref{LEC:Ass5}) and the definition of $\m{potential}$ we have $s_e \composeL g \meetL \fst{a} \not= \emptyset$ and consequently, $s_e \meetL \fst{a} \not= \emptyset$, from (\ref{LEC:Ass2}) we have:
%%
%\begin{align}
%	\fst{\updateFP{a}} \leq s_e /| \fst{\updateFP{a}} \disjoint g\label{LEC:Ass7}
%\end{align}
%% 
%From (\ref{LEC:Ass5}), (\ref{LEC:Ass7}) and the definition of $\m{potential}$ we have:
%%
%\begin{align}
%	\m{potential}(a, s_e) \label{LEC:Ass8}
%\end{align}
%%
%On the other hand, from (\ref{LEC:Ass6}), (\ref{LEC:Ass7}) and the definitions of $a[s_e, g]$ and $\disjoint$, we have: 
%%
%\begin{align}
%	& r' = g \label{LEC:Ass9}\\
%	& a[s_e] = s' \label{LEC:Ass10}
%\end{align}
%%
%Consequently, from (\ref{EC:Ass2}), (\ref{LEC:Ass2}), (\ref{LEC:Ass8}), (\ref{LEC:Ass10}) and the definition of $\strictfences$ we have:
%%
%\begin{align}
%	s' \in  \fence{e}  \label{LEC:Ass11}
%\end{align}
%%
%Finally, from (\ref{LEC:Ass1}), (\ref{LEC:Ass11}) and (\ref{LEC:IH}) we have:
%%
%\begin{align*}
%	\extendsAMUpto{\lmod \cup \lmod_{e} \cup \lmod_0, \gmod}{(n-1)}{s'}{g}{\lmod_e}
%\end{align*}
%%
%and consequently from (\ref{LEC:Ass9})
%%
%\begin{align*}
%	\extendsAMUpto{\lmod \cup \lmod_{e} \cup \lmod_0, \gmod}{(n-1)}{s'}{r'}{\lmod_e}
%\end{align*}
%%
%as required.\\
%%
%%
%%
%%
%%
%%
%%
%%

\noindent\textbf{RTS. \ref{LEC:Goal3}}\\
Pick an arbitrary $\ca{}$, $a = (p, q, c) \in \left(\lmod \cup \lmod_e \cup \lmod_0 \right)(\ca{})$ and $(s', r')$ such that
%
\begin{align}
	& \m{potential}(a, s_e \composeL g)  \label{LEC:ass1}\\
	& (s', r') \in a[s_e, g] \label{LEC:ass2}
\end{align}
%(\ref{LEC:Ass})
Since from (\ref{EC:Ass0}) we know $\lmod_0(\ca{}) = \emptyset$ we know $a \in \lmod_e(\ca{}) |/  a \in\lmod(\ca{})$, and thus there are two cases to consider. \\

%If the first disjunct is the case, the desired result holds trivially since from the definition of $\m{reflected}$ and (\ref{LEC:Ass25}) we have $\m{reflected}(a, s_e \composeL g, \lmod_e(\ca{}))$ as required.
\noindent\textbf{Case 1. } $a \in \lmod(\ca{})$\\ 
From (\ref{LEC:ass1}) and the definition of $\m{potential}$ we have $s_e \composeL g \meetL p \composeL c \not= \emptyset$ and consequently, $g \meetL p \composeL c \not= \emptyset$, from (\ref{LEC:Ass1}) we then have:
%
\begin{align}
	p \leq g /| p \disjoint s_e \label{LEC:ass3}
\end{align}
% 
From (\ref{LEC:ass1}), (\ref{LEC:ass3}) and the definitions of $\m{potential}$ we have:
%
\begin{align}
	\m{potential}(a, g) \label{LEC:ass4}
\end{align}
%
From (\ref{LEC:ass3}) and the definition of $\m{visible}$ we have:
%
\begin{align}
	\neg\m{visible}(a, s_e) 
	\label{LEC:ass5}
\end{align}
%
On the other hand, from (\ref{LEC:ass2}), (\ref{LEC:ass3}) and the definitions of $a[s_e, g]$, $a[g]$ and $\disjoint$, we know: 
%
\begin{align}
	& s' = s_e  \label{LEC:ass5}\\
	& a[g] = r'  \label{LEC:ass6}
\end{align}
%
Consequently, from (\ref{EC:Ass2}), (\ref{LEC:Ass1}), (\ref{LEC:ass4}),  (\ref{LEC:ass6}) and the definition of $\strictfences$ we have:
%
\begin{align}
	r' \in \fence{}  \label{LEC:ass7}
\end{align}
%
Finally, from (\ref{LEC:Ass2}), (\ref{LEC:ass7}), (\ref{LEC:IH}) we have:
%
\begin{align}
	\extendsAMUpto{\lmod \cup \lmod_{e} \cup \lmod_0}{(n-1)}{s_e}{r'}{\lmod_e}
	\label{LEC:ass8}
\end{align}
%
and consequently from (\ref{LEC:ass5}) and (\ref{LEC:ass2})-(\ref{LEC:ass8}) we have
%
\begin{align*}
	& \neg\m{visible}(a, s_e) /| \\
	& \for{(s', r') \in a[s_e, g]} \extendsAMUpto{\lmod \cup \lmod_{e} \cup \lmod_0}{(n-1)}{s_e}{r'}{\lmod_e}
\end{align*}
%
as required.\\
%
%
%
%

\noindent\textbf{Case 2. }$a \in \lmod_e(\ca{})$\\
From the assumption of the case and the definition of $\m{reflected}$ we trivially have
%
\begin{align}
	\m{reflected}(a, s_e \composeL g, \lmod_e(\ca{})
	\label{LEC:ass10}
\end{align}
%
Since from (\ref{LEC:ass1}) and the definition of $\m{potential}$ we have $s_e \composeL g \meetL p \composeL c \not= \emptyset$ and consequently, $s_e \meetL p \composeL c \not= \emptyset$, from (\ref{LEC:Ass2}) we have:
%
\begin{align}
	p \leq s_e /| p \disjoint g\label{LEC:ass11}
\end{align}
% 
From (\ref{LEC:ass1}), (\ref{LEC:ass11}) and the definition of $\m{potential}$ we have:
%
\begin{align}
	\m{potential}(a, s_e) \label{LEC:ass12}
\end{align}
%
On the other hand, from (\ref{LEC:ass2}), (\ref{LEC:ass11}) and the definitions of $a[s_e, g]$ and $\disjoint$, we have: 
%
\begin{align}
	& r' = g \label{LEC:ass13}\\
	& a[s_e] = s' \label{LEC:ass14}
\end{align}
%
Consequently, from (\ref{EC:Ass2}), (\ref{LEC:Ass2}), (\ref{LEC:ass12}), (\ref{LEC:ass14}) and the definition of $\strictfences$ we have:
%
\begin{align}
	s' \in  \fence{e}  \label{LEC:ass15}
\end{align}
%
Finally, from (\ref{LEC:Ass1}), (\ref{LEC:ass15}) and (\ref{LEC:IH}) we have:
%
\begin{align*}
	\extendsAMUpto{\lmod \cup \lmod_{e} \cup \lmod_0}{(n-1)}{s'}{g}{\lmod_e}
\end{align*}
%
and consequently from (\ref{LEC:ass13})
%
\begin{align}
	\extendsAMUpto{\lmod \cup \lmod_{e} \cup \lmod_0}{(n-1)}{s'}{r'}{\lmod_e}
	\label{LEC:ass16}
\end{align}
%
Thus from (\ref{LEC:ass2}) and (\ref{LEC:ass10})-(\ref{LEC:ass16}) we have:
%
\begin{align*}
	& \m{reflected}(a, s_e \composeL g, \lmod_e(\ca{})) /| \\
	& \for{(s', r') \in a[s_e, g]} \extendsAMUpto{\lmod \cup \lmod_e \cup \lmod_0}{(n-1)}{s'}{r'}{\lmod_e}
\end{align*}
%
as required.\\

\end{proof}
\end{lemma}
%
%
%
%
%
%
%
%
%
%
%

\begin{lemma}[\extendRule-closure-2]\label{lem:extend-closure-2}
For all $\lmod_0, \lmod{}, \lmod_{e} \in \AMods$, $s, g, s_e \in \LStates$ and $\fence{}, \fence{e} \in \pset{\LStates}$
%
\[
\begin{array}{l}
	g \in \fence{} /| \fence{} \strictfences \lmod /|
	s_e \in \fence{e} /| \fence{e} \strictfences \lmod_{e} /|
	\extendsAM{\lmod}{s}{g-s}{\lmod_0}\\
	\qquad\implies
	\extendsAM{\lmod \cup \lmod_{e}}{s}{(g-s) \composeL s_e}{\lmod_{0}}
\end{array}
\]
%
\begin{proof} Pick an arbitrary $\lmod_0, \lmod, \lmod_e \in \AMods$, $s, g, s_e \in \LStates$ and $\fence{}, \fence{e} \in \pset{\LStates}$ such that 
%
\begin{align}
	& g \in \fence{} /| s_e \in \fence{e}  \label{EC2:Ass1}\\
	& \fence{} \strictfences \lmod /| \fence{e} \strictfences \lmod{e} \label{EC2:Ass2}\\
	& \extendsAM{\lmod}{s}{g-s}{\lmod_0} \label{EC2:Ass3}
\end{align} 
%
From the definition of $\downarrow$, it then suffices to show
%
\begin{align}
	& \for{n \in \Nats} \isContainedAM{\lmod_0}{n}{g \composeL s_e}{\lmod \cup \lmod_2} \label{EC2:Goal1}\\
	& \for{n \in \Nats} \extendsAMUpto{\lmod \cup \lmod_e}{n}{s}{(g-s) \composeL s_e}{\lmod_0} \label{EC2:Goal2}
\end{align}
%
\noindent\textbf{RTS. (\ref{EC2:Goal1})} \\
%Since from (\ref{EC2:Ass3}) and the definition of $\downarrow$ we have $\lmod_0 \subseteq \lmod$, we can consequently derive $\lmod_0 \subseteq \lmod \cup \lmod_e$ as required.\\
Rather than proving (\ref{EC2:Goal1}) directly, we first establish the following.
%
\begin{align}
	& \for {n \in \Nats} \for{g, s_e \in \LStates} \nonumber\\
	& \quad g \in \fence{} /|  s_e \in \fence{e} /| \isContainedAM{\lmod_0}{n}{g}{\lmod} \implies \nonumber\\
	& \qquad \isContainedAM{\lmod_0}{n}{g \composeL s_e}{\lmod \cup \lmod_e} \label{EC2G1:Goal3}
\end{align}
%
We can then despatch (\ref{EC2:Goal1}) from (\ref{EC2:Ass1}), (\ref{EC2:Ass3}) and (\ref{EC2G1:Goal3}); since for an arbitrary $n \in \Nats$, from (\ref{EC2:Ass3}) and the definition of $\leq_n$ we have $\isContainedAM{\lmod_0}{n}{g}{\lmod}$ and consequently from (\ref{EC2:Ass1}) and (\ref{EC2G1:Goal3}) we derive $\isContainedAM{\lmod_0}{n}{g \composeL s_e}{\lmod \cup \lmod_e} $ as required. \\

\noindent\textbf{RTS. (\ref{EC2G1:Goal3})} \\
We proceed by induction on the number of steps $n$.\\

%\noindent Pick an arbitrary $s_1, s_2, r \in \LStates, \lmod, \lmod', \gmod \in \AMods$.\\
\noindent\textbf{Base case }$n=0$\\
Pick an arbitrary $g, s_e \in \LStates$. We are then required to show	$\isContainedAM{\lmod_0}{0}{g \composeL s_e}{\lmod \cup \lmod_e}$ which follows trivially from the definition of $\leq_0$.\\


\noindent\textbf{Inductive case }\\
Pick an arbitrary $n \in \Nats$ and $g, s_e \in \LStates$ such that
%
\begin{align}
	& g \in \fence{} \label{LEC2G1:Ass1}\\
	& s_e \in \fence{e} \label{LEC2G1:Ass2}\\
	& \isContainedAM{\lmod_0}{n}{g}{\lmod} \label{LEC2G1:Ass4}\\
	& \for{g'', s_e''}  g'' \in \fence{} /| s''_e \in \fence{e} /| \isContainedAM{\lmod_0}{(n-1)}{g''}{\lmod} \nonumber \\
	& \tag{I.H} \qquad \implies \isContainedAM{\lmod_0}{(n-1)}{g''\composeL s_e''}{\lmod \cup \lmod_{e}} \label{LEC2G1:IH}
\end{align}
%
\textbf{RTS.}
%
\begin{align*}  
	& \quad
  \begin{array}{@{} l @{}}
  	\for{\ca{}}\for{a \in \lmod_0(\ca{})}
		\m{reflected}(a,g \composeL s_e, \lmod \cup \lmod_e(\ca{})) \land
		\isContainedAM{\lmod_0}{(n-1)}{a[g \composeL s_e]}{\lmod \cup \lmod_e}
 	\end{array} 
\end{align*}
%
Pick an arbitrary $\ca{}$, $a$ and $p, q, c$ such that
%
\begin{align}
	& a = (p, q, c) \in \lmod_0(\ca{}) \label{LEC2G1:Ass5}
\end{align}
%(\ref{LEC2G1:Ass})
Pick an arbitrary $l$ such that
%
\begin{align}
	& p \composeL c \leq g \composeL s_e \composeL l \label{LEC2G1:Ass6}
\end{align}
%(\ref{LEC2G1:Ass})
Then from (\ref{LEC2G1:Ass4}),(\ref{LEC2G1:Ass6}) we know there exists $a''$ and $c''$ such that
%
\begin{align}
	a'' = (p, q, c'') \in \lmod(\ca{}) \land p \composeL c'' \leq g \composeL s_e \composeL l \label{LEC2G1:Ass7}
\end{align}
%
and thus from (\ref{LEC2G1:Ass6}) and by definitions of $\lmod \cup \lmod_e$ and $\m{reflected}$ we have:
%
\begin{align*}
	\m{reflected}(a, g \composeL s_e, \lmod \cup \lmod_e(\ca{}))
\end{align*}
%
as required. \\
%
On the other hand, from (\ref{LEC2G1:Ass7}) and the definition of action application we have:
%
\begin{align}
	a''[g \composeL s_e] = a[g \composeL s_e] \land a''[g] = a[g] 
	\label{LEC2G1:Ass8}
\end{align}
%
From (\ref{LEC2G1:Ass1}), (\ref{LEC2G1:Ass7}) and (\ref{EC2:Ass2}) we know
%
\begin{align}
	a''[g] \in \fence{} \land p \leq g \label{LEC2G1:Ass9}
\end{align}
%
and consequently from the the definition of action application we have:
%
\begin{align}
	a''[g \composeL s_e] = a''[g] \composeL s_e \label{LEC2G1:Ass10}
\end{align}
%
From (\ref{LEC2G1:Ass4}) and (\ref{LEC2G1:Ass5}) and (\ref{LEC2G1:Ass8}) we have:
%
\begin{align}
	\isContainedAM{\lmod_0}{(n-1)}{a''[g]}{\lmod} \label{LEC2G1:Ass11}
\end{align}
%
Finally from (\ref{LEC2G1:Ass9}), (\ref{LEC2G1:Ass2}), (\ref{LEC2G1:Ass11}) and (\ref{LEC2G1:IH}) we have:
%
\begin{align*}
	\isContainedAM{\lmod_0}{(n-1)}{a''[g] \composeL s_e}{\lmod \cup \lmod_e}
\end{align*}
%
and consequently from (\ref{LEC2G1:Ass10}) and (\ref{LEC2G1:Ass8}) 
%
\begin{align*}
	\isContainedAM{\lmod_0}{(n-1)}{a[g \composeL s_e]}{\lmod \cup \lmod_e}
\end{align*}
%
as required.\\



\noindent\textbf{RTS. (\ref{EC2:Goal2})} \\
Rather than proving (\ref{EC2:Goal2}) directly, we first establish the following.
%
\begin{align}
	& \for {n \in \Nats} \for{s, g, s_e \in \LStates} \nonumber\\
	& \quad g \in \fence{} /|  s_e \in \fence{e} /| \extendsAMUpto{\lmod}{n}{s}{g-s}{\lmod_0} \implies \nonumber\\
	& \qquad \extendsAMUpto{\lmod \cup \lmod_e}{n}{s}{(g-s) \composeL s_e}{\lmod_0} \label{EC2:Goal3}
\end{align}
%
We can then despatch (\ref{EC2:Goal2}) from (\ref{EC2:Ass1})-(\ref{EC2:Ass3}) and (\ref{EC2:Goal3}); since for an arbitrary $n \in \Nats$, from (\ref{EC2:Ass3}) and the definition of $\downarrow$ we have $\extendsAMUpto{\lmod}{n}{s}{g-s}{\lmod_0}$ and consequently from (\ref{EC2:Ass1}) and (\ref{EC2:Goal3}) we derive $\extendsAMUpto{\lmod \cup \lmod_e}{n}{s}{(g-s) \composeL s_e}{\lmod_0} $ as required. \\

\noindent\textbf{RTS. (\ref{EC2:Goal3})} \\
We proceed by induction on the number of steps $n$.\\

%\noindent Pick an arbitrary $s_1, s_2, r \in \LStates, \lmod, \lmod', \gmod \in \AMods$.\\
\noindent\textbf{Base case }$n=0$\\
Pick an arbitrary $s, g, s_e \in \LStates$. We are then required to show	$\extendsAMUpto{\lmod \cup \lmod_e}{0}{s}{(g-s) \composeL s_e}{\lmod_0}$ which follows trivially from the definition of $\downarrow_0$.\\


\noindent\textbf{Inductive case }\\
Pick an arbitrary $n \in \Nats$ and $s, r, g, s_e \in \LStates$ such that
%
\begin{align}
	& g \in \fence{} \label{LEC2:Ass1}\\
	& s_e \in \fence{e} \label{LEC2:Ass2}\\
%	& \gmod' = (\lmod, g) + (\lmod_{e}, s_e) 
	& g = s \composeL r \label{LEC2:Ass3} \\
	& \extendsAMUpto{\lmod}{n}{s}{r}{\lmod_0} \label{LEC2:Ass4}\\
	& \for{s'', g'', s_e''}  g'' \in \fence{} /| s''_e \in \fence{e} /| \extendsAMUpto{\lmod}{(n-1)}{s''}{g''-s''}{\lmod_0} \nonumber \\
	& \tag{I.H} \qquad \implies \extendsAMUpto{\lmod \cup \lmod_{e}}{(n-1)}{s''}{(g''-s'') \composeL s_e''}{\lmod_{0}} \label{LEC2:IH}
\end{align}
%
\textbf{RTS.}
%
\begin{align*}
%	& 
%	\V{\ca{}}  \V{a\in \lmod_0(\ca{})} \nonumber \\
%  &\quad (\m{potential}(a, g \composeL s_e) => \nonumber\\
%  & \quad\qquad\for{(s', r') \in a[s, r \composeL s_e]} \extendsAMUpto{\lmod \cup \lmod_e, \gmod'}{(n-1)}{s'}{r'}{\lmod_0}) \label{LEC2:Goal1}\\
%%   
%  &\quad (\m{enabled}(a,g \composeL s_e)
%  => (g \composeL s_e, a[g \composeL s_e])\in \gmod'(\ca{})) \label{LEC2:Goal2}\\
%  
%  &\V{\ca{}}\V{a\in \left(\lmod \cup \lmod_e \right) (\ca{})}
%  \m{potential}(a,g \composeL s_e) =>\null \nonumber \\
%  &\ \m{reflected}(a, g \composeL s_e, \lmod_0(\ca{})) |/\null \nonumber \\
%%  
%  &\ \neg\m{visible}(a, s) /| \for{(s', r') \in a[s, r \composeL s_e]} \extendsAMUpto{\lmod \cup \lmod_e, \gmod'}{(n-1)}{s'}{r'}{\lmod_0} 
	& \quad
  \begin{array}{@{} l @{}}
		\left(\m{reflected}(a,g \composeL s_e, \lmod_0(\ca{})) |/ \neg\m{visible}(a, s) \ \right) /| \\
		\for{(s', r') \in a[s, r \composeL s_e]} \extendsAMUpto{\lmod \cup \lmod_e}{(n-1)}{s'}{r'}{\lmod_0}
 	\end{array} 
\end{align*}
%
%
Pick an arbitrary $\ca{}$, $a = (p, q, c) \in \left(\lmod \cup \lmod_e \right)(\ca{})$ and $(s', r')$ such that
%
\begin{align}
	& \m{potential}(a, g \composeL s_e) \label{LEC2:Ass5}\\
	& (s', r') \in a[s, r \composeL s_e] \label{LEC2:Ass6}
\end{align}
%(\ref{LEC2:Ass})
Since either $a \in \lmod(\ca{})$ or $a \in \lmod_e(\ca{})$, there are two cases to consider:\\

\noindent\textbf{Case 1. }$a \in \lmod(\ca{})$\\
Since from (\ref{LEC2:Ass5}) and the definition of $\m{potential}$ we have $s_e \composeL g \meetL p \composeL c \not= \emptyset$ and consequently, $g \meetL p \composeL c \not= \emptyset$, from (\ref{LEC2:Ass1}) we have:
%
\begin{align}
	p \leq g /| p \disjoint s_e \label{LEC2:Ass7}
\end{align}
% 
From (\ref{LEC2:Ass5}), (\ref{LEC2:Ass7}) and the definition of $\m{potential}$ we have:
%
\begin{align}
	\m{potential}(a, g) \label{LEC2:Ass8}
\end{align}
%
On the other hand, from (\ref{LEC2:Ass6}), (\ref{LEC2:Ass7}) and the definitions of $a[s, r \composeL s_e]$ and $\disjoint$, we know there exists $r''$: 
%
\begin{align}
	& r' = r'' \composeL s_e \label{LEC2:Ass9}\\
	& (s', r'') \in a[s, r]  \label{LEC2:Ass10}
\end{align}
%
From (\ref{LEC2:Ass10}) and the definitions of $a[s, r]$ and $a[s \composeL r]$, we know $s' \composeL r'' = a[s \composeL r]$. Consequently, from (\ref{EC2:Ass2}), (\ref{LEC2:Ass1}), (\ref{LEC2:Ass8}), the definition of $\strictfences$ and since $g = s \composeL r$ (\ref{LEC2:Ass3}), we have:
%
\begin{align}
	s' \composeL r'' \in \fence{}  \label{LEC2:Ass11}
\end{align}
%(\ref{LEC2:Ass})
On the other hand, from (\ref{LEC2:Ass4}), (\ref{LEC2:Ass8}), (\ref{LEC2:Ass5}) and (\ref{LEC2:Ass10}) we have:
%
\begin{align}
	& \left(\m{reflected}(a, s \composeL r, \lmod_0(\ca{}) ) |/ \neg\m{visible}(a, s) \right) /| \label{LEC2:Ass12}\\
	& \extendsAMUpto{\lmod}{(n-1)}{s'}{r''}{\lmod_0} \label{LEC2:Ass13}
\end{align}
%
From (\ref{LEC2:Ass2}), (\ref{LEC2:Ass11}), (\ref{LEC2:Ass12}) and (\ref{LEC2:IH}) we have:
%
\begin{align*}
	\extendsAMUpto{\lmod \cup \lmod_{e}}{(n-1)}{s'}{r'' \composeL s_e}{\lmod_0}
\end{align*}
%
and thus from (\ref{LEC2:Ass9})
%
\begin{align}
	\extendsAMUpto{\lmod \cup \lmod_{e}}{(n-1)}{s'}{r'}{\lmod_0}
	\label{LEC2:Ass14}
\end{align}
%
Consequently, from (\ref{LEC2:Ass6})-(\ref{LEC2:Ass14}) we have:
%
\begin{align}
	\for{(s', r') \in a[s, r \composeL s_e]} \extendsAMUpto{\lmod \cup \lmod_e}{(n-1)}{s'}{r'}{\lmod_0}
	\label{LEC2:Ass15}
\end{align}
%
From (\ref{LEC2:Ass12}) there are two cases to consider:\\
\textbf{Case 1.1. }$\neg\m{visible}(a, s)$\\
From (\ref{LEC2:Ass15}) and the assumption of the case we have:
%(\ref{LEC2:Ass})
\begin{align*}
	& \neg\m{visible}(a, s) /| \\
	& \for{(s', r') \in a[s, r \composeL s_e]} \extendsAMUpto{\lmod \cup \lmod_e}{(n-1)}{s'}{r'}{\lmod_0}
\end{align*}
% 
as required. \\
%
%
%

\noindent\textbf{Case 1.2. }$\m{reflected}(a, s \composeL r, \lmod_0(\ca{})$\\
Pick an arbitrary $l \in \LStates$ such that 
%(\ref{LEC2:Ass})
\begin{align}
	p \composeL c \leq g \composeL s_e \composeL l \label{LEC2:Ass16}
\end{align} 
%
From the assumption of the case, (\ref{LEC2:Ass3}) and the definition of $\m{reflected}$ we then have
%
\begin{align}
	\exsts{a', c'} a' = (p, q, c') \land a' \in \lmod_0(\ca{}) /| p \composeL c' \leq g \composeL s_e \composeL l \label{LEC2:Ass17}
\end{align}
%
Thus from (\ref{LEC2:Ass16}), (\ref{LEC2:Ass17}) and the definition of $\m{reflected}$ we have:
%
\begin{align}
	\m{reflected}(a, g \composeL s_e, \lmod_0(\ca{}))
	\label{LEC2:Ass18}
\end{align}
%
Thus from (\ref{LEC2:Ass15}) and (\ref{LEC2:Ass18}) we have:
%(\ref{LEC2:Ass})
\begin{align*}
	& \m{reflected}(a, g \composeL s_e, \lmod_0(\ca{})) /| \\
	& \for{(s', r') \in a[s, r \composeL s_e]} \extendsAMUpto{\lmod \cup \lmod_e}{(n-1)}{s'}{r'}{\lmod_0}
\end{align*}
%
as required.\\
%
%
%



%

\noindent\textbf{Case 2. } $a \in \lmod_e(\ca{})$\\
Since from (\ref{LEC2:Ass5}) and the definition of $\m{potential}$ we have $g \composeL s_e \meetL p \composeL c \not= \emptyset$ and consequently, $s_e \meetL p \composeL c \not= \emptyset$, from (\ref{LEC2:Ass2}) and the assumption of the case we have:
%
\begin{align}
	p \leq s_e /| p \disjoint g \label{LEC2:Ass26}
\end{align}
% 
and consequently from the definition of $\m{visible}$ and (\ref{LEC2:Ass3}) we have
%
\begin{align}
	\neg\m{visible}(a, s)
	\label{LEC2:not-visible}
\end{align}
%
From (\ref{LEC2:Ass5}), (\ref{LEC2:Ass26}) and the definition of $\m{potential}$ we have:
%
\begin{align}
	\m{potential}(a, s_e) \label{LEC2:Ass27}
\end{align}
%
From (\ref{LEC2:Ass3}), (\ref{LEC2:Ass26}) and the definitions of $a[s, r \composeL s_e]$ and $\disjoint$, we know there exists $s_e'$ such that: 
%
\begin{align}
	& s' = s /| r' = r \composeL s_e' \label{LEC2:Ass29}\\
	& a[s_e] = s_e'  \label{LEC2:Ass30}
\end{align}
%
Consequently, from (\ref{EC2:Ass2}), (\ref{LEC2:Ass2}), (\ref{LEC2:Ass27}), (\ref{LEC2:Ass30}) and the definition of $\strictfences$ we have:
%
\begin{align}
	s_e' \in \fence{e}  \label{LEC2:Ass31}
\end{align}
%
From (\ref{LEC2:Ass3}), (\ref{LEC2:Ass29}) and \lem~\ref{lem:future-closure} we have:
%
\begin{align}
	\extendsAMUpto{\lmod}{(n-1)}{s'}{r}{\lmod_0}  \label{LEC2:Ass32}
\end{align}
%
From (\ref{LEC2:Ass1}), (\ref{LEC2:Ass31}), (\ref{LEC2:Ass32}), (\ref{LEC:IH}) we have:
%
\begin{align*}
	\extendsAMUpto{\lmod \cup \lmod_{e}}{(n-1)}{s'}{r \composeL s_e'}{\lmod_0}
\end{align*}
%
and thus from (\ref{LEC2:Ass29}) 
%
\begin{align}
	\extendsAMUpto{\lmod \cup \lmod_{e}}{(n-1)}{s'}{r'}{\lmod_0}
	\label{LEC2:Ass33}
\end{align}
%
Finally, from (\ref{LEC2:Ass6}), (\ref{LEC2:not-visible}) and (\ref{LEC2:Ass33}) we have:
%
\begin{align*}
	&\neg\m{visible}(a, s) /|\\
	& \for{(s', r') \in a[s, r \composeL s_e]} \extendsAMUpto{\lmod \cup \lmod_{e}}{(n-1)}{s'}{r'}{\lmod_0}
\end{align*}
%
as required.

\end{proof}
\end{lemma}
