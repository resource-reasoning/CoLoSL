\newpage
\chapter{Examples}\label{chapter:examples}



\section{Concurrent Spanning Tree}\label{sec:CST-example}
Programs manipulating arbitrary graphs present the following significant
challenge for compositional verification: because of deep sharing
between different components of a graph, changes to one subgraph may
affect other subgraphs which may point into it. This makes it hard to
reason about updates to each subgraph in isolation. In a concurrent
setting, this difficulty compounds with the fact that threads working
on different parts of the graph may affect each other in ways that are
difficult to reason about locally to each subgraph. Central to those
issues is the fact that different subgraphs present \emph{unspecified
  sharing} and may overlap in arbitrary ways. We now demonstrate on a
concurrent spanning tree example that \colosl may be just the tool you
need in this case, as it naturally deals with arbitrary overlapping
views of the shared state, and allows one to tailor interferences to a
given subjective view.

Our example, presented in \fig\ref{fig:conSpanningTree}, operates on a
\emph{directed binary graph} (henceforth simply \emph{graph}),
\textit{i.e.} a directed graph where each node has at most two
successors, called its left and right children. The program computes a
in-place spanning tree of the graph, \textit{i.e.} a tree that covers
all nodes of the graphs from a given root, concurrently as follows:
each time a new node is encountered, two new threads are created that
each prune the edges of the left and right children recursively. A
mark bit is associated to each node to keep track of which ones have
already been visited. Each thread returns whether the root
of the subgraph it was operating on had already been visited; if so,
the parent thread removes the link from its root node to the
corresponding child. Intuitively, it is allowed to do so because,
since the child was marked by another thread, the child has to be
already reachable via some other path in the graph.

%% marking the vertices to keep track of those already visited; the
%% return value b records the outcome of marking with \li{b}$=1$ when the
%% top node $x$ is unmarked (not visited yet) and \li{b}$=0$
%% otherwise. Assuming that the root vertex $x$ is unmarked initially,
%% the algorithm continues by first marking $x$ and subsequently spanning
%% the left and right subgraphs concurrently. When the top node of the
%% left subgraph is already marked (\li{!b1}), the edge from $x$ into it
%% is replaced by a null pointer. This corresponds to the case where the
%% node has already been visited by another thread and is thus reachable
%% from the root; \emph{mutatis mutandis} for the right subgraph.

We will prove that, given a shared graph as input, the program always
returns a tree, \textit{i.e.} all sharing and cycles have been
appropriately removed. Pleasingly, the \colosl specification achieves
maximal locality and allows us to reason only on the current subgraph
manipulated by a thread, instead of the whole graph. Because of
arbitrary sharing between the two, a global specification would be
unpleasant indeed!  However, we do not establish that the final tree
indeed spans the original graph. The reason it does is subtle indeed,
as are the invariants required, but in ways unrelated to our main
issue, which is to provide tight specifications for each thread.

To reason about this program, following Hobor and
Villard~\cite{ramification}, we use two representations of graphs. The
first is a mathematical representation $\gamma = (V, E)$ where $V$ is
a finite set of vertices and $E: V \rightarrow (V \uplus
\{\li{null}\}) \times (V \uplus \{\li{null}\})$ is a function
associating each vertex with at most two successors, where \li{null}
denotes the absence of an edge from the node.  We write $n \in \gamma$
for $n \in V$, $\gamma(n)$ for $E(n)$ and $|\gamma|$ for $|V|$.


%% Jules: definition of spanning tree is wrong (what's minimal?) since
%% graphs need not be connected now. Not needed anyway, as we don't
%% show that we have a spanning tree.
%% 
%% Similarly, let $\theta = (V, E)$ denote an \emph{acyclic} DCB graph
%% with no sharing between subgraphs, \textit{i.e.}, a \emph{tree}. For
%% brevity, in this section we refer to a DCB graph and an acyclic DCB
%% graph simply as a graph and a tree, respectively. Given a graph
%% $\gamma$, a tree $\theta$ \emph{spans} $\gamma$ if it includes all
%% vertices of $\gamma$ with a minimal set of edges where every edge in
%% $\theta$ is also an edge in $\gamma$.  For instance,
%% \fig\ref{fig:graphAndTree} depicts a DCB graph and a possible spanning
%% tree where the dashed lines indicate those edges that have been
%% removed from the graph after spanning it.



\begin{wrapfigure}[8]{r}{0.3\columnwidth}
	\centering
	\begin{tabular}{|c |}
		\hline
			\includegraphics[scale=0.27]{Sections/Examples/Images/graph.pdf} \\
		\hline
	\end{tabular}
\caption{A graph.}
\label{fig:graphAndTree}
\end{wrapfigure}
%%
%\begin{figure}
%\hrule
%\begin{tabular}{c c c}
%	\begin{subfigure}[b]{0.3\columnwidth}
%      \centering	
%      \includegraphics[scale=0.3]{Sections/FurtherExamples/Images/graph.pdf}
%    \caption{}
%    \label{subfig:graph}
%    \end{subfigure}
%    &
%    \begin{subfigure}[b]	{0.2\columnwidth}		
%      \centering	
%      \includegraphics[scale=0.3]{Sections/FurtherExamples/Images/tree.pdf}
%    \caption{}
%    \label{subfig:tree}
%    \end{subfigure}
%    &
%    \begin{subfigure}[b]{0.3\columnwidth}
%      \centering	
%      \includegraphics[scale=0.3]{Sections/FurtherExamples/Images/graphWithRootEdge.pdf}
%    \caption{}
%    \label{subfig:graphWithRootEdge}
%    \end{subfigure}
%    \end{tabular}
%\hrule
%\caption{A directed connected binary graph (\subref{subfig:graph}), a possible spanning tree (\subref{subfig:tree}), and the same graph with the logical edge $\rootEdge$ (\subref{subfig:graphWithRootEdge}).}
%\label{fig:graphAndTree}
%\end{figure}
%%

Mathematical graphs are connected to a second, in-memory
representation by an inductive predicate $\graph{x}{\gamma}$, denoting
a spatial (in-heap) graph rooted at address $x$ corresponding to the
mathematical graph $\gamma$. The predicate definition uses the
overlapping conjunction to account for the sharing between the left
and right children and for potential cycles in the graph, as shown in
\fig\ref{fig:globalCST}.  The basic action we allow on spatial graphs
is to \emph{mark} a node, changing its mark field from $0$ to $1$ and
claiming ownership of its left and right pointers in the process. Such
an action is allowed by a \emph{marking} capability of the form
$\markT{n}{e}$ where $n$ denotes the vertex (address) and $e$ the edge
via which vertex $n$ is visited. For instance, the capabilities
associated with marking of vertex $z$ in \fig\ref{fig:graphAndTree}
are $\markT{z}{y.r}$ and $\markT{z}{w.l}$. Note that the
parameterisation of our actions are merely a notational convenience
and can be substituted for their full definitions. Given a graph at
root address $x$, in order to account for the ability to mark the root
vertex $x$, we introduce a logical (virtual) root edge $\rootEdge$
into $x$ as depicted in \fig\ref{fig:graphAndTree} together with its
associated marking capability $\markT{x}{\rootEdge}$. The shared state
contains node $x$ which can be either unmarked ($\unmarked{x}{l}{r}$)
or marked ($\marked{x}$); as well as the left and right subgraphs
captured recursively by $\G{l}{\gamma}$ and $\G{r}{\gamma}$.
%% Note that the two subgraphs and vertex $x$ are combined by
%% the overlapping conjunction $\sepish$ since the graph can be cyclic
%% and each node may be reachable via more than one path.

Each vertex is represented as three consecutive cells in the heap
tracking the mark bit and the addresses of the left ($l$) and right
($r$) subgraphs. For brevity, we write $\cell{x}{m, l, r}$ for
$\cell{x}{m} * \cell{x+1}{l} * \cell{x+2}{r}$, and $x.m$, $x.l$, and
$x.r$ for $x$, $x+1$, and $x+2$, respectively. When vertex $x$ is in
the unmarked state, the whole cell $\cell{x}{0,l,r}$ and the
capabilities to mark the children reside in the shared state. In the
marked state, the shared state only contains $\cell{x.m}{1}$: the left
and right subgraphs (pointers and capabilities) have been claimed by
the thread who marked the node, while other threads need not access
the children of $x$ once they see that $x$ is already marked. The
atomic \li{CAS} instruction prevents several threads to concurrently
mark the same node and claim ownership of the same resource.

The interference associated with the graph is described as the union
of interferences pertaining to the vertices of the graph ($n \in
\gamma$). For each vertex $n \in \gamma$, the only permitted action is
that of marking $n$ which can be carried out by any of the marking
capabilities associated with node $n$ ($\markT{n}{-}$). Note that the
anonymous quantification $-$ is yet another notational shorthand and
can be substituted for the following more verbose definition.
%
\[
\vspace{-1ex}
I(n) \eqdef \bigcup\limits_{p \in \gamma} \left(\bigcup_{e \in \{p.l,
  p.r, \rootEdge\}}\!\!\! \markT{n}{e} : \exsts{l, r} \unmarked{n}{l}{r} \swap \marked{n} \right)
\]
%
%
\begin{figure}
%
\hrule
\[
\begin{array}{r @{\hspace*{2pt}} l}
	\graph{x}{\gamma} \eqdef & \left[\markT{x}{\rootEdge}\right] * \shared{\G{x}{\gamma}}{I_\gamma} \hspace*{0.5cm} I_\gamma \eqdef \bigcup\limits_{n \in \gamma}I(n)\\
%	
	\G{x}{\gamma} \eqdef & (x = \li{null} \land \emp) \lor x \in \gamma \land \exsts{l, r} \gamma(x) = (l, r) \\
	& \land \left( \unmarked{x}{l}{r} \lor \marked{x}\right) \sepish \G{l}{\gamma} \sepish \G{r}{\gamma}\\
%
	\unmarked{x}{l}{r} \eqdef & \cell{x}{0, l, r} * \left[\markT{l}{x.l}\right] * \left[\markT{r}{x.r} \right]\\
%	
	\marked{x} \eqdef & \cell{x}{1}\\
%
	I(n) \eqdef & \left\{ \markT{n}{-}: \exsts{l, r} \unmarked{n}{l}{r} \swap \marked{n}\right\}\\
%
%	\tree{x}{\gamma} \eqdef & \markT{x}{\rootEdge} * \shared{\G{x}{\gamma}}{\bigcup\limits_{n \in \gamma}I(n)}\\
\end{array}
\]
%
\hrule
\caption{Global specification of the graph predicate.}
\label{fig:globalCST}
\end{figure}
%
%
\fig\ref{fig:conSpanningTree} shows an in place concurrent algorithm for calculating a spanning tree of a graph. 
%
\begin{figure}
%
\hrule
\[
\begin{array}{r @{\hspace*{2pt}} l}
	\g{x}{\gamma} \eqdef & (x = \li{null} \land \emp) \lor x \in \gamma \land \exsts{l, r} \gamma(x) = (l, r) \land\\
	& \shared{\unmarked{x}{l}{r} \lor \marked{x}}{I(x)} * \g{l}{\gamma} * \g{r}{\gamma}\\
	
	\tr{x}{\gamma} \eqdef & (x = \li{null} \land \emp) \lor x \in \gamma \land \exsts{l, r} \gamma(x) = (l, r) \land\\
	& \shared{\marked{x}}{I(x)} *  \exsts{l' \in \{l, \li{null}\}} \exsts{r' \in \{r, \li{null}\}}\\
	& \left[\markT{l}{x.l}\right] * \cell{x.l}{l'} * \tr{l'}{\gamma} * \\
	& \left[\markT{r}{x.r}\right] * \cell{x.r}{r'} * \tr{r'}{\gamma}
\end{array}
\]
\hrule
\caption{Local specification of the graph predicate.}
\label{fig:localCST}
\end{figure}
%
The $\graph{x}{\gamma}$ predicate defined in \fig\ref{fig:globalCST} is a \emph{global} account of the graph in that it captures all vertices and the interference associated with them. However, our spanning tree algorithm operates \emph{locally} as it is called upon recursively for each node. That is, for each \li{span(n)} call (where $\cell{\li{n}}{n}$ and $n \in \gamma$), the footprint of the call is limited to node $n$. Moreover, in order to reason about the concurrent recursive calls $\li{span(x.l)} || \li{span(x.r)}$, we need to \emph{split} the state into two $*$-composed states prior to the calls, pass each constituent state onto the relevant thread and combine the resulting states by $*$ composition through an application of the \parRule\ rule. We thus provide a \emph{local} specification of the graph, $\g{x}{\gamma}$ as defined in \fig\ref{fig:localCST} such that for all $n, p \in \gamma$ and $e \in \{p.l, p.r, \rootEdge \}$
%
\[
\begin{array}{@{} c @{} }
	\color{blue}{
	\Big\{
		\cell{\li{n}}{n} * \cell{\li{b}}{-} * 
		\left[\markT{n}{e}\right] * 
		\g{n}{\gamma}
	\Big\} 
	} \\
%	
	\command{b:= span(n)} \\ 
%
	\color{blue}{
	\Big\{
		\cell{\li{n}}{n} *  
		\left[\markT{n}{e}\right] * 
		\left(
%		\begin{array}{@{} l @{}}
			\cell{\li{b}}{1} * \tr{n}{\gamma} \lor
			\cell{\li{b}}{0} *  \tr{\li{null}}{\gamma}
%		\end{array}
		\right)
	\Big\}
	}
\end{array}
\]
%
The definition of the $\g{x}{\gamma}$ predicate is similar to that of $\shared{\G{x}{\gamma}}{I_{\gamma}}$ except that the global view $\shared{\G{x}{\gamma}}{I_{\gamma}}$ that describes the resources associated with all $|\gamma|$ vertices has been replaced by $|\gamma|$ $*$-composed more local views, each describing the resources of a vertex $n \in \gamma$. Moreover, the interference of each local view concerning a vertex $n \in \gamma$ has been shifted from $I_{\gamma}$ to $I(n)$ as to reflect only those actions that affect $n$.  

Similarly, the $\tr{x}{\gamma}$ predicate represents a \emph{tree}
rooted at $x$, as is standard in separation logic~\cite{rey02}, and
consists of $|\gamma|$ subjective views one for each vertex in
$\gamma$. The assertion of each subjective view reflects that the
corresponding vertex ($x$) has been marked
$\shared{\marked{x}}{I(x)}$. The resources associated with each node
$x$, namely the left and right pointers and the corresponding marking
capabilities have been claimed by the marking thread and moved into
the local state. The vertex addressed by the left pointer of $x$
(\textit{i.e.} $l'$) corresponds to either the initial value prior to
marking ($l$ where $\gamma(x) = (l, r)$) or \li{null}\ when $l$ has
more than one predecessors and has been marked by another thread,
making the whole predicate stable against actions of the program and
the environment.

We now demonstrate how to obtain the local specification $\g{x}{\gamma}$ from the global specification of \fig\ref{fig:globalCST}. 
When expanding the definition of $\G{x}{\gamma}$, there are two cases to consider depending on whether or not $x = \li{null}$. In what follows we only consider the case where $x \not= \li{null}$ since the derivation in the case of $x = \li{null}$ is trivial.
Let $P$ and $Q$ predicates be defined as below.
%
\[
\begin{array}{l l}
	P \eqdef & \iterStar_{n \in \gamma} \left( \gamma(n) = (l, r) \land (\unmarked{n}{l}{r} \lor \marked{n}) \; \right)\\
	
	Q \eqdef & \iterStar_{n \in \gamma} \left( \gamma(n) = (l, r) \land \shared{\unmarked{n}{l}{r} \lor \marked{n}}{I(n)} \right)\\
\end{array}	
\]
%
From the definitions of $\G{x}{\gamma}$ and $\g{x}{\gamma}$ we then have:
%
\begin{mathpar}
	\G{x}{\gamma} \iff  P
	
	\g{x}{\gamma} \iff Q
\end{mathpar}
%
%% %
%% \[
%% \begin{array}{l @{\hspace*{1cm}} c @{\hspace*{1cm}} l}
%% 	\G{x}{\gamma} \iff  P & \text{and} & \g{x}{\gamma} \iff Q
%% \end{array}
%% \]
%% %
In order to derive the local specification $\g{x}{\gamma}$ from the global specification $\G{x}{\gamma}$, it thus suffices to show $\shared{P}{I_{\gamma}} \semimplies Q$ as demonstrated below.
%
\begin{align*}
	\shared{P}{I_{\gamma}} &
	\stackrel{(\textsc{Copy})}{\implies}
	\underbrace{\shared{P}{I_{\gamma}} * \cdots * \shared{P}{I_{\gamma}}}_{|\gamma| \text{ times}}\\
	&\stackrel{(\textsc{Forget})}{\implies}
	\iterStar_{n \in \gamma} \left( \gamma(n) = (l, r) \land \shared{\unmarked{n}{l}{r} \lor \marked{n}}{I_{\gamma}}  \right)\\
	& \stackrel{(\textsc{Shift})}{\semimplies}
	\iterStar_{n \in \gamma} \left( \gamma(n) = (l, r) \land \shared{\unmarked{n}{l}{r} \lor \marked{n}}{I(n)}  \right)\\
	&\iffdef Q
\end{align*}
%
%%
%\[
%\begin{array}{@{} c @{} l @{}}
%	&\shared{P}{\bigcup\limits_{n \in S} I(n)}  \\
%	
%	\stackrel{(\textsf{G}\ \defin)}{\implies} & \shared{\exsts{l, r} (\unmarked{x}{l}{r} \lor \marked{x}) \sepish \G{l}{S} \sepish \G{r}{S}}{\bigcup\limits_{n \in S} I(n)} \\
%	
%	\implies &   \exsts{l, r}  \shared{(\unmarked{x}{l}{r} \lor \marked{x}) \sepish \G{l}{S} \sepish \G{r}{S}}{\bigcup\limits_{n \in S} I(n)} \\
%	
%	\stackrel{(\textsc{Copy})}{\implies} &
%	\exsts{l, r}  
%	\shared{(\unmarked{x}{l}{r} \lor \marked{x})  \sepish \G{l}{S} \sepish \G{r}{S}}{\bigcup\limits_{n \in S} I(n)} \\
%	& * \shared{(\unmarked{x}{l}{r} \lor \marked{x}) \sepish \G{l}{S} \sepish \G{r}{S}}{\bigcup\limits_{n \in S} I(n)} \\
%	& * \shared{(\unmarked{x}{l}{r} \lor \marked{x})  \sepish \G{l}{S} \sepish \G{r}{S}}{\bigcup\limits_{n \in S} I(n)} \\
%	
%	
%	\stackrel{(\textsc{Forget})}{\implies} &
%	\exsts{l, r}  
%	\shared{\unmarked{x}{l}{r} \lor \marked{x}  }{\bigcup\limits_{n \in S} I(n)} \\
%	& * \shared{\G{l}{S}}{\bigcup\limits_{n \in S} I(n)} * \shared{\G{r}{S}}{\bigcup\limits_{n \in S} I(n)} \\
%	
%	
%	
%	\stackrel{(?)}{\semimplies} &
%	\exsts{l, r}  
%	\shared{\unmarked{x}{l}{r} \lor \marked{x}}{\bigcup\limits_{n \in S} I(n)} * \g{l}{S} * \g{r}{S}\\
%	
%	
%	\stackrel{(\textsc{Shift})}{\semimplies} &
%	\exsts{l, r}  
%	\shared{\unmarked{x}{l}{r} \lor \marked{x} }{I(x)} * \g{l}{S} * \g{r}{S}\\
%	
%	
%	\iffdef & \g{x}{S}
%	
%\end{array}
%\]
%%
%
\begin{figure}
\hrule     
\begin{lstlisting}
  //<@\codecomment{$\cell{\tx{x}}{x} * \cell{\tx{b}}{-} * \graph{x}{\gamma}$}@>
  //<@\codecomment{$\cell{\tx{x}}{x} *  \cell{\tx{b}}{-} * \markT{x}{\rootEdge} * \shared{\G{x}{\gamma}}{I_{\gamma}}\}$}@>
  //<@\codecomment{$\{\cell{\tx{x}}{x} * \cell{\tx{b}}{-} * \left[\markT{x}{\rootEdge}\right] * \g{x}{\gamma}\}$}@>
  b:= span(x) $\{$
  //<@\codecomment{$\left\{\begin{array}{@{}l@{}} \cell{\tx{x}}{x} * \cell{\tx{b}}{-} * \left[\markT{x}{\rootEdge}\right] * \exsts{l, r} \shared{\unmarked{x}{l}{r} \lor \marked{x}}{I(x)} * \g{l}{\gamma} * \g{r}{\gamma} \end{array} \right\}$}@>
    res:= $\langle$ CAS(x.m, 0, 1) $\rangle$;
    //<@\codecomment{$\left\{\begin{array}{@{} l @{}} \cell{\tx{x}}{x} * \cell{\tx{b}}{-} * \left[\markT{x}{\rootEdge}\right] * \shared{\marked{x}}{I(x)} * \exsts{l, r} \g{l}{\gamma} * \g{r}{\gamma} \\ * \left(\cell{\tx{res}}{0} \lor  \left(\begin{array}{l} \cell{\tx{res}}{1} * \cell{x.l}{l} * \cell{x.r}{r} * \left[\markT{l}{x.l}\right] * \left[\markT{r}{x.r}\right] \end{array}\right)\right)\end{array}\right\}$}@>
    if (res) then $\{$ 
      //<@\codecomment{$\left\{\begin{array}{@{} l @{}} \cell{\tx{x}}{x} * \cell{\tx{b}}{-} * \left[\markT{x}{\rootEdge} \right] * \shared{\marked{x}}{I(x)} * \cell{\tx{res}}{1}\\ * \exsts{l, r} \cell{x.l}{l} * \cell{x.r}{r} * \left[\markT{l}{x.l} \right] * \g{l}{\gamma} * \left[\markT{r}{x.r}\right] * \g{r}{\gamma}  \end{array} \right\}$}@>
      //<@\codecomment{$\left\{ \left[\markT{l}{x.l} \right] * \g{l}{\gamma} * \left[\markT{r}{x.r} \right] * \g{r}{\gamma}  \right\}$}@>
      b1:= span(x.l) || b2:= span(x.r)
      //<@\codecomment{$\left\{\begin{array}{@{} l @{}}  \left[\markT{l}{x.l} \right] * \left( (\cell{\tx{b1}}{1} *  \tr{l}{\gamma}) \lor \cell{\tx{b1}}{0}\right) *\\ \left[ \markT{r}{x.r} \right] * \left( (\cell{\tx{b2}}{1} * \tr{r}{\gamma}) \lor \cell{\tx{b2}}{0} \right)  \end{array}\right\}$}@>
      //<@\codecomment{$\left\{\begin{array}{@{} l @{} }  \cell{\tx{x}}{x} * \cell{\tx{b}}{-} * \left[ \markT{x}{\rootEdge} \right] * \shared{\marked{x}}{I(x)} * \cell{\tx{res}}{1} \\ \begin{array}{@{}l @{\hspace{2pt}} l @{} } * \exsts{l, r} &\cell{x.l}{l} * \left[ \markT{l}{x.l} \right] * \left( (\cell{\tx{b1}}{1} *  \tr{l}{\gamma}) \lor \cell{\tx{b1}}{0} \right) *\\  &\cell{x.r}{r} *\left[ \markT{r}{x.r} \right] * \left( (\cell{\tx{b2}}{1} * \tr{r}{\gamma}) \lor \cell{\tx{b2}}{0} \right) \end{array}   \end{array}\right\}$}@>
      if (!b1) then 
        $\text{[}$x.l$\text{]}$:= null
      if (!b2) then 
        $\text{[}$x.r$\text{]}$:= null
      //<@\codecomment{$\left\{\begin{array}{@{} l @{}}  \cell{\tx{x}}{x} * \cell{\tx{b}}{-} * \left[ \markT{x}{\rootEdge} \right] * \shared{\marked{x}}{I(x)} * \cell{\tx{res}}{1} * \cell{\tx{b1}}{-} * \cell{\tx{b2}}{-}\\  \begin{array}{@{} l @{\hspace{2pt}} l @{}} * \exsts{l, r}  & \exsts{l' \in \{l, \tx{null}\}} \left[ \markT{l}{x.l} \right] * \cell{x.l}{l'} * \tr{l'}{\gamma} *\\ & \exsts{r' \in \{r, \tx{null}\}} \left[ \markT{r}{x.r} \right] * \cell{x.r}{r'} * \tr{r'}{\gamma} \end{array} \end{array}\right\}$}@>
      //<@\codecomment{$\left\{\begin{array}{@{} l @{}}  \cell{\tx{x}}{x} * \cell{\tx{b}}{-} * \cell{\tx{res}}{1} *\cell{\tx{b1}}{-} * \cell{\tx{b2}}{-} * \markT{x}{\rootEdge} *  \tr{x}{\gamma}    \end{array}\right\}$}@>
    $\}$    
    //<@\codecomment{$\left\{ \begin{array}{@{} l @{}} \cell{\tx{x}}{x} * \cell{\tx{b}}{-} * \left[ \markT{x}{\rootEdge} \right]*  \\  (\cell{\tx{res}}{1}  * \tr{x}{\gamma}) \lor \left(\begin{array}{@{} l @{}}\cell{\tx{res}}{0} * \shared{\marked{x}}{I(x)} * \g{l}{\gamma} * \g{r}{\gamma} \end{array}\right) \end{array} \right\}$}@>
    //<@\codecomment{$\left\{ \begin{array}{@{} l @{}} \cell{\tx{x}}{x} * \cell{\tx{b}}{-} * \left[ \markT{x}{\rootEdge} \right] *   (\cell{\tx{res}}{1}  * \tr{x}{\gamma}) \lor (\cell{\tx{res}}{0} ) \end{array} \right\}$}@>
    return res
  $\}$
  //<@\codecomment{$\left\{ \begin{array}{@{} l @{}} \cell{\tx{x}}{x}  * \left[ \markT{x}{\rootEdge} \right] *  ((\cell{\tx{b}}{1}  * \tr{x}{\gamma}) \lor \cell{\tx{b}}{0}) \end{array} \right\}$}@>
\end{lstlisting}
\hrule\vspace*{-6pt}
\caption{Concurrent Spanning Tree Implementation}
\label{fig:conSpanningTree}
\end{figure}
%
%
\vspace{-2ex}
\clearpage\section{Set Module}\label{sec:set-example}
In this section we consider a concurrent set module as described in~\cite{cap-ecoop10}. We first produce the set specification in \colosl and show how to reason about its operations and verify them with respect to their specification. We then compare the \colosl specification of the set module against its specification in Concurrent Abstract Predicates (CAP) described in~\cite{cap-ecoop10}. We demonstrate that our \colosl\ reasoning considerably improves on CAP by producing a \emph{more concise} specification and allowing for \emph{more local} reasoning. 

\subsection*{\colosl Specification}
We implement a set as a sorted singly-linked list with no duplicate elements (since it represents a set) and one lock per node as described in~\cite{cap-ecoop10}. Our set module provides three operations: \li{contains(h, $v$)}, \li{add(h, $v$)} and \li{remove(h, $v$)}.
As the name suggests, the \li{contains(h, $v$)} function checks whether value $v$ is contained in the set at \li{h}; the \li{add(h, $v$)} and \command{remove(h, $v$)} functions add/remove value $v$ to/from the list, respectively.

\fig~\ref{fig:set-locate}-\ref{fig:set-add} illustrate a possible implementation of the set operations. All three operations proceed by traversing the sorted list from the head address and locating the first node in the list holding a value $v'$ greater than or equal to $v$ (through the \li{locate} procedure). The algorithm for locating such a node begins by locking the head node; it then moves down the list by hand-over-hand locking whereby first the node following the one currently held is locked and subsequently the previously locked node is released. No thread can access a node locked by another thread or traverse past it. As such, no thread can overtake an other thread in accessing the list. 
%%
%


%
%%
%%%
%%\[
%%\begin{array}{@{} r @{\hspace*{3pt}} c @{\hspace*{3pt}} l @{}}
%%	\left\{ \inSet{\var{h}}{\var{v}} \right\} & \command{contains(h,v)} & \left\{ \inSet{\var{h}}{\var{v}} * \var{ret} = 1 \right\} \\
%%	
%%	\left\{ \outSet{\var{h}}{\var{v}} \right\} & \command{contains(h,v)} & \left\{ \outSet{\var{h}}{\var{v}} * \var{ret} = 0 \right\} \\
%%	
%%	\left\{ \inSet{\var{h}}{\var{v}} \lor \outSet{\var{h}}{\var{v}}  \right\} & \command{add(h,v)} & \left\{ \inSet{\var{h}}{\var{v}} \right\}\\ 
%%	
%%	\left\{ \inSet{\var{h}}{\var{v}} \lor \outSet{\var{h}}{\var{v}}  \right\} & \command{remove(h,v)} & \left\{ \outSet{\var{h}}{\var{v}} \right\} 
%%\end{array}
%%\]
%%%


\fig\ref{fig:coloslSetExample} shows a specification of the set module in \colosl\ similar to that of Concurrent Abstract Predicates (CAP) defined in \cite{cap-ecoop10}. 
In what follows we first give a description of the predicates of \fig\ref{fig:coloslSetExample} and subsequently contrast our specification with that of CAP.
%
%
\begin{figure}
\hrule
\[
\begin{array}{@{} l @{}}
	\begin{array}{r @{\hspace{3pt}} l}
		\inSet{h}{v} \eqdef &\exsts{\pi} \isLock{h}{\pi} *  \valueC{h}{v} * \inList{h}{v}\\	
		\outSet{h}{v} \eqdef & \exsts{\pi} \isLock{h}{\pi} * \valueC{h}{v} * \outList{h}{v} \vspace{7pt}\\
		
		\sortedList{\in}{h}{v} \eqdef & \exsts{L} \sorted{L} \land v  \in L \land \lsg{h}{\li{null}}{-\infty:: L ++ \{\infty\} }{h}\\	
		\sortedList{\not\in}{h}{v} \eqdef & \exsts{L} \sorted{L} \land v  \not\in L \land \lsg{h}{\li{null}}{-\infty:: L ++ \{\infty\} }{h} \vspace{7pt}\\
		
		\lsg{x}{z}{L}{h} \eqdef & (L = [] \land x = z \land \emp) \lor \\
		& \exsts{y, v, L'} L = v::L' \land \shared{\node{x}{v}{y}}{I(h, x)} * \lsg{y}{z}{L'}{h} \vspace{7pt}\\
		
		\node{x}{v}{y} \eqdef & \unlockedNode{x}{v}{y} \lor \lockedNode{x}{v}{y}\\
		\unlockedNode{x}{v}{y}  \eqdef & \link{x}{y} * \val{x}{v}{y}\\
		\lockedNode{x}{v}{y} \eqdef  & \locked{x} * \val{x}{v}{y} \vspace{7pt}\\
		
		\link{x}{y} \eqdef & \cell{x.next}{y} * \isLock{y}{1} * \modC{x}\\
		\val{x}{v}{y} \eqdef &  \cell{x.value}{v} * \nextC{x}{y}	\\
		
		\isLock{x}{\pi} \eqdef & \lockC{x}[\pi] * \shared{\cell{x.lock}{0} * \unlockC{x} \lor \cell{x.lock}{1}}{L(x)}\\
		\locked{x} \eqdef  & \unlockC{x} * \shared{\cell{x.lock}{1}}{L(x)}
	\end{array}	\vspace{10pt}\\
	
	
	\quad L(x) \eqdef 
	\left\{
	\begin{array}{@{} r @{\hspace{2pt}}  l @{}}
		\color{darkgreen}{\lockT{x}[-]} : &  \color{blue}{\left\{ \cell{x.lock}{0} * \unlockC{x}  \swap \cell{x.lock}{1} \right\} }\\
		\color{darkgreen}{\unlockT{x}} : &  \color{blue}{ \left\{ \cell{x.lock}{1} \swap \cell{x.lock}{0} * \unlockC{x} \right\} }\\
		\color{darkgreen}{\lockT{x} * \unlockT{x}}:  &  \color{blue}{ \left\{ \cell{x.lock}{1} \swap \lockT{x}[1] * \unlockT{x}[1] \right\} }
	\end{array}
	\right\} \vspace{10pt}\\
	

	I(h, x) \eqdef \\
	\quad
	\left\{
	\begin{array}{@{} r @ {\hspace*{2pt}}l @{}}
%%		\text{//Locking}\hspace*{1.4cm}&\\
%		\color{darkgreen}{true} : & \color{blue}{\left\{\exsts{v, y} \unlockedNode{x}{v}{y} \swap \lockedNode{x}{v}{y} \right\}}\vspace{5pt}\\
%%		\color{darkgreen}{\modT{x} } : & \color{blue}{\left\{\exsts{v, y} \lockedNode{x}{v}{y} \swap \unlockedNode{x}{v}{y} \right\}} \vspace{5pt}\\
%		
		\color{darkgreen}{true} : 
		& \color{blue}
		\left\{
		\begin{array}{@{} r @{\hspace{2pt}} l @{\hspace{3pt}} l @{\hspace{3pt}} l @{}}
			\exsts{v, y} & \unlockedNode{x}{v}{y} & \swap & \lockedNode{x}{v}{y} \vspace{5pt}\\
			\exsts{v, y} & \lockedNode{x}{v}{y} & \swap & \unlockedNode{x}{v}{y}
		\end{array}		
		\right\}		\vspace{5pt}\\
		
		
		
		\color{darkgreen}{\modT{x} * \valueT{h}{v}:} & \\		
		&\hspace{-3.7cm}   
		\color{blue}{\left\{
		\begin{array}{@{} l @{\hspace{5pt}} l @{}}
			\exsts{v',y, z} & \lockedNode{x}{v'}{y} * \lockedNode{y}{v}{z}\\
			& \quad \swap \lockedNode{x}{v'}{z} * \lockedNode{y}{v}{z}\\
			
			\exsts{v', w, y} & v' < v /| \lockedNode{w}{v'}{y} * \lockedNode{x}{v}{y}\\
			& \quad \swap \lockedNode{w}{v'}{y}\\
			
			\exsts{v_1,\! v_2,\! y,\! z,\! w} & v_1 < v < v_2 \land \lockedNode{x}{v_1}{y} * \val{w}{v}{y} * \val{y}{v_2}{z} \\
	  	& \quad \swap \lockedNode{x}{v_1}{w} * \val{w}{v}{y} *  \val{y}{v_2}{z}		
		\end{array}
		\right\}
		}

		
	\end{array}
	\right\}


	
%	I(h, a) \eqdef 
%	\left\{
%	\begin{array}{@{} r @ {\hspace*{2pt}}l @{} }
%%		\text{//Locking}\hspace*{1.4cm}&\\
%		\lockT{a}: &\left\{ \exsts{v, b} \unlockedNode{a}{v}{b} \swap \lockedNode{a}{v}{b}\right\} \vspace{5pt}\\
%		
%%		\text{//Unlocking}\hspace*{1.1cm}&\\
%		\unlockT{a}: & \left\{ \exsts{v, b} \lockedNode{a}{v}{b} \swap \unlockedNode{a}{v}{b}\right\} \vspace{5pt}\\ 
%		
%%		\text{//Deletion of } v \hspace*{0.8cm}&\\
%		\unlockT{a} * \valueT{h}{v}: &
%		\left\{
%		\begin{array}{@{} l @{\hspace{5pt}} l @{}}
%			\exsts{v_0, b, c} & \lockedNode{a}{v_0}{b} * \lockedNode{b}{v}{c} \\
%			& \quad \swap \lockedNode{a}{v_0}{c} * \lockedNode{b}{v}{c} \\
%			\exsts{v_0, b, c} & \lockedNode{b}{v_0}{c} * \lockedNode{a}{v}{c} \\
%			& \quad \swap \lockedNode{b}{v_0}{c} 
%			
%		\end{array}
%		\right\} \vspace{5pt}\\ 
%
%		
%%		\text{//Insertion of } v \hspace*{0.7cm}&\\
%		\unlockT{a} * \valueT{h}{v}: &\\
%		&\hspace{-3cm}
%		\left\{
%		\begin{array}{@{} l @{\hspace{4pt}} l @{}}
%			\exsts{v_1, v_2, c, d} & v_1 < v < v_2 \land \lockedNode{a}{v_1}{c} * \lockedNode{c}{v_2}{d} \\
%			& \quad \swap \lockedNode{a}{v_1}{b} * \node{b}{v}{c} *  \lockedNode{c}{v_1}{d}\\
%			 
%			\exsts{v_1, v_2, c, d} & v_1 < v < v_2 \land \lockedNode{a}{v_1}{c} * \unlockedNode{c}{v_2}{d} \\
%			& \quad \swap \lockedNode{a}{v_1}{b} * \node{b}{v}{c} *  \unlockedNode{c}{v_1}{d}
%						
%		\end{array}
%		\right\}\\ 
%		
%	\end{array}
%	\right\}
\end{array}
\]
%
\hrule
\caption{\colosl\ specification of the concurrent set module.}
\label{fig:coloslSetExample}
\end{figure}
%
%

Since \colosl is parametric in the separation algebra of capabilities and their assertions, we instantiate it with a \emph{fractional heap} to represent the separation algebra of capabilities and write heap-like assertions of the form $\setCap{a}[\pi][b]$ to denote capability $a$ with permission $\pi$. Moreover, our capabilities are \emph{stateful} in that they can capture some additional information ($b$). As we demonstrate below, this leads to conciser specifications. For readability we write $\setCap{a}[][b]$ for $\setCap{a}[1][b]$, $\setCap{a}[\pi]$ for $\exsts{b}\setCap{a}[\pi][b]$ and $\setCap{a}$ for $\exsts{b}\setCap{a}[1][b]$.

%\noindent\textbf{\textsf{isLock}($a, \pi$) / \textsf{locked}($a$)} \hspace{0.3cm} 
\paragraph{\textsf{isLock}(\textit{x}, $\pi$) / \textsf{locked}(\textit{x})} 
Every node in the singly-linked list is protected by a lock that is to be acquired prior to its modification. The $\isLock{x}{\pi}$ predicate is similar to that in \cite{cap-ecoop10} and states that the lock at address $x$ can be acquired with permission $\pi \in (0, 1]$. Multiple threads may attempt to acquire the lock at once and thus we use the $\pi$ argument to reflect this sharing where $1$ denotes exclusive right to acquisition while $\pi \in (0, 1)$ accounts for sharing of the lock between multiple threads. The $\isLock{x}{\pi}$ predicate asserts that the thread's local state contains the capability $\lockT{x}[\pi]$ to acquire the lock and that the lock resides in the shared state where at any one point either the lock is unlocked ($\cell{x.lock}{0}$) and the shared state contains the capability to unlock it ($\unlockT{x}$); or it is locked ($\cell{x.lock}{1}$) and the unlocking capability has been claimed by the locking thread. 

The $\locked{x}$ predicate asserts that the thread's local state contains the exclusive capability to unlock the lock ($[\unlockT{x}]$); and that the $x$ is in the locked state ($\cell{x.lock}{1}$). 

\paragraph{\textit{L(x)}} denotes the interference associated with the lock on node at address $x$. When a thread holds a non-zero locking capability on node $x$ ($\lockC{x}[-]$), it can change the lock state from unlocked (0) to locked (1) and claim the exclusive unlocking capability in doing so ($\unlockC{x}$). 
Similarly, a thread holding the full unlocking capability on $x$ ($\unlockC{x}$) can change the lock state from locked to unlocked and return the full unlocking capability to the shared state. 
When removing a value $v$ from the set, the node containing the value $v$ must be removed from the shared state and disposed of. However, recall that \colosl does not have an explicit mechanism for unsharing resources modelling a dual behaviour to that of the \extendRule principle. Instead, resources can be unshared through actions explicitly specified as part of their associated interference. This is the case in the last component of $L(x)$; it is concerned with \emph{unsharing} of node $x$'s lock and making it a local resource. It states that when a thread holds full permission on both locking and unlocking capabilities of $x$, it can claim its lock in exchange for the capabilities. 

\paragraph{\textsf{in}(\textit{h}, \textit{v}) / \textsf{out}(\textit{h}, \textit{v})}
The $\inSet{h}{v}$ predicate states that the set at head address $h$ contains value $v$ and captures exclusive right to alter the set with respect to $v$ by changing whether $v$ belongs to it. It asserts that the thread owns some capability ($\pi$) to acquire the lock associated with head address ($\isLock{h}{\pi}$) and that the thread owns the exclusive capability to alter the set with respect to value $v$ ($\valueC{h}{v}$) should it own the capability to modify the address at which value $v$ is stored (\textit{cf.} the description of $\unlockedNode{x}{v}{y}$ predicate). The underlying singly-inked list is captured by the $\inList{h}{v}$ predicate. 
The $\inList{h}{v}$ predicate uses an auxiliary carrier sorted list $L$ (such that $v \in L$) to capture the contents of the singly-linked list via the $\textsf{lsg}$ predicate. For simpler implementation, we extend $L$ with two sentinel values $-\infty$ and $\infty$, one at each end to avoid corner cases such as removing the first/last element of the list. The $\outSet{h}{v}$ predicate is analogous to $\inSet{h}{v}$ and corresponds to the case where the set does not contain $v$.

\paragraph{\textsf{lsg}(\textit{x}, \textit{z}, \textit{L}, \textit{h})} 
This predicate is defined inductively and describes a \emph{segment} of the list at $h$ that starts at address $x$ and extends upto (but not including) address $z$ and contains the elements in the mathematical list $L$. When $L$ is empty, it corresponds to a void segment and yields no resources (\emp); otherwise, it is defined as the composition of the first node of the segment at address $x$, $\shared{\node{x}{v}{y}}{I(h, x)}$, and the tail of the list segment ($\lsg{y}{z}{L'}{h}$). The $\node{x}{v}{y}$ predicate describes a node at address $x$ with value $v$ and successor $y$ and can be either unlocked ($\unlockedNode{x}{v}{y}$) or locked ($\lockedNode{x}{v}{y}$).

\paragraph{\textsf{U}(\textit{x}, \textit{v}, \textit{y}) / \textsf{L}(\textit{x}, \textit{v}, \textit{y})}
The $\unlockedNode{x}{v}{y}$ predicate states that the node at address $x$ is unlocked, it contains value $v$ and comes before the node at address $y$. The statement of the $\lockedNode{x}{v}{y}$ predicate is analogous and describes the node when locked.
%In both locked and unlocked states, the shared state contains the lock field ($\cell{a}{0}$ or $\cell{a}{1}$), the value field ($\cell{a+1}{v}$) and the capability that records the current successor of the node $\nextC{a}{b}$.
%When a node is in the unlocked state, the shared state contains the next pointer of the node ($\cell{a+2}{b}$), as well as the capability to unlock the node ($[\unlockT{a}]$).
%% A thread can alter the set at $h$ with respect to the node at $a$ with value $v$ only when it holds both the value capability $\valueC{h}{v}$ and the modification capability $\modC{a}$ in its local state. 
%Recall that when traversing the list by hand-over-hand locking, no thread can overtake an other in traversing the list. Consequently, a node can only be locked by a thread that has already acquired the lock associated with the previous node. 
%As such, when the node $a$ is in the unlocked state, the exclusive capability to lock the successor node at address $b$ ($[\lockT{b}]$) also lies in the shared state.
%% 
%
%When a thread successfully locks the node at $a$ it claims the next pointer, the unlocking capability and the locking capability pertaining to its successors. ($\cell{a.next}{b} * [\unlockT{a}] * [\lockT{b}]$) and moves them into its local state.
%\\
%
%A thread can alter the set at $h$ with respect to the node at address $x$ with value $v$, only when it holds both the value capability $\valueC{h}{v}$ and the modification capability $\modC{x}$ in its local state. When the node is in the unlocked state, no thread can modify it and thus the modification capability $\modC{x}$ as well as the next pointer of the node ($\cell{x.next}{y}$) lie in the shared state. 
A thread may modify the next pointer of node at address $x$, only when it holds the modification capability $\modC{x}$ in its local state. When the node is in the unlocked state, no thread can modify it and thus the modification capability $\modC{x}$ as well as the next pointer of the node ($\cell{x.next}{y}$) lie in the shared state. 
%
Recall that when traversing the list by hand-over-hand locking, no thread can overtake an other in traversing the list. Consequently, a node can only be locked by a thread that has already acquired the lock associated with the previous node. Therefore, when the node $x$ is in the unlocked state, the exclusive capability to lock its successor at address $y$ ($\isLock{y}{1}$) also lies in the shared state.
%
When a thread successfully locks the node at $x$ (and is hence in possession of the $\locked{x}$ resource), it can then claim the next pointer, the modification capability and the locking capability pertaining to its successors (captured by $\link{x}{y}$), and move them into its local state. In return, it transfers the $\locked{x}$ resource to the shared state as evidence that it is indeed the locking thread.

In both locked and unlocked states, the shared state contains the value field ($\cell{x.value}{v}$) and the ``successor'' capability $\nextC{x}{y}$. This capability is used to track the current successor of $x$ even when the node is locked and its next pointer has been claimed by the locking thread. 

\paragraph{\textit{I}(\textit{h}, \textit{x})} As discussed above, given a node at address $x$ with successor $y$, any thread in possession of the $\locked{x}$ resource may change its state from unlocked ($\unlockedNode{x}{v}{y}$) to locked ($\lockedNode{x}{v}{y}$) and claim the $\link{x}{y}$ resource in exchange for $\locked{x}$. This is captured by the first action associated with the $\m{true}$ assertion; that is, no additional capability is required for performing this action and holding the $\locked{x}$ resource locally is sufficient for performing the action. Conversely, as described by the second action of $\m{true}$, a thread in possession of the $\link{x}{y}$ resource may change its state from locked ($\lockedNode{x}{v}{y}$) to unlocked ($\unlockedNode{x}{v}{y}$) and return $\link{x}{y}$ to the shared state in return for $\locked{x}$. 

Consider a node at address $y$, with value $v$, successor $z$ and predecessor $x$. A thread may remove value $v$ from the set provided that both the node and its predecessor are locked by the thread; and the thread holds the value capability $\valueC{h}{v}$ as well as the update capabilities pertaining to both the node itself ($\modC{x}$) and its predecessor node ($\modC{y}$). This is captured by the first two actions associated with $\modC{x} * \valueC{h}{v}$ in $I(h, x)$. First, the next pointer of $x$ is redirected to $z$ (the first action of $\modC{x} * \valueC{h}{v}$); and then node $y$ is removed from the shared state (the second action of $\modC{y} * \valueC{h}{v}$).
Similarly, for a thread to insert the value $v$ in between node $x$ and node $y$, the predecessor node $x$ must be locked; and the thread should hold the value capability $\valueC{h}{v}$ as well as the update capability on $x$. This is reflected in the last action of $\modC{x} * \valueC{h}{v}$.

\subsection*{Reasoning about Set Operations}
%
\fig~\ref{fig:set-contains}-\ref{fig:set-locate} illustrate the implementation of the set operations along with an outline of the reasoning steps involved in establishing their correctness against their \colosl specification. In what follows, we give an account of reasoning about the set \li{add} operation as outlined in \fig~\ref{fig:set-add}; we demonstrate that given a set at address \li{h}, where value $v$ may or may not be present in the set, after a call to the \li{add(h, $v$)} operation the set will contain the value $v$. That is, the \li{add(h, $v$)} operation satisfies the following specification.
%
\[
\begin{array}{c}
	\color{blue}\left\{ \inSet{\li h}{v} \lor \outSet{\li h}{v} \right\}\\
	\li{add(h, $v$)}\\
	\color{blue}\left\{ \inSet{\li h}{v} \right\}
\end{array}
\]
%
The implementation of \li{add} proceeds by locating the index at which the new node is to be inserted through a call to the \li{locate(h, $v$)} operation. The \li{locate(h, $v$)} call traverses the list by hand-over-hand locking until it locates the addresses of node $p$ with value $v_p$ and node $c$ with value $v_c$ such that $v_p < v \leq v_c$. It then locks $p$ and claims its next pointer ($\link{\li p}{\li c}$) as allowed by $L(\li{p})$. Note that since the list contains the sentinel values $-\infty$ and $+\infty$ at either end, for any given value $v'$ the locate operation always finds $v_1$ and $v_2$ such that $v_1 < v' \leq v_2$. 
%The following depicts the state of the set upon return from the \li{(p, c):= locate(h, $v$)} call.
%%
%\[
%	\includegraphics[scale=0.24]{Sections/Examples/Images/coloslSet.pdf}
%\]
%%{\centering \includegraphics[scale=0.24]{Sections/Examples/Images/coloslSet.pdf}\\}
%%
If \li{c.value = $v$} and consequently the set contains value $v$, node $p$ is unlocked and the operating thread simply returns. On the other hand if $v_c > v$,  
%\noindent Since \colosl\ allows for \emph{dynamic} extension of the shared state, we do not need to account for capabilities associated with \emph{all} possible addresses. Instead, fresh capabilities are generated dynamically as needed. We demonstrate this by giving an outline of reasoning about the \li{add(}$v'$\li{)} method. 
%
a new node $z$ with value $v$ and successor $c$ is allocated and the shared state is extended by the resources associated with the new node $z$ as follows. First, the shared state is extended by $z$'s lock and in doing so the locking and unlocking capabilities pertaining to $z$ are generated on the fly. That is the $\cell{\li{z.lock}}{0}$ predicate is replaced by $\isLock{z}{1}$ as shown in the following derivation.
%
\small
\begin{align*}
	&\color{blue}
	\left\{
 	\begin{array}{@{} l @{}}
	 	\exsts{L_1, L_2, v_p, v_c, \pi} v_p < v < v_c \land \sorted{L_1 ++ \{v_p\} ++ v_c :: L_2}  \land\\
	 	\isLock{\!\li h}{\pi} * \valueT{\!\li h}{v}  		 	
		* \lsg{\!\li h}{\!\li p}{L_1}{\!\li h} \\		
	 	* \shared{\lockedNode{\!\li p}{v_p}{\!\li c}}{I(\!\li p)} 
	 	* \link{\!\li p}{\!\li c}
	 	* \lsg{\!\li c}{\li{null}}{v_c ::L_2}{\!\li h} 	
 	\end{array}
 	\right\}\\
%
%
	\stackrel{\text{(frame off)}}{}
	& \quad\color{blue}
	\left\{ \emp \right\}\\
%
%
	\stackrel{\text{alloc+assign}}{\semimplies}
	& \quad \color{blue}
	\left\{
		\cell{\!\li{z.value}}{v}	
		* \cell{\!\li{z.next}}{\!\li c}
		* \cell{\!\li{z.lock}}{0}
	\right\}\\
%
%
	\stackrel{\extendRule}{\semimplies}
	& \quad \color{blue}
	\left\{
		\cell{\!\li{z.value}}{v}	
		* \cell{\!\li{z.next}}{\li c}
		* \lockT{\!\li z} 
		* \shared{\cell{\!\li{z.lock}}{0} * \unlockT{\!\li z}}{L(\!\li z)}
	\right\}\\
%
%	
	%
%
	\stackrel{\textsf{isLock} \text{ def.}}{}
	& \quad \color{blue}
	\left\{
		\cell{\!\li{z.value}}{v}	
		* \cell{\!\li{z.next}}{\!\li c}
		* \isLock{\!\li z} {1}
	\right\}\\
%
%	
	\stackrel{\text{(frame on)}}{} 
	&\color{blue}
	\left\{
 	\begin{array}{@{} l @{}}
	 	\exsts{L_1, L_2, v_p, v_c, \pi} v_p < v < v_c \land \sorted{L_1 ++ \{v_p\} ++ v_c :: L_2}  \land\\
	 	\isLock{\!\li h}{\pi} * \valueT{\!\li h}{v}  		 	
		* \lsg{\!\li h}{\!\li p}{L_1}{\!\li h} \\		
	 	* \shared{\lockedNode{\!\li p}{v_p}{\!\li c}}{I(\!\li p)} 
	 	* \link{\!\li p}{\!\li c}
	 	* \lsg{\!\li c}{\li{null}}{v_c ::L_2}{\!\li h} 	\\
	 	* \cell{\!\li{z.value}}{v}	
		* \cell{\!\li{z.next}}{\!\li c}
	 	* \isLock{\!\li z} {1}
 	\end{array}
 	\right\}
%
%
\end{align*}\normalsize
%\[
%	\cell{\li{z.lock}}{0} \semimplies \lockT{\li z} * \shared{\cell{\li{z.lock}}{0} * \unlockT{\li z}}{L(\li z)} \implies \isLock{\li z}{1}
%\]
%
The shared state is then extended with the remaining resources of node $\li z$ and the relevant capabilities are also generated; subsequently, the subjective views of nodes $p$, $z$ and $c$ are combined through an application of the \mergeRule principle as demonstrated by the following derivation. 
%
\small
\begin{align*}
	& 
	\color{blue} 
	\left\{
 	\begin{array}{@{} l @{}}
	 	\exsts{L_1, L_2, v_p, v_c, \pi} v_p < v < v_c \land \sorted{L_1 ++ \{v_p\} ++ v_c :: L_2}  \land\\	 	
	 	\isLock{\!\li h}{\pi} * \valueT{\!\li h}{v}  		 	
		* \lsg{\!\li h}{\!\li p}{L_1}{\!\li h} \\		
	 	* \shared{\lockedNode{\!\li p}{v_p}{\!\li c}}{I(\!\li p)} 
	 	* \link{\!\li p}{\!\li c}
	 	* \lsg{\!\li c}{\li{null}}{v_c ::L_2}{\!\li h}\\	 	
	 	* \cell{\!\li{z.value}}{v} * \cell{\!\li{z.next}}{\!\li c} * \isLock{\li z}{1} 	
 	\end{array}
 	\right\}\\
%	 	
%	 	
	\stackrel{\text{(frame off)}}{} & 
	\quad\color{blue} 
	\left\{ 
		\link{\!\li p}{\!\li c} * \cell{\!\li{z.value}}{v} * \cell{\!\li{z.next}}{\!\li c} 
	\right\}\\
%	
%
	\stackrel{\textsf{link}\text{ def.}}{\implies} & 
	\quad\color{blue} 
	\left\{
		\cell{\!\li{p.next}}{\!\li c} * \modC{\!\li p} * \isLock{\!\li c}{1}  * \cell{\!\li{z.next}}{\!\li c} * \cell{\!\li{z.value}}{v}
	\right\}\\
%	
%
	\stackrel{\extendRule}{\semimplies} & 
	\quad\color{blue} 
	\left\{
		\cell{\!\li{p.next}}{\!\li c} * \modC{\!\li p} * 
		\shared{
			\begin{array}{@{} l @{}}
				\isLock{\!\li c}{1}  * \cell{\!\li{z.next}}{\!\li c} * \modC{\!\li z}\\
				* \cell{\!\li{z.value}}{v} * \nextC{\!\li z}{\li c}
			\end{array}
		}{I(\li z)}
	\right\}\\
%	
%	
	\stackrel{\textsf{U}\text{ def.}}{\implies} & 
	\quad\color{blue} 
	\left\{
		\cell{\!\li{p.next}}{\!\li c} * \modC{\!\li p} * 
		\shared{
			\begin{array}{@{} l @{}}
				\unlockedNode{\li z}{v}{\li c}
			\end{array}
		}{I(\li z)}
	\right\}\\
%	
%	
	\stackrel{\textsf{node}\text{ def.}}{\implies} & 
	\quad\color{blue} 
	\left\{
		\cell{\!\li{p.next}}{\!\li c} * \modC{\!\li p} * 
		\shared{
			\begin{array}{@{} l @{}}
				\node{\li z}{v}{\li c}
			\end{array}
		}{I(\li z)}
	\right\}\\
%	
%	
	\stackrel{\text{(frame on)}}{} 
	& \color{blue} 
	\left\{
 	\begin{array}{@{} l @{}}
	 	\exsts{L_1, L_2, v_p, v_c, \pi} v_p < v < v_c \land \sorted{L_1 ++ \{v_p\} ++ v_c :: L_2}  \land\\ 	
	 	\isLock{\!\li h}{\pi} * \valueT{\!\li h}{v}  		 	
		* \lsg{\!\li h}{\!\li p}{L_1}{\!\li h} \\		
	 	* \shared{\lockedNode{\!\li p}{v_p}{\!\li c}}{I(\!\li p)} 
	 	* \cell{\!\li{p.next}}{\!\li c} * \modC{\!\li p} 
	 	* \lsg{\!\li c}{\li{null}}{v_c ::L_2}{\!\li h}\\ 	
	 	* \shared{
			\begin{array}{@{} l @{}}
				\node{\li z}{v}{\li c}
			\end{array}
		}{I(\li z)}
		* \isLock{\li z}{1}
 	\end{array}
 	\right\}\\
%	
%	
	\stackrel{\textsf{lsg}\text{ def.}}{\implies} 
	& \color{blue} 
	\left\{
 	\begin{array}{@{} l @{}}
	 	\exsts{L_1, L_2, v_p, v_c, \pi, d} v_p < v < v_c \land \sorted{L_1 ++ \{v_p\} ++ v_c :: L_2}  \land\\ 	
	 	\isLock{\!\li h}{\pi} * \valueT{\!\li h}{v}  		 	
		* \lsg{\!\li h}{\!\li p}{L_1}{\!\li h} \\		
	 	* \shared{\lockedNode{\!\li p}{v_p}{\!\li c}}{I(\!\li p)} 
	 	* \cell{\!\li{p.next}}{\!\li c} * \modC{\!\li p} 
	 	* \shared{\lockedNode{\!\li c}{v_c}{\!\li d}}{I(\!\li c)} \\	 	
	 	* \lsg{\!\li d}{\li{null}}{L_2}{\!\li h}
	 	* \shared{
			\begin{array}{@{} l @{}}
				\node{\li z}{v}{\li c}
			\end{array}
		}{I(\li z)}
		* \isLock{\li z}{1}
 	\end{array}
 	\right\} 	\\
%
%	
	\stackrel{\text{\mergeRule}\times 2}{\implies} 
	& \color{blue} 
	\left\{
 	\begin{array}{@{} l @{}}
	 	\exsts{L_1, L_2, v_p, v_c, \pi, d} v_p < v < v_c \land \sorted{L_1 ++ \{v_p\} ++ v_c :: L_2}  \land\\ 	
	 	\isLock{\!\li h}{\pi} * \valueT{\!\li h}{v}  		 	
		* \lsg{\!\li h}{\!\li p}{L_1}{\!\li h} \\		
	 	* \shared{\lockedNode{\!\li p}{v_p}{\!\li c} * \node{\li z}{v}{\li c} * \lockedNode{\!\li c}{v_c}{\!\li d} * }{I(\!\li p) \cup I(\li z) \cup I(\!\li c)} \\
	 	* \cell{\!\li{p.next}}{\!\li c} * \modC{\!\li p} 
	 	* \lsg{\!\li d}{\li{null}}{L_2}{\!\li h}
		* \isLock{\li z}{1}
 	\end{array}
 	\right\} 	
%
% 
\end{align*}	
\normalsize
%
At this point, since the locking thread holds the next pointer of $p$ in its local state ($\cell{\li{p.next}}{\li c}$), it modifies it to point to the new node $z$; and through the action of $\modC{\li p} * \valueC{\li h}{v}$, the $\lockedNode{\li p}{v_p}{\li c}$ predicate is updated as $\lockedNode{\li p}{v_p}{\li z}$. Recall that the $\lockedNode{\li p}{v_p}{\li c}$ predicate states that the node at \li{p} is locked, it holds value $v_p$ and prior to locking its successor was the node at address \li{c}. As such, since the value of the new node $\li{z}$ lies between that of \li{p} and \li{c} ($v_p < v < v_c$), this action allows us to insert \li{z} in between \li{p} and \li{c} provided that \li{z} itself points to \li{c}. This is captured by the following derivation where ($\dagger$) denotes the application of the action associated with the $\modC{\li p} * \valueC{\li h}{v}$ capability in $I(\li p)$.
%That is $\shared{\lockedNode{\li p}{v_p}{\li c} * \node{\li z}{v}{\li c} * \node{\li c}{v_c}{d}}{I(\li p) \cup I(\li z) \cup I(\li c)}$ is updated as $\shared{\lockedNode{\li p}{v_p}{\li z} * \node{\li z}{v}{\li c} * \node{\li c}{v_c}{d}}{I(\li p) \cup I(\li z) \cup I(\li c)}$.
%
\small
\begin{align*}
	& \color{blue} 
	\left\{
 	\begin{array}{@{} l @{}}
	 	\exsts{L_1, L_2, v_p, v_c, \pi, d} v_p < v < v_c \land \sorted{L_1 ++ \{v_p\} ++ v_c :: L_2}  \land\\ 	
	 	\isLock{\!\li h}{\pi} * \valueT{\!\li h}{v}  		 	
		* \lsg{\!\li h}{\!\li p}{L_1}{\!\li h} \\		
	 	* \shared{\lockedNode{\!\li p}{v_p}{\!\li c} * \node{\li z}{v}{\li c} * \lockedNode{\!\li c}{v_c}{\!\li d} * }{I(\!\li p) \cup I(\li z) \cup I(\!\li c)} \\
	 	* \cell{\!\li{p.next}}{\!\li c} * \modC{\!\li p} 
	 	* \lsg{\!\li d}{\li{null}}{L_2}{\!\li h}
		* \isLock{\li z}{1}
 	\end{array}
 	\right\} 	\\
%
% 
	\stackrel{(\!\li{p.next:= z})}{\implies} 
	& \color{blue} 
	\left\{
 	\begin{array}{@{} l @{}}
	 	\exsts{L_1, L_2, v_p, v_c, \pi, d} v_p < v < v_c \land \sorted{L_1 ++ \{v_p\} ++ v_c :: L_2}  \land\\ 	
	 	\isLock{\!\li h}{\pi} * \valueT{\!\li h}{v}  		 	
		* \lsg{\!\li h}{\!\li p}{L_1}{\!\li h} \\		
	 	* \shared{\lockedNode{\!\li p}{v_p}{\!\li c} * \node{\li z}{v}{\li c} * \lockedNode{\!\li c}{v_c}{\!\li d} * }{I(\!\li p) \cup I(\li z) \cup I(\!\li c)} \\
	 	* \cell{\!\li{p.next}}{\!\li z} * \modC{\!\li p} 
	 	* \lsg{\!\li d}{\li{null}}{L_2}{\!\li h}
		* \isLock{\li z}{1}
 	\end{array}
 	\right\} 	\\
%
% 
	\stackrel{\textsf{link}\text{ def.}}{\implies} 
	& \color{blue} 
	\left\{
 	\begin{array}{@{} l @{}}
	 	\exsts{L_1, L_2, v_p, v_c, \pi, d} v_p < v < v_c \land \sorted{L_1 ++ \{v_p\} ++ v_c :: L_2}  \land\\ 	
	 	\isLock{\!\li h}{\pi} * \valueT{\!\li h}{v}  		 	
		* \lsg{\!\li h}{\!\li p}{L_1}{\!\li h} \\		
	 	* \shared{\lockedNode{\!\li p}{v_p}{\!\li c} * \node{\li z}{v}{\li c} * \lockedNode{\!\li c}{v_c}{\!\li d} * }{I(\!\li p) \cup I(\li z) \cup I(\!\li c)} 
	 	* \link{\li p}{\li z}
 	\end{array}
 	\right\} 	\\
%
% 
	\stackrel{(\dagger)}{\semimplies} 
	& \color{blue} 
	\left\{
 	\begin{array}{@{} l @{}}
	 	\exsts{L_1, L_2, v_p, v_c, \pi, d} v_p < v < v_c \land \sorted{L_1 ++ \{v_p\} ++ v_c :: L_2}  \land\\ 	
	 	\isLock{\!\li h}{\pi} * \valueT{\!\li h}{v}  		 	
		* \lsg{\!\li h}{\!\li p}{L_1}{\!\li h} \\		
	 	* \shared{\lockedNode{\!\li p}{v_p}{\!\li z} * \node{\li z}{v}{\li c} * \lockedNode{\!\li c}{v_c}{\!\li d} * }{I(\!\li p) \cup I(\li z) \cup I(\!\li c)} 
	 	* \link{\li p}{\li z}
 	\end{array}
 	\right\} 	
%
% 
\end{align*}
\normalsize
%
All that remains is to unlock the node at \li{p}; to do this, first the state of the node is changed from locked ($\lockedNode{\li{p}}{v_p}{\li{z}}$) to unlocked ($\unlockedNode{\li{p}}{v_p}{\li{z}}$) by applying the action of $\modC{\li p}$ capability. Subsequently, through several applications of \copyRule, \forgetRule and \shiftRule principles, the subjective views of \li{p}, \li{z} and \li{c} nodes are recovered as follows where ($\dagger$) denotes the application of the action associated with the $\modC{\li p}$ capability in $I(\li p)$.
%
\small
\begin{align*}
	& \color{blue} 
		\left\{
	 	\begin{array}{@{} l @{}}
		 	\exsts{L_1, L_2, v_p, v_c, \pi, d} v_p < v < v_c \land \sorted{L_1 ++ \{v_p\} ++ v_c :: L_2}  \land\\ 	
		 	\isLock{\!\li h}{\pi} * \valueT{\!\li h}{v}  		 	
			* \lsg{\!\li h}{\!\li p}{L_1}{\!\li h} \\		
		 	* \shared{\lockedNode{\!\li p}{v_p}{\!\li z} * \node{\li z}{v}{\li c} * \lockedNode{\!\li c}{v_c}{\!\li d} * }{I(\!\li p) \cup I(\li z) \cup I(\!\li c)} 
		 	* \link{\li p}{\li z}
	 	\end{array}
	 	\right\} 	\\
%
% 
	\stackrel{\text{(frame off)}}{}
	& \quad \color{blue} 
		\left\{ 
			\shared{\lockedNode{\!\li p}{v_p}{\!\li c} * \node{\!\li z}{v}{\!\li c} * \node{\!\li c}{v_c}{d}}{I(\!\li p) \cup I(\!\li z) \cup I(\!\li c)}
			* \link{\!\li p}{\!\li z}
		\right\}\\
%	
%	
	\stackrel{(\dagger)}{\semimplies}
	& \quad \color{blue} 
		\left\{
			\shared{\unlockedNode{\!\li p}{v_p}{\!\li z} * \node{\!\li z}{v}{\!\li c} * \node{\!\li c}{v_c}{d}}{I(\!\li p) \cup I(\!\li z) \cup I(\!\li c)}
			* \locked{\!\li p}
		\right\}\\
%
%	
	\stackrel{\textsf{node}\text{ def.}}{\implies} 
	& \quad \color{blue} 
		\left\{
			\shared{\node{\!\li p}{v_p}{\!\li z} * \node{\!\li z}{v}{\!\li c} * \node{\!\li c}{v_c}{d}}{I(\!\li p) \cup I(\!\li z) \cup I(\!\li c)} 
			\!\!* \locked{\!\li p}
		\right\} \vspace{7pt}\\
%
%	
	\stackrel{\copyRule \times 2}{\implies} 
	& \quad \color{blue} 
	\left\{
	\begin{array}{@{} l @{}}
		\shared{\node{\!\li p}{v_p}{\!\li z} * \node{\!\li z}{v}{\!\li c} * \node{\!\li c}{v_c}{d}}{I(\!\li p) \cup I(\!\li z) \cup I(\!\li c)}\\
		* \shared{\node{\!\li p}{v_p}{\!\li z} * \node{\!\li z}{v}{\!\li c} * \node{\!\li c}{v_c}{d}}{I(\!\li p) \cup I(\!\li z) \cup I(\!\li c)}\\
		* \shared{\node{\!\li p}{v_p}{\!\li z} * \node{\!\li z}{v}{\!\li c} * \node{\!\li c}{v_c}{d}}{I(\!\li p) \cup I(\!\li z) \cup I(\!\li c)}\\
		* \locked{\!\li p}
	\end{array}
	\right\} \vspace{7pt}\\
%	
%	
	\stackrel{\forgetRule \times 3}{\implies} 
	& \quad \color{blue} 
	\left\{
	\begin{array}{@{} l @{}}
		\shared{\node{\!\li p}{v_p}{\!\li z}}{I(\!\li p) \cup I(\!\li z) \cup I(\!\li c)}
		* \shared{\node{\!\li z}{v}{\!\li c}}{I(\!\li p) \cup I(\!\li z) \cup I(\!\li c)}\\
		* \shared{\node{\!\li c}{v_c}{d}}{I(\!\li p) \cup I(\!\li z) \cup I(\!\li c)} * \locked{\!\li p}
	\end{array}
	\right\} \vspace{7pt}\\
%	
%	
	\stackrel{\shiftRule \times 3}{\semimplies} 
	& \quad \color{blue} 
	\left\{\shared{\node{\!\li p}{v_p}{\!\li z}}{I(\!\li p)}
	\!\!* \shared{\node{\!\li z}{v}{\!\li c}}{I(\!\li z)}
	\!\!* \shared{\node{\!\li c}{v_c}{d}}{I(\!\li c)}
	\!\!* \locked{\!\li p}\right\}\\
%
%
	\stackrel{\text{(frame on)}}{}	
	& \color{blue} 
		\left\{
	 	\begin{array}{@{} l @{}}
		 	\exsts{L_1, L_2, v_p, v_c, \pi, d} v_p < v < v_c \land \sorted{L_1 ++ \{v_p\} ++ v_c :: L_2}  \land\\ 	
		 	\isLock{\!\li h}{\pi} * \valueT{\!\li h}{v}  		 	
			* \lsg{\!\li h}{\!\li p}{L_1}{\!\li h} \\		
		 	* \shared{\node{\!\li p}{v_p}{\!\li z}}{I(\!\li p)}
			\!\!* \shared{\node{\!\li z}{v}{\!\li c}}{I(\!\li z)}
			\!\!* \shared{\node{\!\li c}{v_c}{d}}{I(\!\li c)}
			\!\!* \locked{\!\li p}
	 	\end{array}
	 	\right\} 	\\
%
% 
	\stackrel{\textsf{lsg}\text{ def.}}{\implies}	
	& \color{blue} 
		\left\{
	 	\begin{array}{@{} l @{}}
		 	\exsts{L, \pi} v \in L \land \sorted{L}  \land
		 	\isLock{\!\li h}{\pi} * \valueT{\!\li h}{v}  		 	
			* \lsg{\!\li h}{\!\li{null}}{L}{\!\li h} \\		
			* \locked{\!\li p}
	 	\end{array}
	 	\right\} 	\\
%
% 
	\stackrel{\textsf{in}\text{ def.}}{\implies}	
	& \color{blue} 
		\left\{
	 	\begin{array}{@{} l @{}}
		 	\inSet{\!\li h}{v}	
			* \locked{\!\li p}
	 	\end{array}
	 	\right\} 	\\
%
% 
\end{align*}
\normalsize
%
Finally, \li{p}'s lock is released by applying the action of $\unlockC{\li p}$ capability in $L(\li p)$ and we obtain $\inSet{\li h}{v}$ as required. This is demonstrated in the following derivation where ($\dagger$) denotes applying the action of $\unlockC{\li p}$.
%
\small
\begin{align*}
	& \color{blue} 
		\left\{
	 	\begin{array}{@{} l @{}}
		 	\inSet{\!\li h}{v}	
			* \locked{\!\li p}
	 	\end{array}
	 	\right\} 	\\
%
% 
	\stackrel{\textsf{locked}\text{ def.}}{\implies}	
	& \color{blue} 
		\left\{
	 	\begin{array}{@{} l @{}}
		 	\inSet{\!\li h}{v}	
			* \unlockC{\li p} * \shared{\cell{\li{p.lock}}{1}}{L(\li p)}
	 	\end{array}
	 	\right\} 	\\
%
% 
	\stackrel{(\dagger)}{\semimplies}	
	& \color{blue} 
		\left\{
	 	\begin{array}{@{} l @{}}
		 	\inSet{\!\li h}{v}	
			* \shared{\cell{\li{p.lock}}{0} * \unlockC{\li p}}{L(\li p)}
	 	\end{array}
	 	\right\} 	\\
%
% 
	\stackrel{}{\implies}	
	& \color{blue} 
		\left\{
	 	\begin{array}{@{} l @{}}
		 	\inSet{\!\li h}{v}	
	 	\end{array}
	 	\right\} 
%
% 
\end{align*}
\normalsize
%
Note that the final implication follows directly from the semantics of \colosl. That is, for all  $P \in \Assertions$ and $I \in \IAssertions$, $\shared{P}{I} \implies \emp$ is valid.\\
%
%
%We can reason about the \li{remove} operation in a similar fashion. Note that the dynamic extension afforded by the \extendRule\ principle allows us to generate new capabilities only when needed and thus gives way to a conciser specification. 
%%
%Moreover, rather than having a distinct capability to modify the element at address $x$, before address $y$, for each possible successor address $y$ (as with $[\text{U}(x, y, v)]$ in CAP), we appeal to a single capability of the form $\nextC{x}{y}$ whereby the capability is modified accordingly to record the changes to the successor address as demonstrated above. 
%%
%Lastly, using the reasoning principles of \mergeRule, \forgetRule, \shiftRule\ and \copyRule, we can grow and shrink our subjective views as needed and consequently at any one point we only view the relevant parts of the shared state. 
%%
%The full \colosl\ specification of the set module as well as the specification proofs are provided in~\cite{colosl-tr14}.
%%in that it limits the modifications on node $x$ with respect to its successor $y$. For instance, in order to remove the node at address $y$, a thread proceeds by first acquiring the locks on both $x$ and $y$ nodes. It then claims $x$'s next pointer and redirects it to $z$. In order to ensure that the thread indeed redirects $x$ to $z$ and not any arbitrary address, the we track the current successor of $x$ with $\nextC{x}{y}$ and consequently, in our reasoning, when the thread attempts to unlock $x$, its new successor is compared against the old one $y$ and only when  t of $y$  nodes at $x$ and $y$ when node $x$ is locked and the its next pointer has been temporarily claimed by the locking thread, 
%
\begin{figure}
%
\hrule
\[
\small
\begin{array}{@{} l @{}}
	\color{blue} //
	\left\{ \inSet{\li h}{v} \lor \outSet{\li h}{v} \right \}\\
	
	\command{contains(h, $v$)}\{\\
	\begin{array}{l}
		
		
		\command{(p, c):= locate(h, $v$); }\\
		
		\color{blue}//
		\left\{
	 	\begin{array}{@{} l @{}}
		 	\exsts{L_1, L_2, v_p, v_c, \pi} v_p < v \leq v_c \land \sorted{L_1 {++}  \{v_p\} {++}   v_c::L_2}  \land\\
		 	\isLock{\li h}{\pi} * \valueT{\li h}{v}
			*\lsg{\li h}{\li p}{L_1}{\li h} \\
			
		 	* \shared{\lockedNode{\li p}{v_p}{\li c}}{I(\li p)} 
		 	* \link{\li p}{\li c}
		 	* \lsg{\li c}{\li{null}}{v_c ::L_2}{\li h}
	 	
	 	\end{array}
	 	\right\}		\\
		
		\command{b:= (c.value == $v$);}\\
		
		\color{blue}//
		\left\{
	 	\begin{array}{@{} l @{}}
		 	\exsts{L_1, L_2, v_p, v_c, \pi} v_p < v \land \sorted{L_1 {++}  \{v_p\} {++}   v_c::L_2}  \land\\
		 	\isLock{\li h}{\pi} * \valueT{\li h}{v}
			*\lsg{\li h}{\li p}{L_1}{\li h} \\
			
		 	* \shared{\lockedNode{\li p}{v_p}{\li c}}{I(\li p)} 
		 	* \link{\li p}{\li c}
		 	* \lsg{\li c}{\li{null}}{v_c ::L_2}{\li h}\\
		 	
		 	* \left((v_c > v \land \li b = false) \lor (v_c = v \land b = true)\right)
	 	
	 	\end{array}
	 	\right\}\\


		 	
		\command{unlock(p);} \\
		

		\color{blue}//
		\left\{
		 	\left((\outSet{\li h}{v} \land \li b = false) \lor (\inSet{\li h}{v} \land b = true)\right)
	 	\right\}\\
		 	
		
	\end{array}\\
	
	\}\\
	
	\color{blue}//
	\left\{
	 	\left((\outSet{\li h}{v} \land \li b = false) \lor (\inSet{\li h}{v} \land b = true)\right)
 	\right\}\\
	
	
\end{array}
\]
%
%
\hrule
\caption{Implementation of the set \li{contains} operation.}
\label{fig:set-contains}
\end{figure}
%
%
%
\begin{figure}
%
\hrule
\[
\small
\begin{array}{@{} l @{}}
	\color{blue} //
	\left\{ \inSet{\!\li h}{v} \lor \outSet{\!\li h}{v} \right \}\\
	
	\command{add(h, $v$)}\{\\
	\begin{array}{l}
		
%		\color{blue} //
%		\left\{ \ownSet{h}{v} \right\}\\
		
		\command{(p, c):= locate(h, $v$); }\\
		
		\color{blue} //
		\left\{
	 	\begin{array}{@{} l @{}}
		 	\exsts{L_1, L_2, v_p, v_c, \pi} v_p < v\leq v_c \land \sorted{L_1 ++ \{v_p\} ++ v_c :: L_2}  \land\\
		 	\isLock{\!\li h}{\pi} * \valueT{\!\li h}{v}  		 	
			* \lsg{\!\li h}{\!\li p}{L_1}{\!\li h} \\
			
		 	* \shared{\lockedNode{\!\li p}{v_p}{\!\li c}}{I(\!\li p)} 
		 	* \link{\!\li p}{\!\li c}
		 	* \lsg{\!\li c}{\li{null}}{v_c ::L_2}{\!\li h}
	 	
	 	\end{array}
	 	\right\}\\
	 	
	 	\command{if (c.value } > \command{v) then $\{$ }\\
	 	\begin{array}{l}

%		 	\command{lock(c.lock) ; }\\
		 	
		 	\color{blue} //
			\left\{
		 	\begin{array}{@{} l @{}}
			 	\exsts{L_1, L_2, v_p, v_c, \pi} v_p < v < v_c \land \sorted{L_1 ++ \{v_p\} ++ v_c :: L_2}  \land\\
			 	\isLock{\!\li h}{\pi} * \valueT{\!\li h}{v}  		 	
				* \lsg{\!\li h}{\!\li p}{L_1}{\!\li h} \\
				
			 	* \shared{\lockedNode{\!\li p}{v_p}{\!\li c}}{I(\!\li p)} 
			 	* \link{\!\li p}{\!\li c}
			 	* \lsg{\!\li c}{\li{null}}{v_c ::L_2}{\!\li h}
		 	
		 	\end{array}
		 	\right\}\\
	 	
		 	
		 	
%		 	\command{z:= c.next ;}\\
%		 	\command{$\text{[}$p.next$\text{]}$:= z ;}\\
%		 	
%		 	\color{blue} //
%		 	\left\{
%		 	\begin{array}{@{} l @{}}
%			 	\exsts{L_1, L_2, v_p, d} \sorted{L_1 ++ \{v_p\} ++ L_2}  \land\\
%			 	\lockT{h}[\pi] * \valueT{h}{v} * 
%				\lsg{h}{p}{L_1}{h} \\
%				
%			 	* \shared{\lockedNode{p}{v_p}{c}}{I(p)} 
%			 	* \modC{p} * \cell{p.next}{d} * \isLock{c}{1}\\
%			 	
%			 	* \shared{\lockedNode{c}{v}{d}}{I(c)} 
%			 	* \modC{c} * \cell{c.next}{d} * \isLock{d}{1}
%			 	* \lsg{d}{\li{null}}{L_2}{h}
%		 	
%		 	\end{array}
%		 	\right\}\\
		 	
		 	
		 	\begin{array}{l}
		 		\color{blue} // \{\emp\}\\
		 	
		 		\command{z:= alloc(3)};\\
		 		\command{z.value:= $v$; z.next:= c; z.lock:= $0$;}\\
		 	
		 		\color{blue}// 
			 	\left\{
			 		\cell{\!\li{z.value}}{v} * \cell{\!\li{z.next}}{\!\li c} * \cell{\!\li{z.lock}}{0} 
			 	\right\}\\
		 	
		 		\color{darkgreen} /* \text{ Use the \extendRule principle. } */\\
		 	
		 	
			 	\color{blue}// 
			 	\left\{
			 		\cell{\!\li{z.value}}{v} * \cell{\!\li{z.next}}{\!\li c} * \isLock{\li z}{1}
			 	\right\}
		 	
		 	\end{array}\\
		 	
		 	\color{blue} //
			\left\{
		 	\begin{array}{@{} l @{}}
			 	\exsts{L_1, L_2, v_p, v_c, \pi} v_p < v < v_c \land \sorted{L_1 ++ \{v_p\} ++ v_c :: L_2}  \land\\
			 	
			 	\isLock{\!\li h}{\pi} * \valueT{\!\li h}{v}  		 	
				* \lsg{\!\li h}{\!\li p}{L_1}{\!\li h} \\
				
			 	* \shared{\lockedNode{\!\li p}{v_p}{\!\li c}}{I(\!\li p)} 
			 	* \link{\!\li p}{\!\li c}
			 	* \lsg{\!\li c}{\li{null}}{v_c ::L_2}{\!\li h}\\
			 	
			 	* \cell{\!\li{z.value}}{v} * \cell{\!\li{z.next}}{\!\li c} * \isLock{\li z}{1}
		 	
		 	\end{array}
		 	\right\}\\
		 	
		 	\color{darkgreen} /* \text{ Use the \extendRule principle followed by \mergeRule principle. } */\\
		 	
		 	\color{blue} //
		 	\left\{
		 	\begin{array}{@{} l @{}}
			 	\exsts{L_1, L_2, v_p, v_c, \pi, d} v_p < v < v_c \land \sorted{L_1 ++ \{v_p\} ++ v_c :: L_2}  \land\\
			 	
			 	\isLock{\!\li h}{\pi} * \valueT{\!\li h}{v}  		 	
				* \lsg{\!\li h}{\!\li p}{L_1}{\!\li h} \\
				
			 	* \shared{\lockedNode{\!\li p}{v_p}{\!\li c} * \node{\!\li z}{v}{\!\li c} * \node{\!\li c}{v_c}{\!\li d}}{I(\!\li p) \cup I(\!\li z) \cup I(\!\li c)} \\
			 	
			 	
			 	* \modC{\!\li p} * \cell{\!\li{p.next}}{\!\li c} 			 	
			 	* \lsg{\!\li d}{\!\li{null}}{L_2}{\!\li h}
			 	* \isLock{\!\li z}{1}
		 	
		 	\end{array}
		 	\right\}\\
		 	
		 	\command{p.next:= z} \\
		 	
		 	
		 	\color{blue} //
		 	\left\{
		 	\begin{array}{@{} l @{}}
			 	\exsts{L_1, L_2, v_p, v_c, \pi, d} v_p < v < v_c \land \sorted{L_1 ++ \{v_p\} ++ v_c :: L_2}  \land\\
			 	
			 	\isLock{\!\li h}{\pi} * \valueT{\!\li h}{v}  		 	
				* \lsg{\!\li h}{\!\li p}{L_1}{\!\li h} \\
				
			 	* \shared{\lockedNode{\!\li p}{v_p}{\!\li c} * \node{\!\li z}{v}{\!\li c} * \node{\!\li c}{v_c}{d}}{I(\!\li p) \cup I(\!\li z) \cup I(\!\li c)} \\
			 	
			 	* \link{\!\li p}{\!\li z}{} 
			 	* \lsg{d}{\!\li{null}}{L_2}{\!\li h}
		 	
		 	\end{array}
		 	\right\}\\
		 	
		 	\color{darkgreen} /* \text{ Apply the actions of } \valueT{h}{v} * \modT{p} \text{ and } \modT{p} \text{ in } I(p) \text{ in order. }*/\\
		 	\color{darkgreen} /* \text{ Use the \copyRule, \forgetRule and \shiftRule principles in order. } */\\
		 	
		 	
		 	\color{blue} //
		 	\left\{
		 	\begin{array}{@{} l @{}}
			 	\inSet{\li h}{v} * \locked{\li p}
		 	\end{array}
		 	\right\}
		 	
		 	
		\end{array}\\
		
		\}\\
		
		\color{blue} //
	 	\left\{
	 	\begin{array}{@{} l @{}}
		 	\inSet{\li h}{v} * \locked{\li p}
	 	\end{array}
	 	\right\}\\
		
		\command{unlock(p) ;}\\
		 	
		 	
	 	\color{blue} //
	 	\left\{
	 	\begin{array}{@{} l @{}}
		 	\inSet{\li h}{v}
	 	
	 	\end{array}
	 	\right\}\\
		 	
		
	\end{array}\\
	
	\}\\
	
	\color{blue} //
	\left\{ \inSet{\li h}{v} \right\}
	
	
\end{array}
\]
%
%
\hrule
\caption{Implementation of the set \li{add} operation.}
\label{fig:set-add}
\end{figure}
%
%
\begin{figure}
%
\hrule
\[
\small
\begin{array}{@{} l @{}}
	\color{blue} //
	\left\{ \inSet{\li h}{v} \lor \outSet{\li h}{v} \right \}\\
	
	\command{remove(h, $v$)}\{\\
	\begin{array}{l}
		
%		\color{blue} //
%		\left\{ \ownSet{h}{v} \right\}\\
		
		\command{(p, c):= locate(h, $v$); }\\
		
		\color{blue} //
		\left\{
	 	\begin{array}{@{} l @{}}
		 	\exsts{L_1, L_2, v_p, v_c, \pi} v_p < v\leq v_c \land \sorted{L_1 ++ \{v_p\} ++ v_c :: L_2}  \land\\
		 	\isLock{\li h}{\pi} * \valueT{\li h}{v}  		 	
			* \lsg{\li h}{\li p}{L_1}{\li h} \\
			
		 	* \shared{\lockedNode{\li p}{v_p}{\li c}}{I(\li p)} 
		 	* \link{\li p}{\li c}
		 	* \lsg{\li c}{\li{null}}{v_c ::L_2}{\li h}
	 	
	 	\end{array}
	 	\right\}\\
	 	
	 	\command{if (c.value == $v$) then $\{$ }\\
	 	\begin{array}{l}

		 	\command{lock(c.lock) ; }\\
		 	
		 	
		 	\color{blue} //
		 	\left\{
		 	\begin{array}{@{} l @{}}
			 	\exsts{L_1, L_2, v_p < v, d, \pi} v \not \in L_2 \land \sorted{L_1 ++ \{v_p\} ++ v :: L_2}  \land\\
			 	
			 	\lockT{\li h}[\pi] * \valueT{\li h}{v} * 
				\lsg{\li h}{\li p}{L_1}{\li h} \\
				
			 	* \shared{\lockedNode{\!\li p}{v_p}{\!\li c}}{I(\!\li p)} 
			 	* \link{\!\li p}{\!\li c}
			 	
			 	* \shared{\lockedNode{\!\li c}{v}{d}}{I(\!\li c)} 
			 	* \link{\!\li c}{d}
			 	* \lsg{d}{\!\li{null}}{L_2}{\!\li h}
		 	
		 	\end{array}
		 	\right\}\\
		 	
		 	\command{z:= c.next ;}\quad
		 	\command{$\text{[}$p.next$\text{]}$:= z ;}\\
		 	
		 	\color{blue} //
		 	\left\{
		 	\begin{array}{@{} l @{}}
			 	\exsts{L_1, L_2, v_p< v}v \not \in L_2 \land  \sorted{L_1 ++ \{v_p\} ++ L_2}  \land\\
			 	\lockT{\li h}[\pi] * \valueT{\li h}{v} * 
				\lsg{\li h}{\li p}{L_1}{\li h} \\
				
			 	* \shared{\lockedNode{\!\li p}{v_p}{\!\li c}}{I(\!\li p)} 
			 	* \modC{\!\li p} * \cell{\!\li{p.next}}{\!\li z} * \isLock{\li c}{1}\\
			 	
			 	* \shared{\lockedNode{\!\li c}{v}{\!\li z}}{I(\!\li c)} 
			 	* \modC{\!\li c} * \cell{\!\li{c.next}}{\!\li z} * \isLock{\!\li z}{1}
			 	* \lsg{\!\li z}{\!\li{null}}{L_2}{\!\li h}
		 	
		 	\end{array}
		 	\right\}\\
		 	
		 	
		 	\color{darkgreen} /* \text{ Use the \mergeRule principle. } */\\
		 	
		 	\color{blue} //
		 	\left\{
		 	\begin{array}{@{} l @{}}
			 	\exsts{L_1, L_2, v_p < v, \pi} v \not \in L_2 \land \sorted{L_1 ++ \{v_p\} ++ L_2}  \land\\
			 	
			 	\lockT{\li h}[\pi] * \valueT{\li h}{v}  
				* \lsg{\li h}{\li p}{L_1}{\li h} \\
				
			 	* \shared{\lockedNode{\!\li p}{v_p}{\!\li c} * \lockedNode{\!\li c}{v}{\!\li z}}{I(\!\li p) \cup I(\!\li c)} 
			 	* \modC{\!\li p} * \cell{\!\li{p.next}}{\!\li z} * \isLock{\!\li c}{1}\\
			 	
			 	* \modC{\li c} * \cell{\li{c.next}}{\!\li z} * \isLock{\!\li z}{1}
			 	* \lsg{\!\li z}{\li{null}}{L_2}{\li h}
		 	
		 	\end{array}
		 	\right\}\\
		 	
		 	\color{darkgreen} /* \text{ Apply the actions of } \valueT{h}{v} * \modT{p} \text{ cpability in } I(p) \text{ in order. } */\\
		 	
		 	\color{blue} //
		 	\left\{
		 	\begin{array}{@{} l @{}}
			 	\exsts{L_1, L_2, v_p < v, \pi} v \not \in L_2 \land \sorted{L_1 ++ \{v_p\} ++ L_2}  \land\\
			 	
			 	\isLock{\!\li h}{\pi} * \valueT{\!\li h}{v}
				* \lsg{\!\li h}{\!\li p}{L_1}{\!\li h} \\
				
				
			 	* \shared{\lockedNode{\!\li p}{v_p}{\!\li z}}{I(\!\li p) \cup I(\!\li c)} 
			 	* \modC{\!\li p} * \cell{\!\li{p.next}}{\!\li z} * \isLock{\!\li c}{1}\\
			 	
			 	* \lockedNode{\!\li c}{v}{\!\li z}
			 	* \modC{\!\li c} * \cell{\!\li{c.next}}{\!\li z} * \isLock{\!\li z}{1}
			 	* \lsg{\!\li z}{\!\li{null}}{L_2}{\!\li h}
		 	
		 	\end{array}
		 	\right\}\\
		 	
		 	
		 	
		 	\color{darkgreen} /* \text{Use the \shiftRule principle; apply the action of } \lockT{c} * \unlockT{c} \text{ in } L(c) \text{.} */\\
		 	
		 	
		 		 	
		 	\color{blue} //
		 	\left\{
		 	\begin{array}{@{} l @{}}
			 	\exsts{L_1, L_2, v_p < v, \pi} v \not \in L_2 \land \sorted{L_1 ++ \{v_p\} ++ L_2}  \land \isLock{\!\li h}{\pi}\\
			 	
			 	* \valueT{\!\li h}{v} * 
				\lsg{\!\li h}{\!\li p}{L_1}{\!\li h} 
			 	* \shared{\lockedNode{\!\li p}{v_p}{\!\li z}}{I(\!\li p)} 
			 	* \link{\li p}{\li z}
			 	* \lsg{\!\li z}{\li{null}}{L_2}{\!\li h}\\
			 	
			 	* \cell{\!\li{c.value}}{v} * \cell{\!\li{c.next}}{\!\li z}  *\cell{\!\li{c.lock}}{1} 
		 	
		 	\end{array}
		 	\right\}\\
		 	
		 	
		 	
		 	
		 	
%		 	\color{blue} //
%		 	\left\{
%		 	\begin{array}{@{} l @{}}
%			 	\outSet{\!\li h}{v}
%				* \cell{\!\li{c.value}}{v} * \cell{\!\li{c.next}}{\!\li z}  *\cell{\!\li{c.lock}}{1} 
%		 	
%		 	\end{array}
%		 	\right\}\\
		 	
		 	\command{dispose(c, 3) ; }\\
		 	
		 	
		 	\color{blue} //
		 	\left\{
		 	\begin{array}{@{} l @{}}
			 	\exsts{L_1, L_2, v_p  < v, \pi} v \not \in L_2 \land \sorted{L_1 ++ \{v_p\} ++ L_2}  \land \isLock{\!\li h}{\pi}\\
			 	
			 	* \valueT{\!\li h}{v}
				* \lsg{\!\li h}{\!\li p}{L_1}{\!\li h} 
			 	* \shared{\lockedNode{\!\li p}{v_p}{\!\li z}}{I(\!\li p)} 
			 	* \link{\li p}{\li z}
			 	* \lsg{\!\li z}{\li{null}}{L_2}{\!\li h}
		 	
		 	\end{array}
		 	\right\}
		 	
		 	
		\end{array}\\
		
		\}\\
		
		\color{blue} //
	 	\left\{
	 	\begin{array}{@{} l @{}}
		 	\exsts{L_1, L_2, v_p  < v, \pi, n} v \not \in L_2 \land \sorted{L_1 ++ \{v_p\} ++ L_2}  \land \isLock{\!\li h}{\pi}\\
		 	
		 	* \valueT{\!\li h}{v}
			* \lsg{\!\li h}{\!\li p}{L_1}{\!\li h} 
		 	* \shared{\lockedNode{\!\li p}{v_p}{\!\li n}}{I(\!\li p)} 
		 	* \link{\li p}{\li n}
		 	* \lsg{\!\li n}{\li{null}}{L_2}{\!\li h}
	 	
	 	\end{array}
	 	\right\}\\
	 	
	 	
	 	\command{unlock(p);}\quad
	 	
	 	\color{blue} //
	 	\left\{
	 	\begin{array}{@{} l @{}}
		 	\outSet{\!\li h}{v}
	 	\end{array}
	 	\right\}
		
	\end{array}\\
	
	\}
	
	\color{blue} //
	\left\{ \outSet{\!\li h}{v} \right\}
	
	
\end{array}
\]
%
%
\hrule
\caption{Implementation of the set \li{remove} operation.}
\label{fig:set-remove}
\end{figure}
%
%
%
\begin{figure}
\hrule
\[
\small
\begin{array}{l}
	
	\color{blue} //
	\left\{\inSet{\vr h}{v} |/ \outSet{\vr h}{v}\right\}\\
	
	\command{(p, c):= locate(h, $v$) $\{$}\\
	\begin{array}{l}
		\command{p:= h}\\
		\command{lock(p.lock)}\\
		\command{c:= p.next}\\
		
		\color{blue}//
		\left\{
		\begin{array}{@{} l @{}}
			\exsts{L, \pi} \sorted{-\infty :: L  {++}   \{+\infty\}} \land\\
			 \isLock{\vr h}{\pi} * \valueT{\vr h}{v}\\
			 
			 * \shared{\lockedNode{\vr p}{-\infty}{\vr c}}{I(\vr p)} * \link{\vr p}{\vr c} * \lsg{\vr h}{\vr c}{L{++}\{\infty\}}{\li{null}}
		\end{array}	 
		\right\}\\
		
		
		\color{blue}//
		\left\{
		\begin{array}{@{} l @{}}
			\exsts{L_1, L_2, v_p, \pi} v_p < v \land \sorted{L_1 {++} \{v_p\}  {++}  L_2}  \land \\
			\isLock{\vr h}{\pi}  * \valueT{\vr h}{v} \\
			* \lsg{\vr h}{\vr p}{L_1}{\vr h} * \shared{\lockedNode{\vr p}{v_p}{\vr c}}{I(\vr p)} * \link{\vr p}{\vr c} * \lsg{\vr h}{\vr c}{L_2}{\li{null}}
		\end{array}
		\right\}\\
		
		\command{while(c.value} < v \command{)}\{ \\
		
			\begin{array}{l}
				\color{blue}//
				\left\{
				\begin{array}{@{} l @{}}
					\exsts{L_1, L_2, v_p, v_c, d, \pi} v_p < v_c < v \land  \sorted{L_1 {++}  \{v_p; v_c\} {++} L_2}  \land\\
					\isLock{\li h}{\pi}  * \valueT{\li h}{v} 
					* \lsg{\li h}{\li p}{L_1}{\li h} * 
				 	\shared{\lockedNode{\li p}{v_p}{\li c}}{I(\li p)} * \link{\li p}{\li c}\\ 
				 	* \shared{\node{\li c}{v_c}{d}}{I(\li c)} * \lsg{d}{\li{null}}{L_2}{\li h}
				 	
			 	\end{array}
			 	\right\}\\
			 	
			 	
			 	\command{lock(c.lock);}\\
			 	
			 	\color{blue}//
			 	\left\{
				\begin{array}{@{} l @{}}
			 		\exsts{L_1, L_2, v_p, v_c, d, \pi} v_p < v_c < v \land \sorted{L_1 {++}  \{v_p; v_c\} {++} L_2}  \land\\
			 		\isLock{\li h}{\pi}  * \valueT{\li h}{v} * 
					\lsg{\li h}{\li p}{L_1}{\li h} * 
			 		\shared{\lockedNode{\li p}{v_p}{\li c}}{I(\li p)} * \link{\li p}{\li c}\\
			 		* \locked{\li c} 
			 		* \shared{\node{\li c}{v_c}{d}}{I(\li c)} * \lsg{d}{\li{null}}{L_2}{\li h}
			 	\end{array}
			 	\right\}\\
			 	
			 	
			 	\color{blue}//
			 	\left\{
				\begin{array}{@{} l @{}}
			 		\exsts{L_1, L_2, v_p, v_c, d, \pi} v_p < v_c < v \land \sorted{L_1 {++}  \{v_p; v_c\} {++} L_2}  \land\\
			 		\isLock{\li h}{\pi}  * \valueT{\li h}{v} * 
					\lsg{\li h}{\li p}{L_1}{\li h} * 
			 		\shared{\lockedNode{\li p}{v_p}{\li c}}{I(\li p)} * \link{\li p}{\li c} \\ 
			 		* \shared{\lockedNode{\li c}{v_c}{d}}{I(\li c)} * \link{\li c}{d} * \lsg{d}{\li{null}}{L_2}{\li h}
			 	\end{array}
			 	\right\}\\
			 	
			 	
			 	\command{unlock(p.lock);}\\
			 	\command{p:= c ;}\\
			 	\command{c:= p.next ;}\\

				\color{blue}//
				\left\{
			 	\begin{array}{@{} l @{}}
				 	\exsts{L_1, L_2, v_p, c, \pi} v_p < v \land  \sorted{L_1 {++}  \{v_p\} {++}  L_2}  \land\\
				 	\isLock{\li h}{\pi} * \valueT{\li h}{v}
					* \lsg{\li h}{\li p}{L_1}{\li h} \\
				 	* \shared{\lockedNode{\li p}{v_p}{\li c}}{I(\li p)} 
				 	* \link{\li p}{c}
				 	* \lsg{\li c}{\li{null}}{L_2}{\li h}
			 	
			 	\end{array}
			 	\right\}
		
			\end{array}
		
	\end{array}\\
	\}\\
	
	\color{blue}//
	\left\{
 	\begin{array}{@{} l @{}}
	 	\exsts{L_1, L_2, v_p, v_c, \pi} v_p < v \leq v_c \land \sorted{L_1 {++}  \{v_p\} {++}   v_c::L_2}  \land\\
	 	\isLock{\li h}{\pi} * \valueT{\li h}{v}
		*\lsg{\li h}{\li p}{L_1}{\li h} \\
		
	 	* \shared{\lockedNode{\li p}{v_p}{\li c}}{I(\li p)} 
	 	* \link{\li p}{\li c}
	 	* \lsg{\li c}{\li{null}}{v_c ::L_2}{\li h}
 	
 	\end{array}
 	\right\}

\end{array}
\]
%
\hrule
\caption{Implementation of the set \li{locate} operation.}
\label{fig:set-locate}
\end{figure}
%
%
%\section*{Set Example}
%%
%\[
%\begin{array}{l}
%	\left\{\inSet{h}{v} \right\}\\
%	\command{locate(h, v) \{}\\
%	\begin{array}{l}
%		\command{p:= h}\\
%		\command{lock(p)}\\
%		\command{c:= p.next}\\
%		
%		\left\{
%		\begin{array}{@{} l @{}}
%			\exsts{L, \pi} \sorted{-\infty :: L ++ \{+\infty\}} \land v \in L \land  \lockT{h}[\pi] * \valueT{h}{v} * \\
%			 \shared{\lockedNode{p}{-\infty}{c}}{I(p)} * \unlockT{p} * \cell{p.next}{c} * \lockT{c} * \lsg{h}{c}{L++\{\infty\}}{\nil}
%		\end{array}	 
%		\right\}\\
%		
%		
%		\left\{
%		\begin{array}{@{} l @{}}
%			\exsts{L_1, L_2, v_p, \pi} \sorted{L_1 ++\{v_p\}  ++ L_2} \land v \in L_2 \land \lockT{h}[\pi] * \valueT{h}{v} * \\
%			\lsg{h}{h}{L_1}{p} * \shared{\lockedNode{p}{v_p}{c}}{I(p)} * \unlockT{p} * \cell{p.next}{c} * \lockT{c} * \lsg{h}{c}{L_2}{\nil}
%		\end{array}
%		\right\}\\
%		
%		\command{while(c.value < v)\{}\\
%			\begin{array}{l}
%				\left\{
%				\begin{array}{@{} l @{}}
%					\exsts{L_1, L_2, v_p, v_c, d, \pi} \sorted{L_1 ++ \{v_p; v_c\} ++L_2} \land v \in L_2 \land \lockT{h}[\pi] * \valueT{h}{v} * \\
%					\lsg{h}{h}{L_1}{p} * 
%				 	\shared{\lockedNode{p}{v_p}{c}}{I(p)} * \unlockT{p} * \cell{p.next}{c} * \lockT{c}\\ 
%				 	* \shared{\node{c}{v_c}{d}}{I(c)} * \lsg{h}{d}{L_2}{\nil}
%				 	
%			 	\end{array}
%			 	\right\}\\
%			 	
%			 	
%			 	\command{lock(c)}\\
%			 	
%			 	\left\{
%				\begin{array}{@{} l @{}}
%			 		\exsts{L_1, L_2, v_p, v_c, d, \pi} \sorted{L_1 ++ \{v_p; v_c\} ++L_2} \land v \in L_2 \land \lockT{h}[\pi] * \valueT{h}{v} * \\
%					\lsg{h}{h}{L_1}{p} * 
%			 		\shared{\lockedNode{p}{v_0}{c}}{I(p)} * \unlockT{p} * \cell{p.next}{c} * \lockT{c}\\ 
%			 		* \shared{\lockedNode{c}{v_c}{d}}{I(c)} * \lsg{h}{d}{L_2}{\nil}
%			 		* \unlockT{c} * \cell{c.next}{d} * \lockT{d}
%			 	\end{array}
%			 	\right\}\\
%			 	
%			 	\command{unlock(p)}\\
%			 	\command{p:= c ;}\\
%			 	\command{c:= p.next ;}\\
%
%				\left\{
%			 	\begin{array}{@{} l @{}}
%				 	\exsts{L_1, L_2, v_p, c, \pi} \sorted{L_1 ++ \{v_p\} ++ L_2} \land  v \in L_2 \land \lockT{h}[\pi] * \valueT{h}{v} * \\
%					\lsg{h}{h}{L_1}{p} 
%				 	* \shared{\lockedNode{p}{v_p}{c}}{I(p)} 
%				 	* \unlockT{p} * \cell{p.next}{c} * \lockT{c}
%				 	* \lsg{h}{c}{L_2}{\nil}
%			 	
%			 	\end{array}
%			 	\right\}
%		
%			\end{array}
%		
%	\end{array}\\
%	\}\\
%	\left\{
% 	\begin{array}{@{} l @{}}
%	 	\exsts{L_1, L_2, v_p, c, \pi} \sorted{L_1 ++ \{v_p\} ++ v :: L_2}  \land \lockT{h}[\pi] * \valueT{h}{v} * \\
%		\lsg{h}{h}{L_1}{p} 
%	 	* \shared{\lockedNode{p}{v_p}{c}}{I(p)} 
%	 	* \unlockT{p} * \cell{p.next}{c} * \lockT{c}
%	 	* \lsg{h}{c}{v ::L_2}{\nil}
% 	
% 	\end{array}
% 	\right\}
%
%\end{array}
%\]
%%
%

%%%
%%\[
%%\begin{array}{l}
%%	\left\{\inSet{h}{v} \right\}\\
%%	\command{locate(h, v) \{}\\
%%	\begin{array}{l}
%%		\command{p:= h}\\
%%		\command{lock(p)}\\
%%		\command{c:= p.next}\\
%%		
%%		\left\{
%%		\begin{array}{@{} l @{}}
%%			\exsts{L, \pi} \sorted{-\infty :: L ++ \{+\infty\}} \land v \in L \land  \lockT{h}[\pi] * \valueT{h}{v} * \\
%%			 \shared{\lockedNode{p}{-\infty}{c}}{I(p)} * \unlockT{p} * \cell{p.next}{c} * \lockT{c} * \lsg{h}{c}{L++\{\infty\}}{\nil}
%%		\end{array}	 
%%		\right\}\\
%%		
%%		
%%		\left\{
%%		\begin{array}{@{} l @{}}
%%			\exsts{L_1, L_2, v_p, \pi} \sorted{L_1 ++\{v_p\}  ++ L_2} \land v \in L_2 \land \lockT{h}[\pi] * \valueT{h}{v} * \\
%%			\lsg{h}{h}{L_1}{p} * \shared{\lockedNode{p}{v_p}{c}}{I(p)} * \unlockT{p} * \cell{p.next}{c} * \lockT{c} * \lsg{h}{c}{L_2}{\nil}
%%		\end{array}
%%		\right\}\\
%%		
%%		\command{while(c.value < v)\{}\\
%%			\begin{array}{l}
%%				\left\{
%%				\begin{array}{@{} l @{}}
%%					\exsts{L_1, L_2, v_p, v_c, d, \pi} \sorted{L_1 ++ \{v_p; v_c\} ++L_2} \land v \in L_2 \land \lockT{h}[\pi] * \valueT{h}{v} * \\
%%					\lsg{h}{h}{L_1}{p} * 
%%				 	\shared{\lockedNode{p}{v_p}{c}}{I(p)} * \unlockT{p} * \cell{p.next}{c} * \lockT{c}\\ 
%%				 	* \shared{\node{c}{v_c}{d}}{I(c)} * \lsg{h}{d}{L_2}{\nil}
%%				 	
%%			 	\end{array}
%%			 	\right\}\\
%%			 	
%%			 	
%%			 	\command{lock(c)}\\
%%			 	
%%			 	\left\{
%%				\begin{array}{@{} l @{}}
%%			 		\exsts{L_1, L_2, v_p, v_c, d, \pi} \sorted{L_1 ++ \{v_p; v_c\} ++L_2} \land v \in L_2 \land \lockT{h}[\pi] * \valueT{h}{v} * \\
%%					\lsg{h}{h}{L_1}{p} * 
%%			 		\shared{\lockedNode{p}{v_0}{c}}{I(p)} * \unlockT{p} * \cell{p.next}{c} * \lockT{c}\\ 
%%			 		* \shared{\lockedNode{c}{v_c}{d}}{I(c)} * \lsg{h}{d}{L_2}{\nil}
%%			 		* \unlockT{c} * \cell{c.next}{d} * \lockT{d}
%%			 	\end{array}
%%			 	\right\}\\
%%			 	
%%			 	\command{unlock(p)}\\
%%			 	\command{p:= c ;}\\
%%			 	\command{c:= p.next ;}\\
%%
%%				\left\{
%%			 	\begin{array}{@{} l @{}}
%%				 	\exsts{L_1, L_2, v_p, c, \pi} \sorted{L_1 ++ \{v_p\} ++ L_2} \land  v \in L_2 \land \lockT{h}[\pi] * \valueT{h}{v} * \\
%%					\lsg{h}{h}{L_1}{p} 
%%				 	* \shared{\lockedNode{p}{v_p}{c}}{I(p)} 
%%				 	* \unlockT{p} * \cell{p.next}{c} * \lockT{c}
%%				 	* \lsg{h}{c}{L_2}{\nil}
%%			 	
%%			 	\end{array}
%%			 	\right\}
%%		
%%			\end{array}
%%		
%%	\end{array}\\
%%	\}\\
%%	\left\{
%% 	\begin{array}{@{} l @{}}
%%	 	\exsts{L_1, L_2, v_p, c, \pi} \sorted{L_1 ++ \{v_p\} ++ v :: L_2}  \land \lockT{h}[\pi] * \valueT{h}{v} * \\
%%		\lsg{h}{h}{L_1}{p} 
%%	 	* \shared{\lockedNode{p}{v_p}{c}}{I(p)} 
%%	 	* \unlockT{p} * \cell{p.next}{c} * \lockT{c}
%%	 	* \lsg{h}{c}{v ::L_2}{\nil}
%% 	
%% 	\end{array}
%% 	\right\}
%%
%%\end{array}
%%\]
%%%
%%
%\[
%\begin{array}{l}
%	\left\{\ownSet{h}{v} \right\}\\
%	\command{locate(h, v) \{}\\
%	\begin{array}{l}
%		\command{p:= h}\\
%		\command{lock(p)}\\
%		\command{c:= p.next}\\
%		
%		\left\{
%		\begin{array}{@{} l @{}}
%			\exsts{L, \pi} \sorted{-\infty :: L ++ \{+\infty\}} \land  \lockT{h}[\pi] * \valueT{h}{v} * \\
%			 \shared{\lockedNode{p}{-\infty}{c}}{I(p)} * \unlockT{p} * \cell{p.next}{c} * \lockT{c} * \lsg{h}{c}{L++\{\infty\}}{\nil}
%		\end{array}	 
%		\right\}\\
%		
%		
%		\left\{
%		\begin{array}{@{} l @{}}
%			\exsts{L_1, L_2, v_p, \pi} v_p < v \land \sorted{L_1 ++\{v_p\}  ++ L_2}  \land \lockT{h}[\pi] * \valueT{h}{v} * \\
%			\lsg{h}{h}{L_1}{p} * \shared{\lockedNode{p}{v_p}{c}}{I(p)} * \unlockT{p} * \cell{p.next}{c} * \lockT{c} * \lsg{h}{c}{L_2}{\nil}
%		\end{array}
%		\right\}\\
%		
%		\command{while(c.value < v)\{}\\
%			\begin{array}{l}
%				\left\{
%				\begin{array}{@{} l @{}}
%					\exsts{L_1, L_2, v_p, v_c, d, \pi} v_p < v_c < v \land  \sorted{L_1 ++ \{v_p; v_c\} ++L_2}  \land \lockT{h}[\pi] * \valueT{h}{v} * \\
%					\lsg{h}{h}{L_1}{p} * 
%				 	\shared{\lockedNode{p}{v_p}{c}}{I(p)} * \unlockT{p} * \cell{p.next}{c} * \lockT{c}\\ 
%				 	* \shared{\node{c}{v_c}{d}}{I(c)} * \lsg{h}{d}{L_2}{\nil}
%				 	
%			 	\end{array}
%			 	\right\}\\
%			 	
%			 	
%			 	\command{lock(c)}\\
%			 	
%			 	\left\{
%				\begin{array}{@{} l @{}}
%			 		\exsts{L_1, L_2, v_p, v_c, d, \pi} v_p < v_c < v \land \sorted{L_1 ++ \{v_p; v_c\} ++L_2}  \land \lockT{h}[\pi] * \valueT{h}{v} * \\
%					\lsg{h}{h}{L_1}{p} * 
%			 		\shared{\lockedNode{p}{v_0}{c}}{I(p)} * \unlockT{p} * \cell{p.next}{c} * \lockT{c}\\ 
%			 		* \shared{\lockedNode{c}{v_c}{d}}{I(c)} * \lsg{h}{d}{L_2}{\nil}
%			 		* \unlockT{c} * \cell{c.next}{d} * \lockT{d}
%			 	\end{array}
%			 	\right\}\\
%			 	
%			 	\command{unlock(p)}\\
%			 	\command{p:= c ;}\\
%			 	\command{c:= p.next ;}\\
%
%				\left\{
%			 	\begin{array}{@{} l @{}}
%				 	\exsts{L_1, L_2, v_p, c, \pi} v_p < v \land  \sorted{L_1 ++ \{v_p\} ++ L_2}  \land \lockT{h}[\pi] * \valueT{h}{v} * \\
%					\lsg{h}{h}{L_1}{p} 
%				 	* \shared{\lockedNode{p}{v_p}{c}}{I(p)} 
%				 	* \unlockT{p} * \cell{p.next}{c} * \lockT{c}
%				 	* \lsg{h}{c}{L_2}{\nil}
%			 	
%			 	\end{array}
%			 	\right\}
%		
%			\end{array}
%		
%	\end{array}\\
%	\}\\
%	\left\{
% 	\begin{array}{@{} l @{}}
%	 	\exsts{L_1, L_2, v_p, c, \pi} v_p < v \leq v_c \land \sorted{L_1 ++ \{v_p\} ++  v_c::L_2}  \land \lockT{h}[\pi] * \valueT{h}{v} * \\
%		\lsg{h}{h}{L_1}{p} 
%	 	* \shared{\lockedNode{p}{v_p}{c}}{I(p)} 
%	 	* \unlockT{p} * \cell{p.next}{c} * \lockT{c}
%	 	* \lsg{h}{c}{v_c ::L_2}{\nil}
% 	
% 	\end{array}
% 	\right\}
%
%\end{array}
%\]
%%
%%
%\[
%\begin{array}{@{} l @{}}
%	\left\{ \ownSet{h}{v} \right \}\\
%	
%	\command{remove($h$, $v$)}\{\\
%	\begin{array}{l}
%		\left\{ \ownSet{h}{v} \right\}\\
%		
%		\command{($p$, $c$):= locate($h$, $v$); }\\
%		
%		\left\{
%	 	\begin{array}{@{} l @{}}
%		 	\exsts{L_1, L_2, v_p, \pi} v_p < v\leq v_c \land \sorted{L_1 ++ \{v_p\} ++ v_c :: L_2}  \land \lockT{h}[\pi] * \valueT{h}{v} * \\
%		 	
%			\lsg{h}{h}{L_1}{p} 
%		 	* \shared{\lockedNode{p}{v_p}{c}}{I(p)} 
%		 	* \unlockT{p} * \cell{p.next}{c} * \lockT{c}
%		 	* \lsg{h}{c}{v_c ::L_2}{\nil}
%	 	
%	 	\end{array}
%	 	\right\}\\
%	 	
%	 	\command{if ($c$.value == $v$) then\{ }\\
%	 	\begin{array}{l}
%	 	
%	
%	 	
%		 	\command{lock($c$) ; }\\
%		 	
%		 	
%		 	\left\{
%		 	\begin{array}{@{} l @{}}
%			 	\exsts{L_1, L_2, v_p, d, \pi} \sorted{L_1 ++ \{v_p\} ++ v_c :: L_2}  \land \lockT{h}[\pi] * \valueT{h}{v} * \\
%				\lsg{h}{h}{L_1}{p} 
%			 	* \shared{\lockedNode{p}{v_p}{c}}{I(p)} 
%			 	* \unlockT{p} * \cell{p.next}{c} * \lockT{c}\\
%			 	
%			 	* \shared{\lockedNode{c}{v}{d}}{I(c)} 
%			 	* \unlockT{c} * \cell{c.next}{d} * \lockT{d}
%			 	* \lsg{h}{d}{L_2}{\nil}
%		 	
%		 	\end{array}
%		 	\right\}\\
%		 	
%		 	\command{z:= $c$.next ;}\\
%		 	\command{$p$.next:= z ;}\\
%		 	
%		 	\left\{
%		 	\begin{array}{@{} l @{}}
%			 	\exsts{L_1, L_2, v_p, d} \sorted{L_1 ++ \{v_p\} ++ L_2}  \land \lockT{h}[\pi] * \valueT{h}{v} * \\
%				\lsg{h}{h}{L_1}{p} 
%			 	* \shared{\lockedNode{p}{v_p}{c}}{I(p)} 
%			 	* \unlockT{p} * \cell{p.next}{d} * \lockT{c}\\
%			 	
%			 	* \shared{\lockedNode{c}{v}{d}}{I(c)} 
%			 	* \unlockT{c} * \cell{c.next}{d} * \lockT{d}
%			 	* \lsg{h}{d}{L_2}{\nil}
%		 	
%		 	\end{array}
%		 	\right\}\\
%		 	
%		 	
%		 	\text{//\textsc{Merge} } \\
%		 	
%		 	\left\{
%		 	\begin{array}{@{} l @{}}
%			 	\exsts{L_1, L_2, v_p, d, \pi} \sorted{L_1 ++ \{v_p\} ++ L_2}  \land \lockT{h}[\pi] * \valueT{h}{v} * \\
%				\lsg{h}{h}{L_1}{p} 
%			 	* \shared{\lockedNode{p}{v_p}{c} * \lockedNode{c}{v}{d}}{I(p) \cup I(c)} 
%			 	* \unlockT{p} * \cell{p.next}{d} * \lockT{c}\\
%			 	
%			 	* \unlockT{c} * \cell{c.next}{d} * \lockT{d}
%			 	* \lsg{h}{d}{L_2}{\nil}
%		 	
%		 	\end{array}
%		 	\right\}\\
%		 	
%		 	\text{//Use } \valueT{h}{v} * \unlockT{p} \text{ cpability.}\\
%		 	
%		 	\left\{
%		 	\begin{array}{@{} l @{}}
%			 	\exsts{L_1, L_2, v_p, d, \pi} \sorted{L_1 ++ \{v_p\} ++ L_2}  \land \lockT{h}[\pi] * \valueT{h}{v} * \\
%				\lsg{h}{h}{L_1}{p} 
%			 	* \shared{\lockedNode{p}{v_p}{d}}{I(p) \cup I(c)} 
%			 	* \unlockT{p} * \cell{p.next}{d} * \lockT{c}\\
%			 	
%			 	* \lockedNode{c}{v}{d}
%			 	* \unlockT{c} * \cell{c.next}{d} * \lockT{d}
%			 	* \lsg{h}{d}{L_2}{\nil}
%		 	
%		 	\end{array}
%		 	\right\}\\
%		 	
%		 	
%		 	\text{//\textsc{Shift} + \textsc{Con}} \\
%		 	
%		 	
%		 	\left\{
%		 	\begin{array}{@{} l @{}}
%			 	\exsts{L_1, L_2, v_p, d, \pi} \sorted{L_1 ++ \{v_p\} ++ L_2}  \land \lockT{h}[\pi] * \valueT{h}{v} * \\
%				\lsg{h}{h}{L_1}{p} 
%			 	* \shared{\lockedNode{p}{v_p}{d} }{I(p)} 
%			 	* \unlockT{p} * \cell{p.next}{d} * \lockT{d}\\
%			 	
%			 	* \cell{c}{1, v, d}  
%			 	* \lsg{h}{d}{L_2}{\nil}
%		 	
%		 	\end{array}
%		 	\right\}\\
%		 	
%		 	
%		 	\command{unlock($p$) ;}\\
%		 	
%		 	
%		 	\left\{
%		 	\begin{array}{@{} l @{}}
%			 	\outSet{h}{v}
%				* \cell{c}{1, v, d}  
%		 	
%		 	\end{array}
%		 	\right\}\\
%		 	
%		 	\command{dispose($c$, 3) ; }\\
%		 	
%		 	
%		 	\left\{
%		 	\begin{array}{@{} l @{}}
%			 	\outSet{h}{v}
%		 	\end{array}
%		 	\right\}
%		 	
%		\end{array}\\
%		
%		\}\\
%		
%	\end{array}\\
%	
%	\}\\
%	
%	\left\{ \outSet{h}{v} \right\}
%	
%	
%\end{array}
%\]
%


%\[
%\begin{array}{@{} l @{}}
%	\left\{ \ownSet{h}{v} \right \}\\
%	
%	\command{add($h$, $v$)}\{\\
%	\begin{array}{l}
%		\left\{ \ownSet{h}{v} \right\}\\
%		
%		\command{($p$, $c$):= locate($h$, $v$); }\\
%		
%		\left\{
%	 	\begin{array}{@{} l @{}}
%		 	\exsts{L_1, L_2, v_p, \pi} v_p < v\leq v_c \land \sorted{L_1 ++ \{v_p\} ++ v_c :: L_2}  \land \lockT{h}[\pi] * \valueT{h}{v} * \\
%		 	
%			\lsg{h}{h}{L_1}{p} 
%		 	* \shared{\lockedNode{p}{v_p}{c}}{I(p)} 
%		 	* \unlockT{p} * \cell{p.next}{c} * \lockT{c}
%		 	* \lsg{h}{c}{v_c ::L_2}{\li{null}}
%	 	
%	 	\end{array}
%	 	\right\}\\
%	 	
%	 	\command{if ($c$.value != $v$) then $\{$ }\\
%	 	\begin{array}{l}
%	 		\left\{\emp\}\\
%	 		\command{z:= alloc(3);}\\
%	 		// \extendRule \text{ twice}\\
%	 		\left\{\}
%	 		\command{z.value:= $v$;}\\
%	 		\command{$p$.next:= z;}\\
%	 		
%		 	
%		 	
%		 	\left\{
%		 	\begin{array}{@{} l @{}}
%			 	\exsts{L_1, L_2, v_p, d, \pi} \sorted{L_1 ++ \{v_p\} ++ v_c :: L_2}  \land \lockT{h}[\pi] * \valueT{h}{v} * \\
%				\lsg{h}{h}{L_1}{p} 
%			 	* \shared{\lockedNode{p}{v_p}{c}}{I(p)} 
%			 	* \unlockT{p} * \cell{p.next}{c} * \lockT{c}\\
%			 	
%			 	* \shared{\lockedNode{c}{v}{d}}{I(c)} 
%			 	* \unlockT{c} * \cell{c.next}{d} * \lockT{d}
%			 	* \lsg{h}{d}{L_2}{\li{null}}
%		 	
%		 	\end{array}
%		 	\right\}\\
%		 	
%		 	\command{z:= $c$.next ;}\\
%		 	\command{$p$.next:= z ;}\\
%		 	
%		 	\left\{
%		 	\begin{array}{@{} l @{}}
%			 	\exsts{L_1, L_2, v_p, d} \sorted{L_1 ++ \{v_p\} ++ L_2}  \land \lockT{h}[\pi] * \valueT{h}{v} * \\
%				\lsg{h}{h}{L_1}{p} 
%			 	* \shared{\lockedNode{p}{v_p}{c}}{I(p)} 
%			 	* \unlockT{p} * \cell{p.next}{d} * \lockT{c}\\
%			 	
%			 	* \shared{\lockedNode{c}{v}{d}}{I(c)} 
%			 	* \unlockT{c} * \cell{c.next}{d} * \lockT{d}
%			 	* \lsg{h}{d}{L_2}{\li{null}}
%		 	
%		 	\end{array}
%		 	\right\}\\
%		 	
%		 	
%		 	\text{//\textsc{Merge} } \\
%		 	
%		 	\left\{
%		 	\begin{array}{@{} l @{}}
%			 	\exsts{L_1, L_2, v_p, d, \pi} \sorted{L_1 ++ \{v_p\} ++ L_2}  \land \lockT{h}[\pi] * \valueT{h}{v} * \\
%				\lsg{h}{h}{L_1}{p} 
%			 	* \shared{\lockedNode{p}{v_p}{c} * \lockedNode{c}{v}{d}}{I(p) \cup I(c)} 
%			 	* \unlockT{p} * \cell{p.next}{d} * \lockT{c}\\
%			 	
%			 	* \unlockT{c} * \cell{c.next}{d} * \lockT{d}
%			 	* \lsg{h}{d}{L_2}{\li{null}}
%		 	
%		 	\end{array}
%		 	\right\}\\
%		 	
%		 	\text{//Use } \valueT{h}{v} * \unlockT{p} \text{ cpability.}\\
%		 	
%		 	\left\{
%		 	\begin{array}{@{} l @{}}
%			 	\exsts{L_1, L_2, v_p, d, \pi} \sorted{L_1 ++ \{v_p\} ++ L_2}  \land \lockT{h}[\pi] * \valueT{h}{v} * \\
%				\lsg{h}{h}{L_1}{p} 
%			 	* \shared{\lockedNode{p}{v_p}{d}}{I(p) \cup I(c)} 
%			 	* \unlockT{p} * \cell{p.next}{d} * \lockT{c}\\
%			 	
%			 	* \lockedNode{c}{v}{d}
%			 	* \unlockT{c} * \cell{c.next}{d} * \lockT{d}
%			 	* \lsg{h}{d}{L_2}{\li{null}}
%		 	
%		 	\end{array}
%		 	\right\}\\
%		 	
%		 	
%		 	\text{//\textsc{Shift} + \textsc{Con}} \\
%		 	
%		 	
%		 	\left\{
%		 	\begin{array}{@{} l @{}}
%			 	\exsts{L_1, L_2, v_p, d, \pi} \sorted{L_1 ++ \{v_p\} ++ L_2}  \land \lockT{h}[\pi] * \valueT{h}{v} * \\
%				\lsg{h}{h}{L_1}{p} 
%			 	* \shared{\lockedNode{p}{v_p}{d} }{I(p)} 
%			 	* \unlockT{p} * \cell{p.next}{d} * \lockT{d}\\
%			 	
%			 	* \cell{c}{1, v, d}  
%			 	* \lsg{h}{d}{L_2}{\li{null}}
%		 	
%		 	\end{array}
%		 	\right\}\\
%		 	
%		 	
%		 	\command{unlock($p$) ;}\\
%		 	
%		 	
%		 	\left\{
%		 	\begin{array}{@{} l @{}}
%			 	\outSet{h}{v}
%				* \cell{c}{1, v, d}  
%		 	
%		 	\end{array}
%		 	\right\}\\
%		 	
%		 	\command{dispose($c$, 3) ; }\\
%		 	
%		 	
%		 	\left\{
%		 	\begin{array}{@{} l @{}}
%			 	\outSet{h}{v}
%		 	\end{array}
%		 	\right\}
%		 	
%		\end{array}\\
%		
%		\}\\
%		
%	\end{array}\\
%	
%	\}\\
%	
%	\left\{ \outSet{h}{v} \right\}
%	
%	
%\end{array}
%\]

%
%
%
%
%
%
\clearpage\subsection*{\colosl vs. CAP Specification}
The definitions of most of the predicates in \fig\ref{fig:coloslSetExample} are analogous to the correspondingly-named CAP predicates specified in~\cite{cap-ecoop10}. The difference however lies in the definitions of the $\textsf{L}_{\in}$, $\textsf{L}_{\not\in}$, and \textsf{lsg} predicates. 
\fig~\ref{fig:capSetExample} presents the CAP definitions of these predicates. 

There are two main advantages to the \colosl\ specification of the set module. In the CAP specification, all elements of the set (represented as a singly-linked list) reside in a \emph{single} shared region labelled $s$.
Each node at a given address $x$, is associated with an (\emph{infinite}) set of update capabilities of the form $\capGapT{x}{y}{v}$ for \emph{all} possible addresses $y$ and \emph{all} possible values $v$. This is to capture all potential successor addresses $y$ and all potential values $v$ that may be stored at address $x$. For instance, since the value $v'$ may be contained in the list at address $a$ before address $b$, there must exist an update capability $\capGapT{a}{b}{v'}$ to accommodate possible modifications of value $v'$ with respect to $a$ and $b$. In order to modify a node, a thread can acquire the lock associated with the node and subsequently claim the relevant update capability. Since the capabilities associated with a region are \emph{all} allocated at the point of region creation, the shared region is required to keep track of \emph{all} (infinitely many) possible update capabilities $\capGapT{x}{y}{v}$ associated with all addresses $x$, all possible successor addresses $y$ and all values $v$. This is captured by the $\gapsCAP{S}{s}$ and $\myGapsCAP{S}{s}$ predicates through the infinite multiplicative star operator $\iterStar$. The $\gapsCAP{S}{s}$ predicate uses an auxiliary mathematical set $S$ to track those nodes of the list that are currently locked and thus infer which \textsc{LGap} capabilities reside in the shared state and which ones have been removed. This results in an inevitably cluttered and a rather counter-intuitive specification since it seems unnatural to account for the capabilities pertaining to addresses not in the domain of the set. 

On the other hand, in the \colosl\ specification we use a combination of the $\modT{x}$ and $\nextT{x}{y}$ capabilities to achieve the same effect for each address $x$. Since \colosl\ allows for dynamic extension of the shared state, the capabilities associated with each address $x$ are generated upon \emph{extension} of the shared state with those addresses, thus avoiding the need to account for capabilities concerning those addresses absent from the list. 

Moreover, since the fractional heaps used to represent the separation algebra of capabilities are \emph{stateful}, rather than having a distinct capability to modify the element at address $x$ before address $y$ for each possible successor address $y$, we appeal to a single capability of the form $\nextT{x}{y}$ whereby the capability is modified accordingly to reflect the changes to the successor address. For instance, when the node at address $x$ is redirected to point from address $y$ to a new location at address $z$, $\nextT{x}{y}$ is also updated to $\nextT{x}{z}$. 

%
\begin{figure*}
\hrule
\[
\begin{array}{@{} l @{}}
	\begin{array}{@{} r @{\hspace*{3pt}} l @{}}
	
		\sortedListCAP{\triangleleft}{h}{v}{s} \eqdef & \exsts{L, S} \sorted{L} \land v  \triangleleft L \land \lsgCAP{h}{\li{null}}{S}{-\infty:: L ++ \{\infty\} }\\
		&  * \shade{\left( \gapsCAP{S}{s} \land \myGapsCAP{v}{s} \right) } \qquad \text{ where } \triangleleft = \in \text{ or } \triangleleft = \not\in \vspace{7pt}\\
	  
	
		\lsg{x}{z}{S}{L} \eqdef & (L = [] \land S = \emptyset \land x = z \land \emp) \lor\\
		&  (\exsts{y, v, L'} L = v::L' \land \unlockedNode{x}{v}{y} * \lsg{y}{z}{S}{L'}) \lor\\
		& 
		\left(
		\begin{array}{@{} l @{\hspace{2pt}} l @{}}
			\exsts{y, v, L', S'} & L = v::L' \land S = \{(x, y)\} \uplus S' \land \\
			& \lockedNode{x}{v}{y} * \lsg{y}{z}{S'}{L'}
		\end{array}
		\right)
		
	\end{array} \vspace{7pt}\\
	
		
		
	\shade{
	\begin{array}{@{} r @{\hspace*{3pt}} l @{}}
		\gapsCAP{S}{s} \eqdef & \iterStar(x, y) \not\in S. \iterStar v. \capGapC{x}{y}{v}{s} *\\
		& \iterStar (x, y) \in S. \exsts{w} \iterStar v \not= w. \capGapC{x}{y}{v}{s} \land \\
		&\for{x, y, w, z} (x, y) \in S \land (w, z) \in S \!\implies\! (x = w \!\iff\! y = z) \vspace{7pt}\\
		
	  \myGapsCAP{v}{s} \eqdef & \iterStar x, y. \capGapC{x}{y}{v}{s} * true \\
		
	\end{array}
	}
\end{array}
\]
\hrule
\caption{CAP predicates of the set module that contrast with the \colosl\ specification of \fig\ref{fig:coloslSetExample}.}
\label{fig:capSetExample}
\end{figure*}



